%%% Setup
%%% Dokumentenklasse Artikel
%%% 12pt
%%% A4
%%% Titelseite
%%% Nummerierung ohne endende Punkte zB. 1. ; 1.1.
%%% einseitig
\documentclass[fontsize=12pt,paper=a4,titlepage,numbers=noenddot,twoside=false,bibliography=totocnumbered,toc=listof,hidelinks,listof=totocnumbered]{scrartcl}

%%% Schriftart, Kodierung und Sprache
\usepackage[T1]{fontenc}
\usepackage[utf8]{inputenc}
\usepackage[ngerman]{babel}

%%% Schriftart
%\usepackage{ae}
\usepackage{times}

%%% Space :)
\usepackage{setspace}

%%% Multirow in Tabellen (miktex only?)
\usepackage{multirow}

%%% Longtable
\usepackage{longtable}

%%% Grapfikpaket
\usepackage{graphicx}

%%% Randeinstellungen
\usepackage[left=30mm,right=30mm,bottom=25mm,top=25mm]{geometry}

%%% Blindtext
\usepackage{lipsum}

%%% Footer und Header
%% headsepline - Headerlinie
%% footsepline - Fusslinie
%% mehr unter http://get-software.net/macros/latex/contrib/koma-script/doc/scrguide.pdf
\usepackage[headsepline,footsepline,automark]{scrlayer-scrpage}

%%% Formelkram falls gebraucht
%\usepackage{amsmath}
%\usepackage{amsfonts}
%\usepackage{amssymb}

%%% Listen
\usepackage{enumerate}
\usepackage{listings}
\usepackage{enumitem}

%%% Listeneinträge sollen alle mit Bullets anfangen
%\renewcommand{\labelitemi}{$\bullet$}
%\renewcommand{\labelitemii}{$\bullet$}
%\renewcommand{\labelitemiii}{$\bullet$}
%\renewcommand{\labelitemiv}{$\bullet$}

%%% Für Programm/Code Listings
\renewcommand{\lstlistingname}{Programm}

%%% Gepunktete Linie im Inhaltsverzeichnis entfernen
\usepackage[titles]{tocloft}
\renewcommand{\cftdot}{}

%%% (Kleinere) Randnotizen
\usepackage{marginnote}
\renewcommand*{\marginfont}{\footnotesize}

%%% Farbenkram
%\usepackage{color}

%%% Gedichte
\usepackage{verse}

%%% pdf zusätzlich einbinden
\usepackage{pdfpages}

%%% PDF Hyperlinks
\usepackage[ngerman,plainpages=false,pdfpagelabels]{hyperref}

%%% biblatex (Quellenverz)
%%% mehr unter http://biblatex.dominik-wassenhoven.de/download/DTK-2_2008-biblatex-Teil1.pdf
%%% benutze biber.exe anstelle von bibtex.exe (miktex)
\usepackage[backend=biber,bibencoding=utf8,style=authortitle-icomp,bibstyle=numeric,pagetracker=false,isbn=false]{biblatex}
\usepackage[babel,german=quotes]{csquotes}
\setlength{\bibitemsep}{1em}

%%% Das Quellenverz. z.B. mit JabRef erstellt. 
\bibliography{bibliography}
\addbibresource{bibliography.bib}

\DeclareNameAlias{sortname}{last-first}
\DeclareNameAlias{default}{last-first}

\AtBeginBibliography{%
  \renewcommand*{\multinamedelim}{\addsemicolon\space}
  \renewcommand*{\finalnamedelim}{\addsemicolon\space}
}

%%% Zeilenabstand 1,5
\onehalfspacing

%%% Extra Kommandos
%% \punkte -> ...
\newcommand*{\punkte}{\dots\unkern}

%\renewcommand\listoffigures{%
    %\section{\listfigurename}% Used to be \section*{\listfigurename}
      %\@mkboth{\MakeUppercase\listfigurename}%
              %{\MakeUppercase\listfigurename}%
    %\@starttoc{lof}%
    %}

\hypersetup{
	pdftitle    = {Schüler haben\grq s schwer, Lehrer aber auch. Außerunterrichtliche Beratungs- und Unterstützungsbedarfe von SchülerInnen im medizinischen und sozialen Bereich und daraus abgeleitete sozialpädagogische Angebote},
	pdfsubject  = {Ermittlung möglicher Problemlagen von Schülern am DRK-Bildungswerk Sachsen; zudem die Sicht der Lehrer auf diese Problematik und daraus schlussfolgernd die Konzeption möglicher Beratungs- und Unterstützungsangebote},
	pdfauthor   = {Doreen Stichel, Daniela Wobst},
	pdfkeywords = {Schulsozialarbeit, Berufsbildende Schulen, Sachsen, DRK-Bildungswerk Sachsen},
	pdfcreator  = {pdflatex},
	pdfproducer = {LaTeX with hyperref}
}

%%% Hier beginnt das Dokument
\begin{document}

\KOMAoption{headsepline}{false} 
\KOMAoption{footsepline}{false}

%%% römische nummerierung
\pagenumbering{roman}

\setcounter{secnumdepth}{5}
\setcounter{tocdepth}{5}

%%% Footer und Header clearen
\clearscrheadfoot

%%% Titelseite
\begin{titlepage}
\begin{center}
\normalsize
\scshape
Technische Universität Dresden\\
\normalsize
\upshape
Fakultät Erziehungswissenschaften\\
Institut für Berufspädagogik und Berufliche Didaktiken\\
Professur für Sozialpädagogik einschließlich ihrer Didaktik\\[2,0cm]
\end{center}

\begin{center}
\Huge
\scshape
\bfseries
Masterarbeit\\[0,5cm]
\normalsize
\mdseries
im Fach Sozialpädagogik\\
\noindent\rule{\textwidth}{1pt}\\[1cm]
\normalsize
\mdseries
\textbf{Schüler haben's schwer}\\
\textbf{Lehrer aber auch}\\[0,5cm]
\normalsize
\mdseries
Außerunterrichtliche Beratungs- und Unterstützungsbedarfe von SchülerInnen im medizinischen und sozialen Bereich\\ und daraus abgeleitete sozialpädagogische Angebote\\
-\\
eine explorative Analyse unter Einbeziehung der Lehrerperspektive am DRK Bildungswerk Sachsen\\[2,0cm]
\end{center}

\begin{center}
\textbf{eingereicht von:}\\
Doreen Stichel\\
Daniela Wobst\\[1,0cm]

\textbf{Erstgutachter:}\\
Prof. Dr. rer. soc. habil. Hans Gängler\\[1,0cm]
\textbf{Zweitgutachter:} \\
Heike Haupt\\[1,0cm]

Dresden, \today
\end{center}

\end{titlepage}

\begin{flushright}
	\includegraphics[scale=0.5]{images/TU-Logo.png}
\end{flushright}

\noindent
\textbf{Doreen Stichel}\\[0,5cm]
\begin{tabular}{lll}
geboren am: & 10.02.1984\\[0,5cm]
Anschrift: & Fetscherstrasse 7\\
& 01307 Dresden\\[0,5cm]
E-Mail: & \flq{}doreens84@aol.com\frq{}\\[0,5cm]
Studiengang: & Höheres Lehramt an berufsbildenden Schulen\\
Berufliche Fachrichtung: & Gesundheit und Pflege\\
Zweites studiertes Fach: & Sozialpädagogik\\[0,5cm]
Matrikelnummer: & 3686656\\[0,5cm]
Immatrikuliert am: & 01.10.2010\\[2cm]
\end{tabular}

\noindent
\textbf{Daniela Wobst}\\[0,5cm]
\begin{tabular}{lll}
geboren am: & 19.06.1979\\[0,5cm]
Anschrift: & Grillparzerstrasse 28\\
& 01157 Dresden\\[0,5cm]
E-Mail: & \flq{}daniela.wobst@aol.com\frq{}\\[0,5cm]
Studiengang: & Höheres Lehramt an berufsbildenden Schulen\\
Berufliche Fachrichtung: & Gesundheit und Pflege\\
Zweites studiertes Fach: & Sozialpädagogik\\[0,5cm]
Matrikelnummer: & 3686834\\[0,5cm]
Immatrikuliert am: & 01.10.2010
\end{tabular}

\include{danksagung}


\begin{center}
	\textbf{Schüler haben\grq s schwer}
\end{center}

\begin{figure}[h]
	\centering
		\includegraphics[width=0.4\textwidth]{images/Lasten-eines-Schuelers.jpg}\\
		\footnotesize Quelle: sonstwoher
	\caption{Schüler}
	\label{fig:Lasten-eines-Schuelers}
\end{figure}

\begin{center}
	\textbf{Lehrer aber auch}
\end{center}

\begin{figure}[h]
	\centering
		\includegraphics[width=0.60\textwidth]{images/karikatur-lehrer-sommerferien.jpg}\\
		\footnotesize Quelle: sonstwoher
	\caption{Lehrer}
	\label{fig:karikatur-lehrer-sommerferien}
\end{figure}







%%%-Hinweis
\vspace*{\fill}

\begin{flushleft}
\textbf{(1) Hinweis zur geschlechtlichen Benennung}
\end{flushleft}
Die unzureichende Bezeichnung weiblicher und männlicher Personen in dieser Arbeit bringt die verfassungsrechtlich gebotene Gleichstellung von Mann und Frau sprachlich nicht angemessen zum Ausdruck. Auf die Verwendung von Doppelformen oder andere Kennzeichnungen für weibliche und männliche Personen wird jedoch verzichtet, um die Lesbarkeit und Übersichtlichkeit zu wahren. Mit allen im Text verwendeten Personenbezeichnungen sind stets beide Geschlechter gemeint.

\begin{flushleft}
\textbf{(2) Hinweis zur beiliegenden CD}
\end{flushleft}
Die beiliegende CD enthält sowohl die PDF-Version dieses Dokumentes als auch die vollständigen Transkripte und Audiodateien der Interviews, welche für die Erarbeitung der Inhalte herangezogen wurden.

\begin{flushleft}
\textbf{(3) Hinweis zur Dokumenterstellung (\LaTeX\ und GitHub)}
\end{flushleft}
Die vorliegende Masterarbeit wurde mit \LaTeX\ erstellt. Zur Versionierung und zum Zweck der Datensicherung wurde der Quellcode des Dokumentes bei GitHub hochgeladen und liegt dort einsehbar unter folgender Adresse vor:\\
\url{https://github.com/virtualdreams/ma-schulsozialarbeit}

%%% Inhaltsverzeichnis
%%%Inhaltsverzeichnis
\pdfbookmark[1]{Inhaltsverzeichnis}{toc}
\tableofcontents


\noindent
\textbf{Zuordnung der Kapitel}\\[0,5cm]
Gemäß §21 Absatz 5 der Prüfungsordnung für den konsekutiven Master-Studiengang Höheres Lehramt an berufsbildenden Schulen kann die Masterarbeit auch in Form einer Gruppenarbeit erfolgen, solange der zu bewertende Einzelbeitrag des Studierenden kenntlich gemacht wird.\\
Dieser Anforderung wird mittels folgender Kennzeichnung entsprochen.\\

\noindent
Doreen Stichel: S\\
Daniela Wobst: W\\

\begin{description}[nosep]
	\item[Kapitel 1:] S/W
	\item[Kapitel 2:] W
	\item[Kapitel 3:] W
	\item[Kapitel 4:] W
	\item[Kapitel 5:] S
	\item[Kapitel 6:] S
	\item[Kapitel 7:] S
	\item[Kapitel 8:] S
	\item[Kapitel 9:] W
	\item[Kapitel 10:] S/W	
\end{description}



\textbf{Abkürzungsverzeichnis}\\[1cm]


\begin{tabular}{lll}
ABS: & Allgemeinbildende Schule\\[0,5cm]
BBS: & Berufsbildende Schule\\[0,5cm]
DRK BWK SN: & Deutsches Rotes Kreuz Bildungswerk Sachsen\\[0,5cm]
SSA: & Schulsozialarbeit\\
\end{tabular}

\KOMAoption{headsepline}{true} 
\KOMAoption{footsepline}{true} 

%%% Seitenzahl ab hier immer rechts
\ofoot{\pagemark}
\ohead{Masterarbeit}
\ihead{\headmark}
\pagestyle{scrheadings}

%%% arabische Nummerierung
\pagenumbering{arabic}

%%% Inhalte
\section{Einleitung}
\label{sec:Einleitung}

Die Lehrerin einer Berufsfachschule für Pflegehilfe tritt nachmittags gegen 15.30 Uhr sichtlich erschöpft aus dem Unterrichtsraum:
\begin{quotation}
\noindent
"`Diese Schülerinnen und Schüler, es ist kaum noch mit ihnen fertig zu werden! Die erste Unterrichtseinheit brauche ich schon für die Klärung von Schulgeldproblemen, danach besprechen wir zum tausendsten Male die Schwierigkeiten im Praktikum und die Unpünktlichkeit oder die unentschuldigten Fehlzeiten. Sobald ich dann endlich mit fachlichen Inhalten anfangen will, ist die Aufmerksamkeit und Disziplin schon so schlecht, dass ich meine Unterrichtsplanung völlig über Bord werfen muss und wieder nicht vorwärts komme. Die Klasse hat doch bald Prüfung! Wir brauchen dringend jemanden, der sich mal mit diesen Problemen beschäftigt [\punkte]!"'
\end{quotation}

\noindent
Diese tatsächlich erlebte Situation ist eine, die sich täglich vermutlich mehrfach in beruflichen Schulen in Sachsen so oder ähnlich abspielen könnte. Sie war daher unter anderem ausschlaggebend, sich im Rahmen der vorliegenden Masterarbeit mit den Problemlagen von Schülern und den sich daraus möglicherweise ergebenden Bedarfen für außerunterrichtliche Beratungs- und Unterstützungsangebote an berufsbildenden Schulen in den Fachrichtungen Gesundheit, Pflege und Sozialwesen auseinanderzusetzen. 

"`Über kaum ein Thema wird in den letzten Jahren so kontrovers diskutiert, wie über die Schule."' stellt unter anderem Drilling \footcite[9]{Drilling2004} fest und auch am berufsbildenden Bereich gehen konzeptionelle, strukturelle, curriculare und fachliche Debatten keinesfalls spurlos vorüber. Im besonderen Spannungsfeld von Theorie und Praxis diagnostizieren und kritisieren die Praktiker eine angeblich zunehmende Ausbildungsunreife der Jugendlichen und eine Nichteignung für die Ausübung eines Berufes (vgl. u. a. Ehrenthal 2005 \footcite[vgl.]{Ehrenthal2005}), während insbesondere die Lehrkräfte der Schulen sich über verstärkende und vielfältige soziale Problemlagen und Lernschwierigkeiten beklagen (vgl. u. a. SMS 2009 \footcite[vgl.][13]{SMSSS2009}). Mit der Aussage [\punkte]

\begin{quotation}
\noindent
"`Erwachsenwerden in der heutigen Zeit fällt vielen Jugendlichen schwer. Der Übergang von der Schule in die Berufswelt ist vielfach von Brüchen gekennzeichnet, familiäre Geborgenheit findet in ökonomischen Zwängen ihre Grenzen, nicht wenige Jugendliche wachsen im Widerspruch zwischen Erlebniswelt und Erfahrungsrealität auf. Die Gesellschaft schwankt zwischen Dramatisierung und Verharmlosung, schiebt mal den Eltern, mal der Schule, mal der Politik die Schuld zu."' 
\end{quotation}

\noindent
[\punkte] versucht Drilling \footcite[19]{Drilling2004} eine kurze Beschreibung dessen, was heute vielfach als "`gesellschaftlicher Wandel"' bezeichnet und diskutiert wird sowie vielfältige Herausforderungen, insbesondere für die berufliche Bildung, mit sich bringt. Als Ergebnis scheinen diese gesellschaftlichen Veränderungsprozesse immer mehr sogenannte "`schwierige Jugendliche"', also junge Menschen in komplexen Problemsituationen, hervorzubringen, mit deren Umgang die Lehrkräfte im berufsbildenden Bereich zunehmend überfordert sind \footcite[vgl.][1]{UniversitaetLeipzig2007}. Titelgemäß könnte man durchaus etwas provokativ fragen: Haben es Schüler also heute besonders schwer? Und Lehrer aber auch!? 

Die eben geschilderten gesellschaftlichen und schulischen Veränderungen lassen verständlicherweise immer wieder den Ruf nach sozialpädagogischer Unterstützung laut werden, der jedoch scheinbar selten wirklich gehört wird bzw. nur vereinzelt in tatsäch\-lichen Angeboten an berufsbildenden Schulen eine wirksame Umsetzung findet. Um diese Annahme zu untermauern, lohnt sich ein Blick in die Übersicht zu Angeboten Sozialer Arbeit an Schulen im Freistaat Sachsen, aus der ersichtlich wird, dass im Jahr 2014 an zwei beruflichen Schulen in öffentlicher Trägerschaft (von im Schuljahr 2014/15 vorhandenen 78) Projekte der Schulsozialarbeit umgesetzt wurden, obwohl eine Bedarfsanalyse von 2004 ergeben hatte, dass in mindestens 43 \% der Schulen Bedarfe vorhanden waren \footcites[vgl.][5]{LSS2004}[vgl.][5ff]{SMSSSV2014}. Augenscheinlich besteht also entweder eine Diskrepanz von Angebot und Nachfrage oder Schüler an berufsbildenden Schulen haben deutlich weniger Problemlagen als subjektiv wahrgenommen und somit gar keinen so relevanten Bedarf an Unterstützungsangeboten!?

Die Praxiserfahrungen der Verfasserinnen als angehende Berufspädagoginnen in den Fachrichtungen Gesundheit, Pflege und Sozialpädagogik lassen jedoch die Vermutung zu, dass auch dieses Berufsfeld von den oben geschilderten Entwicklungen und Veränderungen stark betroffen zu sein scheint. Aus diesen Feststellungen und den Überlegungen bezüglich der Relevanz des Themas für die spätere berufliche Tätigkeit entstand die Idee, mit einer eigenen Forschungstätigkeit die tatsächlichen Problemlagen der viel zitierten "`schwierigen Schüler"' genauer in den Blick zu nehmen. Verstärkt wurde diese Absicht durch die Erkenntnis, dass spezifische wissenschaftliche Forschungsergebnisse für die genannten Fachrichtungen nicht vorhanden sind bzw. nicht aufgefunden werden konnten. Vor dem Beginn der eigenen Forschungstätigkeit bestand jedoch die Herausforderung, theoretische Grundlagen zu den Aspekten der Problemlagen und möglicher sozialpädagogischer Unterstützungsangebote für Schüler in beruflichen Ausbildungen der Gesundheit und Pflege sowie des Sozialwesens zu recherchieren und herauszuarbeiten. Dabei ergab sich schnell, dass die Schulsozialarbeit als theoretischer Bezugsrahmen für die sozial\-pä\-da\-gogische Arbeit in den beruflichen Schulen relevant und demzufolge in der Betrachtung unerlässlich ist. Allerdings wurde dabei klar, dass der berufsbildende Bereich in der ausgewählten Literatur zur Schulsozialarbeit sehr unterrepräsentiert ist und deshalb die theoretischen Grundlagen an den entsprechenden Stellen an die Spezifika bzw. die speziellen Anforderungen angepasst werden mussten. Wesentliche Aussagen wurden daher häufig aus den Darstellungen für die allgemeinbildende Schule abgeleitet. Die gleiche Problematik ergab sich hinsichtlich der Analyse von Problemlagen junger Menschen, die eine Berufsausbildung in den Fachrichtungen Gesundheit, Pflege und Sozialwesen absolvieren. Es liegen zwar insgesamt zahlreiche Publikationen vor, die jedoch entweder sehr allgemein die Probleme Jugendlicher oder insbesondere die Teilnehmer an berufsvorbereitenden Maßnahmen oder Berufsvorbereitungsjahren in den Blick nehmen. Fachrichtungsspezifisch sind weder Problemanalysen noch theoretische Grundlagen oder Konzepte zum Umgang mit den komplexen Problemlagen der Schüler vorhanden. Auf der Basis dieser Erkenntnisse entstanden die ersten Ideen hinsichtlich der erkenntnisleitenden Fragen dieser Arbeit, die zu Beginn vorrangig die Probleme aus der Schülersicht beinhalteten. Zusätzlich bestand die Herausforderung, eine geeignete Schule für das Forschungsvorhaben auszuwählen. Aufgrund des bereits abgegrenzten Berufsfeldes kam dahingehend nur ein Berufliches Schulzentrum für Gesundheit und Soziales oder eine Einrichtung in freier Trägerschaft in Frage, an der die genannten Fachrichtungen gemeinsam vertreten sind. Da eine Forschungsarbeit an einer öffentlichen Schule mit nicht unerheblichen organisatorischen Hürden und Beantragungsverfahren verbunden ist, fiel die Entscheidung relativ schnell zugunsten einer Schule in freier Trägerschaft, wobei das Bildungswerk des Deutschen Rotes Kreuz in Sachsen (nachfolgend nur noch als DRK Bildungswerk SN benannt) durch das langjährige Angestelltenverhältnis einer der Verfasserinnen als geeignet er\-schien. Zeitgleich mit diesen ersten Überlegungen ergaben sich Interessensbekundungen der Geschäftsführung des DRK Bildungswerkes SN an der Thematik, da dort ein verstärktes Auftreten von Schülerproblemen ebenfalls registriert wurde. Zusätzlich verstärkte die Feststellung, dass dahingehend an dieser Bildungsinstitution bisher keine Unterstützungsangebote vorhanden waren, erste Überlegungen, diese Situation möglicherweise zu verändern. Die Ergebnisse der anvisierten Forschungsarbeiten wurden als eventuell dazu nutzbar eingestuft. Da die Einrichtung mit ca. 650 Schülern auch von der Größe her für das Vorhaben geeignet erschien, wurde die Entscheidung getroffen, die Forschungsarbeit dort durchzuführen und auch mögliche konzeptionelle Ansätze auf diese zu fokussieren. Es sind also sowohl die analytischen Ergebnisse, als auch die abgeleiteten Erkenntnisse als exemplarisch für das DRK Bildungswerk SN einzuordnen. Trotzdem sind diese in Ansätzen sicherlich auf andere Schulen bzw. den beruflichen Bereich Gesundheit, Pflege und Sozialwesen übertragbar. 

Neben der primär in den Fokus genommenen Schülersicht auf Problemlagen und mögliche Unterstützungsbedarfe erschien zunehmend auch die Lehrersicht unerlässlich, um ein besseres Gesamtbild zur Thematik zu erhalten. Es erfolgte eine kontinuierliche Beschäftigung mit den möglichen erkenntnisleitenden Fragestellungen, aus denen sich letztendlich die folgenden Forschungsfragen differenzierten:

\begin{enumerate}
	\item Welche persönlichen und sozialen (außerunterrichtlichen) Problemlagen haben Schüler des DRK Bildungswerk SN in sozialen und medizinischen Ausbildungsberufen, die das Unterrichtsgeschehen und den Ausbildungserfolg beeinflussen?
	\item Welche persönlichen und sozialen (außerunterrichtlichen) Problemlagen von Schüler nehmen Lehrkräfte des DRK Bildungswerk SN im Bereich Gesundheit, Pflege und Sozialwesen als besondere Belastung für den Unterricht wahr?
	\item Wie schätzen Schüler und Lehrkräfte anhand der (möglichen) subjektiv wahrgenommenen Problemlagen den Bedarf an sozialpädagogischen und anderweitigen Unterstützungs- und Beratungsangeboten ein und wie könnten passende Angebote aussehen?
\end{enumerate}

\noindent
An dieser Stelle ist darauf hinzuweisen, dass fachliche bzw. unterrichtlich-inhaltliche Problemstellungen ausdrücklich nicht im Fokus der Forschungsarbeit standen, sondern hauptsächlich auf soziale und persönliche Problemlagen eingegangen wurde. Das ist schwerpunktmäßig damit zu begründen, dass die verschiedenen Fachrichtungen berufsspezifisch verschiedene Fachkompetenzbereiche aufweisen, die innerhalb einer interdisziplinären Schülerbefragung inhaltlich nur schwer abzubilden sind. Allgemeine und übergreifende Themen, wie z. B. Überforderung oder Prüfungsangst, wurden jedoch mit berücksichtigt.\\

\noindent
Die Zielstellungen der eigenen Forschungsarbeit waren, insbesondere herauszuarbeiten, 
\begin{itemize}
	\item ob und inwieweit die der Fachliteratur entnommenen allgemeinen Problemlagen Jugendlicher auf den Bereich Gesundheit, Pflege und Soziales zutreffen.
	\item wie die Problemlagen im Einzelnen repräsentiert und verteilt sind.
	\item ob Schüler und Lehrer gleiche bzw. ähnliche Problemlagen wahrnehmen.
	\item inwieweit die Problemlagen das Unterrichtsgeschehen konkret beeinflussen.
	\item ob Bedarfe für  Unterstützungsangebote überhaupt vorhanden sind.
	\item wie mögliche Unterstützungsangebote konkret aussehen könnten. 
\end{itemize}

\noindent
Zur Beantwortung der Forschungsfragen und zur Realisierung der genannten Zielstellungen wurden eine quantitative Befragung an einer Stichprobe von 175 Schülern mittels eines Fragebogens und eine qualitative Studie in Form von fünf Lehrerinterviews durchgeführt. Diese Methoden wurden ausgewählt, da sie hinsichtlich der Betrachtung beider Perspektiven die größtmöglichen Erkenntnisgewinne versprachen. Mittels eines Fragebogens konnte in einem überschaubaren Zeitrahmen eine relativ große Anzahl von Probanden befragt werden, wobei durch eine geschickte Auswahl der Befragungsgruppe gleichzeitig alle Fachrichtungen am DRK Bildungswerk SN berücksichtigt werden konnten. Zusätzlich musste für die Befragung der Schüler eine relativ einfach auswertbare Methode gewählt werden. Der Fokus lag dabei auf der Eruierung der persönlichen Problemlagen und auf dem Erkenntnisgewinn hinsichtlich der Nutzung potentieller Unterstützungsangebote. Da aus der Lehrersicht durch die pädagogischen Fachkompetenzen tendenziell differenziertere und spezifischere Wahrnehmungen zu den Problemlagen und zu möglichen Unterstützungsangeboten zu erwarten waren, eignete sich eine Befragung mittels Fragebogen nicht, so dass das teilstandardisierte Leitfadeninterview als Alternative ausgewählt wurde. 

Nachdem in den vorangegangenen Ausführungen die Schulsozialarbeit bereits als theoretischer Bezugsrahmen herausgestellt wurde, mag es verwundern, dass der Begriff Schulsozialarbeit in den Forschungsfragen bzw. Zielstellungen überhaupt nicht auftaucht. Das ist zum einen, wie bereits oben ausgeführt, damit zu begründen, dass die Schulsozialarbeit sich in ihren theoretischen Grundlagen wenig bis gar nicht auf den berufsbildenden Bereich bezieht und damit zahlreiche Aspekte, wie einige Methoden oder Konzeptionen, für diese Schulart schlichtweg als unrelevant bezeichnet werden können. Zum anderen "`[\punkte] mangelt es dem Arbeitsfeld Schulsozialarbeit sowohl an einem unumstrittenen Begriff als auch an einem relativ klaren inhaltlichen Verständnis."' \footcite[23]{Speck2007}. Erschwerend kommt hinzu, dass, bedingt durch die derzeitige gesetzliche Lage und Trägerstruktur der Schulsozialarbeit, freie Schulträger nicht nur aufgrund der Finanzierungsaspekte selten Schulsozialarbeit nach der theoretischen Definition anbieten \footcite[vgl.][116]{Stuewe2015}. Da in der vorbereitenden Beschäftigung mit der Thematik ersichtlich wurde, dass die Schüler und Lehrkräfte am DRK Bildungswerk SN mit dem Begriff Schulsozialarbeit wenig anfangen konnten und die Einführung von Schulsozialarbeit als Angebotsform fraglich war und ist, wurde die Entscheidung getroffen, den offeneren Begriff "`außerunterrichtliche Beratungs- und Unterstützungsangebote"' zu wählen. Trotzdem wird davon ausgegangen, dass die theoretischen Grundlagen der Schulsozialarbeit auch dafür anwendbar und in bestimmten Anteilen handlungsleitend sind.
 
Zur Bearbeitung der genannten Themen erfolgte eine ausführliche Literaturrecherche, die neben der Suche nach Literatur im WebOPAC der Sächsischen Landes- und Universitätsbibliothek Dresden (SLUB) auch eine umfangreiche internetbasierte Datenrecherche über das Datenbank-Infosystem (DBIS) der SLUB, den Deutschen Bildungsserver und in Fachdatenbanken, wie bspw. SOLIS -- Sozialwissenschaftliches Literaturinformationssystem, sowiport -- das Portal für Sozialwissenschaften und FIS-Bildung beinhaltete. Zusätzlich erfolgten persönliche Anfragen bei der Landesarbeitsgemeinschaft Schulsozialarbeit Sachsen e. V. sowie bei zuständigen Ämtern des Freistaates Sachsen. Bei der Recherchetätigkeit wurden hauptsächlich Schlagworte wie Problemlagen von Schülern, Schulsozialarbeit, Soziale Arbeit an Schulen, Methoden der Sozialen Arbeit in der Schule, Forschung zur Schulsozialarbeit und andere, jeweils bezogen auf den allgemeinbildenden Bereich und die berufsbildende Schule, verwendet. Festzustellen war, dass die Literaturlage zu den genannten Stichworten sehr reichhaltig ist, so dass aktuelle Publikationen und Quellen der letzten zehn Jahre gesichtet und nach inhaltlicher Analyse verwendet werden konnten. Bestehen blieb jedoch das bereits beschriebene Problem der fehlenden Spezifika hinsichtlich der Schulart berufsbildende Schule. Insbesondere die Publikationen von Karsten Speck, Anke Spies und Nicole Pötter sind als besonders gewinnbringend zur Thematik einzuschätzen und wurden daher auch vorrangig, zumindest im Anteil der theoretischen Ausführungen, verwendet. 

Zum Aufbau der vorliegenden Arbeit ist auszuführen, dass grundsätzlich eine Einteilung in einen ersten theoretischen Anteil und einen zweiten praktischen Teil erfolgt; in diesem werden die selbst konzipierten und durchgeführten Studien bzw. Forschungsarbeiten ausführlich vorgestellt und ausgewertet. Der erste Teil beschäftigt sich hingegen zunächst mit wichtigen theoretischen Grundlagen der Schulsozialarbeit, wobei Definitionen, rechtliche Grundlagen, Träger, Zielgruppen, Ziele, Aufgabenfelder, Methoden und Konzeptionen in den Fokus genommen werden. Damit sollen in relativ kurzen und komprimierten Ausführungen aktuelle Erkenntnisse zu diesem Handlungsfeld der sozialen Arbeit vorgestellt werden, die im Wesentlichen dazu dienen, mögliche Unterstützungsangebote theoretisch zu fundieren und zu begründen. Danach erfolgt der Versuch einer allgemeinen Bedarfsanalyse für Schulsozialarbeit bzw. für Unterstützungsangebote, wobei die heutigen spezifischen Problemlagen junger Menschen explizit betrachtet werden und insbesondere für den berufsbildenden Bereich Erkenntnisse gewonnen und vorgestellt werden. Dem schließt sich eine Analyse des aktuellen Standes der Schulsozialarbeit an beruflichen Schulen in Sachsen an, der als Standortbestimmung die bereits eingangs erwähnte Diskrepanz von Angebot und Nachfrage in diesem Bereich verdeutlichen soll und unter anderem versucht darzulegen, inwieweit Erkenntnisse für den Bereich der Schulen in freier Trägerschaft vorhanden sind, von denen in konzeptioneller Hinsicht möglicherweise profitiert werden könnte. Als Hinführung zum praktischen Teil ergänzen eine ausführliche Vorstellung des DRK Bildungswerkes SN als "`Forschungsschule"' und einige überleitende Gedanken und Ausführungen den theoretischen Part der Arbeit. Im zweiten Abschnitt, der überwiegend praktisch orientiert ist, werden zunächst die eigenen Forschungsvorhaben ausführlich vorgestellt und begründet sowie die Herangehensweise und Durchführung offengelegt. Die mittels Schülerbefragung und Lehrerinterviews gewonnenen Daten werden anschließend ausführlich ausgewertet und sowohl zunächst eigenständig als auch in einem direkten Vergleich präsentiert. Die kritische Diskussion der durchgeführten Forschungsarbeiten rundet diesen Anteil der Arbeit ab und betrachtet mögliche Ressourcen und Potentiale. Aus den Ergebnissen der Forschungsarbeiten und den theoretischen Grundlagen werden nun erste mögliche konzeptionelle Ansätze für konkrete Unterstützungsangebote am DRK Bildungswerk SN abgeleitet und vorgestellt. Abschließend erfolgt im letzten Teil der Arbeit eine Zusammenfassung und Bilanzierung der betrachteten Theoriegrundlagen und praktischen Ergebnisse, wobei weiterführende Forschungsbedarfe aufgezeigt werden. 

Es wird darauf hingewiesen, dass die Begriffe Berufsschule, berufliche Schule und berufsbildende Schule als synonyme Bezeichnungen für die Schulart verwendet wurden. Obwohl die genannten Begriffe bei genauer Betrachtung Differenzierungen aufweisen, die jedoch auch in der verwendeten Literatur nicht immer trennscharf berücksichtigt werden, ergeben sich in der vorliegenden Arbeit dadurch keine Zuordnungen zu beruflichen Fachrichtungen bzw. schulischen Unterformen wie Berufsfachschule oder Fachschule.

\section{Schulsozialarbeit in Deutschland}
\label{sec:k2_Schulsozialarbeit in Deutschland}


\section{Bedarfe für Schulsozialarbeit}
\label{sec:BedarfeFürSchulsozialarbeit}

\subsection{Allgemeine Einführung}
\label{sec:AllgemeineEinführung}

In der Literatur zur Thematik Schulsozialarbeit finden sich umfassende Ausführungen zu deren Notwendigkeit und Bedarf, jedoch meist allgemein formuliert und wenig schulartspezifisch. "`Der umfassende gesellschaftliche Wandel der vergangenen Jahrzehnte, der zu einer Veränderung traditioneller Orientierungs- und Lebensmuster geführt hat und auch die Welt der Kinder- und Jugendlichen nachhaltig veränderte, stellt Jugendhilfe und Schule vor neue Herausforderungen, die sie nur durch eine gemeinsame Gestaltung des Lebens und Lernens bewältigen können"' schreibt dazu beispielsweise das Sächsische Staatsministerium für Soziales in seiner vorliegenden Handreichung zur Schulsozialarbeit in Sachsen \footcite[13]{SMSSS2009}. Der Bedarf an unterstützender Schulsozialarbeit scheint dabei vorrangig aus den vielfältigen und vielfach diskutierten Problemlagen junger Menschen in der heutigen Zeit zu resultieren, von denen fehlende Schulabschlüsse, Schwierigkeiten beim Übergang in die berufliche Ausbildung, Drogenprobleme, Kriminalität, Gewalt sowie Schulvermeidung nur einige sind. So konstatiert u.a. dos Santos-Stubbe zutreffend, dass "`[d]ie Biografie von Kindern und Jugendlichen gegenwärtig geprägt ist durch zahlreiche Umbrüche, die eine tiefe Bedeutung besitzen."' \footcite[68]{dosSantos-Stubbe2009} Dadurch werden Jugendhilfe und Schule mit Problemlagen konfrontiert, die im Kontext gesellschaftlicher Veränderungen stehen und die Lebenswelten von Kindern und Jugendlichen sowie deren Familien betreffen. Dies sind insbesondere Sozialisationsdefizite der Familie, erhöhte Leistungsanforderungen der Schule und an die Schule, ein erhöhter Wettbewerbsdruck bei einerseits schwachen Schülerinnen und Schülern angesichts fehlender Ausbildungsplätze sowie drohender Arbeitslosigkeit und andererseits bei starken Schülerinnen und Schülern angesichts großer Marktchancen, Schwierigkeiten beim Übergang in Ausbildung und Arbeit, Schulvermeidung in ihren unterschiedlichen Formen sowie die Belastung des Klimas an vielen Schulen durch zunehmendes delinquentes und deviantes Verhalten der Schülerschaft \footcite[vgl.][17]{SMSSS2009}.

Viele Lehrkräfte sehen sich mit den teilweise komplexen Problemlagen der Schülerinnen und Schüler überfordert, bemängeln die unzureichende Unterstützung und fordern eine Rückbesinnung auf ihr Kerngeschäft -- nämlich den Unterricht \footcite[vgl.][10]{Drilling2009}. Dennoch kann und darf sich Schule heute nicht darauf beschränken, mit Bildungsangeboten auf die Anforderungen des Lebens vorzubereiten, sondern muss aufgrund der oben ausgeführten, teilweise komplexen Problemlagen der Schülerschaft ihren Beitrag zur Bewältigung aller Lebensbereiche leisten. Diesen Anforderungen können Lehrkräfte jedoch nur in begrenztem Maße gerecht werden, zumindest dann, wenn der Unterricht und damit die fachliche Vorbereitung auf Teilaspekte des Lebens, nicht ständig in den Hintergrund gerückt werden soll. Hier kann nur durch die dauerhafte und klar geregelte Kooperation von Lehrerschaft und Schulsozialarbeitern oder Schulsozialpädagogen eine befriedigende Situation für alle beteiligten Akteure gewährleistet werden \footcite[vgl.][9ff]{Drilling2009}. 

\subsection{Berufsbildende Schulen}
\label{sec:BerufsbildendeSchulen}

Zuerst einmal ist jeglichen Ausführungen vorauszuschicken, dass das Feld der Berufsbildung ein so heterogenes ist, dass generelle Aussagen zu diesem wohl genauso schwierig sind wie zur Schulsozialarbeit in dieser Schulart. Möglicherweise tragen diese, noch näher auszuführenden, Strukturmerkmale dazu bei, dass Schulsozialarbeit als wenig existent erscheint, aber vielleicht doch -- zumindest in Ansätzen -- häufiger vorhanden ist, als vordergründig anzunehmen. Einige Begründungen für diese Annahme der Verfasserinnen dieser Arbeit werden in den nachfolgenden Ausführungen an entsprechend passender Stelle gegeben. 

Zunächst lohnt sich eine Beantwortung der Frage, was eigentlich die Heterogenität des Feldes der Berufsbildung ausmacht. Da wiederum bundesländerspezifisch eine vielfältige Anzahl unterschiedlichster Regelungen zur Berufsausbildung existiert, die sich in mannigfaltigen Schularten und Bezeichnungen niederschlagen, wird das Bundesland Sachsen zur näheren Betrachtung ausgewählt. Zu dieser können verschiedene Aspekte herangezogen werden, der Versuch einer überblicksartigen Systematisierung erfolgt in den folgenden Punkten.

\subsubsection{Schularten}
\label{sec:Schularten}

Die berufsbildenden Schularten in Sachsen sind die Berufsschulen, Berufsfachschulen, Fachoberschulen, Fachschulen und beruflichen Gymnasien. Diese sind zumeist in Beruflichen Schulzentren zusammengefasst, zumindest soweit sie sich in öffentlicher Trägerschaft befinden. In allen Schularten können berufsbildende Förderschulen eingerichtet werden.

Die Berufsschule wird von Schülern besucht, die eine duale Berufsausbildung in einem der mehr als 360 anerkannten Ausbildungsberufe absolvieren und sich dazu mit einem Arbeitgeber in einem Ausbildungsverhältnis befinden. Sie enthält auch Angebote für behinderte oder benachteiligte Jugendliche. Die Berufsschulzeit dauert in der Regel drei Jahre. 

Berufsfachschulen führen zu einem bundeseinheitlich anerkannten Berufsabschluss und sind häufig im medizinisch-pflegerischen und sozialen Bereich (z. B. Altenpfleger, Gesund- heits- und Krankenpfleger, Notfallsanitäter, Physiotherapeuten, Sozialassistenten) vorzufinden. Die Ausbildung dauert in der Regel 2 -- 3 Jahre. An Berufsfachschulen werden derzeit etwa 40 Bildungsgänge angeboten, welche meist in vollzeitschulischen Formen zu einem Berufsabschluss führen. Das bedeutet, dass sich die Schüler an Berufsfachschulen in der Regel nicht in einem Ausbildungsverhältnis mit einem Arbeitgeber befinden, sondern praktische Anteile der beruflichen Handlungskompetenz durch Praktika, zumeist in verschiedenen Einrichtungen des Berufsfeldes, erworben werden. Ausnahmen bilden jedoch einige Berufe im medizinisch-pflegerischen Bereich, wie z. B. Altenpfleger, Gesundheits- und Krankenpfleger und Notfallsanitäter. In diesen Ausbildungsberufen befinden sich Berufsfachschüler auch in einem Angestelltenverhältnis mit einem Arbeitgeber.
 
An der Fachoberschule können Jugendliche und Erwachsene die Fachhochschulreife erlangen. Die Ausbildung dauert für Schüler mit Realschulabschluss zwei Jahre, für Schüler mit abgeschlossener Berufsausbildung ein Jahr. 
Fachschulen sind Einrichtungen der beruflichen Weiterbildung. Sie bieten Fachkräften mit bereits abgeschlossener Berufsausbildung und beruflichen Erfahrungen länderspezifische Abschlüsse, die sie für Tätigkeiten im mittleren Funktionsbereich zwischen Facharbeitern bzw. Fachangestellten und Hochschulabsolventen befähigen. Fachschulen finden sich in verschiedenen beruflichen Feldern, im Bereich Sozialwesen werden beispielsweise Erzieher und Heilerziehungspfleger in dieser Schulart ausgebildet. 

Schüler mit Realschulabschluss und guten Leistungen können am beruflichen Gymnasium in drei Jahren die allgemeine Hochschulreife erlangen, die zum Studium an allen Hochschulen berechtigt. Sie erhalten neben allgemein bildendem auch berufsbezogenen Unterricht, der sie an die Berufswelt heranführt \footcites[vgl.]{SBSBSSSK2015}[vgl.][4ff]{SMKSK2013}.

\subsubsection{Berufsvorbereitende Maßnahmen}
\label{sec:BerufsvorbereitendeMassnahmen}

Jugendliche, die nach erfolgreichem Abschluss der Oberschule keinen betrieblichen Ausbildungsplatz erhalten oder die Oberschule ohne Hauptschulabschluss beendet haben, können sich an der Berufsschule in einem Berufsgrundbildungsjahr (BGJ) auf die Aufnahme eines Berufsausbildungsverhältnisses oder eine Berufstätigkeit vorbereiten. Sie können eine berufliche Grundbildung in verschiedenen Berufsbereichen erhalten. Damit wird die Berufsschulpflicht erfüllt. Der erfolgreiche Abschluss des BGJ kann als erstes Ausbildungsjahr auf eine nachfolgende Berufsausbildung angerechnet werden.

Das Berufsvorbereitungsjahr (BVJ) hat die Aufgabe, Jugendliche bei der Berufswahl zu unterstützen und auf die Aufnahme einer Berufsausbildung vorzubereiten. Schüler des BVJ erwerben eine berufliche Orientierung in zwei Berufsbereichen (z. B. Holztechnik und Metalltechnik). Bei erfolgreichem Abschluss wird der Hauptschulabschluss zuerkannt.

In Vorbereitungsklassen mit berufspraktischen Aspekten werden Jugendliche und junge Erwachsene mit Migrationshintergrund auf die Aufnahme einer Berufsausbildung oder den Erwerb eines höheren Bildungsabschlusses (z. B. am Beruflichen Gymnasium oder der Fachoberschule) sprachlich vorbereitet. Im Rahmen der Vorbereitungsklasse nehmen die Schüler entsprechend der individuell angestrebten künftigen beruflichen Ausbildung für zwei Monate am Regelunterricht einer berufsbildenden Schule teil \footcite[vgl.][15ff]{SMKSK2013}.

\subsubsection{Zugänge zu beruflicher Bildung}
\label{sec:ZugängeZuBeruflicherBildung}

Ebenso vielfältig wie die einzelnen Schularten sind die Zugangsvoraussetzungen zu diesen. In die Berufsschulen kann aufgenommen werden kann, wer die Vollzeitschulpflicht erfüllt hat, noch berufsschulpflichtig ist und einen Ausbildungsvertrag abgeschlossen hat. Ein bestimmter Schulabschluss wird nicht vorausgesetzt.

Die Ausbildung an den Berufsfachschulen erfordert in der Regel den Realschulabschluss oder einen gleichwertigen Abschluss. Im Bereich Gesundheit und Pflege ist die gesundheitliche Eignung zwingend erforderlich. Bei den Bildungsgängen in der Alten- und Krankenpflege sowie bei der Ausbildung von Hebammen und Notfallsanitätern ist der Nachweis eines Ausbildungsvertrages erforderlich. Jedoch gibt es auch in dieser Schulart Ausnahmeregelungen, da der Zugang zu einigen Berufen auch mit Hauptschulabschluss möglich ist. Ein Beispiel hierfür ist die Krankenpflegehilfe.
 
An den Fachschulen gibt es für die einzelnen Fachbereiche unterschiedliche Aufnahmevoraussetzungen. In der Regel sind eine abgeschlossene Berufsausbildung und der Nachweis einer beruflichen Tätigkeit notwendig \footcite[vgl.][15ff]{SMKSK2013}.

\subsubsection{Schulträger}
\label{sec:Schulträger}

Neben den staatlichen Schulen gibt es in allen Schularten als Ergänzung der sächsischen Bildungslandschaft auch Schulen in freier Trägerschaft, zum Beispiel von privaten oder kirchlichen Organisationen, Vereinen, Gesellschaften oder Privatpersonen. Sie sind für die Schulgestaltung verantwortlich und können insbesondere über eine spezielle pädagogische, religiöse oder weltanschauliche Prägung entscheiden. Die Träger können außerdem Lehr- und Unterrichtsmethoden sowie Lehrinhalte und die Organisation des Unterrichts auch abweichend von den Vorschriften für die öffentlichen Schulen festlegen. Bei den Schulen in freier Trägerschaft werden Ersatz- und Ergänzungsschulen unterschieden. Die aktuellen gesetzlichen Regelungen zu freien Schulen sind im Sächsischen Gesetz über Schulen in freier Trägerschaft (SächsFrTrSchulG) vom 8. Juli 2015 festgeschrieben.

Ersatzschulen sind Schulen in freier Trägerschaft, die als Ersatz für eine im Freistaat Sachsen vorhandene oder grundsätzlich vorgesehene öffentliche Schule dienen. Die Ersatzschule darf in ihren wesentlichen Merkmalen nicht hinter einer öffentlichen Schule zurückstehen. An Ersatzschulen muss das gleiche Bildungsniveau erreicht werden wie an entsprechenden öffentlichen Schulen. Ersatzschulen verwenden daher in der Regel die sächsischen Lehrpläne. Die Ersatzschulen können ein Schulgeld erheben und werden bei Vorliegen der Voraussetzungen nach Ablauf der vierjährigen Wartefrist durch den Freistaat finanziell unterstützt. 

Ergänzungsschulen sind Schulen in freier Trägerschaft, die nicht als Ersatz für öffentliche Schulen dienen. Ergänzungsschulen haben hinsichtlich ihrer Organisation und ihres Bildungsangebots einen schulischen Charakter, sind aber mit keiner Schulart des öffentlichen Schulwesens vergleichbar und stehen damit außerhalb des Schulaufbaus in Sachsen. An ihnen muss das Bildungsniveau einer vergleichbaren öffentlichen Schule nicht erreicht werden. Ergänzungsschulen verwenden daher in der Regel auch keine sächsischen Lehrpläne. Die vergebenen Abschlüsse entsprechen damit auch nicht den staatlichen Abschlüssen, die an öffentlichen Schulen oder Ersatzschulen vergeben werden. Ergänzungsschulen sind demzufolge nicht berechtigt Zeugnisse auszustellen. Schüler einer Ergänzungsschule erhalten zum Schluss ihrer Ausbildung an der Ergänzungsschule eine Bescheinigung über den Schulbesuch oder ein Zertifikat. Ersatzschulen erhalten keine finanzielle Unterstützung durch den Freistaat, die Höhe des Schulgeldes ist daher nicht begrenzt \footcite[vgl.]{SMKSK2015a}. 

Die soeben dargelegten Informationen haben die Heterogenität der berufsbildenden Schulen in Sachsen nur in einigen wichtigen Punkten vorgestellt. Übertragbar sind diese Faktoren auf andere deutsche Bundesländer dahingehend, dass ähnlich vielfältige Strukturmerkmale vorliegen. Deutlich werden sollte, dass es "`die berufsbildende Schule"', also ein einheitliches Konstrukt, bezogen auf die Voraussetzungen hinsichtlich der schulischen und beruflichen Vorbildung oder auch das Alter der Schüler gar nicht gibt. Es ist durchaus vorstellbar, dass an beruflichen Schulzentren und auch an freien Schulen Schüler ohne bisherigen Schulabschluss gemeinsam mit denen, die einen (Haupt-) Schulabschluss nachweisen, aber noch keinen Ausbildungsplatz gefunden haben, eine berufsvorbereitende Maßnahme absolvieren. In anderen Klassen können Personen vom Hauptschulabsolventen, über den Schüler mit Realschulabschluss oder Fachhochschulreife bis hin zu einem Erwachsenen in der dritten Berufsausbildung in allen möglichen Konstellationen aufeinandertreffen. Demzufolge sind Klassen an berufsbildenden Schulen in allen Bereichen oft in ihrer Altersstruktur, in der Struktur der allgemeinen und beruflichen Vorbildung und auch in der Sozialstruktur sehr heterogen, was letztendlich ganz andere pädagogische und zwischenmenschliche Herausforderungen sowie Probleme mit sich bringt als im allgemeinbildenden Bereich. Diese Gegebenheiten wirken selbstverständlich auch direkt in den Bereich der Schulsozialarbeit innerhalb der Berufsbildung hinein und sind deshalb auf keinen Fall zu vernachlässigen.

\subsection{Problemlagen von Schülern an berufsbildenden Schulen}
\label{sec:ProblemlagenVonSchülernAnBerufsbildendenSchulen}

In der gesichteten Literatur zum Thema Schulsozialarbeit im Kontext der Berufsbildung wird das Thema Berufsorientierung und berufsbezogene Jugendbildung meist als ein wichtiges Aufgabenfeld der sozialpädagogischen Jugendhilfe angesehen und ausgeführt. Dadurch entsteht oft der Eindruck, dass mit der Entscheidung für einen Beruf bzw. eine berufliche Fachrichtung der Auftrag erledigt wäre. In kaum einer Publikation finden sich Aussagen zur Notwendigkeit der Schulsozialarbeit an berufsbildenden Schulen und auch Praxisbeispiele sowie Projektbeschreibungen nehmen fast ausschließlich den allgemeinbildenden Schulbereich in den Fokus. Lediglich das Gebiet berufsvorbereitender Bildungsmaßnahmen, z. B. in Form von Berufsvorbereitungsjahren oder Berufsgrundbildungsjahren wird in Veröffentlichungen aus verschiedenen Bundesländern immer wieder erwähnt, erschöpft sich jedoch meist in Beschreibungen der Konzeption und Umsetzung sozialpädagogischer Angebote an bestimmten Schulen.

Fast erscheint es, als würden die Problemlagen der meisten Jugendlichen zum Stillstand kommen, sobald sie die allgemeinbildende Schule verlassen, was in der Realität ganz sicher nicht der Fall ist. 

\begin{quotation}
\noindent
"`Der genauere Blick macht deutlich, dass das Schulsystem in Gestalt der Risikoschüler eine große Gruppe hervorbringt, die gegebenenfalls trotz entsprechender formaler Abschlüsse, nicht über die realen Kompetenzen verfügen, um eine moderne Berufsausbildung absolvieren zu können. Zugleich werden von Seiten des Beschäftigungssystems jedoch nicht all jene aufgenommen, die über die notwendigen Kompetenzen verfügen. [\punkte] Dies zeigt sich insbesondere bei dem zunehmend krisenhaften Übergang von der Schule in die duale Ausbildung."'
\end{quotation}

\noindent
[\punkte] fassen dazu treffend  Braun und Wetzel zusammen \footcite[181]{Braun2006}. Dadurch begründet, müsste eigentlich die Schulsozialarbeit in berufsbildenden Schulen besonders häufig zu finden sein und einen festen Bestandteil der sozialpädagogischen Arbeit darstellen.

Bezogen auf den Bedarf an sozialpädagogischer Unterstützung in der berufsbildenden Schulen lässt sich feststellen, dass die Problemlagen der dortigen jungen Menschen sich wenig bis gar nicht von denen der allgemeinbildenden Schule unterscheiden, jedoch einige strukturell bedingte Unterschiede zu berücksichtigen sind, die im vorangegangenen Punkt bereits ausführlicher erläutert wurden. Es finden sich sowohl Jugendliche mit einem Ausbildungsplatz im dualen System, als auch solche in vollzeitschulischen Ausbildungen sowie junge Menschen ohne Ausbildungsplatz, z. B. in Berufsvorbereitungsjahren. Insbesondere die Jugendlichen ohne Ausbildungsplatz stellen Lehrkräfte und Personal vor zahlreiche Herausforderungen, gelten sie doch vielfach heute als "`berufsunreif"', "`schwer vermittelbar"' und "`lernunwillig"', obwohl sie noch vor wenigen Jahren ohne größere Schwierigkeiten einen Ausbildungsplatz gefunden hätten. Ihre Ausbildungslosigkeit und die damit verbundene Arbeitslosigkeit wird oftmals als individuelles Versagen definiert und die Jugendlichen werden als Benachteiligte stigmatisiert. Die berufsbildende Schule hat hier insbesondere die Aufgabe diese berufsschulpflichtigen Jugendlichen auf eine ungewisse Zukunft vorzubereiten, indem sie sie durch Unterricht, Praktika und sozialpädagogische Maßnahmen mit sozialen und lebenspraktischen Kompetenzen ausstattet \footcite[vgl.][6]{ASSB2011}. 

Obwohl häufig als weniger problembehaftet dargestellt, weisen auch Jugendliche mit Ausbildungsplatz Unterstützungsbedarfe auf, die von Lehrkräften nicht oder nur in sehr begrenztem Maße geleistet werden können. Die berufliche Schule als zentraler Lernort zur Verknüpfung von Theorie und Praxis kämpft hier insbesondere mit einem permanenten Zeitmangel und dadurch mit geringen Möglichkeiten zur Vermittlung sozialer Kompetenzen und zur Unterstützung einer adäquaten Auseinandersetzung mit individuellen Problemen. Die Vorbereitung auf die Abschlussprüfungen prägt den berufsschulischen Alltag so sehr, dass für pädagogische Zielsetzungen und soziale Lerninhalte meist nur wenig Unterrichtszeit zur Verfügung steht. Weiterhin führen deutlich gestiegene theoretische Anforderungen durch die Neuordnung vieler Ausbildungsberufe häufig zu einer Überforderung der Jugendlichen mit schlechteren Schulabschlüssen. Dadurch scheitern viele Auszubildende mit Lernproblemen und schlechten Sprachkenntnissen, z.B. bedingt durch einen Migrationshintergrund, teilweise bereits während der Ausbildung oder auch an den Abschlussprüfungen, so dass die Quoten der Ausbildungsabbrüche und der nicht bestandenen Prüfungen zum Teil sehr hoch sind. Außerdem sind die betriebliche Realität, der volle Arbeitstag, der Umgang mit Vorgesetzten, Kolleginnen, Kundinnen oder Patientinnen und die hohe Verantwortlichkeit für ihr Tun neue Erfahrungen, die für die Jugendlichen zu verarbeiten sind. Dazu kommen oft noch Erlebnisse mit negativen Arbeitsbedingungen, Nichteinhaltung gesetzlicher Schutzbestimmungen, Überstunden, Verrichtung von ausbildungsfremden Arbeiten, einengende oder überfordernde Tätigkeitsbereiche sowie ungerechte Behandlung durch Vorgesetzte oder Kollegen und vieles mehr \footcite[vgl.][7]{ASSB2011}. 

Ähnliche Befunde und Problemlagen wie der Arbeitskreis Berufsschulsozialarbeit in Bayern diagnostizierte auch die Landesarbeitsgemeinschaft Schulsozialarbeit e. V. in Sachsen im Jahre 2004 \footcite[18]{LSS2004}. 

\begin{quotation}
\noindent
"`Von den Berufsschulen werden massive individuelle Probleme und Lernprobleme geschildert. Bei der genaueren Betrachtung dieser äußern sie sich in teilweise gewalthaften Verhaltensauffälligkeiten, Schulverweigerung, mangelnder sozialer Kompetenz und Suchtproblemen. Bei Lernproblemen werden Motivationsprobleme und fehlende Lerntechniken geschildert."' \footcite[18]{LSS2004}
\end{quotation}

\noindent
Insbesondere, aber nicht ausschließlich wurden derartige Feststellungen in den Berufsvorbereitungsjahren gemacht, zu deren Zweck und Besonderheiten bereits Ausführungen erfolgten. Als besonders problematisch wurde dabei die Tatsache angesehen, dass die Jugendlichen teilweise ihr Scheitern bei der Suche nach einem Ausbildungsplatz ausschließlich als persönliches Defizit und nicht als Auswirkung des Lehrstellenmangels betrachten. Dadurch bilden sich in den beruflichen Schulen mitunter konfliktbeladene Klassen, die überwiegend aus leistungsschwachen und unmotivierten Schülern bestehen und aufgrund weitreichender sozialer Benachteiligungen der Schüler eigentlich ein definitives Arbeitsfeld der Schulsozialarbeit im Sinne des § 13 SGB VIII/KJHG wären \footcite[18]{LSS2004}.

Abweichend von den schulartspezifischen Problemlagen sollen auch noch einige Studien betrachtet werden, die sich dem Thema allgemeiner gewidmet haben, um weitere mögliche Themen herauszuarbeiten, die viele Jugendliche heute tangieren und Bedarfe für Schulsozialarbeit bedingen können. Dazu liegt eine Studie der Universität Leipzig aus dem Jahre 2007 vor, in der Problemlagen von Kindern und Jugendlichen mittels einer Befragung von 34 Fachkräften der Kinder- und Jugendhilfe eruiert wurden. Festgestellt wurde dabei, dass die Problemlagen insbesondere in ihrer Komplexität zunehmen, also zunehmend vielfältiger und unberechenbarer geworden sind \footcite[vgl.][142ff]{UniversitaetLeipzig2007}. Dies zeigt sich unter anderem daran, dass die befragten Fachkräfte eine deutliche Zunahme der Gruppe "`schwieriger"' Jugendlicher beschrieben, die durch mehrere soziale und schulische Auffälligkeiten gekennzeichnet sind. Ebenso wurde ein deutliches Ansteigen psychischer und psychosomatischer Störungs- und Krankheitsbilder verzeichnet. Legale und illegale Formen exzessiven Drogengebrauchs stellen die Fachkräfte vor schwierige Probleme, bei denen sowohl Sachfragen (Informationen über Stoffe, Verbreitungsmuster, Wirkungen etc.) als auch Verständnisfragen (Wissen über soziale und jugendkulturelle Hintergründe, Gebrauchsmuster etc.) eine Rolle spielen. 

\begin{quotation}
\noindent
"`Hier wurde hervorgehoben, wie sehr legale wie auch illegale Drogen unter Jugendlichen heute "`alltagsfähig"' geworden sind. Probleme wie das Sinken des Einstiegsalters und eine gesteigerten Suchtgefahr bei Jugendlichen sind Ausdruck dieser Entwicklung."' \footcite[vgl.][142ff]{UniversitaetLeipzig2007}
\end{quotation}

\noindent
Weiterhin konstatierten die Fachkräfte eine deutliche Veränderung der Altersstruktur hilfebedürftiger Jugendlicher. Als Problemlage wurde eine regelrechte "`Überalterung"' des Klientels beschrieben, die durch fehlende Übergänge in Ausbildung und Erwerbstätigkeit entsteht. Demgegenüber sind im Arbeitsfeld auch immer jüngere Jugendliche mit komplexen Problemlagen zu verzeichnen, für die junge bis sehr junge Mutterschaften nur ein Beispiel darstellen. So bilden sich derzeit neue Zielgruppen für die Jugendhilfe und somit auch für die Schulsozialarbeit heraus, die in der Praxis konzeptionelles Umdenken und veränderte Arbeitsformen erfordern \footcite[vgl.][142ff]{UniversitaetLeipzig2007}.
 
Eine Untersuchung zur Schulsozialarbeit in Sachsen aus dem Jahr 2005 durch eine Arbeitsgruppe der Landesarbeitsgemeinschaft Schulsozialarbeit Sachsen e.V. beschäftigte sich ebenfalls mit Problemlagen der Schülerschaft, der Schulen und Möglichkeiten der Schulsozialarbeit. Befragt wurden 78 SozialarbeiterInnen allgemeinbildender Schulen und Sonderschulen. Berufsbildende Schulen waren in der Untersuchung nicht vertreten. Als besonders gravierende Problemlagen und somit Handlungsfelder der Schulsozialarbeit wurden Mobbing, insbesondere in Klassen ab der Klassenstufe 7, Schulverweigerung, Gewalt, Leistungsschwäche, Verhaltensauffälligkeiten, familiäre Probleme (Armut), Integrationsschwierigkeiten von Schülern mit Migrationshintergrund sowie Alkohol- und Drogenprobleme herausgearbeitet. Festgestellt wurde weiterhin, dass keine klar abgrenzbaren Hauptprobleme zu erkennen waren, sondern die jeweiligen Lebensverhältnisse der Betroffenen zu Überforderungssituationen führten, die besonders im Rahmen der Einzelfallhilfe zu thematisieren sind \footcite[vgl.][63ff]{Lang2010}. Anzunehmen ist, dass die, von den Sozialarbeitern benannten, Problemlagen ebenso auf den Bereich der berufsbildenden Schulen übertragen werden können, da die Schüler nach Beendigung ihrer allgemeinbildenden Schullaufbahn in eine berufsvorbereitende oder berufsausbildende Maßnahme überwechseln.

In den vorgestellten Untersuchungen und Berichten ist nunmehr eine Vielzahl von Problemlagen vorgestellt und beschrieben wurden, die jedoch an dieser Stelle nur einen Ausschnitt tatsächlich im berufsbildenden Bereich vorhandener Problemlagen und Herausforderungen repräsentieren können. Alle benannten Probleme sowie auch weitere finden sich im 14. Kinder- und Jugendbericht \footcite[44f]{BundesministeriumFamilie2013} und im Dritten Sächsischen Kinder- und Jugendbericht \footcite[30ff]{SMSSS2009} an verschiedensten Stellen und unter mannigfaltigen Schwerpunkten. Keinesfalls erheben die hier vorgestellten Problemlagen den Anspruch auf Vollständigkeit, was in Anbetracht der individuellen sozialen, schulischen und beruflichen Gegebenheiten der Schüler im berufsbildenden Bereich auch niemals möglich sein kann. Für die vorliegende Arbeit bedeuten die Ausführungen eine wichtige Grundlage zur Konzeption einer eigenen schulspezifischen Erfassung der Problemlagen von Schülern in medizinischen und sozialen Ausbildungen, da für diese spezielle Schülergruppe und die dazugehörigen Schularten keine Untersuchungen existieren bzw. bei der Literaturrecherche gefunden werden konnten. 

\section[Berufsbildende Schulen in Sachsen]{Berufsbildende Schulen in Sachsen - Eine exemplarische Vorstellung am DRK Bildungswerk Sachsen}
\label{sec:BerufsbildendeSchulenInSachsenEineExemplarischeVorstellungAmDRKBildungswerkSachsen}

\section{Überleitung vom theoretischen zum praktischen Teil der Arbeit}
\label{sec:ÜberleitungVomTheoretischenZumPraktischenTeilDerArbeit}

Im gesamten vorangestellten theoretischen Teil dieser Arbeit, mit Ausnahme des Gliederungspunktes \ref{sec:BerufsbildendeSchulenInSachsenEineExemplarischeVorstellungDesDRKBildungswerkSachsen}, sind Grundlagen der Schulsozialarbeit vorgestellt, interpretiert und bewertet worden, meist jedoch mit dem Hinweis, dass sie nur in bestimmten Anteilen auf den berufsbildenden Bereich anwendbar und übertragbar sind. Vorausschickend auf den nächsten, forschungsbezogenen Abschnitt der Arbeit ist zunächst zuzugeben, dass der Begriff "`Schulsozialarbeit"' nunmehr kaum noch verwendet werden wird. Der Leser könnte sich nun berechtigterweise fragen, warum denn bis hierher ausführlich darüber referiert wurde, wenn der Begriff inklusive seiner Definition und charakterisierenden Merkmale gar keine Verwendung mehr findet. Zum einen ist dies damit zu begründen, dass Schulsozialarbeit in ihrer ganzen Merkmalsbreite und Ausdifferenzierung nach Ansicht der Verfasserinnen auf viele Bereiche der berufsbildenden Schulen gar nicht zutrifft, was an einigen Stellen der theoretischen Ausführungen (siehe Punkt \ref{sec:Konzeptionen}) bereits dargelegt wurde. Bezogen auf die vorgestellte Forschungsschule, das DRK Bildungswerk SN, müssen, bedingt durch strukturelle, personelle und finanzielle Besonderheiten, weitere Abstriche vom eigentlichen Begriff erfolgen. Aus diesen Gründen wurde die Entscheidung für die Bezeichnung "`außerunterrichtliche Beratungs- und Unterstützungsangebote"' getroffen, die jedoch sehr wohl ausgewählte theoretische Grundlagen der Schulsozialarbeit enthält, diese jedoch nicht in ihrer gesamten Breite und Tiefe abbildet. Dementsprechend können diese außerunterrichtlichen Beratungs- und Unterstützungsangebote z. B. in Bezug auf konzeptionelle, methodische sowie ziel- und zielgruppenbezogene Aspekte auf die eigentliche Schulsozialarbeit zurückgreifen. 

Ebenfalls mit einer kurzen Erläuterung zu versehen ist die Tatsache, dass auf eine Vorstellung und theoretisch fundierte Darstellung des Schulsozialarbeiters verzichtet wurde. Zu diesem Berufsbild gäbe es eine ganze Fülle an interessanten Aspekten, möglicherweise auch in der Spezifik des Einsatzfeldes berufsbildende Schule, zu beleuchten. Da jedoch die Einstellung eines Schulsozialarbeiters derzeit nach Aussagen der Geschäftsführung am DRK Bildungswerk SN nicht geplant ist, blieben diese Aspekte unberücksichtigt, zumal die vorliegende Arbeit ja ausdrücklich als nur relevant für die exemplarische Forschungsschule betitelt wurde.

Weiterhin wurde in den theoretischen Grundlagen nicht auf die Fachbereiche Gesundheit, Pflege und Soziales Bezug genommen, obwohl die eigenen Forschungsarbeiten doch dafür ausgerichtet sind. Auch hier muss die Begründung angeführt werden, dass Publikationen und Studien über veränderte Schülerstrukturen und Problemlagen in diesen Bereichen kaum vorhanden sind, was zum Teil ausschlaggebend für die eigene Forschungsidee war. Für zukünftige Lehrkräfte in diesem Feld erscheinen die genannten Themen als durchaus wichtig und relevant für die eigene Arbeit.

Nun könnte leicht festgestellt werden, dass gesellschaftliche Veränderungen, bezogen auf Schülerstrukturen und Problemlagen, selbstverständlich aus anderen Berufsfeldern oder allgemein aus dem beruflichen Bereich abgeleitet werden können, da die Schüler in Gesundheits-, Pflege- und sozialen Berufen den gleichen Lebensbedingungen unterliegen wie alle anderen Berufsschüler auch. Das mag auch zutreffend sein, dennoch liegen nach Ansicht der Verfasserinnen einige weitere Besonderheiten vor, die in diese Berufe und deren Ausbildung hineinwirken. Lehrkräfte beklagen hier, ebenso wie anderswo auch, die seit einigen Jahren deutlich schwierigeren, leistungsschwächeren, demotivierten, problembehafteten und für die Berufe ungeeigneten Schüler. Sind diese Beobachtungen nur als allgemeine, schon immer dagewesenen Lehrerklagen und subjektive Wahrnehmungen zu sehen oder haben sie auch eine begründbare Substanz? Ohne wieder die veränderten Lebensbedingungen und gesellschaftlichen Umbrüche, allgemein oder auch bis in die Tiefe ausgedeutet, heranzuziehen \footcite[vgl.][37ff]{BundesministeriumFamilie2013} und zu sehr in das allgemeine Lamento der "`schwierigen Jugendlichen"' einzustimmen, das die Geschichte der Jugendhilfe und der Pädagogik seit jeher begleitet \footcite[vgl.][144]{UniversitaetLeipzig2007}, sollen einige eigene Gedanken zu diesen wahrgenommenen Veränderungen formuliert werden. 

Die auch als "`Helferberufe"' bezeichneten Tätigkeitsbereiche im Gesundheits-, Pflege- und Sozialwesen unterliegen von jeher besonderen Anforderungen, insbesondere hinsichtlich der Fach- und Sozialkompetenz. Wer in seinem späteren Berufsleben so intensiv mit unterschiedlichsten Menschen aller Altersgruppen, oft in gesundheitlich oder sozial schwierigen Lagen, arbeiten möchte, sollte dafür bestimmte Voraussetzungen mitbringen. Viele praktische Fähigkeiten und Fertigkeiten und theoretische Kenntnisse sind erlernbar, jedoch scheinen auch eine gewisse Grundeignung und eine "`Berufung"' für die Tätigkeiten unabdingbar. Erfahrene Lehrkräfte behaupten, dass dies immer seltener zu beobachten wäre und sich in den letzten zehn bis fünfzehn Jahren stark verändert hätte. "`Die schulischen Voraussetzungen, bzw. personellen Eignungen der Bewerber werden seitens der Einrichtungen jedoch als abnehmend eingeschätzt."' konstatiert dazu das Deutsche Krankenhausinstitut in einer Studie von 2006 \footcite[8]{DeutschesKrankenhausinstitut2006}. Als Untermauerung der genannten Eindrücke können auch am DRK Bildungswerk SN gemachte persönliche Erfahrungen und Beobachtungen von Mobbing und Diskriminierung in Erzieherklassen oder enormen Lernschwierigkeiten in der Physiotherapie gelten, die so wie derzeit vorher kaum berichtet wurden. Dafür scheint vielfach die bereits beschriebene Heterogenität hinsichtlich der Altersstruktur und der schulischen Voraussetzungen in den Klassen mit verantwortlich zu sein. Neben vielen anderen möglichen Gründen, die an dieser Stelle nicht alle aufzuarbeiten sind, können sicherlich die demografische Veränderungen und insbesondere die Veränderungen auf dem Ausbildungsmarkt als Ursachen mitverantwortlich gemacht werden. Noch vor ca. 15-20 Jahren galten die Schüler in den Gesundheits-, Pflege- und sozialen Berufen als "`handverlesen"', am DRK Bildungswerk SN wurden z. B. ausschließlich Abi\-tu\-ri\-en\-ten und gute Realschulabsolventen für die Berufe aufgenommen. 200 Bewerber für eine Klasse Physiotherapeuten waren damals keine Seltenheit, so dass die Auswahl geeigneter Kandidaten mit bereits gefestigten Persönlichkeiten, keine besondere Schwierigkeit war. Die Klassen wiesen aus fachlicher und sozialer Sicht keine so große Heterogenität auf. Heute jedoch hat sich dieses Bild deutlich gewandelt, viele Abiturienten studieren und  gute Realschulabsolventen durchlaufen andere, bevorzugt duale Ausbildungen mit den entsprechenden Ausbildungsvergütungen und guten Aufstiegschancen. Im Ausbildungsjahr 2013/14 blieben beispielsweise 37100 betriebliche Ausbildungsstellen unbesetzt, was die große Auswahl an Möglichkeiten für Bewerber mit guten Schulabschlüssen nur im Ansatz verdeutlicht \footcite[vgl.][15]{BBF2015}. Demgegenüber blieben jedoch auch ca. 20900 Bewerber aus verschiedensten Gründen "`unversorgt"' und stellen damit mögliche Kandidaten für berufsvorbereitende Maßnahmen, Praktika oder eben auch den Berufsfachschulbereich mit niedrigen Zugangsvoraussetzungen (z. B. am DRK Bildungswerk SN die Krankenpflegehilfe mit Zugangsvoraussetzung Hauptschulabschluss) dar \footcite[vgl.][15]{BBF2015}. Diese eben ausgeführte Gesamtsituation zeigt sich auch an den Berufsfachschulen mit einem insgesamt deutlichen Rückgang der Bewerberzahlen und der damit verbundenen geringeren Auswahl fachlich und persönlich geeigneter Schüler sowie damit einhergehend auch einer steigenden Heterogenität in den Klassen. 

Erschwerend kommt hinzu, dass derzeit die "`Helferberufe"' aktuell mit einem niedrigeren Status konfrontiert sind, vermeintlich schlechte Arbeitsbedingungen, Schichtdienste und ein vergleichsweise geringes Einkommen bei hoher Verantwortung sind Schwerpunkte, die bei Jugendlichen und jungen Erwachsenen offenbar eine abschreckende Wirkung ausmachen, was sich besonders an den Bewerberzahlen in der Altenpflege am DRK Bildungswerk SN zeigen lässt. 

\begin{quotation}
\noindent
"`So verwundert es auch nicht, dass das Image der Pflegeberufe in der Öffentlichkeit schlecht und der Beruf bei Schulabgänger/-innen unattraktiv ist."'
\end{quotation}

\noindent 
[\punkte] stellt dazu zutreffend Hall in ihrer Publikation fest \footcite[19]{Hall2012}. Noch drastischer wird diese Entwicklung in einer Studie des ipp Bremen dargestellt, in der Berufe in der Pflege und dabei insbesondere in der Altenpflege als sogenannte "`Out-Berufe"' aus der Sicht der Jugendlichen bezeichnet werden \footcite[18]{BPHP2010}. Mit diesen kurzen Auszügen können im Ansatz Veränderungen in der Ausbildungs- und Bewerbersituation der Gesundheits-, Pflege und Sozialberufe dargestellt werden, mit denen ein scheinbar sinkendes fachliches Niveau und auch soziale Probleme verbunden sind, die in die genannten Ausbildungsbereiche hineinwirken und die Heterogenität innerhalb der Klassen mitbedingen. Viele weitere Fragestellungen und Probleme der beruflichen Ausbildung stehen damit in Zusammenhang, so scheinen bspw. die Lehrpläne noch vielfach an ein nicht mehr vorhandenes Schülerklientel angepasst zu sein, was eine fachliche Überforderung nach sich zieht, die immer wieder beobachtbar wird. Zu all den dargestellten Herausforderungen gesellen sich, z. B. durch das Aufkommen und die Verbreitung der sozialen Netzwerke ganz neue Problemstellungen für die Berufsausbildung, die neben der Ablenkung und Beschäftigung im Unterricht und auch ganz neue Tendenzen für Mobbing und Diskriminierung mit sich gebracht haben und Lehrkräfte sowie ganze Schulen vor gänzlich neue Probleme stellen.

All die genannten und hier kurz vorgestellten Beobachtungen, Wahrnehmungen und Tendenzen und noch viele mehr wirken in die Zusammensetzung der Schülerschaft mit ihren individuellen Problemlagen hinein und bedingen somit auch in die aktuellen Herausforderungen für die berufsbildenden Schulen bzw. möglicherweise auch den Bedarf für außerunterrichtliche Beratungs- und Unterstützungsangebote. Deshalb wurden sie als Hintergrundinformationen dem sich nun anschließenden Forschungsteil vorausgeschickt. 

\section{Vorstellung des Forschungsvorhabens}
\label{sec:VorstellungDesForschungsvorhabens}

\subsection{Das erste eigene Forschungsprojekt - Eine Hinführung}
\label{sec:DasErsteEigeneForschungsprojektEineHinführung}

Lehrer sein. OK. Forscher sein. OK. Aber forschende Lehrer?

Als Student an einer deutschen Hochschule ist man am Ende seines Studiums dazu aufgefordert eine Abschlussarbeit zu verfassen, welche die erworbenen Kenntnisse, Methoden und Wissensstrukturen angemessen widerspiegelt. Auch die Autorinnen standen zum Ende ihres 5-jährigen Studiums für das Höhere Lehramt an berufsbildenden Schulen, sprich als angehende Berufsschullehrer, vor dieser abschließenden Herausforderung. Bereits Monate zuvor macht man sich Gedanken, welche Fachrichtung man dafür heranziehen möchte, welches Themengebiet spannend und erforschbar wäre, welche Betreuer sich der Arbeit annehmen würden und wie man diesen Arbeitsprozess überhaupt organisieren könnte? 

Doch zunächst ein Blick zurück ins Jahr 2010. In diesem Jahr haben sich Daniela Wobst und Doreen Stichel dazu entschlossen, ein Studium für das Lehramt an berufsbildenden Schulen im Bereich Gesundheit und Pflege und Sozialpädagogik an der TU Dresden aufzunehmen. Zu diesem Zeitpunkt hatten beide bereits eine Ausbildung absolviert, entsprechende Berufserfahrung gesammelt und waren in verschiedenen Branchen des Gesundheitswesens tätig. Beide hatten sich mit Aufnahme des Studiums das Ziel gesetzt, den bestehenden beruflichen Qualifikationen neue Wissensbereiche, Kompetenzen und berufliche Möglichkeiten hinzuzufügen. Zum Zeitpunkt der Immatrikulation war das Lehramt für berufsbildende Schulen in Bachelor- und Masterstudiengang organisiert, wohingegen im Jahr 2013 das Staatsexamen erneut eingeführt wurde. 

Zurück im Jahr 2015 stand man nun, wie alle Kommilitonen, vor der Frage, welches Thema für eine wissenschaftliche Bearbeitung im Rahmen einer Masterarbeit geeignet wäre. Unabhängig voneinander haben beide Autorinnen bei der Eingrenzung der möglichen Themen besonderes Interesse für den Gegenstand der Schulsozialarbeit gezeigt. Aufgrund einer guten persönlichen Beziehung und positiver Erfahrungswerte in Bezug auf die Erarbeitung gemeinsamer Seminararbeiten entschloss man sich dazu, das bestehende Forschungsinteresse auf dem Gebiet der Schulsozialarbeit zu nutzen und eine gemeinsame Arbeit in die Wege zu leiten. 

Die wissenschaftliche Bearbeitung einer Problemstellung im Rahmen einer Masterarbeit an der Fakultät Erziehungswissenschaften kann durch zwei verschiedene Herangehensweisen erfolgen. Entweder man nutzt den theoretischen Zugang mittels der literaturbezogenen Bearbeitung einer wissenschaftlichen Problemstellung oder den praktischen Zugang in Form eines eigenen Forschungsprojektes.
Aufgrund persönlicher Interessen und Ansichten erschien ein kleines Forschungsprojekt, welches in der Lage ist, eigene Erkenntnisse zu generieren, äußerst attraktiv und zugleich motivierend für den gemeinsamen Forschungsprozess. Zudem entspricht diese Vorgehensweise dem Anspruch an eine Masterprüfung gemäß § 20 der Prüfungsordnung für das Höhere Lehramt an berufsbildenden Schulen nach dem "`der Studierende die fachlichen Zusammenhänge überblickt sowie die Fähigkeit besitzt, wissenschaftliche sowie gegebenenfalls künstlerische Methoden und Erkenntnisse anzuwenden, und die für den Übergang in den für die Befähigung für das Höhere Lehramt an berufsbildenden Schulen vorgeschriebenen Vorbereitungsdienst bzw. eine Promotion notwendigen gründlichen Fachkenntnisse erworben hat. Ebenso wird festgestellt, dass der Studierende über vertiefte fachliche Kenntnisse und berufsfeldbezogene Qualifikationen als Beschäftigungsbefähigung für eine Tätigkeit in Berufsfeldern des öffentlichen oder privaten Bildungssektors verfügt."' \footcite[13]{TUDresden2010}

Angesicht dieser Ausgangslage entschied man sich im Dialog für ein gemeinsames Forschungsprojekt, welches sich dem Thema der Schulsozialarbeit an berufsbildenden Schulen widmen soll. 

Nach einer grundlegenden Sichtung der zum Zeitpunkt der Arbeit veröffentlichten Literatur zum Thema Schulsozialarbeit an berufsbildenden Schulen zeigte sich eine deutliche geringere Repräsentativität sozialpädagogischer Beratungs- und Unterstützungsangebote an berufsbildenden Schulen im Vergleich zu derer an allgemeinbildenden Schulen beispielsweise in Form der Grund- oder Förderschule; ein Umstand der u. a. gesetzliche Gründe hat (siehe Punkt \ref{sec:RechtlicheGrundlagen}) aber auch Verwunderung hervorrief, korrelierte er doch mit persönlichen Vermutungen und Erfahrungen. Aus diesem Grund setzte man sich nun zum Ziel die Probleme und Unterstützungsbedarfe von Schülern reell zu erfassen, um so eine Aussage über den potentiellen Bedarf von Schulsozialarbeit für Schüler an berufsbildenden Schulen formulieren zu können. Aufgrund der begrenzten zeitlichen und personellen Möglichkeiten entschied man sich für eine exemplarische Untersuchung an einer berufsbildenden Schule in Sachsen.
 
Die für die Realisierung des Forschungsprojekts notwendige Kooperationsvereinbarung mit einer berufsbildenden Schule konnte aufgrund der jahrelangen Tätigkeit von Daniela Wobst am DRK Bildungswerk SN und der freundlichen Genehmigung der dortigen Geschäftsleitung in Person von Hr. Vlodrop, Fr. Hösel und Hr. Eckert umgesetzt werden. Die genauen Abläufe sollen nun im Folgenden vorgestellt werden.

\subsection{Problemdefinition}
\label{sec:Problemdefinition}

\subsubsection{Relevanz des Themas}
\label{sec:RelevanzDesThemas}

Da unter Punkt \ref{sec:BedarfeFürSchulsozialarbeit} die Sachlage zum Thema der Schulsozialarbeit an beruflichen Schulen in Sachsen bereits ausführlich erläutert wurde, wird hier auf eine grundlegende Darstellung zur Relevanz des Themas verzichtet. Die folgenden Ausführungen, eine Kurzfassung bisheriger Inhalte, dienen der Hinführung zum Forschungsprojekt.

Es gibt subjektiv viele Problemlagen, denen sich Jugendliche bzw. junge Menschen in Ausbildung gegenüber sehen, dennoch sind kaum Studien, Bedarfsanalysen oder konkrete Angebote von Schulsozialarbeit zur Lösung dieser Probleme vorhanden. Forschungen und Publikationen zum Thema sind zwar sehr vielfältig und zahlreich, darunter auch umfassende Begründungen zu Notwendigkeit und Bedarf, diese richten sich aber nur an den grundschulischen und allgemeinbildenden Bereich. Trotz intensiver Recherche in Fachliteratur und einschlägigen Datenbanken existieren nur sehr wenige Ausführungen, Studien und Bedarfsanalysen für den berufsbildenden Bereich. Eine Ursache hierfür ist gesetzlich bedingt, denn Schulsozialarbeit ist in Sachsen nur in Schulen mit Berufsvorbereitungsjahr vorgeschrieben (siehe Punkt \ref{sec:RechtlicheGrundlagen}). Dem gegenüberzustellen sind jedoch die der sozialwissenschaftlichen Literatur zu entnehmenden vielfältigen Problemlagen von Jugendlichen und jungen Menschen in Ausbildung, die mit gesellschaftlichen und strukturellen sowie institutionellen Gegebenheiten in Zusammenhang stehen. 

Doch inwiefern machen diese Problemlagen den Einsatz von Schulsozialarbeit an berufsbildenden Schulen prinzipiell erforderlich, wenn man die momentan geringen Angebote für diese Schulform beachtet? 

\subsubsection{Forschungsgegenstand}
\label{sec:Forschungsgegenstand}

Im Rahmen der Masterarbeit sollen die Problemlagen und außerunterrichtlichen Be\-rat\-ungs- sowie Unterstützungsbedarfe von Schülern an einer exemplarischen berufsbildenden Schule in Sachsen erfasst werden.

Zusätzlich soll die Sicht der Lehrkräfte hinsichtlich der Problemlagen von Schülern sowie mögliche Beratungs- und Unterstützungsbedarfe durch Lehrkräfte eruiert werden.

\subsubsection{Nutzen der Forschung/Forschungslücke}
\label{sec:NutzenDerForschungForschungslücke}

Lehrkräfte nehmen Problemlagen von Schülern und individuelle Unterstützungsbedarfe (subjektiv) wahr, sehen sich jedoch mit den teilweise komplexen Problemlagen überfordert und bemängeln die unzureichende Unterstützung in dieser Sachlage. Sie fordern eine Rückbesinnung auf ihr Kerngeschäft -- nämlich den Unterricht.

Die geplanten Datenerhebungen aus Schüler- und Lehrerperspektive untersuchen die tatsächlichen Probleme von Schülern und Bedarfe für Schulsozialarbeit an einer berufsbildenden Schule in Sachsen. Abgeleitet werden sollen sowohl die Bedarfe als auch mögliche Beratungs- und Unterstützungsangebote.

\subsubsection{Vorannahmen}
\label{sec:Vorannahmen}

\begin{itemize}
	\item Die Problemlagen von Schülern an berufsbildenden Schulen unterscheiden sich wenig bis gar nicht von denen an allgemeinbildenden Schulen.
	\item Außerunterrichtliche Problemlagen wirken sich tendenziell negativ auf das Unterrichtsgeschehen aus.
	\item Lehrkräfte sehen Problemlagen bei Schülern und erkennen Bedarfe für sozialpädagogische Unterstützungsangebote. 
\end{itemize}

\subsubsection{Forschungsfragen}
\label{sec:Forschungsfragen}

\begin{enumerate}
	\item Welche persönlichen und sozialen (außerunterrichtlichen) Problemlagen haben Schüler des DRK Bildungswerk SN in sozialen und medizinischen Ausbildungsberufen, die das Unterrichtsgeschehen und den Ausbildungserfolg beeinflussen?
	\item Welche persönlichen und sozialen (außerunterrichtlichen) Problemlagen von Schüler nehmen Lehrkräfte des DRK Bildungswerk SN im Bereich Gesundheit, Pflege und Sozialwesen als besondere Belastung für den Unterricht wahr?
	\item Wie schätzen Schüler und Lehrkräfte anhand der (möglichen) subjektiv wahrgenommenen Problemlagen den Bedarf an sozialpädagogischen und anderweitigen Unterstützungs- und Beratungsangeboten ein und wie könnten passende Angebote aussehen?
\end{enumerate}

\noindent
\textbf{Begründung}\\

\noindent
Es existieren subjektiv wahrgenommene Bedarfe an sozialpädagogischen Unterstützungsangeboten für persönliche außerunterrichtliche Probleme, welche das Unterrichtsgeschehen negativ beeinflussen.\\

\noindent
\textbf{Erläuterung der Forschungsfragen}\\

\noindent
Zu Beginn der vorliegenden Arbeit wurde das Prinzip der Schulsozialarbeit in Deutschland bzw. Sachsen als Ausgangspunkt der weiterführenden Überlegungen grundlegend dargelegt. Dennoch wurde, wie bereits am Titel der Arbeit erkennbar, bei der Formulierung der Forschungsfragen auf eine Verwendung dieses Begriffes verzichtet. Diesen Umstand gilt es aus verschiedenen Sichtweisen zu klären.
 
Wie bereits unter Punkt \ref{sec:AllgemeineTheoretischeGrundlagenZurSchulsozialarbeit} erläutert wurde, ist das Wort Schulsozialarbeit ein sehr umfassender Begriff, welchem eine Vielzahl an Maßnahmen, Zielgruppen, Einsatzmöglichkeiten und Zielstellungen zuzuordnen sind. Viele dieser Maßnahmen und Angebote sind jedoch aufgrund gesetzlicher oder schulspezifischer Gegebenheiten nicht im Kontext der berufsbildenden Schulen, explizit am DRK Bildungswerk SN umsetzbar. Das schließt u. a. die am Zielort zu berücksichtigende Abwesenheit eines Schulsozialarbeiters ein. Unabhängig davon sind sowohl sozialpädagogische als auch anderweitige Angebote, die durch die Lehrkräfte aufgezeigt werden können, im Fokus des Forschungsinteresse. Diese können möglicherweise in Auszügen den genannten Angeboten der Schulsozialarbeit zugeordnet werden (siehe Punkt \ref{sec:Methoden}) und so eine Verbindung zwischen Theorie und Praxis aufzeigen. Zudem ist darauf hinzuweisen, dass die Erfassung subjektiver Problemlagen und die Eruierung potentieller Unterstützungsbedarfe von Schülern im Fokus dieser Untersuchung stehen. Konzeptionelle Überlegungen hinsichtlich möglicher sozialpädagogischer oder anderweitiger Angebote werden als Ergebnis der Bedarfsanalyse erst am Ende der Arbeit formuliert und sind als Anregung/Empfehlung für das DRK Bildungswerk SN zu verstehen.
 
Wie im weiteren Verlauf erkennbar, wurde bei der Formulierung der Umfrage und der Interviews ebenfalls auf den Begriff Schulsozialarbeit verzichtet, da dieser für viele Lehrer und Schüler ein zumeist unbekannter und schwer zu definierender Begriff ist. Schulsozialarbeit hätte aus dieser Vermutung heraus zunächst erst als "`Fremdwort"' grundlegend erläutert werden müssen; dem wurde entsagt.

Aus diesen Gründen wurde im Rahmen der Untersuchung komplett darauf verzichtet, den Begriff Schulsozialarbeit zu verwenden und eher die leicht verständliche Formulierung von "`Beratungs- und Unterstützungsangeboten"' gewählt. Darunter werden alle möglichen Angebote mit beratenden und unterstützenden Charakter subsumiert, welche einen Benefit für die Schüler erkennen lassen.

Desweiteren sollen im Rahmen dieses Forschungsprojektes ausschließlich außerunterrichtliche Probleme und Bedarfe eruiert werden. Das bedeutet, dass nur die Problemlagen Betrachtung finden, die unabhängig von Unterrichtsinhalten entstehen, das schulische Miteinander und Erleben jedoch stark beeinflussen können. Beispiel hierfür sind finanzielle oder familiäre Probleme. Es ist in dieser Studie nicht von Interesse, inwieweit Schüler mit ihren Lehrkräften zufrieden sind oder ob Unterrichtsinhalte interessant und abwechslungsreich vermittelt werden; jegliche Problematiken, die innerhalb des Unterrichtsgeschehens entstehen, werden nicht betrachtet. 

Um ein möglichst realistisches und breitgefächertes Abbild der momentanen Probleme und Unterstützungsbedarfe von Berufsschülern in Sachsen zu erhalten, wurde im Vorfeld der Untersuchung beschlossen, zwei verschiedene Perspektiven in Bezug auf das Forschungsinteresse in die Analyse einzubinden. Neben der Schülerperspektive wird auch die Lehrerperspektive bei der Erfassung der Daten berücksichtigt. Beide Parteien haben trotz des gemeinsam erlebten Schulalltages ggf. eine unterschiedliche Wahrnehmung bezüglich der Probleme und Unterstützungsbedarfe von Schülern. Daher betrachtet die erste Forschungsfrage ausschließlich die Schülerperspektive; wo hingegen die zweite Forschungsfrage ausschließlich die Lehrerperspektive in den Fokus stellt. Beide Sichtweisen sollen schließlich mit der dritten Forschungsfrage zusammengeführt und auf eine Schnittmenge hin untersucht werden.

\subsection{Projektplanung}
\label{sec:Projektplanung}

\subsubsection{Zugang zum Forschungsfeld}
\label{sec:ZugangZumForschungsfeld}

Zu Beginn der Projektplanung gab es mehrere Ideen, diese institutionell umzusetzen. In Sachsen gibt es eine Vielzahl von berufsbildenden Schulen unter privater sowie öffentlicher Trägerschaft. Der Gedanke, mehrere öffentliche Schulen zur Generierung einer großen Datenmenge heranzuziehen, wurde aufgrund der geringen zeitlichen Ressourcen negiert. Desweiteren wurden Überlegungen angestellt, öffentliche und private schulische Einrichtung(en) bei der Auswertung der beabsichtigten Ergebnisse zu vergleichen; dieser Aspekt wurde jedoch aufgrund zeitlicher und organisatorischer Hürden ebenso verworfen. Letztendlich wurde beschlossen, dass nur eine exemplarische private schulische Einrichtung zur Projektrealisierung berücksichtigt werden soll. Diese Entscheidung offeriert den Autorinnen bessere personelle, organisatorische und zeitliche Absprachen und ermöglicht zudem eine fokussierte, abgrenzbare und ganzheitlich angelegte Darstellung einer berufsbildenden Schule hinsichtlich der gewählten Problemstellung. Diese exemplarische  Betrachtungsweise ermöglicht zudem konkrete Empfehlungen hinsichtlich Bedarf und Konzeption möglicher Beratungs- und Unterstützungsangebote für eine schulische Einrichtung.

Wie bereits angesprochen, wurde die notwendige schulische Kooperation für dieses Forschungsprojekt durch die freundliche Genehmigung der Geschäftsleitung des DRK Bildungswerk SN ermöglicht, wobei Daniela Wobst, bereits viele Jahre bei dieser Bildungsstätte angestellt, als "`Gatekeeper"' (Türöffner) fungierte. 

\subsubsection[Selektive Erarbeitung im Rahmen einer universitären Gruppenarbeit]{Darlegung über die selektive Erarbeitung relevanter Inhalte im Rahmen einer universitären Gruppenarbeit}
\label{sec:DarlegungÜberDieSelektiveErarbeitungRelevanterInhalteImRahmenEinerUniversitärenSeminararbeit}

Bestandteile der vorliegenden Ausarbeitung beruhen auf einer universitären Gruppenarbeit; ein Umstand der zu klären ist. Zeitgleich mit Beginn der Masterarbeit wurden Daniela Wobst und Doreen Stichel Teil einer universitären Arbeitsgruppe, die bis Ende September 2015 Bestand hat. Diese Gruppe formierte sich im Rahmen des Seminars "`Forschungsfelder"', einem Modul der Fachrichtung Gesundheit und Pflege unter Leitung von Fr. Thümmler. Im Rahmen dieser Veranstaltung ist eine Prüfungsleistung in Form einer Qualitativen Studie zu erbringen. Im gemeinsamen Dialog entschloss man sich, das bereits bestehende Thema der Autorinnen aufzugreifen und die Sicht der Lehrkräfte hinsichtlich der Problemlagen von Schülern sowie mögliche Beratungs- und Unterstützungsbedarfe durch Lehrkräfte in Bezug auf die o. g. Forschungsfrage in Auszügen für die Seminararbeit heranzuziehen bzw. zu bearbeiten. Die Gruppenmitglieder, bestehend aus Janet Kaiser, Juliane Hemmerling, Anne Krause, Daniela Wobst und Doreen Stichel, erarbeiteten somit gemeinsam Inhalte, die auch in die vorliegende Arbeit eingeflossen sind. Inhalte und Arbeitsschritte, die unter Mitwirkung dieser Arbeitsgruppe erbracht wurden, werden im weiteren Verlauf kenntlich gemacht. Da die entsprechende Seminararbeit mit dem vorläufigen Arbeitstitel "`Eine qualitative Einzelfallstudie am DRK Bildungswerk Sachsen"' zum Zeitpunkt der Abgabe dieser Arbeit noch nicht vorlag, kann hiermit nur eine allgemeine Referenz auf diese erfolgen \footcite{Hemmerling2015}. Alle Gruppenmitglieder haben sich jedoch schriftlich damit einverstanden erklärt, das etwaige Inhalte, die unter Mithilfe derer entstanden sind, in der vorliegenden Masterarbeit verwendet werden dürfen (siehe \ref{sec:Sonstiges} Einverständniserklärung Gruppe).

\subsubsection{Universitäre Betreuung}
\label{sec:UniversitäreBetreuung}

Es ist uns ein Anliegen, die fachkundige, freundliche und unterstützende universitäre Betreuung durch Prof. Dr. Gängler, Dipl. Med-Päd. Haupt und Fr. Thümmler bei der Realisierung des Forschungsprozesses zu betonen. Deren hilfreiche Tipps, Anmerkungen und seminargebundenen Inhalte haben zum Gelingen dieses Projektes maßgeblich beigetragen.

\subsubsection{Grundlegende Organisation des Forschungsprojektes}
\label{sec:GrundlegendeOrganisationDesForschungsprojektes}

Im Folgenden sollen die grundlegenden organisatorischen Schritte dargelegt werden, die zur Realisierung des Forschungsprojektes notwendig waren. 

Die ersten Schritte auf dem Weg zur gemeinschaftlichen Abschlussarbeit waren bereits im Februar 2015 vollzogen. Themengebiet, Problemstellung, Forschungsfrage(n) und wesentliche Inhalte der geplanten Untersuchung wurden im stetigen Dialog durch die Autorinnen festgelegt. Ehe jedoch mit der eigentlichen Projektumsetzung begonnen werden konnte, hielt man zunächst Rücksprache mit den universitären Gutachtern, welche sich dem Thema angenommen hatten. Dank dieser kompetenten und inhaltlich ergiebigen Gespräche konnten neue Details und Gedanken in das geplante Forschungsvorhaben einfließen. Nach Abschluss dieser grundlegenden Arbeitsschritte konnte mit der detaillierten Planung der Erhebung begonnen werden. 
Es ist zu beachten, dass einige Arbeitsschritte, welche im Folgenden beschrieben werden, gleichzeitig erfolgten, was eine klare zeitliche Einordnung erschwert.
Dank dem Engagement von Daniela Wobst, u. a. Fachbereichsleiterin der Diätassistenz am DRK Bildungswerk SN, konnte diese langjährige private Bildungseinrichtung in Dresden als Untersuchungsfeld bereits frühzeitig festgelegt werden. Nach ersten mündlichen Absprachen von Frau Wobst mit der Geschäftsführung, unterbreiteten beide Autorinnen zu einem späteren Zeitpunkt ihr genaues Forschungsvorhaben im Rahmen einer kleinen Präsentation vor der gesamten Geschäftsleitung, so dass Ziel, Methoden und Verwendung der gesammelten Daten transparent erläutert werden konnten. Im Laufe dieses freundlichen Gesprächs konnten zudem organisatorische und inhaltliche Fragen geklärt werden und den Autorinnen wurde anschließend Unterstützung in Form der Bereitstellung der zahlreichen Umfragekopien zugesichert. Dank der freundlichen Projektbewilligung durch die Schulleitung konnte man im Anschluss an die betreffenden Lehrer und Schüler herantreten.

Bereits zuvor wurden Überlegungen angestellt, welche wissenschaftlichen Methoden sich als geeignet zur Beantwortung der gewählten Forschungsfragen hinsichtlich der Schü- ler- und  Lehrerperspektive erweisen könnten. Um möglichst wirklichkeitsnahe Aussagen in Bezug auf etwaige Schülerprobleme und Bedarfe unter Berücksichtigung der zur Verfügung stehenden Zeit zu generieren, entschied man sich für zwei verschiedene Methoden. Der Umfragebogen, eine Methode der quantitativen Forschung, ermöglicht das Zusammentragen potentieller Probleme und Bedarfe für Schulsozialarbeit von möglichst vielen Schülern; das Interview hingegen, eine Methode der qualitativen Forschung, dient der Erfassung subjektiver Aussagen von ausgewählten Lehrkräften hinsichtlich der Probleme und Bedarfe ihrer Schüler. Die genauere theoretische Einbettung und praktische Erarbeitung dieser Methoden wird unter Punkt \ref{sec:DieUmfrage} sowie \ref{sec:DasInterview} vertieft dargelegt.

Im nächsten Schritt wandt man sich nun an die Lehrkräfte des DRK Bildungswerk SN. Zu diesem Zweck kontaktierte Daniela Wobst zunächst alle betreffenden Personen über die berufliche E-Mail-Adresse. Über diesen Weg wurden sie über Inhalt, Ziel und Ablauf des universitären Forschungsprojektes an der Schule informiert, über anstehende Besuche in den Klassen bezüglich der Schülerumfragen in Kenntnis gesetzt, aber auch um persönliche Unterstützung in Form der Interviewteilnahme gebeten. Im Anschluss erreichten Daniela Wobst viele freundliche und interessierte Reaktionen diesbezüglich, aber auch zahlreiche Angebote von Lehrkräften, welche bereitwillig an einem Interview teilnehmen würden. Da die betreffenden Pädagogen zumeist in verschiedenen Ausbildungsgängen tätig sind, entschied man sich für eine Auswahl an Personen, welche mit ihrem Erfahrungshorizont die vorliegenden Ausbildungsrichtungen am DRK Bildungswerk SN möglichst komplett abbilden konnten. Somit konnten nach und nach 5 Lehrpersonen kontaktiert und für ein Interview gewonnen werden. Die individuelle Terminvergabe der Interviews erfolgte erneut per E-Mail oder telefonisch, organisiert durch Doreen Stichel. Bereits während dieser Absprachen zeigten die ausgewählten Lehrkräfte großes Interesse an Inhalt, Durchführung und Ergebnis der gesamten Untersuchung.

Nachdem die Lehrer über das Forschungsprojekt in Kenntnis gesetzt wurden, oblag es ihnen, die betreffenden Klassen auf den Besuch von Doreen Stichel oder Daniela Wobst vorzubereiten. Die Autorinnen nahmen vor Beginn der Untersuchung selbst keinen Kontakt zu den Klassen auf.

Ehe mit der Untersuchung begonnen werden konnte, musste zunächst die genaue Zielgruppe der zu befragenden Schüler am DRK Bildungswerk SN untersucht werden. Um ein möglichst realistisches Bild der vorhandenen Problemlagen und potentiellen Unterstützungsbedarfe zu erhalten, galt es, mit der Stichprobe die Grundgesamtheit der schulischen Einrichtung abzudecken. Von der Umfrage ausgeschlossen wurden jedoch die berufsbegleitenden Klassen. Schüler dieser Klassen zeichnen sich durch ein deutlich höheres Durchschnittsalter aus, besitzen bereits einen Beruf, oftmals eine eigene Familie mit Kindern und haben daher gänzlich andere Problemlagen als Schüler, die ihre Erstausbildung machen und zumeist noch im Elternhaus wohnen. Daher wurden, abzüglich der berufsbegleitenden Klassen, alle zum Zeitpunkt der geplanten Umfrage anwesenden Ausbildungsklassen erfasst. Diese wurden vermerkt, in der Personenanzahl erfasst und die für die Befragung nötige Anzahl ermittelt. Die genaue Darlegung dieses Prozesses, untermauert von einigen theoretischen Ausführungen zu Fragekonzeption und Methode, erfolgt unter Punkt \ref{sec:DieUmfrage}. 

\subsubsection{Angestrebte wissenschaftliche Ziele}
\label{sec:AngestrebteWissenschaftlicheZiele}

\textbf{Ziele im Rahmen des Forschungsinteresses}

\begin{itemize}
	\item Die Bearbeitung des Themas dient dem Abgleich von subjektiv wahrgenommenen Problemlagen von Schülern an berufsbildenden Schulen der Forschungsgruppe und vielen Lehrkräften mit den real beschriebenen Problemlagen von Schülern durch Schüler und Lehrkräfte an einer berufsbildenden Schule.
	\item Das Projekt ermöglicht einen konkreten Einblick in reale Problemlagen der Schüler am DRK Bildungswerk SN, welche den Unterricht negativ beeinflussen bzw. denen die Lehrkräfte in ihrem beruflichen Selbstbild potentiell nicht gewachsen sind.
	\item Aus der Erfassung möglicher Problemlagen könnte ein Bedarf für außerunterrichtliche sozialpädagogische Unterstützungsangebote abgeleitet werden.
\end{itemize}

\noindent
\textbf{Angestrebte Ziele für das DRK Bildungswerk Sachsen}
 
\begin{itemize}
	\item Für Schüler mit möglichen außerunterrichtlichen Problemlagen:
	\\
	Ziel ist es, die Lernleistung durch Minderung oder Lösung persönlicher Problemlagen, die die Aufmerksamkeit auf Schule und Unterricht senken, zu steigern.
	\item Entlastung der Lehrer bei möglichen außerunterrichtlichen Problemlagen:
	 \\
	Ziel ist es, die zusätzliche zeitliche und psychische Belastung der Lehrer zu reduzieren. 
	\item Auf einen möglichen Bedarf, durch das Heranziehen von Beratungs- und Unterstützungsangebote entsprechend zu reagieren.
\end{itemize}

\newpage
\subsubsection{Eigene Vorarbeiten und Expertisen}
\label{sec:EigeneVorarbeitenUndExpertisen}

\noindent
\textbf{Gruppenvorstellung}\\

\noindent
Da einige Inhalte dieser Studie auf einer Gruppenarbeit von 5 Personen beruhen, soll die Expertise aller Gruppenmitglieder hiermit auch berücksichtigt werden.

Alle Teilnehmer der Seminargruppe haben eine abgeschlossene Berufsausbildung, welche sich durch 2 Physiotherapeutinnen sowie 1 Diätassistentin, 1 Operationstechnische Assistentin und 1 Medizinische Fachangestellte näher charakterisieren lässt. Alle Beteiligten haben zum Zeitpunkt der Befragung den Abschluss "`Bachelor of Education"' und streben ihren Masterabschluss für das Höhere Lehramt an berufsbildenden Schulen an. Zudem weisen alle die berufliche Fachrichtung Gesundheit und Pflege im Erstfach auf, jedoch verschiedene Zweitfächer.

Neben den praktischen Lehrerfahrungen aus dem Blockpraktikum A, Blockpraktikum B sowie den Schulpraktischen Übungen im Rahmen der universitären Ausbildung besitzen alle Personen individuelle Erfahrungen mit außerunterrichtlichen Problemlagen durch Mitschüler während der eigenen Ausbildung, aber z. T. auch weitere Erfahrungen durch die Betreuung von Praktikanten und durch eigene Lehrtätigkeiten.

Aufgrund dieser Sachlage sind genügend Situationen vertraut, in welchen die Unterrichtstunde nicht für Lerninhalte, sondern zum Klären von außerunterrichtlichen Problemen genutzt wurde. Das wurde sowohl von Lehrern als auch von nicht betroffenen Schüler als belastend empfunden.

Bekannte Ausbildungsschulen bzw. Praktikumsschulen aus dem Erfahrungsbereich der Gruppenmitglieder wiesen bisher keine sozialpädagogischen Unterstützungsangebote auf, obwohl z. T. subjektiv gesehen Bedarf bestanden hätte.\\

\noindent
\textbf{Vorstellung der Autorinnen}\\

\noindent
Die bisher erfolgten "`gruppenverbindenden"' Angaben sollen hiermit um ein paar individuelle Informationen der Autorinnen erweitert werden.\\

\noindent
\underline{Vorbildung Daniela Wobst}
\begin{itemize}
	\item Erfahrungen als Schülerin im Berufsfeld: 1998-2001 Ausbildung zur Diätassistentin
	\item 2001-2003: Praxisanleiterin für Praktikantinnen in Krankenhaus und Rehaklinik
	\item 2003-2010: Lehrkraft für fachpraktischen Unterricht im Fachbereich Diätassistenz sowie Klassenlehrertätigkeit; QM-Beauftragte des DRK Bildungswerk Sachsen mit Aufgabe der Schülerbetreuung (Klassensprechersitzungen, Zusammenkünfte mit Schülervertretern)
	\item seit 2010: weitere unterrichtliche Tätigkeit (fachpraktischer und theoretischer Bereich) sowie Beginn des Lehramtstudiums an der TU Dresden
	\item seit 2014: Fachbereichsleitung der Berufsfachschule für Diätassistenten am DRK Bildungswerk SN
\end{itemize}

\noindent
\underline{Vorbildung Doreen Stichel}
\begin{itemize}
	\item Erfahrungen als Schülerin im Berufsfeld: 2002-2005 Ausbildung zur Physiotherapeutin
	\item 2006-2012: Praktische Erfahrungen im Bereich der physiotherapeutischen Praxis, der stationären Behandlung, Intensivtherapie sowie in der Fitnessbranche
	\item Erfahrung im Umgang mit Praktikanten und Auszubildenden im Rahmen der beruflichen Tätigkeit
	\item 2010: Aufnahme des Lehramtstudiums an der TU Dresden
\end{itemize}

\newpage
\subsubsection{Arbeitsprogramm (inkl. Zeitplan)}
\label{sec:ArbeitsprogrammInklZeitplan}

\begin{longtable}{l|p{9.8cm}}
	
	\textbf{1. Problemdefinition} & \\
	\emph{Januar 2015} &
	\vspace*{-0.6cm}
	\begin{itemize}[nosep,topsep=-0.6cm]
		\item Bildung der Arbeitsgruppe
		\item Erste Literaturrecherche
		\item Entwicklung der Fragestellung/Zielsetzung
	\end{itemize} \\* 
	
	\emph{Februar 2015} & 
	\vspace*{-0.6cm}
	\begin{itemize}[nosep,topsep=-0.6cm]
		\item Suche nach universitären Betreuern
		\item Vorstellung des Vorhabens bei den Betreuern
		\item Ausformulierung der Forschungsfrage
	\end{itemize} \\
	
	\multicolumn{2}{c}{Kurzzeitige Unterbrechung durch das Blockpraktikum B im März 2015} \\*
	
	\textbf{2. Projektplanung} & \textbf{Klärung Projektbewilligung am DRK Bildungswerk Sachsen} \\
	\emph{April 2015} & 
	\vspace*{-0.6cm}
	\begin{itemize}[nosep,topsep=-0.6cm]
		\item Vorstellung des Forschungsvorhabens bei der Geschäftsleitung
		\item Klären möglicher Fragen bezgl. der geplanten Schülerumfrage und der Lehrerinterviews
		\item Besprechen einer möglichen Ergebnispräsentation nach Fertigstellung der Masterarbeit
	\end{itemize} \\ 
	& \textbf{Ressourcenbestimmung Schüler} \\*
	\emph{Mai 2015} & 
	\vspace*{-0.6cm}
	\begin{itemize}[nosep,topsep=-0.6cm]
		\item Welche Ausbildungsrichtungen wollen wir befragen?
		\item Mögliche Ausschlusskriterien bestimmen
		\item Welche Klassen und Schüler stehen im Juni/Juli am DRK Bildungswerk SN zur Verfügung?
		\item Absprache mit den betreffenden Lehrer zum gegebenen Zeitpunkt
	\end{itemize} \\
	& \textbf{Ressourcenbestimmung Lehrer} \\*
	&
	\vspace*{-0.6cm}
	\begin{itemize}[nosep,topsep=-0.6cm]
		\item Eingrenzen geeigneter Lehrkräfte, um das gewählte Ausbildungsspektrum angemessen zu repräsentieren
		\item Anschreiben der Lehrkräfte bezüglich Forschungsvorhaben und Interviewanfrage
		\item Termine für Interviews festlegen
		\item Interviewer und Beisitzer festlegen
	\end{itemize} \\ 
	& \textbf{Erstellung der Umfragematerialien} \\*
	&
	\vspace*{-0.6cm}
	\begin{itemize}[nosep,topsep=-0.6cm]
		\item Umfragebogen (+ Pretest)
		\item Merkzettel zur Studie
	\end{itemize} \\
	& \textbf{Erstellung der Interviewmaterialien} \\*
	& 
	\vspace*{-0.6cm}
	\begin{itemize}[nosep,topsep=-0.6cm]
		\item Leitfragen des Interviews erstellen (+ Pretest)
		\item Datenschutzerklärung
		\item Einverständniserklärung
		\item Postscript
	\end{itemize} \\
	
	\textbf{3. Projektdurchführung} & \textbf{Datenerhebung} \\*
	\emph{Juni 2015} &
	\vspace*{-0.6cm}
	\begin{itemize}[nosep,topsep=-0.6cm]
		\item Durchführung der Schülerumfrage
		\item Durchführung der Lehrerinterviews I.01 -- I.05
	\end{itemize} \\
	& \textbf{Dateneingabe} \\*
	\emph{Juli 2015} &
	\vspace*{-0.6cm}
	\begin{itemize}[nosep,topsep=-0.6cm]
		\item Dateneingabe der Schülerumfragewerte in LimeSurvey
		\item Transkription der Interviews
	\end{itemize} \\
	& \textbf{Zwischenprojektevaluation} \\*
	\emph{Juli 2015} &
	\vspace*{-0.6cm}
	\begin{itemize}[nosep,topsep=-0.6cm]
		\item regelmäßiger Austausch in der Gruppe über Fortschritte und mögliche Probleme im Forschungsprozess
	\end{itemize} \\
	
	\textbf{4. Evaluation} & \textbf{Datenauswertung} \\*
	\emph{August 2015} &
	\vspace*{-0.6cm}
	\begin{itemize}[nosep,topsep=-0.6cm]
		\item ... der Schülerumfrage mittels Statistikausgabe von LimeSurvey und Gesprächen in der Gruppe
		\item ... der Lehrerinterviews mittels Qualitativer Inhaltsanalyse nach Mayring
	\end{itemize} \\
	
	\textbf{5. Projektbericht} & \\*
	\emph{bis Ende August 2015} &
	\vspace*{-0.6cm}
	\begin{itemize}[nosep,topsep=-0.6cm]
		\item Zusammenfügen aller Daten mit dem Ziel der Beantwortung der Forschungsfragen
		\item konzep. Überlegungen für SSA am DRK BWK SN
		\item Verschriftlichung der Ergebnisse
	\end{itemize} \\
	
	\textbf{6. Abgabe Masterarbeit} & \\*
	\emph{15.09.2015} & \\
	
\end{longtable}

\newpage

\subsubsection{Studiendesign und Methoden}
\label{sec:StudiendesignUndMethoden}

Um den vorliegenden Abschnitt nicht gänzlich mit Inhalten zu überladen, werden die Methoden der Umfrage und des Interview zum Zwecke der besseren Gliederungsmöglichkeit unter Punkt \ref{sec:DieUmfrage} sowie \ref{sec:DasInterview} näher erläutert.

\subsubsection{Erwartungshorizont}
\label{sec:Erwartungshorizont}

Aufgrund einer guten Kommunikation zwischen den Autorinnen und den betreffenden Lehrkräften am DRK Bildungswerk SN im Vorfeld der Befragung wird eine reibungslose Durchführung der Schülerumfrage sowie der Lehrerinterviews erwartet. Da Daniela Wobst aktuell am Bildungsinstitut beschäftigt ist, wird auch die organisatorische Gestaltung in Bezug auf Raumplanung, Bereitstellung von Kopien, kurzzeitige Absprachen mit Interviewpartnern oder das Auffinden der entsprechenden Klassen vermutlich problemlos ablaufen. Doch trotz einer sorgfältigen Planung und Organisation der Untersuchung bleiben Unsicherheiten in Bezug auf eine gelingende Durchführung und der zu gewinnenden Datenmenge mittels der genannten Methoden.

Bezüglich der Schülerumfrage werden folgende Bedenken gehegt: Was geschieht, wenn viele Schüler eine Befragung verweigern?; Wie wird mit fehlerhaft ausgefüllten Umfragebögen umgegangen?; Wie verfährt man, wenn keinerlei Probleme und Unterstützungsbedarfe benannt werden?

Im Hinblick auf die Lehrerinterviews gibt es im Vorfeld ebenso Unsicherheiten: Wie reagiert man auf Desinteresse oder vermeintlich unreflektierte Antworten der Gesprächspartner?; Werden die Lehrkräfte Probleme und Unterstützungsbedarfe für die Schüler wahrnehmen und benennen können? 

\subsubsection{Qualitätssicherung}
\label{sec:Qualitätssicherung}

Folgende Maßnahmen dienen der Qualitätssicherung der Forschungsarbeit:
\begin{itemize}
	\item grundlegende theoretische Fundierung, vor Aufnahme der eigenen Untersuchungen und anschließend zur wissenschaftlichen Einordnung gewonnener Erkenntnisse in bereits vorhandene Wissensstrukturen
	\item inhaltliche oder organisatorische Fragen, Aufgaben oder auch entstehende Probleme werden regelmäßig gemeinsam besprochen, diskutiert, reflektiert und evaluiert (das bezieht sich sowohl auf die gemeinsame Arbeit der Autorinnen als auch auf die partielle Zusammenarbeit mit der Seminargruppe)
	\item bei Fragen oder Problemen wird Rücksprache mit den Betreuern gehalten
	\item im Interviewverlauf wird eine Rückkopplung mit dem Interviewpartnern angestrebt, d. h., inwieweit die Interpretation des Interviewers, den Aussagen des Interviewpartners entspricht
	\item Sowohl vor Beginn der Schülerumfrage als auch vor den Lehrerinterviews wird ein Probelauf (Pretest) erfolgen. Dieser Schritt dient sowohl der Erprobung organisatorischer Abläufe als auch der Überprüfung der erarbeiteten Unterlagen (Umfragebogen; Interviewleitfaden) in Bezug auf Logik, Verständnis, Zeitvorgabe und potentielle Fehler. Zudem werden die Probanden im Anschluss um ein ehrliches Feedback gebeten. Die dabei involvierten Schüler und die Lehrperson werden bei der weiteren Datenerhebung nicht berücksichtigt, um keine Ergebnisverzerrung zu verursachen.
\end{itemize}

\subsection{Die Umfrage}
\label{sec:DieUmfrage}

\subsubsection{Hinführung}
\label{sec:Hinführung}

Nachdem soeben die grundlegende Herangehensweise an das Forschungsprojekt dargelegt worden ist, gilt es nun, den Arbeitsprozess bezüglich Konzeption und Erstellung der Schülerumfrage transparent auszuführen. Dazu soll im Folgenden die wissenschaftliche Methode des Fragebogens parallel zur Konzeption der Schülerumfrage in kurzen Zügen erläutert werden.

Der Fragebogen ist ein Instrument, welches aufgrund seiner inflationären Anwendung in den vergangenen Jahren zunehmend Einzug in das Alltagserleben der Menschen gehalten hat. Er findet Verwendung bei Straßenumfragen, TV-Shows, im Interesse der Kundenakquise bei Energie- oder Handyanbietern, zur Sicherung der Erfolgskontrolle bei Wirtschaftsunternehmen oder auch im Rahmen von Abschlussarbeiten von Studenten und Kommilitonen. Bei dieser mannigfaltigen Präsenz von Fragebögen kommt häufig die systematische und wissenschaftliche Ausarbeitung dieser Methode zu kurz \footcite[vgl.][11]{Kallus2010}. Doch mit dem Anspruch an wissenschaftliches Arbeiten gilt es, bei der Konzeption von Fragebögen gewissen Kriterien zu genügen, um so eine nachvollziehbare Systematik frei von subjektiven Deutungen zu konstruieren \footcite[vgl.][9]{Mayer2013}. Im Rahmen dieser Arbeit wurde versucht, diesen wissenschaftlichen Kriterien im Ansatz gerecht zu werden, wenn auch mangelnde Zeit, geringe Erfahrungswerte der Autorinnen und der exemplarische Charakter der Untersuchung das Unterfangen z. T. erschwerten. Zur Beantwortung der folgenden Forschungsfragen wurde der standardisierte Fragebogen als Erhebungsmethode herangezogen:\\
 

\noindent
\textbf{(1)} Welche persönlichen und sozialen (außerunterrichtlichen) Problemlagen haben Schüler des DRK Bildungswerk SN in sozialen und medizinischen Ausbildungsberufen, die das Unterrichtsgeschehen und den Ausbildungserfolg beeinflussen?\\

\noindent
\textbf{(3)} Wie schätzen Schüler (Anm: Lehrkräfte sind hier ausgenommen) anhand der (möglichen) subjektiv wahrgenommenen Problemlagen den Bedarf an sozialpädagogischen und anderweitigen Unterstützungs- und Beratungsangeboten ein und wie könnten passende Angebote aussehen?\\

\noindent
Zunächst wird die relevante Stichprobe ermittelt. Die Grundgesamtheit (also alle zu befragenden Schüler am DRK Bildungswerk SN) betrug zum Zeitpunkt der Umfragekonzeption 679 Schüler. Aus zeitlichen Gründen war es nicht möglich alle Personen zu befragen. Die eingehende Literaturrecherche ergab unterschiedliche bzw. uneindeutige Angaben hinsichtlich der passenden Stichprobengröße; daher wurde im anschließenden Gespräch mit Universitätsangehörigen, in Person von Prof. Dr. Gängler, Fr. Haase und Hr. Bloße, über eine passende Anzahl an Probanden diskutiert. 30\% wurden schließlich als repräsentative Stichprobe festgelegt. Zudem muss man anmerken, je größer die gewählte Stichprobe, desto größer die Annäherung an die wahre Grundgesamtheit und desto weniger Fehler werden in Kauf genommen \footcite[vgl.][65f]{Mayer2013}. 

Nun gilt es festzulegen, welche Personen bzw. Klassen für die Befragung herangezogen werden. Ziel ist es, "`ohne große Fehler Fehler zu machen [\punkte] von der Stichprobe auf die Grundgesamtheit zu schließen."'\footcite[60]{Mayer2013}. Das bedeutet, alle relevanten Merkmale der Grundgesamtheit von Geschlecht, Ausbildungsgang, über unterschiedlich vorliegende Probleme und Unterstützungsbedarfe müssen von der Stichprobe auf die Grundgesamtheit schließen lassen können. Die Stichprobe ist somit ein verkleinertes Abbild der Grundgesamtheit \footcite [vgl.][197]{Kromrey1995}. Daher werden aus allen Ausbildungsgängen je 30\% der Schüler befragt. Die genaue Anzahl der Schüler ist der folgenden "`Konzeption der Schülerbefragung"' zu entnehmen, welche im Vorfeld der Umfrage u. a. auch mit den Gutachtern besprochen wurde und zum Zwecke der vorliegenden Arbeit nur leicht verändert wurde.

Im weiteren Verlauf wurde die Zufallsauswahl (Random-Verfahren) als zufallsgesteuertes Auswahlverfahren festgelegt, da so die Repräsentativität durch das Verfahren selbst gewährleistet wird \footcite[vgl.][60]{Mayer2013}. Wie bereits erwähnt, werden bei der Umfrage stets ganze Klassen befragt. Die zu berücksichtigenden Fragebögen werden anschließend von einer neutralen Person aus der Gesamtheit der Bögen entsprechend des Ausbildungsganges gezogen. Diese vorherige Unterteilung in Teilgruppen (hier in Form von Fachrichtungen) wird auch als Klumpenauswahl (Cluster Sampling) bezeichnet \footcite[vgl.][63]{Mayer2013}.

\newpage
\subsubsection{Konzeption der Schülerbefragung}
\label{sec:KonzeptionDerSchülerbefragung}

\textbf{Allgemeines}
\begin{itemize}
	\item Vorstellung des DRK Bildungswerk SN: siehe Punkt \ref{sec:BerufsbildendeSchulenInSachsenEineExemplarischeVorstellungAmDRKBildungswerkSachsen} (dementsprechend wird hier auf eine erneute Darstellung der Ausbildungsgänge verzichtet)
	\item Forschungsziel ist die Befragung einer Schülermenge, welche die Grundgesamtheit der Schülerschaft am DRK Bildungswerk SN widerspiegelt.
	\item Befragt werden soll mit einer Zufallsauswahl (Random-Verfahren).
	\item Genaues Verfahren: Klumpen-Auswahl (Cluster Sampling), d.h. die Aufteilung der Grundgesamtheit in Teilgruppen (in diesem Fall Fachrichtungen) 
	\item Begründung: Diese Auswahl repräsentiert unserer Meinung nach die verschiedenen Fachrichtungen am besten, die sich hinsichtlich der Zugangsvoraussetzungen, Altersstruktur und der Ausbildungsanforderungen deutlich unterscheiden.
\end{itemize}

\textbf{Beschreibung der Stichprobenauswahl}
\begin{itemize}
	\item Aktuell gibt es am DRK Bildungswerk SN insgesamt 679 Schüler. Dies ist etwas weniger als zunächst angenommen, da in der Rettungsassistenz und in der Altenpflege im März 2015 insgesamt 4 Klassen verabschiedet wurden.
	\item Ausschlusskriterium: Die berufsbegleitenden Klassen werden bei der Befragung nicht berücksichtigt, da sie nur einen Tag pro Woche zum Unterricht in der Schule sind und sich ihre Altersstruktur sowie Zusammensetzung von den Vollzeitklassen deutlich unterscheiden. Hinzu kommt, dass potentielle Unterstützungsangebote durch sie kaum bis gar nicht genutzt werden könnten.
	\item Abzüglich der berufsbegleitenden Schüler verbleiben 578 Schüler.
	\item Befragt werden sollen nach Absprache mit o.g. Dozenten 30\% der Grundgesamtheit. Das ergibt eine zu befragende Stichprobe von 175 Schülern.
	\item 578 Schüler gesamt -- Stichprobe ergibt 175 Schüler mit folgender Aufteilung (gewichtet nach der Verteilung der jeweiligen Fachrichtung, davon jeweils 30\%). Dadurch erhalten wir ein relativ realistisches Bild der Gesamtgruppe. 
		\begin{itemize}
			\item 32 Personen aus der Altenpflege 
			\item 3 Personen aus der Diätassistenz
			\item 68 Personen aus der Erzieherausbildung 
			\item 13 Personen aus der Heilerziehungspflege 
			\item 21 Personen aus der Physiotherapie
			\item 25 Personen aus der Rettungsassistenz und Notfallsanitäter
		\end{itemize}
	\item Trotz der zum Teil geringen Personenzahl werden jeweils ganze Klassen befragt. Damit entfällt eine komplizierte Vorauswahl, nach welchen Kriterien die Antworten der Schüler berücksichtigt werden und es erfolgt keine Selektion von ausgewählten Schülern. Zusätzlich entfallen ganz praktische Probleme wie z.B. Was passiert, wenn von den Vorausgewählten am Befragungstag nur ein Teil anwesend ist? Zudem können jeweils ganze Klassen persönlich eingewiesen und motiviert werden.
	\item Für die Bedarfsanalyse unberücksichtigt bleiben Geschlecht, Alter und Schulabschluss. Diese Daten werden jedoch für mögliche statistische Auswertungen oder weiterführende Studien bzw. zur Fehleranalyse mit aufgenommen.
	\item Die Fragebögen werden jeweils in die ganze Klasse gegeben, anonym ausgefüllt und aus dem Klassensatz werden dann zufällig die auszuwertenden Bögen, in Höhe der jeweiligen errechneten Stichprobengröße gezogen. Zur Wahrung der Durchführungsobjektivität erfolgt dieser Vorgang durch eine neutrale Person, die dem Projekt nicht angehörig ist. 
	\item Die ausgewählten Bögen werden ausgewertet (LimeSurvey).
	\item Zusätzlich zur Schülerbefragung werden Leitfadeninterviews mit 5 ausgewählten Lehrkräften (Klassenlehrern), darunter die Vertrauenslehrerin, durchgeführt. Diese Personen haben aufgrund ihrer Lehrtätigkeit Erfahrungen in den verschiedensten Berufsfeldern der Gesundheit und Pflege bspw. in der Physiotherapie, Krankenpflegehilfe, Heilerziehungspflege, Erzieher, Altenpflege aber auch bei den Notfallsanitätern und Rettungsassistenten.
	\item Die Interviews erweitern die Informationen hinsichtlich wahrgenommener Problemlagen und möglicher Unterstützungsangebote für Schüler.
\end{itemize}

\subsubsection{Inhaltliche Konzeption}
\label{sec:InhaltlicheKonzeption}

Aufgrund persönlicher und beruflicher Erfahrungen im Umgang mit Schülern entschied man sich einen relativ leicht verständlichen, übersichtlichen und knappen Fragebogen zu erstellen. Nur so ist, aus subjektiven Erfahrungen abgeleitet, zu gewährleisten, dass alle Schüler, unabhängig von Alter und Vorbildung, die Fragebögen verstehen und bearbeiten können und nur so ist der Einsatz im laufenden Unterricht, ohne organisatorische Schwierigkeiten und mit Einverständnis der Lehrer, effektiv möglich. Alle Überlegungen zur inhaltlichen Ausgestaltung des Fragebogens erfolgten während der andauernd-kommunikativen Zusammenarbeit der Autorinnen. Die komplette Schülerumfrage ist dem Anhang unter Punkt \ref{sec:Schülerumfrage} zu entnehmen.

Zu Beginn der Umfrage werden die Schüler von Doreen Stichel bzw. Daniela Wobst bezüglich Sinn, Zweck, Aufbau und Zielstellung aufgeklärt und eingewiesen; dennoch beginnt die Umfrage selbst mit einem erläuternden Textabschnitt, welcher die Zielstellung der kommenden Fragen erneut klar benennt. Der grundlegende Aufbau der Umfrage gliedert sich in 3 Abschnitte, wobei Abschnitt 1: 3 Fragen; Abschnitt 2: 8 Fragen und Abschnitt 3: 4 Fragen beinhaltet. Die Schüler beantworten somit 15 Fragen, wobei der Fokus auf dem Abschnitt 2 liegt. Dieser dient der Ermittlung der persönlichen Problemlagen und thematisiert die Annahme außerunterrichtlicher Unterstützungsangebote. 

Der \textbf{erste Abschnitt} umfasst die allgemeinen Angaben zur Person. Die darin erfragten Inhalte wie Geschlecht, Alter und der höchste bisher erreichte Bildungsabschluss dienen nicht der Beantwortung der Forschungsfragen, sondern werden eher aus statistischen Gründen und persönlichem Interesse erfragt. Zudem haben diese Fragen einen aufschließenden und motivierenden Charakter zu Beginn der Umfrage. Die Angaben zur Person werden in der Ergebnisdarstellung erwähnt, erfahren jedoch keine weitere Interpretation, da sie keine Relevanz für das Forschungsinteresse haben.

Der \textbf{zweite Abschnitt} repräsentiert den Kern des Forschungsinteresses in Bezug auf die Schülerperspektive, da hier die Ermittlung der persönlichen Problemlagen und die potentielle Annahme von außerunterrichtlichen Unterstützungsangeboten ermittelt wird. Im Anschluss an die grundlegende Frage, inwieweit Schüler früher oder aktuell von ausbildungsbeeinflussenden Problemlagen betroffen waren, erfolgt die Differenzierung dieser potentiell vielschichtigen Problemlagen. Den Schülern wird neben der Nennung verschiedener Problemlagen auch die Möglichkeit eingeräumt, keine (ausbildungsbeeinflussenden) Probleme zu haben. Wie bereits unter Punkt \ref{sec:ProblemlagenVonSchülernAnBerufsbildendenSchulen} dargelegt, haben Berufsschüler z. T. ähnliche Probleme wie Schüler an allgemeinbildenden Schulen. Dennoch weichen viele Problemlagen aufgrund der heterogenen Struktur in berufsbildenden Klassen, beispielhaft erkennbar an der Altersspanne, unterschiedlicher Vorbildung und Familienstand, z. T. stark von denen "`klassischer"' Schüler ab. Mit Hilfe eingehender Literaturrecherche, die sich in Bezug auf Problemlagen an Berufsschulen recht bedeckt hält und persönlicher Erfahrungen im Lehrberuf hat man sich mit Unterstützung einer Expertenrunde am DRK Bildungswerk SN auf 19 Problemlagen beschränkt, die das Schulerleben zumeist negativ beeinflussen. Aus subjektiver Sicht decken diese Bereiche einen Großteil der aktuellen Problemlagen und Anforderungsbereiche von Berufsschülern ab; ohne Anspruch auf  Vollständigkeit. Um allen Schülern einen leichten Zugang zur Beantwortung der Frage(n) zu ermöglichen, wurde auf eine komplizierte Benennung der Problemlagen durch Fachvokabular verzichtet und eine alltagssprachliche Formulierung gewählt.\\

\newpage
\noindent
Die gewählten Begriffe gilt es nun möglichst genau und eindeutig zu bestimmen:\\

\noindent
\underline{Arbeitsdefinitionen:}
\begin{enumerate}
	\item \textbf{Konflikte in der Klasse:} Eine Unterrichtsklasse vereinigt Menschen mit unterschiedlichen Persönlichkeiten, welche verschiedenen Meinungen und Interessenlagen hervorbringt. Dieser Umstand kann zu Konflikten in Form von Meinungsverschiedenheiten führen, die verbal wie körperlich ausgefochten werden und unterschiedlichste Ausmaße annehmen können. Diese Belastung kann zu einer Beeinträchtigung einzelner oder mehrerer Personen führen.
	\item \textbf{Mobbing:} Mobbing umschreibt das andauernde Schikanieren von Mitschülern, was seelische Verletzungen zur Folge haben kann. Es kann direkt und offen, aber auch über soziale Netzwerke wie Facebook ausgeübt werden.
	\item \textbf{Diskriminierung:} Diskriminierung in der Klasse umschreibt das Schikanieren von Mitschülern, welche ein vermeintlich "`ausgrenzendes"' Merkmal besitzen. Das könnte z. B. die ethnische Herkunft, Religionsangehörigkeit, sexuelle Orientierung oder eine bestimmte Lebensweise sein.
	\item \textbf{Gewalt in der Schule:} Dieser Punkt umschreibt jegliche körperliche Gewalt gegen Mitschüler z.B. in Form von treten, schubsen, schlagen etc. und beeinträchtigt das Schulerleben stets negativ.
	\item \textbf{Probleme aufgrund von un-/entschuldigten Fehlzeiten:} Fehlzeiten kommen aufgrund verschiedenster Konstellationen zustande. Diese beinhalten bspw. schulische Demotivation, Verschlafen, Probleme mit öffentlichen Verkehrsmitteln, familiäre Verpflichtungen (z. B. Kinder in die Kita), berufliche Überschneidungen (wenn Schüler einen Nebenjob ausüben (müssen)), lange Anfahrtswege uvm. Bei gehäuftem Auftreten dieser entschuldigten oder auch unentschuldigten Fehlzeiten kann es zu Problemen mit der Ausbildung kommen, da hier eine Höchstfehlzeit festgeschrieben und somit eine Gefahr der Versetzung oder des Abschlusses möglich ist.
	\item \textbf{Krankheitsbedingte Probleme:} Schüler können, unabhängig von ihrem zumeist jungen Alter, eine Vielzahl an Krankheiten aufweisen. Dieses persönliche Problem führt meist zu weiteren Problemen wie Fehlzeiten, Konflikten in der Klasse oder zu Schwierigkeiten im Praktikum. Diese können somit die Ausbildung und das Schulerleben erschweren.
	\item \textbf{Konflikte im häuslichen Umfeld bzw. familiäre Probleme:} Jeder Schüler entstammt unterschiedlichen familiären Strukturen. Diese können bezgl. menschlicher Wärme, Unterstützung, Alltagsstrukturen, Konflikten und Anforderungen z. T. stark voneinander abweichen. Diese Probleme haben ebenfalls Zugang zum Schulgeschehen. 
	\item \textbf{Überhöhter Alkoholkonsum:} Die meisten Berufsschüler befinden sich am Übergang vom Jugendlichen zum Erwachsen, mitsamt der einhergehenden neuen Herausforderungen und Möglichkeiten. Viele Schüler testen dabei ihre Grenzen aus und treten auch über diese hinweg. Der Konsum von Alkohol oder auch Drogen führt jedoch meist zu weiteren Problemen wie Fehlzeiten, Versetzungsgefährdung (durch schlechte Leistungen) oder auch zu Konflikten in der Klasse. Alkohol und Drogen sind an Schulen grundlegend nicht gestattet; daher drohen bei Nachweis Disziplinarstrafen oder auch die Kündigung.
	\item \textbf{Drogenkonsum:} siehe 8
	\item \textbf{Mediale Süchte (TV, Spiele, Smartphone, soziale Netzwerke etc.):} Der Gebrauch von modernen Kommunikationstechnologien gehört für viele Schüler zum normalen Alltag. Die massive Nutzung dieser Techniken kann jedoch zu Suchtverhalten führen, was wiederum andere Probleme wie Fehlzeiten oder Demotivation hervorrufen kann.
	\item \textbf{Finanzielle Probleme/Schulden:} Ähnlich wie bereits unter Punkt 8 angeführt, befinden sich die meisten Schüler in einer Art Übergangsphase vom Jugendlichen hin zum Erwachsenen in der Grenzen ausgetestet und z. T. überschritten werden; der Umgang mit Geld ist ein Lernprozess, der ebenfalls zu Schwierigkeiten führen kann. Es gibt aber auch weitere beispielhafte Ursachen für Geldprobleme: Jugendliche können armen Verhältnissen entstammen und haben keinen direkten Zugang zu Geld; andere haben aufgrund übersteigerter Ansprüche keine finanziellen Ressourcen und weitere haben anderweitige Verpflichtungen wie z. B: eigene Kinder. Zudem ist anzumerken, dass diese vollzeitschulische Ausbildung Schulgeld kostet und das Gros der Schüler über kein oder wenig "`Lehrgeld"' verfügt.
	\item \textbf{Probleme in der praktischen Ausbildung/Praktikum:} Diese Probleme können vielseitig sein und sich beispielsweise wie folgt äußern: es gibt persönliche Differenzen mit Ausbildern, konträre Auffassungen von Praktikumsinhalten zwischen Schüler und Ausbildungsstätte, unangemessene Anforderungen im Praktikum oder auch physische wie psychische Belastungssituationen. Einige Schüler stellen bei Problemen in der praktischen Ausbildung ihre generelle Ausbildung infrage und gefährden einen Abschluss.  
	\item \textbf{Probleme bei der Wissensaneignung/fehlende Lernstrategien:} Manche Schüler haben nie gelernt, wie man lernt und haben daher Probleme bei der Wissensaneignung und somit mit der Ausbildung.
	\item \textbf{Fehlende Motivation in Bezug auf Schule/Ausbildung:} Schüler können aus den verschiedensten Gründen heraus für die Ausbildung ungenügend motiviert sein: einige wurden zu dieser Ausbildung überredet; manche hatten ggf. eine falsche Vorstellung vom Berufsbild; andere sind durch mangelnde Erfolgserlebnisse demotiviert und weitere haben aktuell mehr Interesse an anderen Lebensbereichen.
	\item \textbf{Versetzungsgefährdung und/oder schlechte Leistungen, die den Ausbildungserfolg gefährden:} Bedingt durch andere Probleme kann es zu schlechten schulischen Leistungen kommen, die einem erfolgreichen Abschluss entgegenstehen. 
	\item \textbf{Überforderung in der Ausbildung (zu hohes fachliches Anforderungsniveau):} Schüler sind mit dem fachlichen Anforderungsniveau bspw. mit Stoffmenge oder Fachvokabular überfordert.
	\item \textbf{Überbelastung (vielfältige Aufgaben und Anforderungen in der Ausbildung sind trotz Bemühen nicht zu bewältigen):} Schüler können die vielfältigen Anforderungen in der Ausbildung trotz Bemühen nicht bewältigen, beispielsweise in Form von Praktika, regelmäßigem Lernen, Klausuren, Hausaufgaben und Präsentationen, welche gegebenenfalls parallel zu weiteren individuellen Anforderungsbereichen im Privatleben erledigt werden müssen.
	\item \textbf{Prüfungsangst:} Prüfungsangst beinhaltet die (Versagens-)Angst vor dem Nichtbestehen von Prüfungsleistungen, die für den erfolgreichen Abschluss der Ausbildung notwendig sind. Diese Angst kann weitere Ängste generieren wie Zukunftsangst, finanzielle Sorgen, familiäre Konsequenzen etc.
	\item \textbf{Zukunftsängste in Bezug auf Arbeitsplatz:} Arbeitslosigkeit und Hartz IV sind aktuelle Themen, denen sich Jugendliche frühzeitig stellen müssen. Der erfolgreiche Abschluss wird daher nicht unbedingt mit einer Anstellung und finanzieller Absicherung assoziiert, sondern wirft eher neue Fragen und Sorgen auf.
\end{enumerate}

\noindent
Nachdem die Schüler ihre potentiellen Probleme per Ankreuzen dargelegt haben, geht es im Weiteren um die Inanspruchnahme möglicher Unterstützungsangebote für die betreffende Person. Es wird zunächst erfragt, ob Gespräche mit bestimmten Personen in der zurückliegenden Ausbildungszeit in Anspruch genommen wurden. Da Personen aus dem Familien- oder Freundeskreis oftmals die ersten Ansprechpartner für betroffene Schüler darstellen, werden diese neben schulischen Personen wie Klassenlehrer, Schulleiter und anderen Lehrkräften ebenso zur Auswahl gestellt. Im folgenden Punkt wird das Augenmerk auf die Inanspruchnahme des Vertrauenslehrers gelegt. Die Schüler weisen mit ihrer Beantwortung darauf hin, ob dieses bestehende Konzept der Unterstützung am DRK Bildungswerk SN angenommen wird und weisen gleichzeitig mögliche Gründe für eine Nichtinanspruchnahme auf. Neben der Ermittlung der Problemlagen hat es sich diese Studie zur Aufgabe gemacht, etwaige Unterstützungsbedarfe zu eruieren und darauf gestützt mögliche weiterführende Beratungs- und Unterstützungsbedarfe am DRK Bildungswerk SN zu konzipieren bzw. anzuregen. Um den Erfolg möglicher Angebote vorab einschätzen zu können, wird im nächsten Punkt ermittelt, ob Schüler überhaupt solche Angebote bei Bedarf in Anspruch nehmen würden. Darauf aufbauend schließt der zweite Abschnitt der Schülerumfrage mit der Frage nach dem Personenkreis, von welchem ein Unterstützungsangebot angenommen werden würde, und wie die Rahmenbedingungen dieser Angebote aussehen sollten, auch wenn man selbst diese vielleicht nicht nutzen würde. Mit dem zweiten Abschnitt können so die subjektiven Problemlage und individuellen Beratungs- und Unterstützungsangebote der Schüler erfasst werden. 

Im Anschluss erfolgte der \textbf{dritte und letzte Abschnitt} der Befragung. Dieser soll die subjektiv wahrgenommenen allgemeinen Probleme und Unterstützungsbedarfe der Mitschüler am DRK Bildungswerk SN erfassen. Die 4 Fragen, die dafür herangezogen werden, sind den Schülern im Wortlaut bereits bekannt, da sie bereits im zweiten Abschnitt für die persönliche Erfassung von Problemen und Bedarfen genutzt wurden. Die Schüler sollen jedoch jetzt den allgemeinen Bedarf für Beratungs- und Unterstützungsangebote für Mitschüler am DRK Bildungswerk SN benennen und anschließend die Problemlagen aufzeigen, für die es Angebote geben sollte. Zum Schluss wird erneut der Personenkreis erfragt, der diese Angebote bereitstellen sollte, aber auch die passenden Rahmenbedingungen eruiert. Ziel ist, das Selbst- und Fremdbild der Schüler zu überprüfen und so ein möglichst realistisches Bild der Probleme und Bedarfe aller Schüler zu gewinnen. Seitens der Autorinnen wird der Verdacht gehegt, dass viele Schüler gegebenenfalls keine eigenen Probleme und Bedarfe benennen, diese aber verstärkt für ihre Mitschüler aufzeigen. Es wäre ebenfalls denkbar, dass eigene Probleme und Bedarfe auf andere Schüler projiziert werden, da man sich diese persönlich nicht eingestehen möchte. Dieser Aspekt ist aber im Rahmen der aktuellen Untersuchung nicht aufzuklären und bleibt hypothetisch. 

Die Erstellung des Fragebogens war aufgrund der bisher geringfügigen Erfahrungswerte in diesem Wissenschaftszweig eine echte Herausforderung für die Autorinnen und unterlag einem länger andauernden Prozess, wobei die Ergebnisse regelmäßig verworfen und neu überdacht wurden.
 
Bei der Formulierung der Fragen, in Bezug auf Verständlichkeit und Komplexität, wurde versucht, die unterschiedlichen Voraussetzungen der zu befragenden Schüler zu berücksichtigen. Zudem wurden die folgenden 5  Aspekte bei der Formulierung der Fragen bedacht, welche dem Werk von Kallus zu entnehmen sind \footcite[vgl.][63-66]{Kallus2010}. Diese sollen hier kurz vorgestellt werden:

\begin{itemize}
	\item \textbf{Lesbarkeit und klares Design:} Die Aussagen sollten gut lesbar und als Einheit erkennbar sein. 
	\item \textbf{Verständlichkeit:} Die Aussagen sollten durch eine klare und eindeutige Wortwahl leicht verständlich formuliert werden. Es ist auf eine neutrale Formulierung zu achten; keine Suggestivfragen!
	\item \textbf{Einfache Beantwortbarkeit:} Die Fragen sollten leicht zu beantworten sein und keine komplexen Gedankenwege von den Schülern verlangen. Komplexe Sachverhalte sind auf ein einfacheres Niveau "`hinunterzubrechen"'.
	\item \textbf{Neutraler Bezug zum Personen-Lebensumfeld/-kontext:} Die Aussagen sollen einen Bezug zum Erleben und Verhalten der antwortenden Personen herstellen. Dem wurde mittels Einleitung sowie einleitenden Sätzen bei entsprechend komplexerer Fragen entsprochen.
	\item \textbf{Eindeutigkeit und Klarheit:} Fragen sollten möglichst so formuliert werden, dass sie auch ohne andere Fragen aus dem Kontext zu beantworten wären.
\end{itemize}

\noindent
Zum Schluss überlegte man sich verschiedene Wege der Umfragedurchführung -- zur Diskussion standen die Online-Umfrage und die klassische Umfrage auf Papier. Aufgrund des begrenzten Zeitfensters und negativer Erfahrungen im Vorfeld in Bezug auf die aktive Teilnahme an Online-Umfragen am DRK Bildungswerk SN wurde diese Idee zugunsten der Umfrage auf Papier entschieden.

\noindent
\textbf{Merkzettel zur Studie}\\

\noindent
Ergänzend zur Umfrage wurde ein kleiner Merkzettel für die Schüler erstellt. Trotz einer klaren Einweisung und Belehrung vor Beginn der Umfrage soll so auch nach Beendigung der Umfrage die Möglichkeit gegeben sein, sich an die für das Projekt verantwortlichen Personen wenden zu können (Hinweis: Dieses Angebot wurde bis zur Abgabe der vorliegenden Arbeit nicht in Anspruch genommen.). Neben grundlegenden Informationen zur Umfrage werden auch die Kontaktdaten von Daniela Wobst und Doreen Stichel auf dem Zettel vermerkt. Dieses Schriftstück in A5-Größe wird an jeden beteiligten Schüler ausgegeben und ist dem Anhang beigefügt (siehe Punkt \ref{sec:Schülerumfrage}). \\

\noindent
\textbf{Pretest}\\

\noindent
Ein entwickeltes Instrument, in diesem Fall die Schülerumfrage, sollte vor dem konkreten Einsatz getestet (pilotiert) werden. Diese Pilotierung dient der Eignungsfeststellung bei der Zielgruppe, deckt etwaige Verständnisprobleme auf und dient der Erprobung des später geplanten Ablaufs. Dieser Testdurchlauf sollte bei einer Probandengruppe aus der Grundgesamtheit erfolgen \footcite[vgl.][275]{Krueger2014}. Aus diesem Grund wurde vor Beginn der Schülerumfrage ein Pretest, im  Sinne des wissenschaftlichen Arbeitens, von den Autorinnen durchgeführt. Die entsprechenden Schüler wurden zufällig ausgewählt und hatten vor Testbeginn keine Vorkenntnis, warum dieser kurzfristige Termin anberaumt wurde. Zu Beginn wurden die 10 Schüler über Sinn und Zweck der Untersuchung aufgeklärt und begannen anschließend mit der Umfrage. Im Anschluss an die Bearbeitungszeit wurden die Fragebögen nicht sogleich eingesammelt, da die Schüler mit Hilfe des Bogens nun Anmerkungen und Feedback an die Autorinnen richten konnten. Dabei wurden inhaltliche und optische Bestandteile des Fragebogens besprochen, Verständnisprobleme erfragt aber auch das subjektive Interesse an den Inhalten diskutiert. Aufgrund des mehrheitlichen positiven Feedbacks wurden im Anschluss keine tiefgreifenden Änderungen an der Umfrage vorgenommen. Es erfolgten lediglich kleinere Anpassungen bei der Formulierung der Fragestellungen und Ergänzungen zu Antwortmöglichkeiten. Die geplanten 20 Minuten Bearbeitungszeit konnten nach dieser Untersuchung auf 10-15 Minuten reduziert werden. Dieser Pretest hatte somit einen sehr konstruktiven und motivierenden Charakter und bestärkte die Autorinnen für die reale Umsetzung dieses Instrumentes.

\subsection{Das Interview}
\label{sec:DasInterview}

Wie bereits eingangs erwähnt, ist ein Teil der Materialien, welche für die vorliegende Masterarbeit herangezogen wurde, im Rahmen einer universitären Gruppenarbeit in der Fachrichtung Gesundheit und Pflege unter Leitung von Fr. Thümmler erarbeitet worden. Die im Folgenden beschriebene Erarbeitung des Interviewleitfadens für die Befragung der Lehrkräfte am DRK Bildungswerk SN fand durch diese fünfköpfige Arbeitsgruppe statt. Diese Zusammenarbeit währte mehrere Wochen und kann daher nur an wesentlichen Schritten verdeutlicht werden.

\noindent
Grundlegendes Ziel der Interviews ist die Beantwortung der folgenden Forschungsfragen:\\

\noindent
\textbf{(2)} Welche persönlichen und sozialen (außerunterrichtlichen) Problemlagen von Schüler nehmen Lehrkräfte des DRK Bildungswerk SN im Bereich Gesundheit, Pflege und Sozialwesen als besondere Belastung für den Unterricht wahr?\\

\noindent
\textbf{(3)} Wie schätzen Lehrkräfte (Anm.: Schüler sind hier ausgenommen) anhand der (möglichen) subjektiv wahrgenommenen Problemlagen den Bedarf an sozialpädagogischen und anderweitigen Unterstützungs- und Beratungsangeboten ein und wie könnten passende Angebote aussehen?\\

\noindent
Um die subjektive Wahrnehmung der Lehrkräfte hinsichtlich der Problemlagen ihrer Schüler und deren Auswirkungen auf den Schulalltag beschreiben zu können, wurde als Studiendesign die qualitative Einzelfallstudie gewählt. Da die vorliegende Arbeit die Perspektive der Schüler und Lehrer am DRK Bildungswerk SN untersucht, bildet der Forschungsschwerpunkt eine Schnittstelle zwischen Berufsbildungs- und Sozialforschung.

Als Erhebungsmethode zur Datengewinnung wurde das leitfadengestützte Interview ausgewählt. Im Gegensatz zum narrativen Interview, welches ein offenes Erzählverfahren darstellt, werden durch das leitfadengestützte Interview konkrete Aussagen über einen Gegenstand zum Ziel der Datenerhebung \footcite[vgl.][37]{Mayer2013}. Kennzeichnend für ein Leitfadeninterview sind offen formulierte Fragen, denen der Befragte frei antworten kann. Durch den konsequenten Einsatz des Leitfadens können Gespräche vorab strukturiert und daraus resultierend eine Vergleichbarkeit der Ergebnisse der Einzelinterviews ermöglicht werden \footcites[vgl.][112]{Flick1999}[vgl.][376f]{Friebertshaeuser1997}. Dennoch offeriert diese Form der Erhebung dem Interviewer einen gewissen individuellen  Freiraum in der Befragung. "`Diese Einzelentscheidungen, die nur in den Interviewsituationen selbst getroffen werden können, verlangen vom Interviewer ein großes Maß an Sensibilität für den konkreten Interviewverlauf und für den Interviewten."' \footcite[113]{Flick1999} 

Die Stichprobe der Interviewpartner wurde in einer Vorab-Festlegung "`absichtsvoll"' begründet. D. h. die Personen wurden nach bestimmten Kriterien aus der Grundgesamtheit der zur Verfügung stehenden Lehrkräfte festgesetzt \footcite[vgl.][39]{Mayer2013}. Wie bereits in der Unterrichtsplanung erwähnt, stellte sich eine größere Anzahl von Lehrkräften per E-Mail freiwillig zur Verfügung an diesen Interview teilzunehmen. Die Autorinnen entschieden sich für 5 Personen, die Unterschiede in Geschlecht, Alter, Lehrerfahrung und Fachrichtungszugehörigkeit aufwiesen. Trotz der geringen Probandenzahl sollte so ein möglichst breites Erfahrungswissen der Lehrkräfte in Bezug auf Problemlagen und Unterstützungsbedarfe der Schüler erschlossen werden. Ungeachtet der Bestrebungen der Autorinnen war eine größere Stichprobenerhebung in der Kürze der Zeit nicht zu realisieren.

Aufgrund der o. g. spezifisch formulierten Forschungsfragen waren Bestandteile des Leitfadens vorab bereits festgelegt. Ziel der Befragung war die umfassende Berücksichtigung des zu behandelnden Realitätsausschnittes in Form der Erfassung von Problemlagen und Unterstützungsbedarfen der Schüler. Daran anknüpfend galt es nun, Themenkomplexe zu konzipieren, denen Nachfrage-Themen zugeordnet sind \footcite[vgl.][45]{Mayer2013}. Es war eine Herausforderung, den Leitfaden unter Beachtung aller interessierenden Themen nicht zu überladen, da sonst eine nicht zu bewältigende Fülle von Daten die Folge gewesen wäre. In einem längeren Denk- und Arbeitsprozess wurden folgende Fragen und etwaige Unterfragen erarbeitet (siehe Punkt \ref{sec:Lehrerinterviews}). Diese sollen nun in Kurzform erläutert werden:

\begin{enumerate}
	\item Eröffnungsfrage: Beschreiben Sie eine Klasse die Ihnen jetzt spontan einfällt.
	\begin{itemize}
		\item Diese Frage dient als freier Einstieg, um die Lehrkraft für die weitere Befragung zu öffnen und zu motivieren. Zudem können bereits hier Hinweise auf Probleme der Schüler genannt werden z. B. aufgrund der Heterogenität in der Klasse oder in Bezug auf Mitarbeit.
	\end{itemize}
	\item Welche Problemlagen nehmen Sie hauptsächlich als ausbildungsbeeinflussend bei Ihren Lernenden wahr?
	\begin{itemize}
		\item Hier werden die konkreten Problemlagen der Schüler in Erfahrung gebracht, welche zur Beantwortung der Forschungsfragen benötigt werden.
	\end{itemize}
	\item Wie und in welchem Umfang beeinflussen die Schülerprobleme Ihren Unterricht?
	\begin{itemize}
		\item Durch diese Frage kann eruiert werden inwieweit die benannten Probleme der Schüler Einfluss auf die Ausbildung haben und damit auch auf den Arbeitsalltag des Lehrers. Die gewonnenen Ergebnisse untermauern die ausbildungsbeeinflussende Gestalt der genannten Problemlagen.
		\item Sollten hier wenige Anmerkungen generiert werden, sind Nachfragen möglich wie z. B. "`Wie war Ihr persönlicher Umgang mit den Erlebnis(sen)?"'; "`Gibt es besonders prägnante Erinnerungen an bestimmte Problemlagen?"'
	\end{itemize}
	\item Mit welchen Erwartungen kommen die SchülerInnen auf Sie zu?
	\begin{itemize}
		\item Aussagen die hier gewonnen werden, sind nicht primär für die Beantwortung der Forschungsfragen notwendig, offerieren jedoch erste Unterstützungsbedarfe, die Schüler an ihre Lehrer herantragen oder von ihnen einfordern.
		\item Eine Nachfrage wäre inwieweit die Lehrkräfte um konkrete Hilfe von den Schülern gebeten werden oder ob sich diese eher "`Dinge von der Seele reden"' wollen.
	\end{itemize}
	\item Inwieweit sehen Sie sich persönlich den Anforderungen der Schüler an Sie gewachsen? 
	\begin{itemize}
		\item Unterfrage: Wie lassen sich diese Anforderungen mit Ihrem Selbstverständnis/Rollenverständnis als Lehrkraft in Einklang bringen?
		\item Aussagen die für diesen Themenkomplex benannt werden, geben Auskunft über einen möglichen externen Unterstützungsbedarf falls sich die Lehrkräfte den Anforderungen der Schüler nicht gewachsen fühlen. Zudem ermöglicht es einen Einblick in das (Vertrauens-) Verhältnis zwischen Lehrer und Schüler. Mögliche Unterstützungsbedarfe für Lehrkräfte können hier ebenso eruiert werden, stehen jedoch nicht im Fokus der vorliegenden Arbeit und sind daher maximal Grundlage für weitere Überlegungen.
		\item Die Frage nach dem persönlichen Umgang der Lehrperson bezüglich der Problemlagen der Schüler wurde nach dem Pretest den vorhandenen Unterfragen hinzugefügt. Sie dient primär dem persönlichen Forschungsinteresse der Autorinnen.
	\end{itemize}
	\item Inwieweit wünschen Sie sich Angebote, die sie hinsichtlich des Umgangs mit solchen Problemlagen der SchülerInnen unterstützen?
	\begin{itemize}
		\item Unterfrage: Welche außerunterrichtlichen Beratungs- und Unterstützungsangebote können Sie sich vorstellen? Wie könnten sich diese auf das Unterrichtsgeschehen auswirken?
		\item Diese Frage ist bedeutend für die Beantwortung der Forschungsfragen und zeigt neben der Nennung möglicher Unterstützungsangebote auch den Kenntnisstand der Lehrkraft in Bezug auf sozialpädagogische Beratungs- und Unterstützungsangebote auf.
		\end{itemize}
	\item Auch wenn es nicht explizit im Fragebogen angegeben ist, wird am Ende des Interviews dem Gesprächspartner nochmal die Gelegenheit gegeben, etwaige Ideen, Vorschläge oder Anmerkungen, die im Laufe des Gesprächs aufgekommen sind, loszuwerden. So endet das Interview für den Befragten nicht so abrupt und das Gespräch kann angenehm ausgleiten.
\end{enumerate}

\noindent
\textbf{Pretest}\\

\noindent
Analog zur Schülerumfrage wurde nach der Erstellung des Leitfadens im Sinne des wissenschaftlichen Arbeitens ein Pretest mit einer weiteren Lehrperson des DRK Bildungswerkes SN durchgeführt. Dieses Testgespräch diente dazu, potentielle Unklarheiten in der Fragestellung, technische Schwierigkeiten aber auch den Fragestil des Interviewers zu prüfen und im Anschluss gegebenenfalls abzuändern. Das Gespräch verlief freundlich und motivierte die Autorinnen für den konkreten Einsatz bei den kommenden Lehrkräften. Im gemeinsamen Gespräch wurden anschließend kleinere Änderungen bei der Formulierung der Fragen vorgenommen und die Frage nach dem persönlichen Umgang der Lehrkräfte mit den Problemlagen der Schüler als interessante Steuerungsfrage in den Leitfaden übernommen. Die vorab ausgearbeiteten Fragen wurden nicht weiter verändert, da der Pretest insgesamt ein zufriedenstellendes Resultat lieferte. 

Anzumerken ist jedoch, dass mit Beginn der Interviews stets 2 Aufnahmegeräte, in Form von Smartphones, bereitlagen, da im Pretest die Aufnahme aus ungeklärten Gründen plötzlich abbrach und so das Gespräch leider ungenutzt verhallte.\\

\noindent
\textbf{Datenschutzerklärung, Einverständniserklärung und Postscript}\\

\noindent
Die interviewbegleitenden Dokumente in Form der Datenschutzerklärung, Einverständniserklärung und des Postscriptes entstammen in ihrer grundlegenden Gestaltung den Seminarmaterialien von Frau Thümmler, welche im Rahmen des Seminars "`Forschungsfelder"' an der TU Dresden bereit gestellt wurden. Diese wurden anschließend nach persönlichen Gusto leicht abgeändert und durchgängig bei allen 5 Interviews verwendet.

\include{kapitel6}
\section{Ergebnisdarstellung}
\label{sec:k7_Ergebnisdarstellung}


\section{Konzepterstellung}
\label{sec:k8_Konzepterstellung}

\section{Kritische Diskussion}
\label{sec:KritischeDiskussion}



\section[Konzeptionelle Überlegungen für das DRK Bildungswerk Sachsen]{Aus dem Forschungsprojekt resultierende konzeptionelle Überlegungen für Beratungs- und Unterstützungsangebote am DRK Bildungswerk Sachsen}
\label{sec:AusDemForschungsprojektResultierendeKonzeptionelleÜberlegungenFürBeratungsUndUnterstützungsangeboteAmDRKBildungswerkSachsen}

Wie bereits der Titel der vorliegenden Arbeit vermuten lässt, kann und soll an dieser Stelle keine Konzeption im Sinne eines planungs- und durchführungsfähigen Angebotes für das DRK Bildungswerk SN erarbeitet werden. Ob überhaupt die hier aufgezeigten Schwerpunkte in irgendeiner Art und Weise zur Anwendung gebracht werden können und die Ergebnisse des beschriebenen Forschungsvorhabens dazu genutzt werden, hängt von vielen institutionellen Rahmenbedingungen ab und muss im Vorfeld einer möglichen Konzeption genau überdacht werden. Die hier zu machenden Ausführungen können dazu Denkanstöße geben und bedenkenswerte Optionen, auch unter Einbeziehung von Erfahrungswerten anderer berufsbildender Schulen, aufzeigen. Selbstverständlich sollen dabei die eruierten Ergebnisse der Schüler- und Lehrerbefragungen berücksichtigt werden. Nicht berücksichtigt werden an dieser Stelle die von den Lehrkräften benannten Ideen und Möglichkeiten fachlicher Unterstützung der Schüler, da der Schwerpunkt der Arbeit von vornherein im Bereich persönlicher und sozialer Problemlagen und außerunterrichtlicher Angebote lag. 

Wie in den theoretischen Ausführungen zur Schulsozialarbeit bereits angeklungen, liegt der Schwerpunkt möglicher Angebote im berufsbildenden Bereich tendenziell in Varianten der Beratung und Unterstützung, methodisch umgesetzt mittels Einzelfallhilfe, die mit verschiedenen Formen von problembezogener Beratung und individueller Begleitung der Schüler einen hohen Stellenwert einnimmt. Zusätzlich wäre auch die Methode der sozialen Gruppenarbeit, insbesondere bei klassen- und gruppenbezogenen Problemstellungen, wie z. B. Mobbing, möglich. Anhand der Fülle der ermittelten Problemlagen bei Schülern im DRK Bildungswerk SN und dem wahrgenommenen Bedarf für Unterstützung und Beratung sollten diese beiden Methoden nicht unberücksichtigt bleiben \footcite[vgl.][10ff]{LSS2004}. Die Einzelfallhilfe hat vor allem die Funktion einer Begleitung bei der Gestaltung des Übergangs in das Arbeitsleben und bei der Lösung individueller Konflikte und Defizite \footcite[vgl.][74]{Stuewe2015}. Konzeptionell ist also am ehesten von einem oder mehreren möglichen Projekten mit einer problembezogenen fürsorgerischen Ausrichtung auszugehen (siehe Punkt \ref{sec:Konzeptionen}). Die Zielgruppe bilden daher vorrangig (sozial) benachteiligte und/oder individuell beeinträchtigte Schüler mit individuellen Problemlagen \footcite[vgl.][25f]{Speck2006}. 

Wer möglicherweise die Einzelfallhilfe oder soziale Gruppenarbeit bzw. außerunterrichtliche Beratungs- und Unterstützungsangebote im DRK Bildungswerk SN implementieren und anbieten könnte, kann aktuell nicht vorausgesagt werden. Da die Einstellung eines ausgebildeten Schulsozialarbeiters derzeit nicht geplant ist, wäre es ratsam, diese Haltung entweder zu überdenken oder die Einzelfallhilfe in Form von außerunterrichtlichen Unterstützungs- und Beratungsangeboten bspw. durch interessierte und speziell fortgebildete Lehrkräfte anzubieten bzw. vorerst modellhaft zu erproben. In diesem Zusammenhang ist durchaus zu berücksichtigen, dass die Annahme solcher Angebote und die Aufnahme von Verbindungen zu Beratungs- oder Vertrauenslehrern, nach den Erfahrungen aus Berichten oder Evaluationen von Schulsozialarbeit, nicht immer besonders hoch sind \footcite[vgl.][17f]{LSS2004}. Die Gründe dafür sind nicht bekannt, allerdings weisen die Ergebnisse der Schülerbefragung, in denen ein Großteil der Schüler die Nutzung von Beratungs- und Unterstützungsangeboten und Angeboten der Vertrauenslehrer im schulischen Kontext für sich selbst ablehnt, auf eine vergleichbare Situation im DRK Bildungswerk SN hin. 

Hinsichtlich der Erreichbarkeit sprach sich ein Großteil der  befragten Schüler für eine ständige Erreichbarkeit im Schulalltag aus, was ohnehin mit institutionell bedingten Schwierigkeiten behaftet ist und bei einer Realisierung von Angeboten mittels fortgebildeter Lehrkräfte nahezu unmöglich sein dürfte. Keinesfalls empfohlen wird für den berufsbildenden Bereich aber auch eine Ausdehnung von Angeboten in den  Nachmittag \footcite[vgl.][17f]{LSS2004}.

\begin{quotation}
\noindent
"`Einige Schulsozialarbeiter beklagten einen Interessenkonflikt, da die Lehrkräfte die Schüler während ihres Unterrichts ungern zur Beratung gehen lassen und die Schüler hingegen ihre Freizeit nicht unbedingt für das Beratungsangebot "`opfern"' wollen."' \footcite[93]{Ganser2004}
\end{quotation}

Solche Ergebnisse sollten in die konzeptionellen Überlegungen in jedem Falle einfließen. Dennoch sprechen die Befunde zur sozialpädagogischen Arbeit immer wieder auch dafür, niederschwellige Angebote mit verschiedensten Möglichkeiten der Kontaktaufnahme durch Schüler zu realisieren, keine festen Sprechzeiten anzubieten, sondern eher eine dahingehend offene Kultur zu pflegen und breite Präsenz innerhalb der Schule zu zeigen \footcite[vgl.][48]{Essers2012}. Diese Erkenntnisse sprechen deutlich gegen eine Übernahme der Beratungs- und Unterstützungsfunktion durch Lehrkräfte.
  
Die Themen Vertrauen, Vertraulichkeit und Schweigepflicht scheinen für den berufsbildenden Bereich eine besondere Bedeutung zu besitzen, was die Ergebnisse der Schüler- und Lehrerbefragung verdeutlichen und andere Autoren ebenso bestätigen \footcite[vgl.][49]{Essers2012}. Hier könnten verschiedene vertrauensbildende Maßnahmen, wie sie bspw. von den befragten Lehrkräften in Form von erlebnispädagogischen Angeboten für ganze Klassen beschrieben wurden, durchaus hilfreich sein, zumindest wenn Vertrauensprobleme und Konflikte innerhalb des Klassengefüges auftauchen. Jedoch ist auch das derzeitige Angebot einer Vertrauenslehrerin, die einem überwiegenden Teil der Schülerschaft nicht bekannt zu sein scheint bzw. bei persönlichen Problemlagen nicht angesprochen werden würde, zu überdenken. Den Bekanntheitsgrad, z. B. mittels Besuchen und Vorstellungen in den einzelnen Klassen zu erhöhen, erscheint unproblematisch. Beachtenswert ist jedoch auch der Ansatz, die Vertrauenslehrertätigkeit auf mehrere Schultern zu verteilen und aus den einzelnen Fachbereichen Vertrauens- oder Beratungslehrer zu generieren. Diese sollten selbstverständlich die Aufgabe freiwillig übernehmen und dafür zeitliche sowie räumliche Ressourcen, z. B. in Form von Abminderungsstunden zur Verfügung gestellt bekommen. Denkbar wäre es auch, mehrere Vertrauens- und Beratungslehrer in den Fachbereichen für die Schüler zur Wahl zu stellen, um eine frühzeitige Beteiligung der Lernenden zu gewährleisten und das Problem des Bekanntheitsgrades von vornherein zu umgehen. 

Das Thema Netzwerkarbeit scheint für mögliche Beratungs- und Unterstützungsangebote ebenfalls ein wichtiges zu sein, welches auf mehreren Ebenen betrachtet werden kann. Einige Lehrkräfte betonten in den Interviews die teilweise notwendige Weitervermittlung von Schülern an andere Institutionen bzw. Kooperationspartner. Diese können sehr zahlreich sein und von Ämtern, über Beratungsstellen, Psychologen und Ärzte bis hin zu Partnern der praktischen Ausbildung reichen \footcites[vgl.][49ff]{Essers2012}[vgl.][21]{NiedersaechsischesKultusministerium2004}.

\begin{quotation}
\noindent
"`Der Aufbau eines solchen Netzwerkes mit außerschulischen Partnern erfordert Kontaktfreudigkeit, Flexibilität und Organisationsfähigkeit."' \footcite[21]{NiedersaechsischesKultusministerium2004}
\end{quotation}

In erster Linie bedarf diese Netzwerkarbeit aber auch zeitlicher Ressourcen, wenn sie gewinnbringend für beratungs- und unterstützungsbedürftige Schüler sein soll, weshalb hier wiederum die Leistbarkeit solcher Angebote durch Lehrkräfte im DRK Bildungswerk SN neben der unterrichtlichen Tätigkeit in Frage gestellt werden muss. Bezogen auf die bessere und intensivere Zusammenarbeit mit Praxiseinrichtungen, die auf Basis der von Schülern angegebenen zahlreichen Problemlagen in Bezug auf die praktische Ausbildung sowie auf der von Lehrern dargelegten Bedarfe erfolgen sollte, sind jedoch ausgewählte und erfahrene Lehrkräfte als  Netzwerkpartner zu befürworten.

Hinsichtlich möglicher sozialer Gruppenarbeit als schulsozialarbeitsbezogener Methode haben die durchgeführten Befragungen (speziell durch die Themen Mobbing und Konflikte innerhalb der Klassen) durchaus entsprechende Bedarfe offengelegt. Dazu wäre es z. B. vorstellbar eine Fachkraft bzw. eine speziell fortgebildete Lehrkraft in Form eines "`Mobbingbeauftragten"' im DRK Bildungswerk SN zu implementieren oder auch externe Angebote in Anspruch zu nehmen. Alle literaturbezogenen Befunde weisen darauf hin, dass das Thema Mobbing, direkt oder über soziale Netzwerke, immer größere Bedeutung erlangt und Lehrkräfte im Umgang damit häufig an ihre Grenzen stoßen. Mit bestimmten Methoden der sozialen Gruppenarbeit könnten von einer spezialisierten Fachkraft im unterrichtlichen Rahmen thematische Einheiten gestaltet und Konflikte bearbeitet werden. Vorstellbare Effekte davon könnten positive Auswirkungen auf den Unterricht und den Lernerfolg der Schüler sowie besseres Klassenklima und die Förderung der Sozialkompetenz, als wichtige berufsbezogene Kompetenz, sein \footcites[vgl.][51]{Essers2012}[vgl.][20]{NiedersaechsischesKultusministerium2004}. Weiterhin wäre die interdisziplinäre Projektarbeit eine Methode der sozialen Gruppenarbeit, die durch Fachkräfte oder weitergebildete Lehrkräfte innerhalb des DRK Bildungswerk SN positive Entwicklungsmöglichkeiten für die Schüler bereithalten könnte. 

Nach den bisher genannten Möglichkeiten für außerunterrichtliche Beratungs- und Unterstützungsangebote sollen an dieser Stelle noch einige Ideen und Ansätze benannt werden, die den Befragungen entstammen, jedoch weniger Theoriebezüge aufweisen.

Besonders naheliegend wäre es, im DRK Bildungswerk SN innerverbandliche Ressourcen zu nutzen, die bspw. im Fachbereich Jugendhilfe des DRK Landesverbandes zu finden sind. Die fachliche Expertise des Referates soziale Arbeit könnte eine wertvolle Hilfe bei Entscheidungs- und Umsetzungsprozessen hinsichtlich möglicher Angebote leisten. Weiterhin wäre auch eine Intensivierung der Zusammenarbeit im Sinne einer verstärkten Netzwerkarbeit mit DRK-Einrichtungen zu befürworten. 

In den Interviews mit Lehrkräften wurden immer wieder Fortbildungsbedarfe thematisiert, die sinnvolle Unterstützung für die unterrichtenden Berufspädagogen sein könnten. Insbesondere die Themen Supervision und kollegiale Fallberatung kamen mehrfach zur Sprache. Da davon auszugehen ist, dass rein schulorganisatorisch bedingt, Klassenlehrer meist die ersten Ansprechpartner von Schülern bei Problemen sind und es auch bleiben werden, sind solche Fortbildungen ausdrücklich zu befürworten. In den Interviews zeigt sich, dass Lehrkräfte allein schon durch die unterrichtlichen Verpflichtungen in sehr engem Kontakt mit Schülern stehen und mit deren Problemlagen konfrontiert werden. Das würde sich auch nicht ändern, wenn die Zuständigkeiten und personellen Ressourcen innerhalb des DRK Bildungswerkes SN eine "`Überweisung"' oder "`Weiterleitung"' von "`Problemfällen"' zur Bearbeitung ermöglichen würden. Daher sollten entsprechende Fortbildungsmöglichkeiten und die regelmäßige Implementierung des kollegialen Austausches nicht unberücksichtigt bleiben.

Im Abschnitt \ref{sec:Perspektiven} wurde das geplante "`Development-Center"' im Sinne einer strukturierten und kompetenzbezogenen Schülerauswahl bereits kurz vorgestellt. Dadurch könnten sich positive Aspekte hinsichtlich der Prävention von Orientierungslosigkeit und Demotivation der Schüler ergeben. Einige Lehrkräfte gaben in den Interviews an, dass Schüler vermehrt Berufe im Bereich von Gesundheit, Pflege und Sozialwesen ergreifen, ohne dass sie selbst das eigentlich wollen oder wissen, welche Kompetenzen, Fähigkeiten und Fertigkeiten sowie beruflichen Tätigkeiten eigentlich von ihnen verlangt werden. Insbesondere in diesem speziellen Feld, wo Beruf immer auch ein Stück "`Berufung"' sein muss, um den Anforderungen gerecht zu werden, sind solche Entwicklungen fatal und das "`Absitzen"' von Unterrichtszeiten sowie das Erlernen eines Berufes, weil die Eltern diesen als "`arbeitsmarktsicher"' (vgl. Lehrerinterviews) erachten, nicht förderlich. Daher könnten eine theoriegeleitete  und verbesserte Schülerauswahl und die Orientierung auf eventuell passendere Berufe durchaus positive Effekte nach sich ziehen, sofern die Bewerbersituation dies haushaltsbedingt überhaupt zulässt. 

Nunmehr sind mögliche, für die Verfasserinnen relevante, Ansätze für außerunterrichtliche Beratungs- und Unterstützungsangebote am DRK Bildungswerk SN dargelegt, welche z. T. durch theoretische Grundlagen der Schulsozialarbeit begründet werden konnten. Eine Grundsatzentscheidung sowie die Auswahl möglicher praktikabler Ansätze und die Konzeption von Angeboten obliegen nun der Geschäftsführung, wobei die Entscheidung für oder gegen eine sozialpädagogische Fachkraft eine der relevantesten sein wird. Die Verfasserinnen plädieren nach der umfangreichen Beschäftigung mit der Thematik ausdrücklich dafür. 


\section{Fazit}
\label{sec:Fazit}

\begin{quotation}
\noindent
"`Sie nennen sie Schulen, fabulieren vom "`Haus des Lernens"' und tünchen damit die ganztägigen Verwahrungsanstalten für Kinder und Jugendliche, denen in unserer Gesellschaft ein Zuhause abhandenkam."'
\end{quotation}

\noindent
[\punkte] schreibt der Autor Raymond Walden in einer seiner Sammlungen von Sinnsprüchen \footcite[71]{Walden2005} und verdeutlicht damit eines der sich scheinbar verstärkenden Probleme unserer Gesellschaft. Sich verändernde und unsichere Lebensbedingungen, von Brüchen und Neuorientierungen gekennzeichnete Bildungs- und Arbeitsbiografien sowie anders und weniger direkt verlaufende Übergänge als noch vor einigen Jahrzehnten prägen heute den Charakter und die Aufgaben der Schule und der Jugendhilfe mit ihren Unterstützungsangeboten, bspw. in Form der Schulsozialarbeit \footcite[vgl.][9ff]{Bolder2010}. Aus der berufspädagogischen Perspektive heraus ist es schwer zu akzeptieren, unsere Schulen als "`Verwahrungsanstalten"' zu bezeichnen. Jedoch könnte eine solche Einschätzung durchaus getroffen werden, wenn Jugendliche mit ihren aktuellen Problemen lediglich "`beschult"' oder ausgebildet, nicht aber lebens- und alltagsorientiert unterstützt werden. Für diese erforderliche Hilfe gibt es offensichtlich jedoch leider keine Patentrezepte.
 
Zunächst einmal kann festgehalten werden, dass in der Institution berufsbildende Schule immer wieder die viel beschriebenen komplexen Problemlagen, welche die Schüler heute "`schwieriger"' machen und den Unterrichtsprozess nachhaltig beeinflussen, eine große Rolle spielen \footcite[vgl.][1]{UniversitaetLeipzig2007}. Diese theoretische Aussage hat sich durch die eigenen Forschungsarbeiten für die Fachbereiche Gesundheit, Pflege und Soziales am DRK Bildungswerk SN durchaus bestätigt. Die Lehrkräfte beklagten hier, ebenso wie anderswo auch, die seit einigen Jahren deutlich schwierigeren, leistungsschwächeren, demotivierten, problembehafteten und teilweise für die Berufe ungeeigneten Schüler. Die Substanz dieser Aussagen sollte im Rahmen eigener Studien untersucht werden. Erschwerend kam hinzu, dass Erhebungen zu den Problemlagen von Schülern für die Fachbereiche Gesundheit, Pflege und Sozialwesen bisher absolut fehlen und daher nur auf vorherige Forschungsergebnisse allgemeiner Art zurückgegriffen werden konnte. Zusammenfassend lässt sich titelgemäß feststellen: Ja, Schüler haben es heute schwer. Lehrer aber auch! Die Resultate der quantitativen Schülerbefragung sowie der qualitativen Lehrerinterviews sind in den entsprechenden Unterpunkten ausführlich vorgestellt und auch miteinander verglichen worden. Die Forschungsergebnisse zeigen mannigfaltige Problemlagen, die Schüler am DRK Bildungswerk SN für sich persönlich, wie auch für ihre Mitschüler wahrnehmen. Es scheinen also keinesfalls "`problemfreie"' Persönlichkeiten zu sein, die sich in das Berufsfeld mit hohen Anforderungen an Fach- und Sozialkompetenz hineinwagen. Die Lehrerinterviews bestätigten sowohl die Sicht auf die Problemlagen als auch die damit direkt im Zusammenhang stehenden, tendenziell negativen, Einflüsse auf das Unterrichtsgeschehen. Es ergeben sich daraus, sowohl aus der Schüler- als auch aus der Lehrerperspektive betrachtet, definitiv Unterstützungsbedarfe, die jedoch differenziert betrachtet werden müssen. Zum einen, weil unklar ist, wer überhaupt potentielle Unterstützungsangebote leisten könnte, zum anderen, weil die Art und Form der Angebote vorerst nur einen Ideencharakter aufzuweisen hat und konzeptionell ausgearbeitet werden muss. Denkbar wären dabei durchaus individuell ausgestaltete Formen von Berufsschulsozialarbeit, deren wichtigste theoretische Grundlagen im Rahmen der vorliegenden Arbeit aufgezeigt werden konnten. Zu berücksichtigen gilt es jedoch auch, dass 98 von 154 Schülern in der Befragung angaben, persönlich keine außerunterrichtlichen Beratungs- und Unterstützungsangebote im DRK Bildungswerk SN in Anspruch nehmen zu wollen. Die Annahme möglicher Projekte ist demzufolge als fraglich zu bezeichnen, eine Erprobungsphase könnte sich als zweckmäßig erweisen.
 
Überraschend erschienen, neben den sehr zahlreich wahrgenommenen und benannten eigenen und die Mitschüler betreffenden Problemlagen, die Rollenverständnisse der Lehrkräfte in Bezug auf Schülerprobleme. Die Ansicht Drillings, dass viele Lehrkräfte sich mit den teilweise komplexen Problemlagen überfordert fühlen \footcite[vgl.][10]{Drilling2004}, konnte exemplarisch für das DRK Bildungswerk SN so nicht nachgewiesen werden. Offensichtlich erkennen, zumindest die interviewten Lehrkräfte, die Unterstützung der Schüler als Teil ihrer beruflichen Aufgabe, wobei individuelle Grenzen durchaus gegeben sind. Möglicherweise könnte dies auch ein Ausdruck dessen sein, dass Lehrkräfte, ebenso wie Schüler, Berufsschulsozialarbeit oder andere Unterstützungsangebote selbst bisher nicht erlebt haben und auf keine entsprechenden Erfahrungswerte zurückgreifen können. 

Auf die konkrete Beantwortung der Forschungsfragen wurde bereits im Abschnitt \ref{sec:ZusammenfassungDerKategorien} und \ref{sec:GegenüberstellungDerErgebnisse} ausführlich eingegangen. Die Zielstellungen der eigenen Forschungsarbeit, welche in der Einleitung formuliert wurden, sind nach Ansicht der Verfasserinnen weitestgehend erreicht worden. Es gelang herauszuarbeiten, dass die der Fachliteratur entnommenen allgemeinen Problemlagen Jugendlicher auf den Bereich Gesundheit, Pflege und Soziales ebenfalls zutreffen. Hinsichtlich der einzelnen Problemlagen konnte dargestellt werden, in welcher Häufigkeit sie vorliegen, woraus sich Kernprobleme und weniger präsente Aspekte ableiten ließen. Ebenfalls gelang eine Einschätzung hinsichtlich der Wahrnehmung der Lehrkräfte, da sich die geschilderten Problemlagen mit denen der Schüler decken, jedoch personenbezogen unterschiedliche Gewichtungen und vordergründige Kernprobleme deutlich werden. Der Einfluss der Schülerprobleme auf das Unterrichtsgeschehen konnte nachvollziehbar durch die Lehrkräfte beschrieben und somit hinsichtlich der Ausbildungsbeeinflussung als hoch eingeschätzt werden. Abschließend zeigten sich durchaus Bedarfe für Unterstützungsangebote, wenngleich die Ergiebigkeit in Bezug auf verstellbare konkrete Angebote nicht so deutlich ausfiel, wie im Vorfeld angenommen. Viele Vorschläge der Lehrkräfte fokussierten immer wieder den fachlichen Bereich, nicht jedoch den persönlich-sozialen Lebensbereich der Schüler. Ein möglicher Erklärungsversuch durch mangelnde Erfahrungswerte der Lehrkräfte wurde oben bereits ausgeführt. 

Die positive Bilanz hinsichtlich der Forschungsergebnisse für das DRK Bildungswerk SN lässt sich jedoch nicht unbedingt auf weitere Betrachtungen innerhalb der vorliegenden Arbeit übertragen. Weitestgehend ungeklärt bleiben nach wie vor die Gründe der Diskrepanz von, aus den komplexen Problemlagen resultierenden, Unterstützungsbedarfen und tatsächlichen Angeboten im berufsbildenden Bereich allgemein. Dies konnte durch die Bestandsaufnahme der Schulsozialarbeit an öffentlichen berufsbildenden Schulen in Sachsen deutlich aufgezeigt werden. Möglicherweise sind die Begründungen in der geringen gesetzlichen Verankerung von Unterstützungsangeboten für Schüler im sächsischen Schulgesetz und in weiteren fehlenden Vorschriften zu suchen. Hier ist lediglich die sozialpädagogische Betreuung im Berufsvorbereitungsjahr vorgeschrieben, wozu aktuell ca. 15 Personalstellen zur Verfügung stehen. Diese derzeit praktizierte Einengung der sächsischen Schulsozialarbeit auf das Berufsvorbereitungsjahr zeigt sich als wenig praxisorientiert und erzeugt Ungerechtigkeiten gegenüber Jugendlichen mit Unterstützungsbedarf in anderen Bildungsgängen, welche durchaus ähnliche Problemlagen aufweisen. Vermutet werden kann jedoch auch, dass größtenteils Aspekte der Finanzierung dazu führen, dass einem großen Bedarf an Unterstützung nur wenige Angebote der Schulsozialarbeit gegenüberstehen. Die "`Freie Presse"' schrieb im Juni 2015, passend zu dieser Thematik, dass die Träger der sächsischen Schulsozialarbeit die derzeitige Förderpraxis als Desaster kritisieren. Entgegen aller politischen Willensbekundungen soll ab dem Schuljahr 2015/16 die Förderung im Programm "`Soziale Schule -- Kompetenzentwicklung für Schüler"', aus Mitteln des Europäischen Sozialfonds (ESF) und der Sächsische Aufbaubank (SAB), auf zehn Monate beschränkt werden und insgesamt weniger Geld bereitstehen. Die Folge davon sei, dass statt der bisher 19 in Projekten vertretenen Trägern nur noch fünf Projektanträge einreichen wollen \footcite[vgl.]{FreiePresse2015}. Selbstverständlich ist die Finanzierung möglicher Beratungs- und Unterstützungsangebote auch für Schulen in freier Trägerschaft eine immense Herausforderung, der sich auch das DRK Bildungswerk SN stellen muss, insofern tatsächlich praktische Angebote aus der vorliegenden Arbeit abgeleitet werden sollen. Ein weiterer Aspekt für die negative Bilanz von Bedarfen und Angeboten könnte zusätzlich darin begründet liegen, dass nicht alle Schulen, welche Unterstützungsleistungen bereithalten, diese unter dem Begriff Schulsozialarbeit anbieten. Diese Vermutung ergibt sich aus der im Gliederungspunkt \ref{sec:ErhebungenZurSchulsozialarbeitAnBerufsbildendenSchulen} bereits thematisierten Angabe von anderweitigen internen und externen Angeboten an insgesamt sechzehn berufsbildenden Schulen in Sachsen. Die Nichtberücksichtigung solcher anderweitigen und unbeschriebenen Projekte in Bestandserhebungen verzerrt hier möglicherweise die Analyse, wäre jedoch aufgrund der Umstrittenheit des Begriffes Schulsozialarbeit und seiner mangelnden inhaltlichen Klarheit durchaus verständlich \footcite[vgl.][23]{Speck2007}.

Aus der vorliegenden Arbeit ergeben sich, nicht zuletzt aufgrund der soeben geschilderten Aspekte, einige weitere Forschungsdesiderata. Dringend notwendig erscheint es, die bestehende Datenlage zur Schulsozialarbeit oder entsprechenden Projekten unbedingt auszudehnen und zu aktualisieren sowie dabei den berufsschulischen Bereich stärker als bisher in den Fokus zu nehmen. Zur Überprüfung und Erweiterung der hier vorgestellten Ergebnisse sind vergleichende Untersuchungen an mehreren Schulen notwendig, bei denen sowohl öffentliche Schulen als auch Schulen in freier Trägerschaft aus dem Berufsfeld Gesundheit, Pflege und Sozialwesen in den Blick genommen werden müssten. Zusätzlich könnte es gewinnbringend sein, Schulen mit bereits bestehender Berufsschulsozialarbeit oder anderweitigen Unterstützungsangeboten aus dem genannten Berufsfeld (soweit vorhanden) gezielt aufzusuchen und die praktische Umsetzung der Maßnahmen sowie die Annahme der Schüler unter wissenschaftlichen Gesichtspunkten zu prüfen. Die daraus möglicherweise abzuleitenden Forschungsergebnisse könnten sich für zahlreiche Schulen als Handlungsgrundlagen eignen und zur Aufhebung der Unterrepräsentiertheit der Fachrichtungen Gesundheit, Pflege und Sozialwesen in Bezug auf Evaluationsberichte oder Best Practise-Beispiele beitragen. Das gesamte Thema Schulsozialarbeit für den berufsbildenden Bereich stellt sich, wie an mehreren Stellen der Arbeit ausgeführt, ebenfalls als relevantes wissenschaftliches Forschungsfeld dar. Indem es gelänge, zielgruppenbezogene theoretische Fundierungen unter Berücksichtigung des berufsbildenden Bereiches und der Fachrichtungen mit ihren Spezifika herauszuarbeiten, wären möglicherweise auch die notwendigen Konzeptionen für konkrete Angebote erleichtert. Für Schulen, die mehr und mehr in den Zugzwang geraten, sich aufgrund der Problemlagen ihrer Schüler mit entsprechenden Unterstützungsleistungen auseinanderzusetzen, würde dies eine enorme Arbeitsunterstützung bieten. Letztlich ist noch zu erwähnen, dass berufspädagogisch tragfähige Konzepte zum Umgang mit der zunehmenden Heterogenität im genannten Berufsfeld anscheinend bisher fehlen und sich allein daraus enorme Forschungsbedarfe ableiten lassen, die zu einer Verbesserung der Situation in den berufsbildenden Schulen und vor allem zu einer Unterstützung der Lehrkräfte und Schüler in diesem Bereich entscheidend beitragen könnten \footcite[vgl.][23ff]{Grassi2012}.
 
Abschließend soll jedoch auch kritisch darauf hingewiesen werden, dass mögliche Formen der berufsschulischen Sozialarbeit oder anderweitiger Beratungs- und Unterstützungsangebote keinesfalls überzogenen Erwartungen ausgesetzt werden sollen bzw. als Allheilmittel zur Rettung junger Menschen und Problemlösung der Schule angesehen werden dürfen \footcite[vgl.][14]{Engelberg2011}.
\begin{quotation}
\noindent
"`Differenziert betrachtet ist Schulsozialarbeit heute ein wichtiges Instrument zur Förderung von Bildung und Erziehung -- insbesondere bezogen auf die Zielgruppe der benachteiligten Jugendlichen. [\punkte] Allerdings kann Schulsozialarbeit dies nicht allein. Benachteiligte Jugendliche müssen auch außerhalb der Schule gefördert werden."'
\end{quotation}
\noindent
[\punkte] stellen Engelberg und Schattmann dazu treffend fest \footcite[14]{Engelberg2011}. Untermauert wird dies durch die Aussage von Spies, die zu bedenken gibt, dass Unterstützungsangebote empirisch belegt, optimierende Potentiale aufweisen, und unzweifelhafte Berechtigung haben, insofern realistische Ziele formuliert werden \footcite[vgl.][16]{Spies2013}. Ferner sollte auch der präventive Ansatz nicht in Vergessenheit geraten und Beratungs- und Unterstützungsangebote nicht nur dazu vorgehalten werden, bestehende Problemlagen zu beheben \footcite[vgl.][14]{Engelberg2011}. 

Das DRK Bildungswerk SN stellt sich einer großen Herausforderung, sollten die Ergebnisse der vorliegenden Arbeit wirklich zum Anlass genommen werden, praktische Angebote für Schüler zu konzipieren. Viele Fragen sind dazu aufgeworfen worden und im Vorfeld noch zu bedenken. Gleichzeitig wurden jedoch auch konzeptionelle Vorschläge aus der Theorie und den Forschungsergebnissen abgeleitet, die eine Grundlage für Beratungs- und Unterstützungsangeboten darstellen könnten. 

\begin{quotation}
\noindent
"`Schulsozialarbeit reagiert, anders als Schule, flexibler und schneller auf die sich den Heranwachsenden darbietenden Lebensbedingungen, indem sie die lebensweltlichen alltäglichen privaten, sowie innerschulischen als auch außerschulischen Probleme der Kinder und Jugendlichen zum Gegenstand ihrer Arbeit macht."'
\end{quotation}

\noindent
[\punkte] formuliert Speck zutreffend zur Thematik \footcite[75]{Speck2007}. Diese Aussage könnte durchaus auf andere abgeleitete Angebote ausgeweitet werden und eine gute Handlungsmaxime für die zukünftige konzeptionelle Arbeit darstellen. Die Verfasserinnen sind gespannt, ob und inwieweit die hier vorgestellten Ergebnisse ihrer Forschungsarbeit wirklich genutzt und in weiterführende Überlegungen oder sogar konkrete Resultate überführt werden. 


%%% Ab hier kommen die Anhänge
\appendix

%%% Quellenverzeichnis
\nocite{*}
\printbibliography

%%% Anhang
\listoffigures

\newpage

\section{Anhang}
\label{sec:Anhang}

\subsection{Schülerumfrage}
\label{sec:Schülerumfrage}

\begin{itemize}
	\item Umfrage zum Bedarf ausserunterrichtlicher Unterstützungsangebote am DRK BWK SN 
	\item Merkzettel zur Studie 
\end{itemize}

\includepdf[pages=-]{attachments/LimeSurvey-Umfrage-zum-Bedarf-ausserunterrichtlicher-Unterstuetzungsangebote-am-DRK-Bildungswerk-Sachsen.pdf}
\includepdf[pages=-]{attachments/Merkzettel-zur-Studie.pdf}

\subsection{Lehrerinterviews}
\label{sec:Lehrerinterviews}

\begin{itemize}
	\item Vorlage Datenschutzerklärung
	\item Vorlage Einverständniserklärung 
	\item Interview-Leitfaden 
	\item Vorlage Postscript
\end{itemize}

\includepdf[pages=-]{attachments/Datenschutzerklaerung.pdf}
\includepdf[pages=-]{attachments/Einverstaendniserklaerung.pdf}
\includepdf[landscape=true,pages=-]{attachments/Struktur-Leitfaden.pdf}
\includepdf[pages=-]{attachments/Postscript.pdf}

\subsection{Ergebnisauswertung}
\label{sec:Ergebnisauswertung}

	\begin{itemize}
	\item Statistik der Schülerumfrage (LimeSurvey)
	\item Vorlage Paraphrasen und Generalisierungen
	\item Vorlage Reduktion 1 und Reduktion 2
	\item Paraphrasen und Generalisierungen I.01-I.05
	\item Reduktion 1 und Reduktion 2 I.01-I.05
	\end{itemize}
	
\includepdf[pages=-]{attachments/Statistik-der-Schuelerumfrage.pdf}
\includepdf[landscape=true,pages=-]{attachments/Vorlage-Paraphrasen-und-Generalisierungen.pdf}
\includepdf[pages=-]{attachments/Vorlage-Reduktion1-und-Reduktion2.pdf}

\subsection{Sonstiges}
\label{sec:Sonstiges}

\begin{itemize}
	\item Einverständniserklärung der Kommilitonen
	\item Transkriptionsprotokolle I.01-I.05 (CD!)
	\item Audiodateien I.01-I.05 (CD!)
\end{itemize}









%%% Selbstständigkeitserklärung
\section{Selbstständigkeitserklärung}
\label{sec:Selbstständigkeitserklärung}

Ich versichere hiermit, dass ich meinen Anteil an der Arbeit selbstständig verfasst und keine anderen als die angegebenen Quellen und Hilfsmittel benutzt habe.\\[2cm]

$\overline{Stichel, Doreen\hphantom{spaces}\hphantom{spaces}}$ \hfill $\overline{Unterschrift\hphantom{spaces}\hphantom{spaces}}$\\[1cm]

$\overline{Wobst, Daniela\hphantom{spaces}\hphantom{spaces}}$ \hfill $\overline{Unterschrift\hphantom{spaces}\hphantom{spaces}}$\\[2cm]

Dresden, \today

\end{document}
