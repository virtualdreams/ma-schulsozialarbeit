\section{Allgemeine theoretische Grundlagen zur Schulsozialarbeit}
\label{sec:AllgemeineTheoretischeGrundlagenZurSchulsozialarbeit}

\subsection{Annäherung an eine Definition}
\label{sec:AnnäherungAnEineDefinition}

Bei der Sichtung der aktuellen Literatur zur Thematik Schulsozialarbeit wird vergleichsweise schnell deutlich, dass eine klare Definition sich durchaus als schwierig bzw. unmöglich darstellt. Nahezu alle relevanten und im weiteren Fortgang der Arbeit zitierten Autoren stellen dies in ihren Ausführungen fest. Für ein einheitliches und unumstrittenes Begriffsverständnis scheint das Arbeitsfeld, welches zwar recht intensiv beforscht worden ist, zu heterogen und inhaltlich immer noch zu unklar \footcite[vgl.][7]{Speck2010}. Bereits bei der reinen Bezeichnung zeigen sich große Unterschiede, da einige Autoren für den Begriff der "`Sozialen Arbeit an Schulen"', andere jedoch für "`Schulsozialarbeit"' plädieren und gleichzeitig auch noch zahlreiche weitere Wortschöpfungen wie zum Beispiel "`schulbegleitende Sozialarbeit"', "`Jugendsozialarbeit an Schulen"', "`schulalltagsorientierte Sozialpädagogik"' oder "`Schoolwork"' verwendet und mit inhaltlichen Differenzierungen versehen werden \footcites[vgl.][14]{Spies2011}[vgl.][17f]{Stuewe2015}. Diese teilweise verwirrende Vielfalt lässt sich vermutlich mit unterschiedlichen theoretischen Ansätzen und praktischen Anforderungen erklären. Zur Begründung der begrifflichen Fülle wird, insbesondere von Speck \footcite[vgl.][23]{Speck2007}, angeführt, dass grundsätzlich die föderale Struktur der Bundesrepublik und die Förderpolitik der einzelnen Bundesländer zu zahlreichen synonym verwendeten Begrifflichkeiten führt. Eine gewisse Vorbelastung der Schulsozialarbeit, die stärkere Betonung des Jugendhilfecharakters, die angestrebte Verknüpfung präventiver und intervenierender Angebote sowie die Vermeidung einer einseitigen und etikettierenden Zielgruppenfokussierung sind zusätzliche Gründe, welche zahlreiche Wortschöpfungen in der fachlichen Diskussion hervorbringen \footcite[vgl.][23]{Speck2007}. Dabei merkt Karsten Speck kritisch an, dass verschiedenste eingeführte Bezeichnungen selbst von den Autoren diverser Fachpublikationen nicht konsequent eingehalten werden, klare Aussagen zu Zielen und inhaltlichen Schwerpunktsetzungen oftmals fehlen und "`[a]ngenommen werden kann, dass durch die Begriffsvielfalt [\punkte] ein fachlicher Austausch, die notwendige Konzeptdiskussion und Profilschärfung sowie die Transparenz und Durchsetzung des Arbeitsfeldes in der Fachöffentlichkeit deutlich erschwert sind."' \footcite[24]{Speck2007}. Vor dem Hintergrund der genannten Begründungen und in Anlehnung an Speck, als Autor der meisten relevanten Fachpublikationen im Arbeitsfeld, soll daher in den nachfolgenden Ausführungen der Begriff "`Schulsozialarbeit"' verwendet werden, den er 2006 wie folgt definiert:

\begin{quotation}
\noindent
"`Unter Schulsozialarbeit wird im Folgenden ein Angebot der Jugendhilfe verstanden, bei dem sozialpädagogische Fachkräfte kontinuierlich am Ort Schule tätig sind und mit Lehrkräften auf einer verbindlich vereinbarten und gleichberechtigten Basis zusammenarbeiten, um junge Menschen in ihrer individuellen, sozialen, schulischen und beruflichen Entwicklung zu fördern, dazu beizutragen, Bildungsbenachteiligungen zu vermeiden und abzubauen, Erziehungsberechtigte und LehrerInnen bei der Erziehung und dem erzieherischen Kinder- und Jugendschutz zu beraten und zu unterstützen sowie zu einer schülerfreundlichen Umwelt beizutragen. Zu den sozialpädagogischen Angeboten und Hilfen der Schulsozialarbeit gehören insbesondere die Beratung und Begleitung von einzelnen SchülerInnen, die sozialpädagogische Gruppenarbeit, die Zusammenarbeit und Beratung der LehrerInnen und Erziehungsberechtigten, offene Gesprächs-, Kontakt- und Freizeitangebote, die Mitwirkung in Unterrichtsprojekten und in schulischen Gremien sowie die Kooperation und Vernetzung mit dem Gemeinwesen \footcite[28]{Speck2007}."'
\end{quotation}

\noindent
Auf diese umfangreiche Definition, die zugleich bereits einige Zielstellungen und Aufgabenschwerpunkte beinhaltet und die Schulsozialarbeit sehr breit und insbesondere als Kooperation von schulischen und sozialpädagogischen Fachkräften interpretiert, soll an diversen Stellen in dieser Arbeit nochmals zurückgegriffen und an sie angeknüpft werden. 

Eine weitere und deutlich neuere Begriffsbestimmung liefert Pötter im Jahr 2014, wobei die Kooperation verschiedener Professionen und die engere Verzahnung von außerschulischem und schulischem Leben von Kindern und Jugendlichen in den Fokus genommen werden:

\begin{quotation}
\noindent
"`Schulsozialarbeit ist das Ergebnis von Kooperationen zwischen den verschiedenen Akteuren des Systems Schule -- insbesondere zwischen den sozialpädagogischen und den schulpädagogischen Fachkräften -- mit dem Ziel, "`Anschlussfähigkeit"' zwischen den Funktionssystemen -- insbesondere dem Erziehungs- und dem Bildungssystem -- und den Lebenswelten der Kinder und Jugendlichen sicherzustellen und zu unterstützen."'\footcite[23]{Poetter2014}
\end{quotation}

\noindent
Möglicherweise liegt dieser Definition ein ganzheitlicheres Verständnis, also die weniger strenge Trennung von Schule und "`Nichtschule"' im Sinne der Lebensweltorientierung zugrunde. Dies könnte sich insbesondere bei der Betrachtung persönlicher, außerschulischer Problemlagen von Kindern und Jugendlichen und deren Einfluss auf den schulischen Bereich als durchaus sinnvoll erweisen. Daher und aufgrund ihrer Aktualität wird diese Interpretation in die theoretischen Ausführungen aufgenommen.
 
Abschließend zu diesem Teil soll noch eine letzte Begriffsbestimmung ergänzt werden, welche aufgrund des direkten Bezuges zum Bundesland Sachsen interessant für die weiteren Ausführungen erscheint. In der Handreichung des Landesjugendamtes "`Schulsozialarbeit im Freistaat Sachsen"' aus dem Jahr 2008 wird folgende ausführliche Definition zur Thematik gegeben: 

\begin{quotation}
\noindent
"`Schulsozialarbeit zielt auf Begleitung der Schülerinnen und Schüler in ihrem Prozess des Erwachsenwerdens, auf Unterstützung bei einer für sie befriedigenden Lebensbewältigung sowie auf Förderung ihrer Kompetenzen zur Lösung von persönlichen und/oder sozialen Problemen. Dabei berücksichtigt Schulsozialarbeit, dass die gesellschaftliche Teilhabe über berufliche Eingliederung (Ausbildung, Arbeit) für junge Menschen von zentraler Bedeutung ist. Die berufliche Eingliederung wiederum setzt Schulerfolg, also entsprechende Schulabschlüsse, voraus. Schulsozialarbeit als Leistungsangebot der Jugendhilfe vereint die unterschiedlichen Methoden von sozialer Arbeit "`Einzelhilfe"', "`Gruppenarbeit"' sowie "`Gemeinwesenarbeit"' innerhalb eines sozialpädagogischen Gesamtkonzeptes. Dabei sind Einzelhilfe und Gruppenarbeit konstitutive Elemente des Gesamtkonzeptes.

Durch ihren niedrigschwelligen und aufsuchenden Charakter ist Schulsozialarbeit "`Prävention und Intervention vor Ort"' und hat schwerpunktmäßig die Schülerinnen und Schüler im Blick, die aufgrund sozialer Benachteiligungen und/oder individueller Beeinträchtigungen auf besondere Unterstützung angewiesen sind. Schulsozialarbeit fördert die schulische Ausbildung und die soziale Integration. Sie trägt damit ergänzend und erweiternd zur Verwirklichung des Erziehungsauftrages der Schule bei."' \footcite[10]{SMSSSL2008}
\end{quotation}

\noindent
Auffällig in dieser Definition ist insbesondere die starke Betonung der Bedeutung beruflicher Eingliederung im Sinne von Ausbildung und Arbeit, die sich in den Ausführungen anderer Autoren so nicht findet. Weiterhin bilden die Erwähnungen der Schulsozialarbeit im Sinne von Prävention und Intervention, die ausführlichen Darstellungen der Methoden sowie die Aussagen zur Zielgruppe eine Besonderheit, auf die in den weiteren Ausführungen noch einzugehen sein wird. 

Die vorgestellten Definitionen bilden, wie bereits eingangs ausgeführt, nur eine Auswahl aus einer Fülle vielfältiger begrifflicher Erläuterungen, welche sich in einigen Punkten ähneln und vielfach auch unterscheiden. Ausgewählt wurden die hier verwendeten aus unterschiedlichen Gründen. Zum einen sollten verschiedene Herangehensweisen an das Arbeitsfeld von mehreren Autoren vorgestellt werden. Zum anderen erschien es wichtig, dass Karsten Speck als Verfasser zahlreicher, viel zitierter und grundlegender Publikationen zum Thema Schulsozialarbeit vertreten ist. Die Definition von Pötter aus dem Jahr 2014 soll aufgrund ihrer Aktualität nicht außen vor gelassen werden und die des Sächsischen Staatsministeriums für Soziales repräsentiert die ministeriale Grundauffassung zur Schulsozialarbeit im Bundesland Sachsen. Da die von den Verfasserinnen dieser Arbeit selbst durchgeführten Untersuchungen sich auf eine berufliche Schule in Sachsen fokussieren, muss diese theoretische Grundlage unbedingt mit berücksichtigt werden. 

\subsection{Rechtliche Grundlagen}
\label{sec:RechtlicheGrundlagen}

Aktuelle und prägnant-zusammenfassende Aussagen zu den Rechtsgrundlagen von Schulsozialarbeit bieten die Autoren Stüwe, Ermel und Haupt in ihrer Publikation von 2015 \footcite[24ff]{Stuewe2015}, aus der die meisten der folgenden Aussagen entnommen sind. Zunächst muss festgehalten werden, dass derzeit keine verbindliche und bundeseinheitliche gesetzliche Regelung existiert. Daraus entstehen erhebliche länderspezifische Unterschiede in der Umsetzung, Wirksamkeit und Finanzierung von Schulsozialarbeit. Unter anderem ist es diese Heterogenität, welche bis in die differenzierten Bezeichnungen für Schulsozialarbeit hineinwirkt, die grundlegende Probleme im Arbeitsfeld bedingt (vgl. Punkt \ref{sec:AnnäherungAnEineDefinition} in Anlehnung an \cite[23]{Speck2007}).

Gesetzliche und bundesrechtliche Einflussfaktoren auf die Schulsozialarbeit sind der originäre Bildungs- und Erziehungsauftrag der Schule (Artikel 7 des Grundgesetzes in Verbindung mit dem Schulgesetz), der Erziehungs- und Bildungsauftrag der Eltern (Artikel 6 des Grundgesetzes) sowie das Wächteramt des Staates (Artikel 6 des Grundgesetzes), ausgeführt durch Schule und Jugendamt \footcite[vgl.][25]{Stuewe2015}.

\begin{quotation}
\noindent
"`Es gibt keine eigene Gesetzesnorm für Schulsozialarbeit, sondern dieses Aufgabengebiet wird abgeleitet aus dem SGB VIII (Sozialgesetzbuch) auf Bundesebene und den schulrechtlichen Regelungen in der Autorität der jeweiligen Bundesländer. Vor diesem Hintergrund müssen die rechtlichen Grundlagen der Kinder- und Jugendhilfe und die schulrechtliche Situation des jeweiligen Bundeslandes berücksichtigt werden und es gilt zu prüfen, inwieweit im entsprechenden Schulgesetz des Bundeslandes eine Kooperation von Schule mit der Kinder- und Jugendhilfe verankert ist."' \footcite[vgl.][25]{Stuewe2015}
\end{quotation}

\noindent
Ergänzend zu dieser Aussage ist zu erwähnen, dass in den Fachgremien der einzelnen Bundesländer sowie in der Expertenschaft allgemein eine große Uneinigkeit bezüglich der Ableitung der rechtlichen Auftragsgrundlage aus dem SBG VIII/KJHG (Kinder- und Jugendhilfegesetz) vorherrscht. Entscheidend in dieser Frage sind die §§ 11 (Jugendarbeit) und 13 (Jugendsozialarbeit) des SGB VIII/KJHG. 

\begin{quotation}
\noindent
"`Hinter dieser rechtlichen Diskussion steht letztlich die viel entscheidendere Frage, ob die Schulsozialarbeit ausschließlich einen intervenierenden Auftrag (§ 13) oder ausdrücklich auch einen primären Präventionsauftrag (§ 11) hat. Aus der Beantwortung dieser Frage, ergeben sich beachtliche Konsequenzen für die Ziele, Zielgruppen und Angebote der anvisierten Schulsozialarbeit."' \footcite[22]{Speck2006}
\end{quotation}

\noindent
Zur Verdeutlichung der oben ausgeführten Schwerpunkte wird nachfolgend das Schulgesetz des Freistaates Sachsen (SchulG) exemplarisch bezüglich der genannten Aussagen geprüft. Dabei ist festzustellen, dass sich konkrete Regelungen zur Schulsozialarbeit im Schulgesetz nicht finden. Eine kurze Erwähnung findet diese, unter Verwendung verschiedener Bezeichnungen, lediglich in folgenden Abschnitten: 
\begin{itemize}
	\item § 8 Berufsschule, Absatz 4: Die Berufsschule kann für Jugendliche, die zu Beginn der Berufsschulpflicht ein Berufsausbildungsverhältnis nicht nachweisen, als einjährige Vollzeitschule (Berufsvorbereitungsjahr) geführt werden. Jugendliche im Berufsvorbereitungsjahr sind \textbf{sozialpädagogisch zu betreuen}. \footcite[7]{SMKSK2010}
	\item § 16 Ganztagsangebote, Absatz 2: Zulässige Formen von Ganztagsangeboten sind insbesondere Schulklubs, Arbeitsgemeinschaften, zusätzlicher Förderunterricht oder \textbf{Angebote der Schuljugendarbeit}. \footcite[9]{SMKSK2010}
	\item § 17 Bildungsberatung, Absatz 2: Zur Unterstützung der Erziehung und Hilfe bei der Lebensbewältigung der Schüler durch die Eltern und Lehrer wird eine schulpsychologische Beratung ermöglicht, die schulartübergreifend durch Schulpsychologen mit Hilfe von Beratungslehrern erfolgt und die \textbf{Schulsozialarbeit} einbezieht. \footcite[9]{SMKSK2010}
	\item § 35b Zusammenarbeit: Die Schulen arbeiten mit den \textbf{Trägern der öffentlichen und der freien Jugendhilfe} und mit außerschulischen Einrichtungen, insbesondere Betrieben, Vereinen, Kirchen, Kunst- und Musikschulen und Einrichtungen der Weiterbildung, sowie mit Partnerschulen im In- und Ausland zusammen. \footcite[17]{SMKSK2010}
\end{itemize}

\noindent
Weitere Ausführungen zur Regelung der Schulsozialarbeit in Sachsen finden sich in diversen Verwaltungsvorschriften und Erlassen, beispielsweise in der Förderzuständigkeitsverordnung des Sächsischen Ministeriums für Kultus (SMKFördZuVO) im § 4 Förderprogramme zur Erfüllung besonderer schulischer Aufgaben, Abschnitt 2 (Maßnahmen zur sozialpädagogischen Betreuung Jugendlicher im Berufsvorbereitungsjahr) und Abschnitt 3 (Maßnahmen der Schuljugendarbeit) \footcite[1f]{SMKSK2015b}. Eine spezielle Verwaltungsvorschrift, ein Erlass oder eine Verordnung zur Schulsozialarbeit existiert derzeit nicht.

Stüwe, Ermel und Haupt weisen in ihren Ausführungen darauf hin, dass die rechtliche Verankerung und prinzipielle Regelung von Schulsozialarbeit in Deutschland derzeit als unzureichend bewertet werden kann \footcite[30]{Stuewe2015}. 

\begin{quotation}
\noindent
"`Anzustreben ist daher eine verbindlichere Erwähnung bzw. Verankerung [\punkte] sowohl im SGB VIII, als auch in den entsprechenden Schulgesetzen, da dies zu einer Verstetigung und Professionalisierung dieses Handlungsfeldes beitragen könnte."' \footcite[30]{Stuewe2015}
\end{quotation}

\subsection{Träger der Schulsozialarbeit}
\label{sec:TrägerDerSchulsozialarbeit}

Aus den im letzten Abschnitt aufgeführten rechtlichen Grundlagen der Schulsozialarbeit ergeben sich keinesfalls automatisch zuständige "`Ausführende"' im Sinne von festgelegten Behörden oder Institutionen. Auch hier ist innerhalb Deutschlands und innerhalb der Bundesländer eine Vielzahl von unterschiedlichen Modellen und Konstellationen vorzufinden. 

\begin{quotation}
\noindent
"`Für die Schulsozialarbeit ist es nicht unerheblich, unter welchen Trägerkonstellationen sie zukünftig stattfindet. Mit der Wahl eines Trägers werden in der Regel sowohl die Dienst- und Fachaufsicht bestimmt als auch konzeptionelle und inhaltliche Prämissen gesetzt."' \footcite{BIVSD2013}
\end{quotation}

\noindent
Die in der Praxis hauptsächlich vorkommenden Trägerkonstrukte lassen sich in drei Schwerpunkte einteilen, welche im Folgenden stichpunktartig kurz vorgestellt werden sollen (alle Informationen entstammen der Bundesweiten Informations- und Vernetzungsseite zur Schulsozialarbeit in Deutschland \footcite{BIVSD2013}):\\

\noindent
\textbf{Örtlicher Schulträger (Ministerium oder Schulamt) bzw. Einzelschule}
\begin{itemize}
	\item Anstellung des Schulsozialarbeiters beim örtlichen Schulträger bzw. der Einzelschule 
	\item Schulsozialarbeiter unterliegt der Schulhierarchie
	\item Dienstaufsicht und zumeist auch Fachaufsicht liegt bei der Schule bzw. dem Schulträger
	\item Vorteile: enge Einbindung der Schulsozialarbeit in den Arbeits- und Kooperationszusammenhang der Schule, keine Barrieren und Vorbehalte gegen einen engen Einbezug von Schulsozialarbeitern in unterrichtliche und außerunterrichtliche Arbeitszusammenhänge und Entscheidungsgremien
	\item Nachteile: mögliche Vereinnahmung und Unterordnung der Schulsozialarbeiter unter schulische Zwecke, Gefahr der Überladung mit schulischen Anforderungen (Feuerwehrfunktion, Vertretung bei Unterrichtsausfall und Betreuungsdienste), sozialpädagogische Ziele, Aufgaben und Arbeitsprinzipien zweitrangig
	\item Hinweis: Fachaufsicht aus der schulischen Hierarchie herauslösen, an einen Träger der Jugendhilfe (z. B. das Jugendamt) delegieren
\end{itemize}

\noindent
\textbf{Örtliches Jugendamt}
\begin{itemize}
	\item Schulsozialarbeiter ist Angestellter des Jugendamtes
	\item örtliches Jugendamt als Dienst- und Fachaufsicht 
	\item Vorteile: enge Verbindung von Schule und Jugendhilfe, gemeinsame Arbeitsgrundlage für Schulsozialarbeiter verschiedener Schulen ist gegeben 
	\item Nachteile: Lehrer fühlen sich durch die Tätigkeit der Jugendhilfe kontrolliert bzw. indirekt kritisiert 
\end{itemize}

\noindent
\textbf{Freier Träger der Jugendhilfe}
\begin{itemize}
	\item Freier Träger kann ein etablierter Wohlfahrtsverband (Arbeiterwohlfahrt, Deutscher Caritasverband, Deutsches Rotes Kreuz, Diakonisches Werk, Deutscher Paritätischer Wohlfahrtsverband, Zentralwohlfahrtsstelle der Juden), ein Jugendverband (z. B. Evangelische Jugend Deutschlands, Feuerwehrjugend, Sportjugend, Deutscher Pfandfinderbund, Gewerkschaftsbund) oder ein kleiner örtlicher bzw. sogar für die Schulsozialarbeit initiierter Träger (z. B. eingetragener Verein, Elterninitiative, Schulverein etc.) sein.
	\item Anstellung des Schulsozialarbeiters beim freien Träger
	\item Vorteile: fachliche Erfahrungen und Kompetenzen bei großen wohlfahrts- und jugendverbandlichen Trägern (auch in verwandten Arbeitsfeldern), bei Konflikten mit der Institution Schule großer (fach-)politischen Einfluss, besonderes freiwilliges Engagement kleiner Träger, gute Beziehungen zum örtlichen Schulumfeld, größere Flexibilität
	\item Nachteile: strukturelle Schwäche kleinerer Träger in finanzieller, personeller und organisatorischer Hinsicht, Probleme bei der langfristigen Absicherung und Stabilität der Schulsozialarbeit 
\end{itemize}

\noindent
Zusätzlich zu den genannten Trägerkonstrukten sind auch Mischformen, beispielsweise eine Trägerschaft durch sowohl einen öffentlichen oder freien Jugendhilfeträger in Kombination mit dem Schulamt möglich \footcite[vgl.][63]{Spies2011}. Grundsätzlich lässt sich laut Speck \footcite{BIVSD2013} keine Trägerkonstellation eindeutig favorisieren.

\begin{quotation}
\noindent
"`Die Wahl eines möglichen Trägers von Schulsozialarbeit sollte daher in erster Linie aufgrund seiner fachlichen Kompetenz in Bezug auf die Schulsozialarbeit erfolgen. Zur Trägerkompetenz gehören unter anderem die Entwicklung eines sozialpädagogisch fundierten Konzeptes, ein offensives und vorurteilsfreies Zugehen auf die Schulen, die Implementierung von fachlichen Qualitätsstandards in den Schulen, die fachliche Anleitung, Begleitung, Fortbildung und "`Rückendeckung"' für die sozialpädagogischen Fachkräfte in Schulen, die Vernetzung der Fachkräfte mit anderen SchulsozialarbeiterInnen sowie die Unterstützung der Evaluation und Qualitätsentwicklung der Schulsozialarbeit."' \footcite{BIVSD2013}
\end{quotation}

\noindent
Zur Sicherstellung dieser vielfältigen Anforderungen muss der jeweilige Träger über personelle, zeitliche und fachliche Ressourcen verfügen \footcite[vgl.]{BIVSD2013}. Weiterhin weist Speck auf die Bedeutung von Kooperationen bei der Initiierung von Schulsozialarbeit zwischen dem Träger, der Schule, dem Schulträger und dem Jugendamt hin, welche er als überaus sinnvoll erachtet. Spies und Pötter empfehlen in Anlehnung an zahlreiche Autoren sozialpä\-da\-gogischer Veröffentlichungen, dass die Schulsozialarbeit bestenfalls bei einem Jugendhilfeträger anzusiedeln ist \footcite[64]{Spies2011}. 

\begin{quotation}
\noindent
"`Ein Hauptargument dafür ist, dass die Dienst- und Fachaufsicht für die SozialarbeiterInnen bei einem Träger liegen sollte, der das fachliche -- also sozialpädagogische -- Know-How besitzt und außerhalb der stark hierarchisch organisierten Schulstruktur steht."' \footcite[64]{Spies2011}
\end{quotation}

\noindent
Dennoch werden auch Projekte mit einer kooperativen Trägerschaft von Jugendhilfe und Schule als sinnvoll erachtet, da oftmals eine bessere Zusammenarbeit mit der Institution Schule gewährleistet wird, von Beginn an größere Ressourcen beider Partner in die Kooperation einfließen und die Schule eine größere formale Verantwortung mit verstärktem Interesse am Erfolg des Projektes übernimmt \footcite[vgl.][64]{Spies2011}.

Inwieweit die genannten und beschriebenen Trägerkonstrukte für Projekte der Schulsozialarbeit an berufsbildenden Schulen eine Bedeutung haben, kann aus der Fachliteratur nicht herausgearbeitet werden. Es kann an dieser Stelle nur davon ausgegangen werden, dass die genannten Vor- und Nachteile sowie die Nützlichkeit von Kooperationen für diese Schulform ebenso zutreffen wie auf alle anderen. 

\subsection{Zielgruppen}
\label{sec:Zielgruppen}

Als Zielgruppen der Schulsozialarbeit im Allgemeinen können die Adressaten der Jugendhilfe verstanden werden, zu denen alle Kinder und Jugendlichen, Eltern und Lehrkräfte gehören \footcite[vgl.][31]{Speck2007}. Auch hierüber herrscht in der Fachwelt keinesfalls Einigkeit. Kontrovers diskutiert wird von einigen Autoren die Zugehörigkeit der Lehrkräfte zur Zielgruppe. Vielmehr verorten einige diese wiederum unter dem Begriff Kooperationspartner \footcite[vgl.][50]{Spies2011}. 

\begin{quotation}
\noindent
"`Wie es unter Kooperationspartnern üblich ist, kann und sollte man sich gegenseitig informieren, beraten und unterstützen. Eventuell kann man in dem einen oder anderen Fall auch zwischen Schülern und Lehrern oder Eltern und Lehrern vermitteln [\punkte]."' \footcite[vgl.][50]{Spies2011}
\end{quotation}

\noindent
Würde man jedoch die Beratung und Unterstützung von Lehrkräften als immanenten Tätigkeitsbereich der Schulsozialarbeit begreifen, könnten die Lehrenden auch wiederum als Zielgruppe verstanden werden.

Lange Zeit richteten sich die Angebote der Jugendhilfe vorwiegend an benachteiligte junge Menschen, diese Orientierung wurde jedoch durch die Neuausrichtung des Sozialgesetzbuches VIII/KJHG von 1991 eigentlich weitestgehend aufgehoben. Stattdessen betont die gültige gesetzliche Grundlage einen stark präventiven, partizipatorischen und freiwilligen Dienstleistungscharakter der Jugendhilfe. Insbesondere die Formulierungen "`junge Menschen in der sozialen Entwicklung fördern und dazu beitragen, Benachteiligungen zu vermeiden und abzubauen"' sowie "`positive Lebensbedingungen für junge Menschen und deren Familien schaffen"' unterstreichen diese Feststellungen \footcites[vgl.][30f]{Speck2007}[vgl.][46]{Spies2011}. 

Dennoch herrscht, wie im Punkt rechtliche Grundlagen bereits ausgeführt, Uneinigkeit über die Umsetzung dieser gesetzlichen Grundlage und die sich daraus ergebenden primären Zielgruppen der Schulsozialarbeit. Erstaunlicherweise fokussiert die im Abschnitt \ref{sec:AnnäherungAnEineDefinition} vorgestellte Definition von Schulsozialarbeit des Sächsischen Landesjugendamtes insbesondere auf "`[\punkte] Schülerinnen und Schüler [\punkte], die aufgrund sozialer Benachteiligungen und/oder individueller Beeinträchtigungen auf besondere Unterstützung angewiesen sind."' \footcite[10]{SMSSSL2008} Hier ergibt sich eine jedoch nur scheinbar eine sächsische Besonderheit, da ein Vergleich der jeweiligen Länderpraxen von Speck aufzeigt, dass in fast allen Bundesländern Schwerpunkte bei Jugendlichen, die am Übergang von der Schule in die Erwerbstätigkeit zu scheitern drohen und Schülern in Hauptschulbildungsgängen gesetzt werden \footcite[vgl.][19ff]{Speck2007}. Dabei konstatiert Speck auch eine besondere Verpflichtung der Schulsozialarbeit, mit ihren Angeboten benachteiligte und beeinträchtigte SchülerInnen zu erreichen und diese zu unterstützen, gleichzeitig soll sich diese jedoch keinesfalls auf diesen problem- oder defizitorientierten Ansatz beschränken \footcite[vgl.][46]{Speck2007}. 
\begin{quotation}
\noindent
"`Bei einem solchen Ansatz sind a) Defizitorientierung und Stigmatisierung der Schüler, b) eine geringe Mitwirkungsbereitschaft der Schüler sowie c) problematische konzeptionelle Beschränkungen, die dem Bild einer präventiven, lebensweltorientierten und anwaltschaftlichen Jugendhilfe widersprechen, vorprogrammiert. Zudem würden Ressourcen, Kooperationsmöglichkeiten und Wirkungspotentiale der Schulsozialarbeit ungenutzt bleiben."'\footcite[vgl.][46]{Speck2007}
\end{quotation}
 
\noindent
Weiterführende Aussagen zur Thematik formulieren Spies und Pötter, indem sie feststellen, dass "`[s]chulpflichtige Kinder und Jugendliche aller Altersstufen und unabhängig von der Schulform und der Trägerschaft der Schule [\punkte] grundsätzlich die Zielgruppe der Schulsozialarbeit sind."' \footcite[46]{Spies2011} Allerdings reichen die Projekte und Möglichkeiten der Schulsozialarbeit gar nicht aus, um überhaupt allen genannten Zielpersonen ein Angebot zu machen, was sich als durchaus problematisch erweist. Dies zwingt die im Handlungsfeld tätigen Personen letztlich dazu, ihre Adressaten einzugrenzen und erfordert Schwerpunktsetzungen, die praktisch zur Folge haben, dass doch wieder vermeintlich sozial benachteiligte und individuell beeinträchtigte Kinder und Jugendliche im Fokus der Arbeit stehen bzw. stehen müssen \footcite[vgl.][47]{Spies2011}.
 
Bezogen auf die beruflichen Schulen lässt sich festhalten, dass die dort stattfindende soziale Arbeit grundsätzlich alle Schülerinnen und Schüler ansprechen sollte, jedoch aufgrund der oben geschilderten praktischen Problemstellungen ähnlichen institutionellen Bedingungen unterworfen ist wie an allgemeinbildenden Schulen. Dabei  bilden ganz allgemein formuliert Jugendliche, sowohl mit Ausbildungsplatz als auch ohne, die Zielgruppe der berufsschulischen Sozialarbeit \footcite[vgl.][5]{ASSB2011}. Einzig Stüwe, Ermel und Haupt nehmen von allen Autoren, die sich mit theoretischen Grundlagen zur Thematik beschäftigen, in ihren Ausführungen überhaupt auf diese Schulform Bezug:

\begin{quotation}
\noindent
"`Eine besondere Rolle nimmt die Schulsozialarbeit an den berufsbildenden Schulen ein. Hier hat sie vor allem die Funktion einer Begleitung bei der Gestaltung des Übergangs in das Arbeitsleben und bei der Lösung individueller Konflikte und Defizite. Durch die einzelfallbezogenen Schwerpunktsetzungen sind somit die Zielgruppenfragen anders zu stellen."' \footcite[74]{Stuewe2015}
\end{quotation}

\noindent
Über die tatsächliche Fülle der Angebote für die genannten Zielgruppen, speziell bezogen auf den Freistaat Sachsen, wird an anderer Stelle der vorliegenden Arbeit noch ausführlich eingegangen. 

\subsection{Ziele}
\label{sec:Ziele}

Grundlegende Zielstellungen der Schulsozialarbeit finden sich relativ wenige in der vorhandenen Literatur, was dem geschuldet sein könnte, dass je nach definitorischem und zielgruppenbezogenem Ansatz durchaus sehr individuelle Ziele möglich sein könnten. Weiterhin spielen auch die jeweilige Schulform, das Schülerklientel, die institutionellen und finanziellen Gegebenheiten sowie die gesetzlichen Grundlagen des jeweiligen Bundeslandes bei der Betrachtung der Zielstellungen eine Rolle. 

Allgemeingültige Aussagen lassen sich primär aus den in Punkt \ref{sec:AnnäherungAnEineDefinition} benannten Definitionen herausarbeiten und sind auch dabei durchaus als heterogen zu bezeichnen. Laut Speck bestehen sie vordergründig in der Förderung junger Menschen in ihrer individuellen, sozialen, schulischen und beruflichen Entwicklung. Weiterhin soll und muss Schulsozialarbeit dazu beizutragen, Bildungsbenachteiligungen zu vermeiden und abzubauen sowie Erziehungsberechtigte und Lehrkräfte bei der Erziehung und dem erzieherischen Kinder- und Jugendschutz zu beraten und zu unterstützen sowie zu einer schülerfreundlichen Umwelt beizutragen \footcite[vgl.][28]{Speck2007}.

Folgt man hingegen der Definition von Pötter, so ist das vordergründige "`[\punkte] Ziel, Anschlussfähigkeit zwischen den Funktionssystemen -- insbesondere dem Erziehungs- und dem Bildungssystem -- und den Lebenswelten der Kinder und Jugendlichen sicherzustellen und zu unterstützen."' \footcite[23]{Poetter2014} Diese Anschlussfähigkeit definiert Pötter als wichtigen Parameter in einem jeweiligen spezifischen Kontext, "`[\punkte] der die Lebenswelten der Kinder und Jugendlichen und die gesellschaftlichen Strukturen und Anforderungen am Ort Schule miteinander verknüpft und dessen Aufgabe es ist, einerseits auf kommende Strukturen vorzubereiten und der andererseits bereits Chancen und Möglichkeiten innerhalb dieser Strukturen verteilt."' \footcite[24]{Poetter2014} Abgeleitet werden kann daraus, dass bei dieser Betrachtungsweise das Ziel der Schulsozialarbeit in der vielfältigen Nutzbarmachung von unterschiedlichen Angeboten und Chancen für alle Schüler in der derzeitigen Schulform liegt und gleichzeitig auch individuell gelingende Übergänge in weiterführende Schulformen oder in die berufliche Ausbildung sichergestellt, unterstützt und begleitet werden sollen.
 
Das Landesjugendamt des Freistaates Sachsen formuliert ganz ähnliche Zielstellungen, jedoch etwas einfacher und praktischer:

\begin{quotation}
\noindent
"`Schulsozialarbeit zielt auf Begleitung der Schülerinnen und Schüler in ihrem Prozess des Erwachsenwerdens, auf Unterstützung bei einer für sie befriedigenden Lebensbewältigung sowie auf Förderung ihrer Kompetenzen zur Lösung von persönlichen und/oder sozialen Problemen. Dabei berücksichtigt Schulsozialarbeit, dass die gesellschaftliche Teilhabe über berufliche Eingliederung (Ausbildung, Arbeit) für junge Menschen von zentraler Bedeutung ist."' \footcite[15]{SMSSSL2008}
\end{quotation}

\noindent
Neben der Begleitung Heranwachsender stehen hier insbesondere die Kompetenzförderung bei individuellen Problemlagen und die Förderung der Anschlussfähigkeit im Sinne der Eingliederung in Ausbildung und Berufstätigkeit im Vordergrund. 

Spezifische Zielstellungen zur Schulsozialarbeit an berufsbildenden Schulen existieren derzeit nicht. Bezüglich der genannten Aussagen ist jedoch festzustellen, dass diese zweifelsohne auch für den berufsbildenden Bereich gelten können. Neben der Förderung und Herausbildung beruflicher Handlungskompetenzen kann in dieser Schulform immer wieder auch die Förderung persönlicher Kompetenzen und die Unterstützung von Anschlussfähigkeit in eine weiterführende Ausbildung oder eine Erwerbstätigkeit als Ziel verstanden werden. Ob insbesondere die persönliche Problemlösekompetenz und die Unterstützung der Anschlussfähigkeit als Ziele auf die Schulsozialarbeit beschränkt sind oder nicht auch Ziele der Schule und der pädagogischen Fachkräfte sein können und müssen, könnte kritisch diskutiert werden. 

\subsection{Aufgabenfelder}
\label{sec:Aufgabenfelder}

Wie bereits in den Definitionsversuchen zur Schulsozialarbeit ausgeführt, finden sich auch zu den Arbeitsbereichen und Aufgabenfeldern in der verwendeten Fachliteratur unterschiedlichste Darstellungen, Leistungsschwerpunkte und Kategorien. Insbesondere Speck weist darauf hin, dass es für die notwendige Etablierung von Schulsozialarbeit unerlässlich ist, ein klares Arbeitsprofil herauszuarbeiten, welches bestimmte Kernleistungen (gemäß dem Förderauftrag für die schulische und außerschulische Entwicklung) beinhalten sollte, die als ergänzungsoffene Pflichtaufgaben zu verstehen sind \footcite[vgl.][62]{Speck2007}. Orientiert am Konzept der lebensweltorientierten Schulsozialarbeit im Rahmen der Jugendhilfe und an den Befunden dessen wissenschaftlicher Begleitung lassen sich nach Speck folgende sechs Kernleistungen bzw. -aufgaben herauskristallisieren: 
\begin{itemize}
	\item \textbf{Beratung und Begleitung von einzelnen SchülerInnen} (z. B. Einzelfallhilfe, Beratungsgespräche, Einzelförderung, Sprechstunden)
	\item \textbf{sozialpädagogische Gruppenarbeit} (z. B. berufsorientierende Angebote, erlebnispädagogische Maßnahmen, soziales Kompetenztraining, außerunterrichtliche Projekte, offenes Förderangebot)
	\item \textbf{offene Gesprächs-, Kontakt- und Freizeitangebote} (z. B. Schülerclub, offener Treff, Freizeitangebote)
	\item \textbf{Mitwirkung an Unterrichtsprojekten und in schulischen Gremien} (z. B. Gesamtkonferenzen, Klassenkonferenzen, Schulprogrammarbeit)
	\item \textbf{Zusammenarbeit mit und Beratung der LehrerInnen und Erziehungsberechtigungen} (z. B. Beratungsgespräche, Fortbildungen, Elterngespräche, Elternabende und -besuche)
	\item \textbf{Kooperation und Vernetzung mit dem Gemeinwesen} (z. B. Jugendamt, Arbeitsverwaltung, Ämter, freie Träger der Jugendhilfe, Aufbau von Hilfestrukturen, Integration von Personen, Unternehmen und Institutionen aus dem Gemeinwesen \footcite[vgl.][63f]{Speck2007}.
\end{itemize}

\noindent
Eine hilfreiche Erweiterung der Thematik bieten Spies und Pötter, indem sie konstatieren, dass "`[s]tets Ermöglichung von Schulerfolgen, Abbau von Lernbarrieren, Sicherung verwertbarer Abschlüsse und Gewährleistung von schulischer Haltekraft die Leitgedanken sind, die hinter der schulsozialarbeiterischen Deklination jedes Arbeitsbereiches stehen"' \footcite[93]{Spies2011}. Die Autorinnen unterteilen die bereits von Speck formulierten Kernleistungen der Schulsozialarbeit dazu in drei Arbeitsbereiche, nämlich "`Bildungsbedingungen"', "`individuelle Orientierung und Hilfe"' sowie "`Förderung des sozialen Lernens"' und erweitern sie um die fachlichen Hintergründe der jeweiligen Aufgabenfelder. Dabei werden der "`sozialpädagogischen Expertise der Jugendhilfe"' die individuelle Förderung, sozialpädagogische Gruppenarbeit, offene Angebote und sozialpädagogische Einzelfallberatung, dem "`schulischen Hoheitsbereich"' das  Schulprogramm und dessen Entwicklung, gemeinsam mit der Berufsorientierung sowie der "`schulischen Zuständigkeit mit sozialpädagogischem Fach- und Methodeninput"' die Kooperation mit Eltern, schulbezogene Hilfen und Konfliktbewältigung zugeteilt. Unternommen wird der Versuch, "`die tatsächlichen Tätigkeiten in ihrer Bildungsfunktion abzubilden, ohne die dafür nötige Fachlichkeit zu eng an die sozialpädagogische Expertise zu binden."' \footcite[91]{Spies2011} Mit dieser Positionierung wird die Herangehensweise unterstrichen, die Schulsozialarbeit als ein Produkt der Kooperation verschiedener, an Schulen tätiger Disziplinen, insbesondere der Schul- und der Sozialpädagogik zu verstehen und davon ausgehend auch die Aufgabenbereiche der Schulsozialarbeit abzubilden.

\begin{quotation}
\noindent
"`Schulsozialarbeit dient der Sicherstellung und Unterstützung von Anschlussfähigkeit von Schülerinnen und Schülern mit unterschiedlichen sozialen und ethischen Hintergründen, deren Bildungsbedingungen sie verbessert, indem sie mit bildungswissenschaftlicher und sozialpädagogischer Expertise dazu beiträgt, bestehende Benachteiligungen abzubauen, die individuelle und soziale Entwicklung zu fördern und Empowerment-Strategien zu vermitteln. Sie ist als interinstitutionelle Vermittlungsinstanz im Bildungs- und Hilfesystem zu verstehen, die zur Verbesserung der Lern- und Lebensraumbedingungen von Kindern und Jugendlichen in Kontexten des Bildungssystems unter zu Hilfenahme eines breiten Handlungsspektrums von Lern- und Bildungssettings agiert."' \footcite[92]{Spies2011}
\end{quotation}

\noindent
Den genannten Aufgabenfeldern gibt es aus der Sicht der beruflichen Schule nichts hinzuzufügen. Die klassische sozialpädagogische Elternarbeit, geschlechtsspezifische Arbeit (Projektarbeit, Mädchenförderung) und offene Arbeit (insbesondere in Form von Kontakt- und Freizeitangeboten) rücken jedoch im genannten Handlungsfeld zugunsten von individueller Beratung, Begleitung und Unterstützung, sozialem Lernen und Projektarbeit eher in den Hintergrund  \footcite[vgl.][10]{LSS2004}.

\subsection{Methoden}
\label{sec:Methoden}

Nachdem verschiedene Grundlagen der Schulsozialarbeit bis hierhin erläutert wurden, soll nun im vorletzten dazugehörigen Abschnitt ein Überblick über die dazu zur Verfügung stehenden methodischen Möglichkeiten gegeben werden. Unter Methoden werden dabei "`[\punkte] zielgerichtete wissenschaftsgestützte und handlungserprobte Techniken verstanden, die sich im Rahmen einer Konzeption, orientiert an ethischen Prinzipien begründen lassen."' \footcite[25]{Kilb2009}. Weiterhin folgen sie definierten Prinzipien und vollziehen sich in bestimmten Arbeitsschritten. Bei jedem Arbeitsschritt werden Techniken und Instrumente berücksichtigt, die am besten geeignet sind, um die jeweiligen Ziele zu erreichen und als Einzelelemente von Methoden zu verstehen sind \footcite[vgl.][30]{Kilb2009}. Das fachliche Handeln des Schulsozialpädagogen orientiert sich dabei stets an den Zielgruppen, Zielen und Aufgabenfeldern und muss flexibel, variabel und bedarfsorientiert für den jeweiligen Schulstandort sein. Dabei erfordert die Praxis durch die bereits ausgeführte und insbesondere im Punkt \ref{sec:BedarfeFürSchulsozialarbeit} (Bedarfe für Schulsozialarbeit) noch auszuführende Vielfältigkeit ein großes Methodenrepertoire \footcite[vgl.][94]{Stuewe2015}. 

Festzustellen ist nach der Analyse mehrerer Quellen, dass keine spezifischen Methoden theoretisch ausgeführt und praktisch angewandt werden, da die Schulsozialarbeit unter dem methodischen Blickwinkel gesehen keine autonomes Arbeitsfeld ist \autocites[vgl.][94]{Stuewe2015}[vgl.][32]{Kilb2009}[vgl.][63]{Speck2007}. 
\begin{quotation}
\noindent
"`Für die Schulsozialarbeit bieten die Methoden der Sozialen Arbeit den fachlichen Hintergrund, der allerdings in den schulischen Kontext übertragen werden muss. Außerdem kann sie durchaus auf das methodische Repertoire der Schulpädagogik zurückgreifen -- sofern sie dies [\punkte] gründlich geprüft hat."' \footcite[67]{Spies2011}. 
\end{quotation}

\noindent
Das Methodenspektrum der Schulsozialarbeit ist damit äußerst vielfältig und kann in klassische, direkt klientenbezogene Methoden, zu denen die Beratung und Einzelfallhilfe, die soziale Gruppen- und Projektarbeit und die inner- und außerschulische Gemeinwesenarbeit zählen, sowie indirekt klientenbezogene, wie Supervision und Selbstevaluation, gegliedert werden \footcites[vgl.][22]{Kilb2009}[vgl.][64]{Speck2007}. Die drei genannten klientenbezogenen Schwerpunkte sind als klassisches Methodentrio der Sozialen Arbeit bekannt \footcite[vgl.][95]{Stuewe2015}. 

\begin{quotation}
\noindent
"`Da professionelles methodisches Handeln weit über diese drei Methoden hinausgeht, ist jedoch eine feinere Kategorisierung notwendig. Schulsozialarbeit zieht zur Erreichung ihrer Ziele sehr unterschiedliche Methoden heran, die in der Theorie der Sozialen Arbeit in zwei mal zwei Kategorien unterteilt werden können."' (\cite[vgl.][95]{Stuewe2015} in Anlehnung an \cite[64f]{Speck2007})
\end{quotation}

\noindent
Direkt wirkende Angebote der Schulsozialarbeit mit Interventionsbezug sind dabei:
\begin{itemize}
	\item \textbf{Einzelfall- und primärgruppenbezogene Methoden} dienen einem gezielten und überprüfbaren Unterstützungsprozess zwischen Fachkraft und einem Kind bzw. Jugendlichen. Beispiele dafür sind die Einzelfallhilfe, sozialpädagogisch-klientenzen-\\
	trierte, systemische und lösungsorientierte Beratung, multiperspektivische Fallarbeit, Case-Management, Mediation, Familientherapie und Methoden der Gesprächsführung. 
	\item \textbf{Sekundärgruppen- und sozialraumbezogene Methoden} dienen ebenfalls einem gezielten und überprüfbaren Unterstützungsprozess, beziehen jedoch das soziale Netzwerk der Klienten (Eltern, Gruppenangehörige, Klasse, Lehrkräfte usw.), andere Gruppenmitglieder oder das Gemeinwesen mit ein. Zu dieser Methodengruppe gehören die Soziale Gruppenarbeit, Erlebnispädagogik, themenzentrierte Interaktion, themen- und zielgruppenspezifische Methoden (Suchtprävention, Mädchen- und Jungenarbeit), Sozialkompetenztraining, Empowerment, soziale Netzwerkarbeit und Gemeinwesenarbeit. 
\end{itemize}

\noindent
Neben diesen direkten wirkenden Angeboten gibt es weitere, indirekt wirkende Leistungen der Schulsozialarbeit, welche die Zielerreichung und die angestrebte Wirkung der Schulsozialarbeit unterstützen können bzw. Voraussetzungen für die Handlungsfähigkeit schaffen:
\begin{itemize}
	\item \textbf{Indirekte unterstützungsbezogene Methoden} dienen der Erhaltung und Verbesserung von Handlungsfähigkeit, Arbeitsstruktur und Professionalität der Schulsozialarbeiter sowie der Qualitätsentwicklung und -sicherung. Das Arbeitsfeld wird dabei systematisch evaluiert und reflektiert, beispielsweise mittels Supervision, Selbst\-eva\-luation, Fort- und Weiterbildung und Konzeptentwicklung. 
	\item \textbf{Struktur- und organisationsbezogene Methoden} haben das Ziel, strukturelle und organisatorische Voraussetzungen, Rahmenbedingungen und Grundlagen für konkrete Angebote zu schaffen und dienen der Abstimmung und Planung von Hilfestrukturen vor der eigentlichen Intervention. Beispielhaft können dafür Sozialmanagement, Jugendhilfeplanung sowie interdisziplinäre Diskussionen und Vernetzungen genannt werden (\cite[vgl.][96ff]{Stuewe2015} in Anlehnung an \cite[64f]{Speck2007}).
\end{itemize}

\noindent 
Der Einzelfallhilfe als Beispiel der einzelfall- und primärgruppenspezifischen Methoden scheint in der Praxis eine zentrale Bedeutung zuzukommen, da die "`[i]ndividuelle Begleitung von Schülerinnen und Schülern mit den Bezugspunkten "`Hilfen bei der Alltagsbewältigung"' und "`Biografie-Begleitung"' vor dem Hintergrund von Schulschwierigkeiten, Problemen im Elternhaus sowie Fragen, die den Übergang von der Schule in Ausbildung bzw. Ausbildung in Arbeit betreffen, häufig im Vordergrund der sozialpädagogischen Arbeit stehen. In jedem Fall sollte die Einzelfallhilfe sich am Prinzip "`Hilfe zur Selbsthilfe"' orientieren."' \footcite[23]{SMSSS2009} Die soziale Gruppenarbeit hingegen hat eine überwiegend erzieherische Intention und orientiert sich an den Problemen und Verhaltensmustern ihrer Zielgruppe. Durch die Gruppe in ihrem Gesamtgefüge und gruppenspezifische Aktivitäten können unter sozialpädagogischer Anleitung positive Sozialisationseffekte erzielt werden, wodurch die Gruppenarbeit das geeignete Übungsfeld für soziales Lernen im Rahmen der Schulsozialarbeit bildet \footcite[vgl.][24]{SMSSS2009}. Abschließend stellt die Gemeinwesenarbeit im Rahmen der Schulsozialarbeit eine Methode dar, die es sich zur Aufgabe macht, Lebenszusammenhänge und Probleme von jungen Menschen nicht nur individuell zu verstehen, sondern sie in einem Wirkungssystem zwischen Schule und dem jeweiligen sozialem Umfeld zu begreifen. Dabei wird angestrebt, Schule und Gemeinwesen besser zu vernetzen, um die Öffnung der Schule nach außen zu unterstützen und Kooperationen mit außerschulischen Einrichtungen und Institutionen zu fördern \footcite[vgl.][25]{SMSSS2009}. 

Die Einzelfallhilfe in verschiedenen Formen von Beratung und individueller Begleitung, sowie die soziale Gruppenarbeit und Projektarbeit scheinen methodisch insbesondere an beruflichen Schulen eine große Bedeutung zu haben \footcite[vgl.][10ff]{LSS2004} (vgl. Punkt \ref{sec:Aufgabenfelder}).

\subsection{Konzeptionen}
\label{sec:Konzeptionen}

Wie bei allen bisher ausgeführten Punkten der theoretischen Grundlagen zur Schulsozialarbeit gibt es auch hinsichtlich der Konzepte bzw. Konzeptionen zahlreiche verschiedene Aussagen und Ansatzpunkte, weshalb im Folgenden nur die wichtigsten Aspekte betrachtet werden, die für die weiteren Ausführungen der vorliegenden Arbeit von Bedeutung sein könnten. 

Nach Stüwe, Ermel und Haupt ist ein Konzept ein Handlungsmodell, welches Strukturen und Wege beschreibt, mit denen Ziele und erwünschte Ergebnisse erreicht werden können \footcite[vgl.][148]{Stuewe2015}. Zu diesem Zweck werden Inhalte, Methoden und Verfahren in einen sinnvollen Zusammenhang gebracht. Eine Konzeption hingegen wird als Zusammensetzung aufeinander bezogener Konzepte verstanden. Erst ein kommunales, gegebenenfalls trägerspezifisches Konzept und ein davon ausgehendes spezifisches Schulstandortkonzept können zusammen dem Anspruch an eine Konzeption gerecht werden. Die Entwicklung der einzelnen genannten Konzepte wird als komplexer und anspruchsvoller Prozess bezeichnet, der einer intensiven Planung und Beteiligung der Schule, der Anstellungsträger und der Zielgruppen bedarf. Neben einem träger- und schulartspezifischen Rahmenkonzept und dem schon benannten individuellen Schulstandortkonzept können weiterhin themenspezifische, angebotsspezifische, schlüsselsituative und kooperationsbedingte Konzepte die verschiedenen Ebenen der Schulsozialarbeit abbilden und in einer Gesamtkonzeption zusammenfließen. Die Zielstellungen bei der Erstellung von Konzeptionen und Konzepten sind ebenfalls als multidimensional zu bezeichnen. Diese können von der Praxissteuerung mittels Orientierung für die Fachkräfte, über die Transparenz nach innen und außen durch eine klare Beschreibung des Tätigkeitsprofils bis hin zu Profilbildung, kontinuierlicher systematischer Weiterentwicklung der Praxis und Verbesserung der Fachlichkeit reichen \footcite[vgl.][148ff]{Stuewe2015}.

Speck konstatiert hinsichtlich der Konzepte und Konzeptionen eine erhebliche Unübersichtlichkeit und versucht sich an einer sinnvollen fachlichen Systematisierung, die die wichtigsten Konzeptionsmodelle von Projekten der Schulsozialarbeit abbildet \footcite[vgl.][25f]{Speck2006}. Diese Modelle werden nachfolgend kurz vorgestellt:
\begin{description}
	\item[Projekte mit stark freizeitpädagogischer Ausrichtung] integrieren die Schulsozialarbeit häufig aufgrund fehlender Betreuungs- und Freizeitangebote in der Schule und zielen auf sozialpädagogische Betreuung und Kommunikation im außerunterrichtlichen Bereich ab. Die Zielgruppe bilden grundsätzlich alle Schüler. Die Angebote sind meist gruppenbezogen konzipiert und werden zum Beispiel als Schülerclubs, Arbeitsgemeinschaften sowie gemeinsame Freizeitaktivitäten umgesetzt. Lehrkräfte und sozialpädagogische Fachkräfte kooperieren meist wenig. 
	\item[Projekte mit einer problembezogenen fürsorgerischen Ausrichtung] begrün-\\ 
den den Einsatz von Schulsozialarbeit mit den gestiegenen Belastungen und vielfältigen Problemlagen von Kindern und Jugendlichen, bedingt durch familiäre, schulische und gesellschaftliche Veränderungen oder durch problematische Verhaltensweisen. Die Zielgruppe bilden daher vorrangig sozial benachteiligte und/oder individuell beeinträchtigte Schüler. Als wichtige Einsatzbereiche für diese Art von Projekten gelten z. B. "`Brennpunktschulen"' oder bestimmte Schultypen (z. B. Hauptschule oder Berufsschulen). Inhaltlich werden hauptsächlich Angebote zur Beratung und Unterstützung unterbreitet, die methodisch vorwiegend in Form der Einzelfallhilfe und der sozialen Gruppenarbeit umgesetzt werden. Zwischen Lehrkräften und sozialpädagogischen Fachkräften können verschiedene Kooperationsstufen gegeben sein. Möglicherweise arbeiten beide Gruppen mit unterschiedlichen Zuständigkeiten ohne relevante Berührungspunkte, wichtig ist aber auch eine Zusammenarbeit in Form der Überweisung problematischer Schüler durch Lehrkräfte an die Schulsozialarbeiter.
	\item[Projekte mit einer integriert sozialpädagogischen Ausrichtung] verweisen auf die Notwendigkeit der Schulsozialarbeit aufgrund der wünschenswerten niedrigschwelligen und lebensweltorientierten Verzahnung von Jugendhilfe und Schule. Zu den Zielgruppen gehören daher alle Schüler unter besonderer Berücksichtigung solcher mit sozialen Benachteiligungen und/oder individuellen Beeinträchtigungen sowie Eltern und Lehrkräfte. Es werden offene, präventiv ausgerichtete Betreuungs- und Freizeitangebote mit einzelfall- und gruppenbezogenen Probleminterventionen sowie der Gemeinwesenarbeit kombiniert. Weiterhin kann von einer intensiven und gleichberechtigten Kooperation von Schulsozialarbeitern und Lehrkräften ausgegangen werden \footcite[vgl.][25f]{Speck2006}.
\end{description}

\noindent
Aussagen zu den spezifischen Konzeptionen und der konzeptionellen Ausgestaltung von Angeboten im berufsbildenden Bereich finden sich in der gesichteten Literatur wiederum wenige bis keine. Lediglich Speck verweist in seinen Ausführungen zur Schulartspezifik auf die Bedeutung der Projekte mit einer problembezogenen fürsorgerischen Ausrichtung für den Bereich der Berufsschulen \footcite[vgl.][25f]{Speck2006}. Dem kann ohne Frage nur zugestimmt werden, da die freizeitpädagogisch ausgerichteten Konzeptionen aufgrund der Alters- und Schülerstruktur an berufsbildenden Schulen schon von vornherein als unbedeutsam eingestuft werden können. Projekte mit einer integriert sozialpädagogischen Ausrichtung erscheinen zwar auch für die genannte Schulart, wie für andere Schularten auch, als wünschenswert, sind wohl jedoch aufgrund von Umsetzungshürden (z. B. Finanzierung der für die zahlreichen möglichen Angebote dieser Konzeptionsform) oder der heterogenen, auch gesetzlich bedingten Definition der Zielgruppen (in Punkt \ref{sec:RechtlicheGrundlagen} und \ref{sec:Zielgruppen} ausführlich ausgeführt) praktisch wenig vorzufinden. 
