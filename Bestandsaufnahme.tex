\section{[Bestandsaufnahme zur Schulsozialarbeit]{Bestandsaufnahme zur Schulsozialarbeit an berufsbildenden Schulen in Sachsen}
\label{sec:BestandsaufnahmeZurSchulsozialarbeitAnBerufsbildendenSchulenInSachsen}

\subsection{Vorbemerkungen}
\label{sec:Vorbemerkungen}

Nach den, auf den vorherigen Seiten, dargestellten theoretischen Grundlagen zur Schulsozialarbeit und den Problemlagen von Schülern soll nunmehr der Blick auf die tatsächlich vorhandenen Angebote gerichtet werden. Die Feststellung weitreichender Problemlagen, die vermutlich auch im Bereich der berufsbildenden Schulen vorzufinden und insbesondere auch in den berufsvorbereitenden Maßnahmen ausgeprägt sein könnten, lässt die Annahme zu, dass Schulsozialarbeit in dieser Schulart ebenso reichlich vorzufinden sein müsste, wie in anderen Schularten. Dieser Abschnitt soll dabei wiederum hauptsächlich die Verhältnisse im Bundesland Sachsen betrachten, da die Ergebnisse möglicherweise relevant für die eigenen Forschungen an der dafür ausgewählten Schule sein könnten. Erinnert werden soll an dieser Stelle noch einmal an die rechtlichen Grundlagen zur Schulsozialarbeit in Sachsen, die im Abschnitt \ref{sec:RechtlicheGrundlagen} ausführlich beschrieben wurden. Insbesondere die gesetzliche Vorgabe der sozialpädagogischen Betreuung von Jugendlichen im Berufsvorbereitungsjahr \footcite[vgl.][7]{SMKSK2010} scheint hier von Relevanz zu sein. 

\subsection{Statistische Angaben}
\label{sec:StatistischeAngaben}

Bevor Angaben zur Schulsozialarbeit an berufsbildenden Schulen erfolgen, sollen vorab die Schülerzahlen und die Aufgliederung in relevante Schulformen einen Überblick über mögliche Zielgruppen bzw. Adressaten der Schulsozialarbeit bieten. Im Schuljahr 2014/15 wurden an berufsbildenden Schulen in Sachsen insgesamt 99499 Schüler unterrichtet. Dazu standen 259 berufsbildende Schulen bzw. Schulzentren mit mehreren integrierten Schulformen zur Verfügung, die sich in 78 unter öffentlicher und 181 unter freier Trägerschaft stehende aufteilten. An den öffentlichen Einrichtungen wurde eine Schülerzahl von 71000 erreicht, die verbleibenden 28499 Schüler wurden an den, in der deutlichen Überzahl befindlichen, Einrichtungen in freier Trägerschaft beschult (vgl. Statistisches Landesamt des Freistaates Sachsen 2015, S. 10 ff).

Hinsichtlich der Fachrichtungen Gesundheit, Pflege und Soziales sind neben diesen allgemeinen Angaben auch die Anteile der Berufsfachschulen und Fachschulen interessant. An diesen wurden insgesamt 31686 Schüler beschult, dazu standen 175 Berufsfachschulen (20244 Schüler) und 106 Fachschulen (11442 Schüler) zur Verfügung. Auf öffentliche Berufsfachschulen entfielen 35 Schulen mit 5127 Lernenden, auf die in freier Trägerschaft 140 Schulen mit 15117 Lernenden. In Fachschulen in öffentlicher Trägerschaft lernten an 37 Einrichtungen 4272 Teilnehmer, an den 69 freien Einrichtungen 7170 (vgl. ebd., S. 286 ff). 

Abschließend sollen noch die Angaben zu den Berufsvorbereitungsjahren (BVJ) mit explizit vorgeschriebener sozialpädagogischer Betreuung betrachtet werden. In diesen befanden sich insgesamt 3106 sächsische Schüler, an öffentlichen Schulen 2322 und an Schulen in freier Trägerschaft 784 (vgl. ebd.). 

\subsection{Erhebungen zur Schulsozialarbeit an berufsbildenden Schulen}
\label{sec:ErhebungenZurSchulsozialarbeitAnBerufsbildendenSchulen}

Eine erste verfügbare Bestandsaufnahme zum Stand der Schulsozialarbeit in Sachsen wurde 2002 durch die Landesarbeitsgemeinschaft Schulsozialarbeit e. V. durchgeführt und 2004 veröffentlicht. In dieser wurden auch die beruflichen Schulen, zumindest in öffentlicher Trägerschaft, untersucht und dabei aus 46 dieser Schulen Daten erhoben. Notwendig erachteten die berufsbildenden Schulen dabei insbesondere Beratungsangebote durch die Schulsozialarbeit, soziales Lernen und Projektarbeit (vgl. Landesarbeitsgemeinschaft Schulsozialarbeit Sachsen e. V. 2004, S. 6 ff). Der festgestellte Bedarf bei 43\% der Schulen (höhere Bedarfe zeigten sich nur an Förderschulen mit 73\% und an Mittelschulen mit 66\%) lässt nun den Verdacht zu, dass in dieser Schulart relativ viele sozialpädagogische Fachkräfte zur Umsetzung von Angeboten der Schulsozialarbeit eingesetzt werden. Demgegenüber steht jedoch die Tatsache, dass 2002 nur in drei der genannten 46 Berufsschulen Schulsozialarbeit stattfand (vgl. ebd.). Dabei ist jedoch zu berücksichtigen, dass die gesetzlich vorgeschriebene sozialpädagogische Betreuung im Berufsvorbereitungsjahr unerfasst blieb. Interessant erscheint weiterhin die Tatsache, dass einige berufsbildende Schulen keinen Bedarf für Angebote der Schulsozialarbeit signalisierten, da diese intern an dreizehn und extern an drei Schulen bereits abgedeckt wurde (ebd., S. 5). Dies lässt die Vermutung zu, dass sozialpädagogische Angebote in irgendeiner Art und Weise existieren, leider werden keine Angaben zu Form und Umfang gemacht. Von einer Bedarfsdeckung geht die Landesarbeitsgemeinschaft mit den erhobenen Daten auf keinen Fall aus, es wird im Gegenteil von einem "`Tropfen auf den heißen Stein"' gesprochen und die Unterausstattung und der Ausbaubedarf des Handlungsfeldes klar benannt (vgl. ebd., Vorwort). Weiterhin konstatiert die Landesarbeitsgemeinschaft folgende interessante Tatsache: 

\begin{quotation}
\noindent
	"`Schulsozialarbeit wird nach wie vor von der Schule in der Regel nicht als wichtiger, geschweige denn gleichwertiger Partner bei der Lösung sozialer Probleme von SchülerInnen akzeptiert. Entsprechend unzureichend ist die Einbindung von Schulsozialarbeit in die institutionellen Strukturen und Prozesse der Schule [\punkte]."' (ebd.)
\end{quotation}

\noindent
Eine weitere Erhebung zum Stand wurde 2006 im Auftrag der Bundesarbeitsgemeinschaft Jugendsozialarbeit e. V. von einer Mitarbeiterin der Jugendberufshilfe Thüringen e. V. als Überblick über die Berufsschulsozialarbeit in den einzelnen Bundesländern erstellt. In dieser Bestandsaufnahme führt die Autorin aus, dass das Land Sachsen nicht über eine vollständige Auflistung von Projekten der Schulsozialarbeit verfügt, die in Berufsvorbereitungsjahren und beruflichen Schulen allgemein stattfinden. Aus den Daten der Landratsämter ging jedoch hervor, dass an sechs beruflichen Schulzentren Sozialpädagogen und Schulsozialarbeiter angestellt waren und weitere Projekte ausgeschlossen werden konnten (vgl. Laßmann 2006, S. 28).

Weiterhin konnte eine Publikation des Sächsischen Landesjungendamtes über die Projekte der Schulsozialarbeit im Freistaat Sachsen aus dem Jahr 2007 zu Tage bringen, dass Kooperationen mit einer Berufsschule bestanden (vgl. Sächsisches Landesamt für Familie und Soziales 2007, S. 4). Hinsichtlich des Berufsvorbereitungsjahres konnte ergänzend folgende Aussage aufgefunden werden: 

\begin{quotation}
\noindent
	"`Das Sächsische Staatsministerium für Kultus informiert im Januar 2007 im Rahmen der Beantwortung einer entsprechenden Landtagsanfrage über den Einsatz von sozialpädagogischem Fachpersonal an den einzelnen Beruflichen Schulzentren im Freistaat Sachsen. Demnach arbeiten 15 sozialpädagogische Fachkräfte an 15 Beruflichen Schulzentren."' (ebd., S. 13)
\end{quotation}

\noindent
Die aktuellste Publikation zur Thematik liegt aus dem Jahr 2014 vor. Zu bemerken ist, dass diese "`Übersicht zur Schulsozialarbeit im Freistaat Sachsen"' die Angebote zur sozialpädagogischen Betreuung im Berufsvorbereitungsjahr wiederum nicht berücksichtigt, da diese auf der gesetzlichen Grundlage des Schulgesetzes basieren und nicht dem SGB VIII zuzuordnen sind (vgl. SMS 2014, S. 3). Hinsichtlich der berufsbildenden Schulen wurde festgestellt, dass Schulsozialarbeit an zwei Einrichtungen stattfand (vgl. ebd., S. 5). 

In allen auffindbaren Erhebungen der Ministerien und der Landesarbeitsgemeinschaft ist zu berücksichtigen, dass ausschließlich Angebote der Schulsozialarbeit und der sozialpädagogischen Betreuung an öffentlichen Schulen Beachtung fanden. Deshalb wurden zur Vorbereitung der vorliegenden Arbeit weitere eigene Versuche unternommen, um insbesondere auch mögliche Angebote an Schulen in freier Trägerschaft zu eruieren bzw. weitere Gesamtangebote aufzudecken. Dazu erfolgte im Mai 2015 eine Anfrage bei der Landesarbeitsgemeinschaft Schulsozialarbeit Sachsen e. V., die lange unbeantwortet blieb. Auf erneute Rückfrage wurde mitgeteilt, dass keine aktuellen Daten vorliegen, nur an beruflichen Schulzentren mit Berufsvorbereitungsjahr eine sozialpädagogische Betreuung vorgeschrieben ist und das Schulverwaltungsamt genauere Auskünfte erteilen könnte. Die nachfolgende Anfrage beim Schulverwaltungsamt blieb leider bis heute unbeantwortet. Zusätzlich wurde im Juni die Gewerkschaft Erziehung und Wissenschaft (GEW) -- Landesverband Sachsen angefragt, da das dortige Referat Jugendhilfe und Soziale Arbeit sich ebenfalls mit der Thematik beschäftigt. Eine Antwort erfolgte zügig telefonisch, erbrachte leider keine neuen Erkenntnisse zum aktuellen Stand, jedoch den Hinweis einer Anfrage an das SMK, Referat 4, berufsbildende Schulen sowie einer stichprobenartigen Internetrecherche zur Schulsozialarbeit und sozialpädagogischen Betreuung auf den Homepages der beruflichen Schulzentren. Dort fanden sich in den Stichproben der beruflichen Schulzentren Plauen, Aue, Chemnitz, Dresden und Leipzig keinerlei Angaben zur genannten Thematik, so dass auch dieser Rechercheversuch erfolglos blieb. 

Die abschließende Anfrage im SMK ergab, dass aus Datenschutzgründen keine konkreten Angaben zum Thema erfolgen können. Somit können zu möglichen weiteren Angeboten der Schulsozialarbeit und insbesondere zu der an Schulen in freier Trägerschaft leider keine Ausführungen erfolgen. 

\subsection{Fazit der Bestandsaufnahme}
\label{sec:FazitDerBestandsaufnahme}

Folgt man der letzten Bestandaufnahme zur Schulsozialarbeit aus dem Jahr 2014, so finden sich an zwei der 78 öffentlichen berufsbildenden Schulen in Sachsen Projekte der Schulsozialarbeit. Davon ausgehend, dass die Landesarbeitsgemeinschaft Schulsozialarbeit in Sachsen e. V. bereits im Jahr 2004 die vorhandenen Angebote, damals immerhin drei an 46 Schulen, als "`Tropfen auf den heißen Stein"' beschrieb und diese sich seitdem offenbar nochmals verringert haben, hat sich die Situation an sächsischen Berufsschulen wiederum verschlechtert. Obwohl zahlreiche Studien und der Dritte Sächsische Kinder- und Jugendbericht eine stetige Zunahme der Problemlagen feststellen, werden die Angebote der Schulsozialarbeit weniger und können wohl ohne weitere Begründung als unzureichend und nicht bedarfsdeckend bezeichnet werden. Die Gründe für diese Diskrepanz sind nicht nachvollziehbar und entziehen sich, zumindest in der vorliegenden Literatur, jeglicher fachlicher Erklärungsansätze und Argumentationen. Spekulationen dazu sollen an dieser Stelle vermieden werden. Leider fehlen auch jegliche Bestandsaufnahmen zu den 181 Schulen in freier Trägerschaft, an denen immerhin 28499 sächsische Schüler berufsbezogen unterrichtet werden. Mutmaßlich finden sich auch dort Angebote der Schulsozialarbeit oder sozialpädagogischen Betreuung, die beachtenswert sein könnten. Möglicherweise wird Schulsozialarbeit aber auch, sowohl im öffentlichen Bereich, als auch bei freien Schulträgern, begrifflich gar nicht als solche verortet, was aufgrund der zahlreichen Konzepte, Inhalte, Träger und Ziele durchaus denkbar sein könnte. Die Aussage der internen und externen Abdeckung von Bedarfen außerhalb der Schulsozialarbeit (vgl. Punkt \ref{sec:ErhebungenZurSchulsozialarbeitAnBerufsbildendenSchulen}) deutet zumindest auf Projekte oder Unterstützungsangebote hin, die sich den bisherigen Erhebungen entziehen. 

Interessant erscheint auch die Tatsache, dass 2007 scheinbar durch fünfzehn Fachkräfte die gesamte vorgeschriebene sozialpädagogische Betreuung im Berufsvorbereitungsjahr realisiert wurde. Vergleicht man diese personelle Ausstattung rein theoretisch mit den Schülerzahlen des letzten Schuljahres, so hätte jeder Sozialarbeiter 207 Schüler (oder ca. acht bis zehn Klassen) dieses, als durchaus schwierig beschriebenen, Klientels zu betreuen gehabt. Bedenkt man dabei noch die rückläufigen Schülerzahlen der letzten Jahre, ist davon auszugehen, dass 2007 der Anteil zu betreuender Schüler sogar noch größer gewesen sein dürfte. Ob mit einer solchen personellen Ausstattung eine qualifizierte Begleitung und Unterstützung gewährleistet werden kann, darf bezweifelt werden.

Hinsichtlich der Wirksamkeit von Schulsozialarbeit an berufsbildenden Schulen gibt es für Sachsen keine Erhebungen. Deshalb kann lediglich auf Ausführungen zu dieser Schulart aus Thüringen Bezug genommen werden. Dort wurde, im Jahr 2000 beginnend, ein fünfjähriges Modellprojekt begleitet und evaluiert, welches flächendeckend Schulsozialarbeit an den entsprechenden Schulen implementierte und ausbaute (vgl. Bauer 2010, S. 119). Auch hier stand das Berufsvorbereitungsjahr wieder sehr stark im Fokus der Betrachtungen. Zusammenfassend lässt sich feststellen, dass der Schulsozialarbeit aus der Perspektive aller beteiligten Akteure eine bedeutsame Rolle bei der situativen Bearbeitung von komplexen Problemlagen zugeschrieben wird. Insbesondere betrifft dies Probleme, die unmittelbar aus dem schulischen Geschehen entstehen und die schulischen Ziele, wie Schulabschluss oder Ausbildungsplatz, beeinträchtigen (vgl. ebd., S. 129). 

\begin{quotation}
\noindent
	"`Eine wirksame sozialpädagogische Bearbeitung [\punkte] erschien aus Sicht aller Beteiligten als ein wichtiger Beitrag zur Erreichung der Ziele. [\punkte] Erfolge und Wirkungen von Schulsozialarbeit lassen sich in diesem Bereich daher überwiegend als erfolgreiche Bearbeitung von "`Problemfällen"' beschreiben."' (ebd. S. 131)
\end{quotation}

\noindent
 Dem gegenüber ließen sich Effekte auf einer umfassenderen gruppen- oder schulbezogenen Ebene, wie z. B. die Verminderung und Beseitigung allgemeiner schulischer Probleme (Gewalt, Drogen, Unterrichtsstörungen, Schulverweigerung) kaum oder gar nicht nachweisen (vgl. ebd.). 

\begin{quotation}
\noindent
	"`Gründe dafür liegen sicherlich in der geringen personellen und materiellen Ausstattung der Schulsozialarbeit, in der gewachsenen Komplexität von Problemlagen der SchülerInnen an berufsbildenden Schulen, aber auch in methodologischen und messtheoretischen Problemen [\punkte]"'. (ebd., S. 131 f)
\end{quotation}

\noindent
 Trotz der offensichtlichen Bedarfe wurde die Sozialarbeit an vielen Schulen mit dem Auslaufen der Mittel des Landes Thüringen und des Europäischen Sozialfonds nach Abschluss des Projektes wieder eingestellt, nur einige Kommunen übernahmen die Trägerschaft und führten diese weiter (vgl. ebd., S. 119). 