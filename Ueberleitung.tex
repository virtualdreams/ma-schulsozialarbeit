\section{Überleitung vom theoretischen zum praktischen Teil der Arbeit}
\label{sec:ÜberleitungVomTheoretischenZumPraktischenTeilDerArbeit}

Im gesamten vorangestellten theoretischen Teil dieser Arbeit, mit Ausnahme des Gliederungspunktes 4, sind Grundlagen der Schulsozialarbeit vorgestellt, interpretiert und bewertet worden, meist jedoch mit dem Hinweis, dass sie nur in bestimmten Anteilen auf den berufsbildenden Bereich anwendbar und übertragbar sind. Vorausschickend auf den nächsten, forschungsbezogenen Abschnitt der Arbeit ist zunächst zuzugeben, dass der Begriff "`Schulsozialarbeit"' nunmehr kaum noch verwendet werden wird. Der Leser könnte sich nun berechtigterweise fragen, warum denn bis hierher ausführlich darüber referiert wurde, wenn der Begriff inklusive seiner Definition und charakterisierenden Merkmale gar keine Verwendung mehr findet. Zum einen ist dies damit zu begründen, dass Schulsozialarbeit in ihrer ganzen Merkmalsbreite und Ausdifferenzierung nach Ansicht der Verfasserinnen auf viele Bereiche der berufsbildenden Schulen gar nicht zutrifft, was an einigen Stellen der theoretischen Ausführungen (siehe Punkt \ref{sec:Konzeptionen}) bereits dargelegt wurde. Bezogen auf die vorgestellte Forschungsschule, das DRK Bildungswerk SN, müssen, bedingt durch strukturelle, personelle und finanzielle Besonderheiten, weitere Abstriche vom eigentlichen Begriff erfolgen. Aus diesen Gründen wurde die Entscheidung für die Bezeichnung "`außerunterrichtliche Beratungs- und Unterstützungsangebote"' getroffen, die jedoch sehr wohl ausgewählte theoretische Grundlagen der Schulsozialarbeit enthält, diese jedoch nicht in ihrer gesamten Breite und Tiefe abbildet. Dementsprechend können diese außerunterrichtlichen Beratungs- und Unterstützungsangebote z. B. in Bezug auf konzeptionelle, methodische sowie ziel- und zielgruppenbezogene Aspekte auf die eigentliche Schulsozialarbeit zurückgreifen. 

Selbstverständlich könnte weiterhin gefragt werden, warum nicht auf andere und passendere theoretische Konstrukte eingegangen wurde. Auch hier ist die Antwort kurz und prägnant zu geben -- weil solche schlichtweg, für den berufsbildenden Bereich und insbesondere für Gesundheit, Pflege und Soziales, nicht vorhanden sind. Zusätzlich zu diesen genannten Aspekten ist bei den Vorbereitungen zur eigenen Forschungsarbeit, in Gesprächen mit Lehrkräften und Schülern, aufgefallen, dass kaum bis gar keine Vorstellungen von Schulsozialarbeit und ihren Möglichkeiten, Angeboten, Leistungen und Fachkräften vorhanden war, so dass auch dies für eine andere, leichter zu erläuternde, Begriffsverwendung sprach. Eine kurze Begriffsdefinition wird an entsprechender Stelle im Forschungsteil enthalten sein. 

Weiterhin wurde in den theoretischen Grundlagen nicht auf die Fachbereiche Gesundheit, Pflege und Soziales Bezug genommen, obwohl die eigenen Forschungsarbeiten doch dafür ausgerichtet sind. Auch hier muss die Begründung angeführt werden, dass Publikationen und Studien über veränderte Schülerstrukturen und Problemlagen in diesen Bereichen kaum vorhanden sind, was zum Teil ausschlaggebend für die eigene Forschungsidee war. Für zukünftige Lehrkräfte in diesem Feld erscheinen die genannten Themen als durchaus wichtig und relevant für die eigene Arbeit.

Nun könnte leicht festgestellt werden, dass gesellschaftliche Veränderungen, bezogen auf Schülerstrukturen und Problemlagen, selbstverständlich aus anderen Berufsfeldern oder allgemein aus dem beruflichen Bereich abgeleitet werden können, da die Schüler in Gesundheits-, Pflege- und sozialen Berufen den gleichen Lebensbedingungen unterliegen wie alle anderen Berufsschüler auch. Das mag auch zutreffend sein, dennoch liegen nach Ansicht der Verfasserinnen einige weitere Besonderheiten vor, die in diese Berufe und deren Ausbildung hineinwirken. Lehrkräfte beklagen hier, ebenso wie anderswo auch, die seit einigen Jahren deutlich schwierigeren, leistungsschwächeren, demotivierten, problembehafteten und für die Berufe ungeeigneten Schüler. Sind diese Beobachtungen nur als allgemeine, schon immer dagewesenen Lehrerklagen und subjektive Wahrnehmungen zu sehen oder haben sie auch eine begründbare Substanz? Ohne wieder die veränderten Lebensbedingungen und gesellschaftlichen Umbrüche, allgemein oder auch bis in die Tiefe ausgedeutet, heranzuziehen \footcite[vgl.][37ff]{BundesministeriumFamilie2013} und zu sehr in das allgemeine Lamento der "`schwierigen Jugendlichen"' einzustimmen, das die Geschichte der Jugendhilfe und der Pädagogik seit jeher begleitet \footcite[vgl.][144]{UniversitaetLeipzig2007}, sollen einige eigene Gedanken zu diesen wahrgenommenen Veränderungen formuliert werden. 

Die auch als "`Helferberufe"' bezeichneten Tätigkeitsbereiche im Gesundheits-, Pflege- und Sozialwesen unterliegen von jeher besonderen Anforderungen, insbesondere hinsichtlich der Fach- und Sozialkompetenz. Wer in seinem späteren Berufsleben so intensiv mit unterschiedlichsten Menschen aller Altersgruppen, oft in gesundheitlich oder sozial schwierigen Lagen, arbeiten möchte, sollte dafür bestimmte Voraussetzungen mitbringen. Viele praktische Fähigkeiten und Fertigkeiten und theoretische Kenntnisse sind erlernbar, jedoch scheinen auch eine gewisse Grundeignung und eine "`Berufung"' für die Tätigkeiten unabdingbar. Erfahrene Lehrkräfte behaupten, dass dies immer seltener zu beobachten wäre und sich in den letzten zehn bis fünfzehn Jahren stark verändert hätte. "`Die schulischen Voraussetzungen, bzw. personellen Eignungen der Bewerber werden seitens der Einrichtungen jedoch als abnehmend eingeschätzt."' konstatiert dazu das Deutsche Krankenhausinstitut in einer Studie von 2006 \footcite[8]{Krankenhausinstitut2006}. Als Untermauerung der genannten Eindrücke können auch am DRK Bildungswerk SN gemachte persönliche Erfahrungen und Beobachtungen von Mobbing und Diskriminierung in Erzieherklassen oder enormen Lernschwierigkeiten in der Physiotherapie gelten, die so wie derzeit vorher kaum berichtet wurden. Dafür scheint vielfach die bereits beschriebene Heterogenität hinsichtlich der Altersstruktur und der schulischen Voraussetzungen in den Klassen mit verantwortlich zu sein. Neben vielen anderen möglichen Gründen, die an dieser Stelle nicht alle aufzuarbeiten sind, können sicherlich die demografische Veränderungen und insbesondere die Veränderungen auf dem Ausbildungsmarkt als Ursachen mitverantwortlich gemacht werden. Noch vor ca. 15-20 Jahren galten die Schüler in den Gesundheits-, Pflege- und sozialen Berufen als "`handverlesen"', am DRK Bildungswerk SN wurden z. B. ausschließlich Abiturienten und gute Realschulabsolventen für die Berufe aufgenommen. 200 Bewerber für eine Klasse Physiotherapeuten waren damals keine Seltenheit, so dass die Auswahl geeigneter Bewerber mit bereits gefestigten Persönlichkeiten, keine besondere Schwierigkeit war. Die Klassen wiesen aus fachlicher und sozialer Sicht keine so große Heterogenität auf. Heute jedoch hat sich dieses Bild deutlich gewandelt, viele Abiturienten studieren und  gute Realschulabsolventen durchlaufen andere, bevorzugt duale Ausbildungen mit den entsprechenden Ausbildungsvergütungen und guten Aufstiegschancen. Im Ausbildungsjahr 2013/14 blieben beispielsweise 37100 betriebliche Ausbildungsstellen unbesetzt, was die große Auswahl an Möglichkeiten für Bewerber mit guten Schulabschlüssen nur im Ansatz verdeutlicht \footcite[vgl.][15]{BBF2015}. Demgegenüber blieben jedoch auch ca. 20900 Bewerber aus verschiedensten Gründen "`unversorgt"' und stellen damit mögliche Kandidaten für berufsvorbereitende Maßnahmen, Praktika oder eben auch den Berufsfachschulbereich mit niedrigen Zugangsvoraussetzungen (z. B. am DRK Bildungswerk SN die Krankenpflegehilfe mit Zugangsvoraussetzung Hauptschulabschluss) dar \footcite[vgl.][15]{BBF2015}. Diese eben ausgeführte Gesamtsituation zeigt sich auch an den Berufsfachschulen mit einem insgesamt deutlichen Rückgang der Bewerberzahlen und der damit verbundenen geringeren Auswahl fachlich und persönlich geeigneter Schüler sowie und somit auch einer steigenden Heterogenität in den Klassen. 

Erschwerend kommt hinzu, dass derzeit die "`Helferberufe"' aktuell mit einem niedrigeren Status konfrontiert sind, vermeintlich schlechte Arbeitsbedingungen, Schichtdienste und ein vergleichsweise geringes Einkommen bei hoher Verantwortung sind Schwerpunkte, die bei Jugendlichen und jungen Erwachsenen offenbar eine abschreckende Wirkung ausmachen, was sich besonders an den Bewerberzahlen in der Altenpflege am DRK Bildungswerk SN zeigen lässt. 

\begin{quotation}
\noindent
"`So verwundert es auch nicht, dass das Image der Pflegeberufe in der Öffentlichkeit schlecht und der Beruf bei Schulabgänger/-innen unattraktiv ist."'
\end{quotation}

\noindent 
[\punkte] stellt dazu zutreffend Hall in ihrer Publikation fest \footcite[19]{Hall2012}. Noch drastischer wird diese Entwicklung in einer Studie des ipp Bremen dargestellt, in der Berufe in der Pflege und dabei insbesondere in der Altenpflege als sogenannte "`Out-Berufe"' aus der Sicht der Jugendlichen bezeichnet werden \footcite[18]{BPHP2010}. Mit diesen kurzen Auszügen können im Ansatz Veränderungen in der Ausbildungs- und Bewerbersituation der Gesundheits-, Pflege und Sozialberufe dargestellt werden, mit denen ein scheinbar sinkendes fachliches Niveau und auch soziale Probleme verbunden sind, die in die genannten Ausbildungsbereiche hineinwirken und die Heterogenität innerhalb der Klassen mitbedingen. Viele weitere Fragestellungen und Probleme der beruflichen Ausbildung stehen damit in Zusammenhang, so scheinen bspw. die Lehrpläne noch vielfach an ein nicht mehr vorhandenes Schülerklientel angepasst zu sein, was eine fachliche Überforderung nach sich zieht, die immer wieder beobachtbar wird. Zu all den dargestellten Herausforderungen gesellen sich, z. B. durch das Aufkommen und die Verbreitung der sozialen Netzwerke ganz neue Problemstellungen für die Berufsausbildung, die neben der Ablenkung und Beschäftigung im Unterricht und auch ganz neue Tendenzen für Mobbing und Diskriminierung mit sich gebracht haben und Lehrkräfte sowie ganze Schulen vor gänzlich neue Probleme stellen.

All die genannten und hier kurz vorgestellten Beobachtungen, Wahrnehmungen und Tendenzen und noch viele mehr wirken in die Zusammensetzung der Schülerschaft mit ihren individuellen Problemlagen hinein und bedingen somit auch in die aktuellen Herausforderungen für die berufsbildenden Schulen bzw. möglicherweise auch den Bedarf für außerunterrichtliche Beratungs- und Unterstützungsangebote. Deshalb wurden sie als Hintergrundinformationen dem sich nun anschließenden Forschungsteil vorausgeschickt. 
