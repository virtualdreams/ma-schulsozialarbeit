\section{Einleitung}
\label{sec:Einleitung}

Eine Lehrerin einer Berufsfachschule für Krankenpflegehilfe tritt nachmittags gegen 15.30 Uhr sichtlich erschöpft aus dem Unterrichtsraum:
\begin{quotation}
\noindent
"`Diese Schülerinnen und Schüler, es ist kaum noch mit ihnen fertig zu werden! Die erste Unterrichtseinheit brauche ich schon für die Klärung von Schulgeldproblemen, danach besprechen wir zum tausendsten Male die Schwierigkeiten im Praktikum und die Unpünktlichkeit oder die unentschuldigten Fehlzeiten. Sobald ich dann endlich mit fachlichen Inhalten anfangen will, ist die Aufmerksamkeit und Disziplin schon so schlecht, dass ich meine Unterrichtsplanung völlig über Bord werfen muss und wieder nicht vorwärts komme. Die Klasse hat doch bald Prüfung! Wir brauchen dringend jemanden, der sich mal mit diesen Problemen beschäftigt [\punkte]!"'
\end{quotation}

\noindent
Diese tatsächlich erlebte Situation ist eine, die sich täglich vermutlich mehrfach in beruflichen Schulen in Sachsen so oder ähnlich abspielen könnte. Sie war daher unter anderem ausschlaggebend, sich im Rahmen der vorliegenden Masterarbeit mit den Problemlagen von Schülern und den sich daraus möglicherweise ergebenden Bedarfen für außerunterrichtliche Beratungs- und Unterstützungsangebote an berufsbildenden Schulen in den Fachrichtungen Gesundheit, Pflege und Sozialwesen auseinanderzusetzen. 

"`Über kaum ein Thema wird in den letzten Jahren so kontrovers diskutiert, wie über die Schule."' stellt unter anderem Drilling \footcite[9]{Drilling2004} fest und auch am berufsbildenden Bereich gehen konzeptionelle, strukturelle, curriculare und fachliche Debatten keinesfalls spurlos vorüber. Im besonderen Spannungsfeld von Theorie und Praxis diagnostizieren und kritisieren die Praktiker eine angeblich zunehmende Ausbildungsunreife der Jugendlichen und eine Nichteignung für die Ausübung eines Berufes \footcite[vgl.]{Ehrenthal2005}, während insbesondere die Lehrkräfte der Schulen sich über verstärkende und vielfältige soziale Problemlagen und Lernschwierigkeiten beklagen \footcite[vgl.][13]{SMSSS2009}. Mit der Aussage [\punkte]

\begin{quotation}
\noindent
"`Erwachsenwerden in der heutigen Zeit fällt vielen Jugendlichen schwer. Der Übergang von der Schule in die Berufswelt ist vielfach von Brüchen gekennzeichnet, familiäre Geborgenheit findet in ökonomischen Zwängen ihre Grenzen, nicht wenige Jugendliche wachsen im Widerspruch zwischen Erlebniswelt und Erfahrungsrealität auf. Die Gesellschaft schwankt zwischen Dramatisierung und Verharmlosung, schiebt mal den Eltern, mal der Schule, mal der Politik die Schuld zu."' 
\end{quotation}

\noindent
[\punkte] versucht Drilling \footcite[19]{Drilling2004} eine kurze Beschreibung dessen, was heute vielfach als "`gesellschaftlicher Wandel"' bezeichnet und diskutiert wird sowie vielfältige Herausforderungen, insbesondere für die berufliche Bildung, mit sich bringt. Als Ergebnis scheinen diese gesellschaftlichen Veränderungsprozesse immer mehr sogenannte "`schwierige Jugendliche"', also junge Menschen in komplexen Problemsituationen, hervorzubringen, mit deren Umgang die Lehrkräfte im berufsbildenden Bereich zunehmend überfordert sind \footcite[vgl.][1]{UniversitaetLeipzig2007}. Titelgemäß könnte man durchaus etwas provokativ fragen: Haben es Schüler also heute besonders schwer? Und Lehrer aber auch!? 

Die eben geschilderten gesellschaftlichen und schulischen Veränderungen lassen verständlicherweise immer wieder den Ruf nach sozialpädagogischer Unterstützung laut werden, der jedoch scheinbar selten wirklich gehört wird bzw. nur vereinzelt in tatsächlichen Angeboten an berufsbildenden Schulen eine wirksame Umsetzung findet. Um diese Annahme zu untermauern, lohnt sich ein Blick in die Übersicht zu Angeboten Sozialer Arbeit an Schulen im Freistaat Sachsen, aus der ersichtlich wird, dass im Jahr 2014 an zwei beruflichen Schulen in öffentlicher Trägerschaft (von im Schuljahr 2014/15 vorhandenen 78) Projekte der Schulsozialarbeit umgesetzt wurden, obwohl eine Bedarfsanalyse von 2004 ergeben hatte, dass in mindestens 43 Prozent der Schulen Bedarfe vorhanden waren \footcites[vgl.][5]{LSS2004}[vgl.][5ff]{SMSSSV2014}. Augenscheinlich besteht also entweder eine Diskrepanz von Angebot und Nachfrage oder Schüler an berufsbildenden Schulen haben deutlich weniger Problemlagen als subjektiv wahrgenommen und somit gar keinen so relevanten Bedarf an Unterstützungsangeboten!?

Die Praxiserfahrungen der Verfasserinnen als angehende Berufspädagoginnen in den Fachrichtungen Gesundheit, Pflege und Sozialpädagogik lassen jedoch die Vermutung zu, dass auch dieses Berufsfeld von den oben geschilderten Entwicklungen und Veränderungen stark betroffen zu sein scheint. Aus diesen Feststellungen und den Überlegungen bezüglich der Relevanz des Themas für die spätere berufliche Tätigkeit, entstand die Idee, mit einer eigenen Forschungstätigkeit die tatsächlichen Problemlagen der viel zitierten "`schwierigen Schüler"' genauer in den Blick zu nehmen. Verstärkt wurde diese Absicht durch die Erkenntnis, dass spezifische wissenschaftliche Forschungsergebnisse für die genannten Fachrichtungen nicht vorhanden sind bzw. nicht aufgefunden werden konnten. Vor dem Beginn der eigenen Forschungstätigkeit bestand jedoch die Herausforderung, theoretische Grundlagen zu den Aspekten der Problemlagen und möglicher sozialpädagogischer Unterstützungsangebote für Schüler in beruflichen Ausbildungen der Gesundheit und Pflege sowie des Sozialwesens zu recherchieren und herauszuarbeiten. Dabei ergab sich schnell, dass die Schulsozialarbeit als theoretischer Bezugsrahmen für die sozialpädagogische Arbeit in den beruflichen Schulen relevant und demzufolge in der Betrachtung unerlässlich ist. Allerdings wurde dabei klar, dass der berufsbildende Bereich in der ausgewählten Literatur zur Schulsozialarbeit sehr unterrepräsentiert ist und deshalb die theoretischen Grundlagen an den entsprechenden Stellen an die Spezifika bzw. die speziellen Anforderungen angepasst werden mussten. Wesentliche Aussagen wurden daher häufig aus den Darstellungen für die allgemeinbildende Schule abgeleitet. Die gleiche Problematik ergab sich hinsichtlich der Analyse von Problemlagen junger Menschen, die eine Berufsausbildung in den Fachrichtungen Gesundheit, Pflege und Sozialwesen absolvieren. Es liegen zwar insgesamt zahlreiche Publikationen vor, die jedoch entweder sehr allgemein die Probleme Jugendlicher oder insbesondere die Teilnehmer an berufsvorbereitenden Maßnahmen oder Berufsvorbereitungsjahren in den Blick nehmen. Fachrichtungsspezifisch sind weder Problemanalysen, noch theoretische Grundlagen oder Konzepte zum Umgang mit den komplexen Problemlagen der Schüler vorhanden. Auf der Basis dieser Erkenntnisse entstanden die ersten Ideen hinsichtlich der erkenntnisleitenden Fragen dieser Arbeit, die zu Beginn vorrangig die Probleme aus der Schülersicht beinhalteten. Zusätzlich bestand die Herausforderung, eine geeignete Schule für das Forschungsvorhaben auszuwählen. Aufgrund des bereits abgegrenzten Berufsfeldes kam dahingehend nur ein Berufliches Schulzentrum für Gesundheit und Soziales oder eine Einrichtung in freier Trägerschaft in Frage, an der die genannten Fachrichtungen gemeinsam vertreten sind. Da eine Forschungsarbeit an einer öffentlichen Schule mit nicht unerheblichen organisatorischen Hürden und Beantragungsverfahren verbunden ist, fiel die Entscheidung relativ schnell zugunsten einer Schule in freier Trägerschaft, wobei das Bildungswerk des Deutschen Rotes Kreuz in Sachsen (nachfolgend nur noch als DRK Bildungswerk SN benannt) durch das langjährige Angestelltenverhältnis einer der Verfasserinnen als geeignet erschien. Zeitgleich mit diesen ersten Überlegungen ergaben sich Interessensbekundungen der Geschäftsführung des DRK Bildungswerkes SN an der Thematik, da dort ein verstärktes Auftreten von Schülerproblemen ebenfalls registriert wurde. Zusätzlich verstärkte die Feststellung, dass dahingehend dort bisher keine Unterstützungsangebote vorhanden waren, erste Überlegungen diese Situation möglicherweise zu verändern und die Ergebnisse der anvisierten Forschungsarbeiten dazu zu nutzen. Da die Einrichtung mit ca. 650 Schülern auch von der Größe her für das Vorhaben geeignet erschien, wurde die Entscheidung getroffen, die Forschungsarbeit dort durchzuführen und auch mögliche konzeptionelle Ansätze auf diese zu fokussieren. Es sind also sowohl die analytischen Ergebnisse, als auch die abgeleiteten Erkenntnisse als exemplarisch für das DRK Bildungswerk SN einzuordnen. Trotzdem sind diese in Ansätzen sicherlich auf andere Schulen bzw. den beruflichen Bereich Gesundheit, Pflege und Sozialwesen übertragbar. 

Neben der primär in den Fokus genommenen Schülersicht auf Problemlagen und mögliche Unterstützungsbedarfe erschien zunehmend auch die Lehrersicht unerlässlich, um ein besseres Gesamtbild zur Thematik zu erhalten. Es erfolgte eine kontinuierliche Beschäftigung mit den möglichen erkenntnisleitenden Fragestellungen, aus denen sich letztendlich die folgenden Forschungsfragen differenzierten:

\begin{enumerate}
	\item Welche persönlichen und sozialen (außerunterrichtlichen) Problemlagen haben Schüler des DRK Bildungswerk SN in sozialen und medizinischen Ausbildungsberufen, die das Unterrichtsgeschehen und den Ausbildungserfolg beeinflussen?
	\item Welche persönlichen und sozialen (außerunterrichtlichen) Problemlagen von Schüler nehmen Lehrkräfte des DRK Bildungswerk SN im Bereich Gesundheit, Pflege und Sozialwesen als besondere Belastung für den Unterricht wahr?
	\item Wie schätzen Schüler und Lehrkräfte anhand der (möglichen) subjektiv wahrgenommenen Problemlagen den Bedarf an sozialpädagogischen und anderweitigen Unterstützungs- und Beratungsangeboten ein und wie könnten passende Angebote aussehen?
\end{enumerate}

\noindent
An dieser Stelle ist darauf hinzuweisen, dass fachliche bzw. unterrichtlich-inhaltliche Problemstellungen ausdrücklich nicht im Fokus der Forschungsarbeit standen, sondern hauptsächlich auf soziale und persönliche Problemlagen eingegangen wurde. Das ist schwerpunktmäßig damit zu begründen, dass die verschiedenen Fachrichtungen berufsspezifisch verschiedene Fachkompetenzbereiche aufweisen, die innerhalb einer interdisziplinären Schülerbefragung inhaltlich nur schwer abzubilden sind. Allgemeine und übergreifende Themen, wie z. B. Überforderung oder Prüfungsangst, wurden jedoch mit berücksichtigt.\\

\noindent
Die Zielstellungen der eigenen Forschungsarbeit waren, insbesondere herauszuarbeiten, 
\begin{itemize}
	\item ob und inwieweit die der Fachliteratur entnommenen allgemeinen Problemlagen Jugendlicher auf den Bereich Gesundheit, Pflege und Soziales zutreffen.
	\item wie die Problemlagen im Einzelnen repräsentiert und verteilt sind.
	\item ob Schüler und Lehrer gleiche bzw. ähnliche Problemlagen wahrnehmen.
	\item inwieweit die Problemlagen das Unterrichtsgeschehen konkret beeinflussen.
	\item ob Bedarfe für  Unterstützungsangebote überhaupt vorhanden sind.
	\item wie mögliche Unterstützungsangebote konkret aussehen könnten. 
\end{itemize}

\noindent
Zur Beantwortung der Forschungsfragen und zur Realisierung der genannten Zielstellungen wurden eine quantitative Befragung an einer Stichprobe von 175 Schülern mittels eines Fragebogens und eine qualitative Studie in Form von fünf Lehrerinterviews durchgeführt. Diese Methoden wurden ausgewählt, da sie hinsichtlich der Betrachtung beider Perspektiven die größtmöglichen Erkenntnisgewinne versprachen. Mittels eines Fragebogens konnte in einem überschaubaren Zeitrahmen eine relativ große Anzahl von Probanden befragt werden, wobei durch eine geschickte Auswahl der Befragungsgruppe gleichzeitig alle Fachrichtungen am DRK Bildungswerk SN berücksichtigt werden konnten. Zusätzlich musste für die Befragung der Schüler eine relativ einfach auswertbare Methode gewählt werden. Der Fokus lag dabei auf der Eruierung der persönlichen Problemlagen und auf dem Erkenntnisgewinn hinsichtlich der Nutzung potentieller Unterstützungsangebote. Da aus der Lehrersicht durch die pädagogischen Fachkompetenzen tendenziell differenziertere und spezifischere Wahrnehmungen zu den Problemlagen und zu möglichen Unterstützungsangeboten zu erwarten waren, eignete sich eine Befragung mittels Fragebogen nicht, so dass das teilstandardisierte Leitfadeninterview als Alternative ausgewählt wurde. 

Nachdem in den vorangegangenen Ausführungen die Schulsozialarbeit bereits als theoretischer Bezugsrahmen herausgestellt wurde, mag es verwundern, dass der Begriff Schulsozialarbeit in den Forschungsfragen bzw. Zielstellungen überhaupt nicht auftaucht. Das ist zum einen, wie bereits oben ausgeführt, damit zu begründen, dass die Schulsozialarbeit sich in ihren theoretischen Grundlagen wenig bis gar nicht auf den berufsbildenden Bereich bezieht und damit zahlreiche Aspekte, wie einige Methoden oder Konzeptionen, für diese Schulart schlichtweg als unrelevant bezeichnet werden können. Zum anderen "`[\punkte] mangelt es dem Arbeitsfeld Schulsozialarbeit, sowohl an einem unumstrittenen Begriff als auch an einem relativ klaren inhaltlichen Verständnis."' \footcite[23]{Speck2007}. Erschwerend kommt hinzu, dass, bedingt durch die derzeitig gesetzliche Lage und Trägerstruktur der Schulsozialarbeit, freie Schulträger nicht nur aufgrund der Finanzierungsaspekte selten Schulsozialarbeit nach der theoretischen Definition anbieten \footcite[vgl.][116]{Stuewe2015}. Da in der vorbereitenden Beschäftigung mit der Thematik ersichtlich wurde, dass die Schüler und Lehrkräfte am DRK Bildungswerk SN mit dem Begriff Schulsozialarbeit wenig anfangen konnten und die Einführung von Schulsozialarbeit als Angebotsform fraglich war und ist, wurde die Entscheidung getroffen, den offeneren Begriff "`außerunterrichtliche Beratungs- und Unterstützungsangebote"' zu wählen. Trotzdem wird davon ausgegangen, dass die theoretischen Grundlagen der Schulsozialarbeit auch dafür anwendbar und in bestimmten Anteilen handlungsleitend sind.
 
Zur Bearbeitung der genannten Themen erfolgte eine ausführliche Literaturrecherche, die neben der Suche nach Literatur im WebOPAC der Sächsischen Landes- und Universitätsbibliothek Dresden (SLUB) auch eine umfangreiche internetbasierte Datenrecherche über das Datenbank-Infosystem (DBIS) der SLUB, den Deutschen Bildungsserver und in Fachdatenbanken, wie bspw. SOLIS -- Sozialwissenschaftliches Literaturinformationssystem, sowiport -- das Portal für Sozialwissenschaften und FIS-Bildung beinhaltete. Zusätzlich erfolgten persönliche Anfragen bei der Landesarbeitsgemeinschaft Schulsozialarbeit Sachsen e. V. sowie bei zuständigen Ämtern des Freistaates Sachsen. Bei der Recherchetätigkeit wurden hauptsächlich Schlagworte wie Problemlagen von Schülern, Schulsozialarbeit, soziale Arbeit an Schulen, Methoden der Sozialen Arbeit in der Schule, Forschung zur Schulsozialarbeit und andere, jeweils bezogen auf den allgemeinbildenden Bereich und die berufsbildende Schule, verwendet. Festzustellen war, dass die Literaturlage zu den genannten Stichworten sehr reichhaltig ist, so dass aktuelle Publikationen und Quellen der letzten zehn Jahre gesichtet und nach inhaltlicher Analyse verwendet werden konnten. Bestehen blieb jedoch das bereits beschriebene Problem der fehlenden Spezifika hinsichtlich der Schulart berufsbildende Schule. Insbesondere die Publikationen von Karsten Speck, Anke Spies und Nicole Pötter sind als besonders gewinnbringend zur Thematik einzuschätzen und wurden daher auch vorrangig, zumindest im Anteil der theoretischen Ausführungen, verwendet. 

Zum Aufbau der vorliegenden Arbeit ist auszuführen, dass grundsätzlich eine Einteilung in einen ersten theoretischen Anteil und einen zweiten praktischen Teil erfolgt; in diesem werden die selbst konzipierten und durchgeführten Studien bzw. Forschungsarbeiten ausführlich vorgestellt und ausgewertet. Der erste Teil beschäftigt sich hingegen zunächst mit wichtigen theoretischen Grundlagen der Schulsozialarbeit, wobei Definitionen, rechtliche Grundlagen, Träger, Zielgruppen, Ziele, Aufgabenfelder, Methoden und Konzeptionen in den Fokus genommen werden. Damit sollen, in relativ kurzen und komprimierten Ausführungen aktuelle Erkenntnisse zu diesem Handlungsfeld der sozialen Arbeit vorgestellt werden, die im Wesentlichen dazu dienen, mögliche Unterstützungsangebote theoretisch zu fundieren und zu begründen. Danach erfolgt der Versuch einer allgemeinen Bedarfsanalyse für Schulsozialarbeit bzw. für Unterstützungsangebote, wobei die heutigen spezifischen Problemlagen junger Menschen explizit betrachtet werden und insbesondere für den berufsbildenden Bereich Erkenntnisse gewonnen und vorgestellt werden. Dem schließt sich eine Analyse des aktuellen Standes der Schulsozialarbeit an beruflichen Schulen in Sachsen an, der als Standortbestimmung die bereits eingangs erwähnte Diskrepanz von Angebot und Nachfrage in diesem Bereich verdeutlichen soll und unter anderem versucht darzulegen, inwieweit Erkenntnisse für den Bereich der Schulen in freier Trägerschaft vorhanden sind, von denen in konzeptioneller Hinsicht möglicherweise profitiert werden könnte. Als Hinführung zum praktischen Teil ergänzen eine ausführliche Vorstellung des DRK Bildungswerkes SN als "`Forschungsschule"' und einige überleitende Gedanken und Ausführungen den theoretischen Part der Arbeit. Im zweiten Abschnitt, der überwiegend praktisch orientiert ist, werden zunächst die eigenen Forschungsvorhaben ausführlich vorgestellt und begründet sowie die Herangehensweise und Durchführung offengelegt. Die, mittels Schülerbefragung und Lehrerinterviews, gewonnenen Daten werden anschließend ausführlich ausgewertet und sowohl zunächst eigenständig, als auch in einem direkten Vergleich präsentiert. Die kritische Diskussion der durchgeführten Forschungsarbeiten rundet diesen Anteil der Arbeit ab und betrachtet mögliche Ressourcen und Potentiale. Aus den Ergebnissen der Forschungsarbeiten und den theoretischen Grundlagen werden nun erste mögliche konzeptionelle Ansätze für konkrete Unterstützungsangebote am DRK Bildungswerk Sachsen abgeleitet und vorgestellt. Abschließend erfolgt im letzten Teil der Arbeit eine Zusammenfassung und Bilanzierung der betrachteten Theoriegrundlagen und praktischen Ergebnisse, wobei weiterführende Forschungsbedarfe aufgezeigt werden. 

Es wird darauf hingewiesen, dass die Begriffe Berufsschule, berufliche Schule und berufsbildende Schule als synonyme Bezeichnungen für die Schulart verwendet wurden. Obwohl die genannten Begriffe bei genauer Betrachtung Differenzierungen aufweisen, die jedoch auch in der verwendeten Literatur nicht immer trennscharf berücksichtigt werden, ergeben sich in der vorliegenden Arbeit dadurch keine Zuordnungen zu beruflichen Fachrichtungen bzw. schulischen Unterformen wie Berufsfachschule oder Fachschule.
