\section{Fazit}
\label{sec:Fazit}

\begin{quotation}
\noindent
"`Sie nennen sie Schulen, fabulieren vom "`Haus des Lernens"' und tünchen damit die ganztägigen Verwahrungsanstalten für Kinder und Jugendliche, denen in unserer Gesellschaft ein Zuhause abhandenkam."'
\end{quotation}

\noindent
[\punkte] schreibt der Autor Raymond Walden in einer seiner Sammlungen von Sinnsprüchen \footcite[71]{Walden2005} und verdeutlicht damit eines der sich scheinbar verstärkenden Probleme unserer Gesellschaft. Sich verändernde und unsichere Lebensbedingungen, von Brüchen und Neuorientierungen gekennzeichnete Bildungs- und Arbeitsbiografien sowie anders und weniger direkt verlaufende Übergänge als noch vor einigen Jahrzehnten prägen heute den Charakter und die Aufgaben der Schule und der Jugendhilfe mit ihren Unterstützungsangeboten, bspw. in Form der Schulsozialarbeit \footcite[vgl.][9ff]{Bolder2010}. Aus der berufspädagogischen Perspektive heraus ist es schwer zu akzeptieren, unsere Schulen als "`Verwahrungsanstalten"' zu bezeichnen. Jedoch könnte eine solche Einschätzung durchaus getroffen werden, wenn Jugendliche mit ihren aktuellen Problemen lediglich "`beschult"' oder ausgebildet, nicht aber lebens- und alltagsorientiert unterstützt werden. Für diese erforderliche Hilfe gibt es offensichtlich jedoch leider keine Patentrezepte.
 
Zunächst einmal kann festgehalten werden, dass in der Institution berufsbildende Schule immer wieder die viel beschriebenen komplexen Problemlagen, welche die Schüler heute "`schwieriger"' machen und den Unterrichtsprozess nachhaltig beeinflussen, eine große Rolle spielen \footcite[vgl.][1]{UniversitaetLeipzig2007}. Diese theoretische Aussage hat sich durch die eigenen Forschungsarbeiten für die Fachbereiche Gesundheit, Pflege und Soziales am DRK Bildungswerk SN durchaus bestätigt. Die Lehrkräfte beklagten hier, ebenso wie anderswo auch, die seit einigen Jahren deutlich schwierigeren, leistungsschwächeren, demotivierten, problembehafteten und teilweise für die Berufe ungeeigneten Schüler. Die Substanz dieser Aussagen sollte im Rahmen eigener Studien untersucht werden. Erschwerend kam hinzu, dass Erhebungen zu den Problemlagen von Schülern für die Fachbereiche Gesundheit, Pflege und Sozialwesen bisher absolut fehlen und daher nur auf vorherige Forschungsergebnisse allgemeiner Art zurückgegriffen werden konnte. Zusammenfassend lässt sich titelgemäß feststellen: Ja, Schüler haben es heute schwer. Lehrer aber auch! Die Resultate der quantitativen Schülerbefragung sowie der qualitativen Lehrerinterviews sind in den entsprechenden Unterpunkten ausführlich vorgestellt und auch miteinander verglichen worden. Die Forschungsergebnisse zeigen mannigfaltige Problemlagen, die Schüler am DRK Bildungswerk SN für sich persönlich, wie auch für ihre Mitschüler wahrnehmen. Es scheinen also keinesfalls "`problemfreie"' Persönlichkeiten zu sein, die sich in das Berufsfeld mit hohen Anforderungen an Fach- und Sozialkompetenz hineinwagen. Die Lehrerinterviews bestätigten sowohl die Sicht auf die Problemlagen als auch die damit direkt im Zusammenhang stehenden, tendenziell negativen, Einflüsse auf das Unterrichtsgeschehen. Es ergeben sich daraus, sowohl aus der Schüler- als auch aus der Lehrerperspektive betrachtet, definitiv Unterstützungsbedarfe, die jedoch differenziert betrachtet werden müssen. Zum einen, weil unklar ist, wer überhaupt potentielle Unterstützungsangebote leisten könnte, zum anderen, weil die Art und Form der Angebote vorerst nur einen Ideencharakter aufzuweisen hat und konzeptionell ausgearbeitet werden muss. Denkbar wären dabei durchaus individuell ausgestaltete Formen von Berufsschulsozialarbeit, deren wichtigste theoretische Grundlagen im Rahmen der vorliegenden Arbeit aufgezeigt werden konnten. Zu berücksichtigen gilt es jedoch auch, dass 98 von 154 Schülern in der Befragung angaben, persönlich keine außerunterrichtlichen Beratungs- und Unterstützungsangebote im DRK Bildungswerk SN in Anspruch nehmen zu wollen. Die Annahme möglicher Projekte ist demzufolge als fraglich zu bezeichnen, eine Erprobungsphase könnte sich als zweckmäßig erweisen.
 
Überraschend erschienen, neben den sehr zahlreich wahrgenommenen und benannten eigenen und die Mitschüler betreffenden Problemlagen, die Rollenverständnisse der Lehrkräfte in Bezug auf Schülerprobleme. Die Ansicht Drillings, dass viele Lehrkräfte sich mit den teilweise komplexen Problemlagen überfordert fühlen \footcite[vgl.][10]{Drilling2004}, konnte exemplarisch für das DRK Bildungswerk SN so nicht nachgewiesen werden. Offensichtlich erkennen, zumindest die interviewten Lehrkräfte, die Unterstützung der Schüler als Teil ihrer beruflichen Aufgabe, wobei individuelle Grenzen durchaus gegeben sind. Möglicherweise könnte dies auch ein Ausdruck dessen sein, dass Lehrkräfte, ebenso wie Schüler, Berufsschulsozialarbeit oder andere Unterstützungsangebote selbst bisher nicht erlebt haben und auf keine entsprechenden Erfahrungswerte zurückgreifen können. 

Auf die konkrete Beantwortung der Forschungsfragen wurde bereits im Abschnitt \ref{sec:ZusammenfassungDerKategorien} und \ref{sec:GegenüberstellungDerErgebnisse} ausführlich eingegangen. Die Zielstellungen der eigenen Forschungsarbeit, welche in der Einleitung formuliert wurden, sind nach Ansicht der Verfasserinnen weitestgehend erreicht worden. Es gelang herauszuarbeiten, dass die der Fachliteratur entnommenen allgemeinen Problemlagen Jugendlicher auf den Bereich Gesundheit, Pflege und Soziales ebenfalls zutreffen. Hinsichtlich der einzelnen Problemlagen konnte dargestellt werden, in welcher Häufigkeit sie vorliegen, woraus sich Kernprobleme und weniger präsente Aspekte ableiten ließen. Ebenfalls gelang eine Einschätzung hinsichtlich der Wahrnehmung der Lehrkräfte, da sich die geschilderten Problemlagen mit denen der Schüler decken, jedoch personenbezogen unterschiedliche Gewichtungen und vordergründige Kernprobleme deutlich werden. Der Einfluss der Schülerprobleme auf das Unterrichtsgeschehen konnte nachvollziehbar durch die Lehrkräfte beschrieben und somit hinsichtlich der Ausbildungsbeeinflussung als hoch eingeschätzt werden. Abschließend zeigten sich durchaus Bedarfe für Unterstützungsangebote, wenngleich die Ergiebigkeit in Bezug auf verstellbare konkrete Angebote nicht so deutlich ausfiel, wie im Vorfeld angenommen. Viele Vorschläge der Lehrkräfte fokussierten immer wieder den fachlichen Bereich, nicht jedoch den persönlich-sozialen Lebensbereich der Schüler. Ein möglicher Erklärungsversuch durch mangelnde Erfahrungswerte der Lehrkräfte wurde oben bereits ausgeführt. 

Die positive Bilanz hinsichtlich der Forschungsergebnisse für das DRK Bildungswerk SN lässt sich jedoch nicht unbedingt auf weitere Betrachtungen innerhalb der vorliegenden Arbeit übertragen. Weitestgehend ungeklärt bleiben nach wie vor die Gründe der Diskrepanz von, aus den komplexen Problemlagen resultierenden, Unterstützungsbedarfen und tatsächlichen Angeboten im berufsbildenden Bereich allgemein. Dies konnte durch die Bestandsaufnahme der Schulsozialarbeit an öffentlichen berufsbildenden Schulen in Sachsen deutlich aufgezeigt werden. Möglicherweise sind die Begründungen in der geringen gesetzlichen Verankerung von Unterstützungsangeboten für Schüler im sächsischen Schulgesetz und in weiteren fehlenden Vorschriften zu suchen. Hier ist lediglich die sozialpädagogische Betreuung im Berufsvorbereitungsjahr vorgeschrieben, wozu aktuell ca. 15 Personalstellen zur Verfügung stehen. Diese derzeit praktizierte Einengung der sächsischen Schulsozialarbeit auf das Berufsvorbereitungsjahr zeigt sich als wenig praxisorientiert und erzeugt Ungerechtigkeiten gegenüber Jugendlichen mit Unterstützungsbedarf in anderen Bildungsgängen, welche durchaus ähnliche Problemlagen aufweisen. Vermutet werden kann jedoch auch, dass größtenteils Aspekte der Finanzierung dazu führen, dass einem großen Bedarf an Unterstützung nur wenige Angebote der Schulsozialarbeit gegenüberstehen. Die "`Freie Presse"' schrieb im Juni 2015, passend zu dieser Thematik, dass die Träger der sächsischen Schulsozialarbeit die derzeitige Förderpraxis als Desaster kritisieren. Entgegen aller politischen Willensbekundungen soll ab dem Schuljahr 2015/16 die Förderung im Programm "`Soziale Schule -- Kompetenzentwicklung für Schüler"', aus Mitteln des Europäischen Sozialfonds (ESF) und der Sächsische Aufbaubank (SAB), auf zehn Monate beschränkt werden und insgesamt weniger Geld bereitstehen. Die Folge davon sei, dass statt der bisher 19 in Projekten vertretenen Trägern nur noch fünf Projektanträge einreichen wollen \footcite[vgl.]{FreiePresse2015}. Selbstverständlich ist die Finanzierung möglicher Beratungs- und Unterstützungsangebote auch für Schulen in freier Trägerschaft eine immense Herausforderung, der sich auch das DRK Bildungswerk SN stellen muss, insofern tatsächlich praktische Angebote aus der vorliegenden Arbeit abgeleitet werden sollen. Ein weiterer Aspekt für die negative Bilanz von Bedarfen und Angeboten könnte zusätzlich darin begründet liegen, dass nicht alle Schulen, welche Unterstützungsleistungen bereithalten, diese unter dem Begriff Schulsozialarbeit anbieten. Diese Vermutung ergibt sich aus der im Gliederungspunkt \ref{sec:ErhebungenZurSchulsozialarbeitAnBerufsbildendenSchulen} bereits thematisierten Angabe von anderweitigen internen und externen Angeboten an insgesamt sechzehn berufsbildenden Schulen in Sachsen. Die Nichtberücksichtigung solcher anderweitigen und unbeschriebenen Projekte in Bestandserhebungen verzerrt hier möglicherweise die Analyse, wäre jedoch aufgrund der Umstrittenheit des Begriffes Schulsozialarbeit und seiner mangelnden inhaltlichen Klarheit durchaus verständlich \footcite[vgl.][23]{Speck2007}.

Aus der vorliegenden Arbeit ergeben sich, nicht zuletzt aufgrund der soeben geschilderten Aspekte, einige weitere Forschungsdesiderata. Dringend notwendig erscheint es, die bestehende Datenlage zur Schulsozialarbeit oder entsprechenden Projekten unbedingt auszudehnen und zu aktualisieren sowie dabei den berufsschulischen Bereich stärker als bisher in den Fokus zu nehmen. Zur Überprüfung und Erweiterung der hier vorgestellten Ergebnisse sind vergleichende Untersuchungen an mehreren Schulen notwendig, bei denen sowohl öffentliche Schulen als auch Schulen in freier Trägerschaft aus dem Berufsfeld Gesundheit, Pflege und Sozialwesen in den Blick genommen werden müssten. Zusätzlich könnte es gewinnbringend sein, Schulen mit bereits bestehender Berufsschulsozialarbeit oder anderweitigen Unterstützungsangeboten aus dem genannten Berufsfeld (soweit vorhanden) gezielt aufzusuchen und die praktische Umsetzung der Maßnahmen sowie die Annahme der Schüler unter wissenschaftlichen Gesichtspunkten zu prüfen. Die daraus möglicherweise abzuleitenden Forschungsergebnisse könnten sich für zahlreiche Schulen als Handlungsgrundlagen eignen und zur Aufhebung der Unterrepräsentiertheit der Fachrichtungen Gesundheit, Pflege und Sozialwesen in Bezug auf Evaluationsberichte oder Best Practise-Beispiele beitragen. Das gesamte Thema Schulsozialarbeit für den berufsbildenden Bereich stellt sich, wie an mehreren Stellen der Arbeit ausgeführt, ebenfalls als relevantes wissenschaftliches Forschungsfeld dar. Indem es gelänge, zielgruppenbezogene theoretische Fundierungen unter Berücksichtigung des berufsbildenden Bereiches und der Fachrichtungen mit ihren Spezifika herauszuarbeiten, wären möglicherweise auch die notwendigen Konzeptionen für konkrete Angebote erleichtert. Für Schulen, die mehr und mehr in den Zugzwang geraten, sich aufgrund der Problemlagen ihrer Schüler mit entsprechenden Unterstützungsleistungen auseinanderzusetzen, würde dies eine enorme Arbeitsunterstützung bieten. Letztlich ist noch zu erwähnen, dass berufspädagogisch tragfähige Konzepte zum Umgang mit der zunehmenden Heterogenität im genannten Berufsfeld anscheinend bisher fehlen und sich allein daraus enorme Forschungsbedarfe ableiten lassen, die zu einer Verbesserung der Situation in den berufsbildenden Schulen und vor allem zu einer Unterstützung der Lehrkräfte und Schüler in diesem Bereich entscheidend beitragen könnten \footcite[vgl.][23ff]{Grassi2012}.
 
Abschließend soll jedoch auch kritisch darauf hingewiesen werden, dass mögliche Formen der berufsschulischen Sozialarbeit oder anderweitiger Beratungs- und Unterstützungsangebote keinesfalls überzogenen Erwartungen ausgesetzt werden sollen bzw. als Allheilmittel zur Rettung junger Menschen und Problemlösung der Schule angesehen werden dürfen \footcite[vgl.][14]{Engelberg2011}.
\begin{quotation}
\noindent
"`Differenziert betrachtet ist Schulsozialarbeit heute ein wichtiges Instrument zur Förderung von Bildung und Erziehung -- insbesondere bezogen auf die Zielgruppe der benachteiligten Jugendlichen. [\punkte] Allerdings kann Schulsozialarbeit dies nicht allein. Benachteiligte Jugendliche müssen auch außerhalb der Schule gefördert werden."'
\end{quotation}
\noindent
[\punkte] stellen Engelberg und Schattmann dazu treffend fest \footcite[14]{Engelberg2011}. Untermauert wird dies durch die Aussage von Spies, die zu bedenken gibt, dass Unterstützungsangebote empirisch belegt, optimierende Potentiale aufweisen, und unzweifelhafte Berechtigung haben, insofern realistische Ziele formuliert werden \footcite[vgl.][16]{Spies2013}. Ferner sollte auch der präventive Ansatz nicht in Vergessenheit geraten und Beratungs- und Unterstützungsangebote nicht nur dazu vorgehalten werden, bestehende Problemlagen zu beheben \footcite[vgl.][14]{Engelberg2011}. 

Das DRK Bildungswerk SN stellt sich einer großen Herausforderung, sollten die Ergebnisse der vorliegenden Arbeit wirklich zum Anlass genommen werden, praktische Angebote für Schüler zu konzipieren. Viele Fragen sind dazu aufgeworfen worden und im Vorfeld noch zu bedenken. Gleichzeitig wurden jedoch auch konzeptionelle Vorschläge aus der Theorie und den Forschungsergebnissen abgeleitet, die eine Grundlage für Beratungs- und Unterstützungsangeboten darstellen könnten. 

\begin{quotation}
\noindent
"`Schulsozialarbeit reagiert, anders als Schule, flexibler und schneller auf die sich den Heranwachsenden darbietenden Lebensbedingungen, indem sie die lebensweltlichen alltäglichen privaten, sowie innerschulischen als auch außerschulischen Probleme der Kinder und Jugendlichen zum Gegenstand ihrer Arbeit macht."'
\end{quotation}

\noindent
[\punkte] formuliert Speck zutreffend zur Thematik \footcite[75]{Speck2007}. Diese Aussage könnte durchaus auf andere abgeleitete Angebote ausgeweitet werden und eine gute Handlungsmaxime für die zukünftige konzeptionelle Arbeit darstellen. Die Verfasserinnen sind gespannt, ob und inwieweit die hier vorgestellten Ergebnisse ihrer Forschungsarbeit wirklich genutzt und in weiterführende Überlegungen oder sogar konkrete Resultate überführt werden. 
