\section{Datenauswertung}
\label{sec:Datenauswertung}

\subsection{Auswertung der Schülerumfrage}
\label{sec:AuswertungDerSchülerumfrage}

Nach Fertigstellung des Umfragebogen überlegte man sich Wege diesen an die Schüler herangetragen. Zur Diskussion standen die Online-Umfrage und die klassische Umfrage auf Papier. Aufgrund des begrenzten Zeitfensters und negativer Erfahrungen im Vorfeld, in Bezug auf die aktive Teilnahme an Online-Umfragen am DRK Bildungswerk SN, wurde diese Idee zugunsten der Umfrage auf Papier entschieden. Für die ansprechende Gestaltung des Umfragebogens wurde LimeSurvey \footcite {LimeSurvey2015}, eine kostenlose Umfrageserviceplattform im Internet, genutzt. Diese Software dient primär der Erstellung von Online-Umfragen, ist relativ leicht zugänglich und ermöglicht zudem eine Form der Datenanalyse bzw. Statistikausgabe.

Demzufolge wurde die Umfrage neben der reinen Papierressource auch in Form einer Onlineressource erstellt. Die Ergebnisse der ausgefüllten Fragebögen wurden nach Beendigung der Schülerumfrage in die online erstellte Vorlage händig eingetragen. Im Anschluss wurde mit Hilfe der Software eine Ergebnisdarstellung vorgenommen welche im folgenden Punkt näher dargestellt wird.\\

\textbf{Ausschluss fehlerhaft ausgefüllter Fragebögen}\\
Trotz Pretest, intensiver Belehrung der Schüler vor Beginn der Umfrage sowie den Informationen auf dem gereichten Merkzettel, wurden 21 der 175 ausgefüllten Umfragebögen falsch oder unvollständig beantwortet. Das entspricht 12\% von 100\%.  Diese 21 Bögen wurden aussortiert und nicht für die Ergebnisdarstellung herangezogen um keine Antwort- bzw. Ergebnisverzerrung zu generieren. Bei den aussortierten Bögen wurden z.T. Fragen garnicht beantwortet oder es wurden Mehrfachnennungen vorgenommen obwohl nur eine Antwort erlaubt gewesen wäre. Alle Schüler wurden jedoch vor Beginn der Umfrage nachdrücklich auf den Umfang und die korrekte Beantwortung in Form von Einzel- oder Mehrfachnennungen hingewiesen. Mögliche Gründe für falsch ausgefüllte Bögen können nur erahnt werden, doch die Vermutungen reichen von Desinteresse, Unaufmerksamkeit über verbale Verständnisprobleme bis hin zu Leseschwierigkeiten. Diese Angaben sind jedoch rein hypothetisch und können nicht belegt werden. Sollte die Umfrage erneut durchgeführt werden, würden die Autorinnen verstärkter auf diese Fehlerquellen hinweisen. 

Aufgrund dessen basieren die Ergebnisse der Schülerumfrage lediglich auf \textbf{154} von \textbf{175} Fragebögen. Statt der angestrebten 30\% der Grundgesamtheit, konnten somit nur 26,64\% aller Schüler bei der Ergebnisdarstellung berücksichtigt werden. Die bei der Beschreibung der Stichprobenauswahl bezifferte Aussagesicherheit von 95\%und eine Stichprobenfehlerquote von 6,5\% kann somit nicht gehalten und muss bezgl. der Aussagesicherheit reduziert werden \ref{sec:DieUmfrage}. Dennoch kann mit 26,64\% der Grundgesamtheit immer noch ein relativ realistisches Bild der Problemlagen und Unterstützungsbedarfe von Berufsschülern in exemplarischer Weise aufgezeigt werden.

\subsection{Auswertung der Lehrerinterviews}
\label{sec:AuswertungDerLehrerinterviews}

Bevor die Lehrerinterviews ausgewertet werden konnten, mussten zunächst die fünf Tonbandaufnahmen nach dem Transkriptionssystem transkribiert werden. I.01-I.03 wurden im Rahmen der universitären Arbeitsgruppe transkribiert und ausgewertet, die verbleibenden Interviews von den Autorinnen. Die vollständigen Transkripte sind der beiliegenden CD zu entnehmen \ref{sec:Sonstiges}.

Die Datenauswertung erfolgte nach der Qualitativen Inhaltsanalyse nach Mayring.
Die Analyseeinheit ist folgendermaßen definiert:
Auswertungseinheit: chronologisch
Kontexteinheit: alle drei Interviews
Kodiereinheit: einzelne Wortgruppen
Selektionskriterien: alle Textstellen, die der Beantwortung der Forschungsfrage dienen, d.h. Problemlagen der Schüler, Einfluss der Problemlagen auf das Unterrichtsgeschehen, Erwartungen der Schüler an Lehrkraft und Umgang der Lehrkraft sowohl mit den Problemen als auch mit den Erwartungen, Angebote zur Unterstützung bzw. Entlastung der Lehrkraft.

In den Transkripten wurden alle relevanten Textstellen, die eine Aussage zur Fragestellung beinhalten, markiert. Dann bildeten wir aus diesen Textstellen Paraphrasen, welche im Anschluss generalisiert wurden. Mit dem nächsten Schritt, der ersten Reduktion, ordnete man alle inhaltlich zusammengehörenden Generalisierungen einem bestimmten Themenbereich, der Hauptkategorie, zu. In der zweiten Reduktion kam es zur Bildung von Unterkategorien, da sowohl mehrere Aussagen als auch ähnliche Aussagen zu einem Gegenstand vorhanden waren. Den Abschluss bildeten die Interpretation hinsichtlich der Forschungsfrage und grafischen Darstellung der Ergebnisse.
Da nur drei Interviews durchgeführt wurden, ist noch keine Sättigung an Daten erreicht worden. Die Validität der erzielten Ergebnisse muss deshalb kritisch betrachtet werden.
/TODO

