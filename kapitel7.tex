\section{Datenauswertung}
\label{sec:Datenauswertung}

\subsection{Auswertung der Schülerumfrage}
\label{sec:AuswertungDerSchülerumfrage}

Nach Fertigstellung des Umfragebogen überlegte man sich Wege diesen an die Schüler herangetragen. Zur Diskussion standen die Online-Umfrage und die klassische Umfrage auf Papier. Aufgrund des begrenzten Zeitfensters und negativer Erfahrungen im Vorfeld, in Bezug auf die aktive Teilnahme an Online-Umfragen am DRK Bildungswerk SN, wurde diese Idee zugunsten der Umfrage auf Papier entschieden. Für die ansprechende Gestaltung des Umfragebogens wurde LimeSurvey (siehe LimeSurvey \footcite {LimeSurvey2015}), eine kostenlose Umfrageserviceplattform im Internet, genutzt. Diese Software dient primär der Erstellung von Online-Umfragen, ist relativ leicht zugänglich und ermöglicht zudem eine Form der Datenanalyse bzw. Statistikausgabe.

Demzufolge wurde die Umfrage neben der reinen Papierressource auch in Form einer Onlineressource erstellt. Die Ergebnisse der ausgefüllten Fragebögen wurden nach Beendigung der Schülerumfrage in die online erstellte Vorlage händisch eingetragen. Im Anschluss wurde mit Hilfe der Software eine Ergebnisdarstellung vorgenommen welche im folgenden Punkt näher dargestellt wird.\\

\noindent
\textbf{Ausschluss fehlerhaft ausgefüllter Fragebögen}\\

\noindent
Trotz Pretest, intensiver Belehrung der Schüler vor Beginn der Umfrage sowie den Informationen auf dem gereichten Merkzettel, wurden 21 der 175 ausgefüllten Umfragebögen falsch oder unvollständig beantwortet. Das entspricht 12\% von 100\%. Diese 21 Bögen wurden aussortiert und nicht für die Ergebnisdarstellung herangezogen um keine Antwort- bzw. Ergebnisverzerrung zu generieren. Bei den aussortierten Bögen wurden z. T. Fragen nicht beantwortet oder es wurden Mehrfachnennungen vorgenommen obwohl nur eine Antwort erlaubt gewesen wäre. Alle Schüler wurden jedoch vor Beginn der Umfrage nachdrücklich auf den Umfang und die korrekte Beantwortung in Form von Einzel- oder Mehrfachnennungen hingewiesen. Mögliche Gründe für falsch ausgefüllte Bögen können nur erahnt werden, doch die Vermutungen reichen von Desinteresse, Unaufmerksamkeit über verbale Verständnisprobleme bis hin zu Leseschwierigkeiten. Diese Angaben sind jedoch rein hypothetisch und können nicht belegt werden. Sollte die Umfrage erneut durchgeführt werden, würden die Autorinnen verstärkter auf diese Fehlerquellen hinweisen. 

Aufgrund dessen basieren die Ergebnisse der Schülerumfrage lediglich auf \textbf{154} von \textbf{175} Fragebögen. Statt der angestrebten 30\% der Grundgesamtheit, konnten somit nur $\approx$ 26,64\% aller Schüler bei der Ergebnisdarstellung berücksichtigt werden. Dennoch kann mit $\approx$ 26,64\% der Grundgesamtheit immer noch ein recht realistisches Bild der Problemlagen und Unterstützungsbedarfe von Berufsschülern in exemplarischer Weise aufgezeigt werden.

\subsection{Auswertung der Lehrerinterviews}
\label{sec:AuswertungDerLehrerinterviews}

"`Die qualitative Analyse von Daten lässt sich grundsätzlich in drei Schritte gliedern: Erhebung [\punkte], Aufbereitung und Auswertung des Materials."' \footcite[135]{Krueger2014}. Dementsprechend erfolgte nach Beendigung der Interviews (Erhebung) die Transkription der ca. 3-stündigen Tonbandaufnahmen (Aufbereitung); diese können vollständig oder auch sequenziell angefertigt werden. Die Autorinnen entschieden sich für eine vollständige Transkription um jegliche inhaltliche Interpretation vorab zu vermeiden. Die Wortprotokollierung der Tonbandaufnahmen erfolgte unbereinigt, d. h. vorliegende Dialekte, Wiederholungen, Satzbaufehler und Pausen wurden erfasst um die Authentizität der jeweiligen Situation zu bewahren \footcite[vgl.][136]{Krueger2014}. Die 5 Interviews wurden anschließend eindeutig kenntlich gemacht (z. B. I.02 als Kennung für das zweite Interview) und mit Zeilennummern versehen (z. B. I.03 Z 115). So wird eine präzise Zuordnung von Inhalten bei Auswertung und Interpretation der Ergebnisse gewährleistet. Die vollständigen Transkriptionsprotokolle und Audiodateien der geführten 5 Interviews sind der beiliegenden CD zu entnehmen.

Die Datenauswertung, basierend auf den Interviewtranskripten, erfolgte mittels der Qualitativen Inhaltsanalyse nach Mayring. Es handelt sich dabei um eine systematische Methode zur Textanalyse, welche eine regelgeleitete Auswertung von Interviewdaten ermöglicht \footcite[vgl.][133]{Krueger2014}. Diese Methode wurde im Rahmen des Seminars "`Forschungsfelder"' von Fr. Thümmler besprochen und dementsprechend bei der Datenauswertung der Interviews angewandt. Aufgrund der Komplexität kann eine vollständige Darlegung dieses Auswertungsinstrumentes im Rahmen der vorliegenden Arbeit nicht erfolgen, soll jedoch grundlegend an folgenden Schritten veranschaulicht werden.

Zunächst wurden alle 5 Interviewtranskripte gemäß dem Erwartungshorizont analysiert. Hier ist anzumerken, das ein Teil der Ergebnisse (Interview 1 - 3) im Rahmen der universitären Gruppenarbeit erarbeitet wurde. Interview 4 - 5 wurden alleinig von den Autorinnen ausgewertet. Die zu erfassenden Inhalte wurden durch die konkreten  Fragestellungen im Vorfeld festgelegt.\\

\noindent
Die Analyseeinheit selbst wird wie folgt definiert:

\begin{itemize}
	\item \textbf{Auswertungseinheit:} chronologisch
	\item \textbf{Kontexteinheit:} alle 5 Interviews
	\item \textbf{Kodiereinheit:} einzelne Wortgruppen
	\item \textbf{Selektionskriterien:} alle Textstellen, die der Beantwortung der Forschungsfrage dienen, d.h. Problemlagen der Schüler, Einfluss der Problemlagen auf das Unterrichtsgeschehen, Erwartungen der Schüler an Lehrkraft und Umgang der Lehrkraft sowohl mit den Problemen als auch mit den Erwartungen, Angebote zur Unterstützung bzw. Entlastung der Lehrkraft
\end{itemize}

\noindent
In den Transkripten wurden alle relevanten Textstellen, die eine Aussage zur Fragestellung beinhalten, markiert. Aus den ausgewählten Textstellen wurden anschließend Paraphrasen gebildet. Hierbei kommt es zu einer Glättung der Äußerungen, wobei der Sprachstil des Probanden erhalten bleibt \footcite[vgl.][138]{Krueger2014}. Im Anschluss wurden diese Paraphrasen generalisiert, d. h. die Aussagen werden verallgemeinert formuliert, ohne sich jedoch zu sehr von der Originalaussage zu entfernen. Mit dem nächsten Schritt, der ersten Reduktion, ordnet man die Aussagen nach thematischen Sinneinheiten, d. h. es wird eine Bündelung bedeutungsgleicher oder --ähnlicher Aussagen vorgenommen \footcite[vgl.][139]{Krueger2014}. Diese wurden so einem bestimmten Themenbereich in Form der Hauptkategorie zugeordnet. In der zweiten Reduktion wurden nun Unterkategorien gebildet, wenn sowohl mehrere Aussagen als auch ähnliche Aussagen zu einem Gegenstand vorhanden waren (alle benannten Zwischenschritte von der Paraphrasierung bis hin zur Reduktion 2 können dem Anhang entnommen werden (siehe Punkt \ref{sec:Ergebnisauswertung}).

Im Gegensatz zu den Interviews I.01/I.03 und I.05 haben die verbleibenden Gespräche nur in geringem Umfang neue Daten zur Beantwortung der Forschungsfrage generiert. Aufgrund der geringen Probandenzahl ist die Untersuchung nicht repräsentativ, liefert jedoch wertvolle Informationen für das DRK Bildungswerk SN. Parallelen hinsichtlich der Ergebnisse sind jedoch auch für andere Schulen denkbar.
