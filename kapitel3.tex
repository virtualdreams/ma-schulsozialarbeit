\section{Bedarfe für Schulsozialarbeit}
\label{sec:BedarfeFürSchulsozialarbeit}

\subsection{Allgemeine Einführung}
\label{sec:AllgemeineEinführung}

In der Literatur zur Thematik Schulsozialarbeit finden sich umfassende Ausführungen zu deren Notwendigkeit und Bedarf, jedoch meist allgemein formuliert und wenig schulartspezifisch. "`Der umfassende gesellschaftliche Wandel der vergangenen Jahrzehnte, der zu einer Veränderung traditioneller Orientierungs- und Lebensmuster geführt hat und auch die Welt der Kinder- und Jugendlichen nachhaltig veränderte, stellt Jugendhilfe und Schule vor neue Herausforderungen, die sie nur durch eine gemeinsame Gestaltung des Lebens und Lernens bewältigen können"' schreibt dazu beispielsweise das Sächsische Staatsministerium für Soziales in seiner vorliegenden Handreichung zur Schulsozialarbeit in Sachsen \footcite[13]{SMSSS2009}. Der Bedarf an unterstützender Schulsozialarbeit scheint dabei vorrangig aus den vielfältigen und vielfach diskutierten Problemlagen junger Menschen in der heutigen Zeit zu resultieren, von denen fehlende Schulabschlüsse, Schwierigkeiten beim Übergang in die berufliche Ausbildung, Drogenprobleme, Kriminalität, Gewalt sowie Schulvermeidung nur einige sind. So konstatiert u.a. dos Santos-Stubbe zutreffend, dass "`[d]ie Biografie von Kindern und Jugendlichen gegenwärtig geprägt ist durch zahlreiche Umbrüche, die eine tiefe Bedeutung besitzen."' \footcite[68]{dosSantos-Stubbe2009} Dadurch werden Jugendhilfe und Schule mit Problemlagen konfrontiert, die im Kontext gesellschaftlicher Veränderungen stehen und die Lebenswelten von Kindern und Jugendlichen sowie deren Familien betreffen. Dies sind insbesondere Sozialisationsdefizite der Familie, erhöhte Leistungsanforderungen der Schule und an die Schule, ein erhöhter Wettbewerbsdruck bei einerseits schwachen Schülerinnen und Schülern angesichts fehlender Ausbildungsplätze sowie drohender Arbeitslosigkeit und andererseits bei starken Schülerinnen und Schülern angesichts großer Marktchancen, Schwierigkeiten beim Übergang in Ausbildung und Arbeit, Schulvermeidung in ihren unterschiedlichen Formen sowie die Belastung des Klimas an vielen Schulen durch zunehmendes delinquentes und deviantes Verhalten der Schülerschaft \footcite[vgl.][17]{SMSSS2009}.

Viele Lehrkräfte sehen sich mit den teilweise komplexen Problemlagen der Schülerinnen und Schüler überfordert, bemängeln die unzureichende Unterstützung und fordern eine Rückbesinnung auf ihr Kerngeschäft -- nämlich den Unterricht \footcite[vgl.][10]{Drilling2009}. Dennoch kann und darf sich Schule heute nicht darauf beschränken, mit Bildungsangeboten auf die Anforderungen des Lebens vorzubereiten, sondern muss aufgrund der oben ausgeführten, teilweise komplexen Problemlagen der Schülerschaft ihren Beitrag zur Bewältigung aller Lebensbereiche leisten. Diesen Anforderungen können Lehrkräfte jedoch nur in begrenztem Maße gerecht werden, zumindest dann, wenn der Unterricht und damit die fachliche Vorbereitung auf Teilaspekte des Lebens, nicht ständig in den Hintergrund gerückt werden soll. Hier kann nur durch die dauerhafte und klar geregelte Kooperation von Lehrerschaft und Schulsozialarbeitern oder Schulsozialpädagogen eine befriedigende Situation für alle beteiligten Akteure gewährleistet werden \footcite[vgl.][9ff]{Drilling2009}. 

\subsection{Berufsbildende Schulen}
\label{sec:BerufsbildendeSchulen}
   
Zuerst einmal ist jeglichen Ausführungen vorauszuschicken, dass das Feld der Berufsbildung ein so heterogenes ist, dass generelle Aussagen zu diesem wohl genauso schwierig sind wie zur Schulsozialarbeit in dieser Schulart. Möglicherweise tragen diese, noch näher auszuführenden, Strukturmerkmale dazu bei, dass Schulsozialarbeit als wenig existent erscheint, aber vielleicht doch -- zumindest in Ansätzen - häufiger vorhanden ist, als vordergründig anzunehmen. Einige Begründungen für diese Annahme der Verfasserinnen dieser Arbeit werden in den nachfolgenden Ausführungen an entsprechend passender Stelle gegeben. 

Zunächst lohnt sich eine Beantwortung der Frage, was eigentlich die Heterogenität des Feldes der Berufsbildung ausmacht. Da wiederum bundesländerspezifisch eine vielfältige Anzahl unterschiedlichster Regelungen zur Berufsausbildung existiert, die sich in mannigfaltigen Schularten und Bezeichnungen niederschlagen, wird das Bundesland Sachsen zur näheren Betrachtung ausgewählt. Zu dieser können verschiedene Aspekte herangezogen werden, der Versuch einer überblicksartigen Systematisierung erfolgt in den folgenden Punkten:\\

\textbf{Schularten}\\
Die berufsbildenden Schularten in Sachsen sind die Berufsschulen, Berufsfachschulen, Fachoberschulen, Fachschulen und beruflichen Gymnasien. Diese sind zumeist in Beruflichen Schulzentren zusammengefasst, zumindest soweit sie sich in öffentlicher Trägerschaft befinden. In allen Schularten können berufsbildende Förderschulen eingerichtet werden.

Die Berufsschule wird von Schülern besucht, die eine duale Berufsausbildung in einem der mehr als 360 anerkannten Ausbildungsberufe absolvieren und sich dazu mit einem Arbeitgeber in einem Ausbildungsverhältnis befinden. Sie enthält auch Angebote für behinderte oder benachteiligte Jugendliche. Die Berufsschulzeit dauert in der Regel drei Jahre. 

Berufsfachschulen führen zu einem bundeseinheitlich anerkannten Berufsabschluss und sind häufig im medizinisch-pflegerischen und sozialen Bereich (z.B. Altenpfleger,  Gesundheits- und Krankenpfleger, Notfallsanitäter, Physiotherapeuten, Sozialassistenten) vorzufinden. Die Ausbildung dauert in der Regel 2-3 Jahre. An Berufsfachschulen werden derzeit etwa 40 Bildungsgänge angeboten, welche meist in vollzeitschulischen Formen zu einem Berufsabschluss führen. Das bedeutet, dass sich die Schüler an Berufsfachschulen in der Regel nicht in einem Ausbildungsverhältnis mit einem Arbeitgeber befinden, sondern praktische Anteile der beruflichen Handlungskompetenz durch Praktika, zumeist in verschiedenen Einrichtungen des Berufsfeldes, erworben werden. Ausnahmen bilden jedoch einige Berufe im medizinisch-pflegerischen Bereich, wie z. B. Altenpfleger, Gesundheits- und Krankenpfleger und Notfallsanitäter. In diesen Ausbildungsberufen befinden sich Berufsfachschüler auch in einem Angestelltenverhältnis mit einem Arbeitgeber.
 
An der Fachoberschule können Jugendliche und Erwachsene die Fachhochschulreife erlangen. Die Ausbildung dauert für Schüler mit Realschulabschluss zwei Jahre, für Schüler mit abgeschlossener Berufsausbildung ein Jahr. 
Fachschulen sind Einrichtungen der beruflichen Weiterbildung. Sie bieten Fachkräften mit bereits abgeschlossener Berufsausbildung und beruflichen Erfahrungen länderspezifische Abschlüsse, die sie für Tätigkeiten im mittleren Funktionsbereich zwischen Facharbeitern bzw. Fachangestellten und Hochschulabsolventen befähigen. Fachschulen finden sich in verschiedenen beruflichen Feldern, im Bereich Sozialwesen werden beispielsweise Erzieher und Heilerziehungspfleger in dieser Schulart ausgebildet. 
 
Schüler mit Realschulabschluss und guten Leistungen können am beruflichen Gymnasium in drei Jahren die allgemeine Hochschulreife erlangen, die zum Studium an allen Hochschulen berechtigt. Sie erhalten neben allgemein bildendem auch berufsbezogenen Unterricht, der sie an die Berufswelt heranführt \footcite[vgl.]{SBSBSSSK2015} \footcite[vgl.][4ff]{SMKSK2013}.


\textbf{Berufsvorbereitende Maßnahmen}\\
Jugendliche, die nach erfolgreichem Abschluss der Oberschule keinen betrieblichen Ausbildungsplatz erhalten oder die Oberschule ohne Hauptschulabschluss beendet haben, können sich an der Berufsschule in einem Berufsgrundbildungsjahr (BGJ) auf die Aufnahme eines Berufsausbildungsverhältnisses oder eine Berufstätigkeit vorbereiten. Sie können eine berufliche Grundbildung in verschiedenen Berufsbereichen erhalten. Damit wird die Berufsschulpflicht erfüllt. Der erfolgreiche Abschluss des BGJ kann als erstes Ausbildungsjahr auf eine nachfolgende Berufsausbildung angerechnet werden.

Das Berufsvorbereitungsjahr (BVJ) hat die Aufgabe, Jugendliche bei der Berufswahl zu unterstützen und auf die Aufnahme einer Berufsausbildung vorzubereiten. Schüler des BVJ erwerben eine berufliche Orientierung in zwei Berufsbereichen (z.B. Holztechnik und Metalltechnik). Bei erfolgreichem Abschluss wird der Hauptschulabschluss zuerkannt.

In Vorbereitungsklassen mit berufspraktischen Aspekten werden Jugendliche und junge Erwachsene mit Migrationshintergrund auf die Aufnahme einer Berufsausbildung oder den Erwerb eines höheren Bildungsabschlusses (z.B. am Beruflichen Gymnasium oder der Fachoberschule) sprachlich vorbereitet. Im Rahmen der Vorbereitungsklasse nehmen die Schüler entsprechend der individuell angestrebten künftigen beruflichen Ausbildung für zwei Monate am Regelunterricht einer berufsbildenden Schule teil \footcite[vgl.][15ff]{SMKSK2013}.


\textbf{Zugänge zu beruflicher Bildung}\\
Ebenso vielfältig wie die einzelnen Schularten sind die Zugangsvoraussetzungen zu diesen. In die Berufsschulen kann aufgenommen werden kann, wer die Vollzeitschulpflicht erfüllt hat, noch berufsschulpflichtig ist und einen Ausbildungsvertrag abgeschlossen hat. Ein bestimmter Schulabschluss wird nicht vorausgesetzt.

Die Ausbildung an den Berufsfachschulen erfordert in der Regel den Realschulabschluss oder einen gleichwertigen Abschluss. Im Bereich Gesundheit und Pflege ist die gesundheitliche Eignung zwingend erforderlich. Bei den Bildungsgängen in der Alten- und Krankenpflege sowie bei der Ausbildung von Hebammen und Notfallsanitätern ist der Nachweis eines Ausbildungsvertrages erforderlich. Jedoch gibt es auch in dieser Schulart Ausnahmeregelungen, da der Zugang zu einigen Berufen auch mit Hauptschulabschluss möglich ist. Ein Beispiel hierfür ist die Krankenpflegehilfe.
 
An den Fachschulen gibt es für die einzelnen Fachbereiche unterschiedliche Aufnahmevoraussetzungen. In der Regel sind eine abgeschlossene Berufsausbildung und der Nachweis einer beruflichen Tätigkeit notwendig \footcite[vgl.][15ff]{SMKSK2013}.



