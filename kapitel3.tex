\section{Bedarfe für Schulsozialarbeit}
\label{sec:BedarfeFürSchulsozialarbeit}

\subsection{Allgemeine Einführung}
\label{sec:AllgemeineEinführung}

In der Literatur zur Thematik Schulsozialarbeit finden sich umfassende Ausführungen zu deren Notwendigkeit und Bedarf, jedoch meist allgemein formuliert und wenig schulartspezifisch. "`Der umfassende gesellschaftliche Wandel der vergangenen Jahrzehnte, der zu einer Veränderung traditioneller Orientierungs- und Lebensmuster geführt hat und auch die Welt der Kinder- und Jugendlichen nachhaltig veränderte, stellt Jugendhilfe und Schule vor neue Herausforderungen, die sie nur durch eine gemeinsame Gestaltung des Lebens und Lernens bewältigen können"' schreibt dazu beispielsweise das Sächsische Staatsministerium für Soziales in seiner vorliegenden Handreichung zur Schulsozialarbeit in Sachsen \footcite[13]{SMSSS2009}. Der Bedarf an unterstützender Schulsozialarbeit scheint dabei vorrangig aus den vielfältigen und vielfach diskutierten Problemlagen junger Menschen in der heutigen Zeit zu resultieren, von denen fehlende Schulabschlüsse, Schwierigkeiten beim Übergang in die berufliche Ausbildung, Drogenprobleme, Kriminalität, Gewalt sowie Schulvermeidung nur einige sind. So konstatiert u.a. dos Santos-Stubbe zutreffend, dass "`[d]ie Biografie von Kindern und Jugendlichen gegenwärtig geprägt ist durch zahlreiche Umbrüche, die eine tiefe Bedeutung besitzen."' \footcite[68]{dosSantos-Stubbe2009} Dadurch werden Jugendhilfe und Schule mit Problemlagen konfrontiert, die im Kontext gesellschaftlicher Veränderungen stehen und die Lebenswelten von Kindern und Jugendlichen sowie deren Familien betreffen. Dies sind insbesondere Sozialisationsdefizite der Familie, erhöhte Leistungsanforderungen der Schule und an die Schule, ein erhöhter Wettbewerbsdruck bei einerseits schwachen Schülerinnen und Schülern angesichts fehlender Ausbildungsplätze sowie drohender Arbeitslosigkeit und andererseits bei starken Schülerinnen und Schülern angesichts großer Marktchancen, Schwierigkeiten beim Übergang in Ausbildung und Arbeit, Schulvermeidung in ihren unterschiedlichen Formen sowie die Belastung des Klimas an vielen Schulen durch zunehmendes delinquentes und deviantes Verhalten der Schülerschaft \footcite[vgl.][17]{SMSSS2009}.

Viele Lehrkräfte sehen sich mit den teilweise komplexen Problemlagen der Schülerinnen und Schüler überfordert, bemängeln die unzureichende Unterstützung und fordern eine Rückbesinnung auf ihr Kerngeschäft -- nämlich den Unterricht \footcite[vgl.][10]{Drilling2009}. Dennoch kann und darf sich Schule heute nicht darauf beschränken, mit Bildungsangeboten auf die Anforderungen des Lebens vorzubereiten, sondern muss aufgrund der oben ausgeführten, teilweise komplexen Problemlagen der Schülerschaft ihren Beitrag zur Bewältigung aller Lebensbereiche leisten. Diesen Anforderungen können Lehrkräfte jedoch nur in begrenztem Maße gerecht werden, zumindest dann, wenn der Unterricht und damit die fachliche Vorbereitung auf Teilaspekte des Lebens, nicht ständig in den Hintergrund gerückt werden soll. Hier kann nur durch die dauerhafte und klar geregelte Kooperation von Lehrerschaft und Schulsozialarbeitern oder Schulsozialpädagogen eine befriedigende Situation für alle beteiligten Akteure gewährleistet werden \footcite[vgl.][9ff]{Drilling2009}. 

\subsection{Berufsbildende Schulen}
\label{sec:BerufsbildendeSchulen}

Zuerst einmal ist jeglichen Ausführungen vorauszuschicken, dass das Feld der Berufsbildung ein so heterogenes ist, dass generelle Aussagen zu diesem wohl genauso schwierig sind wie zur Schulsozialarbeit in dieser Schulart. Möglicherweise tragen diese, noch näher auszuführenden, Strukturmerkmale dazu bei, dass Schulsozialarbeit als wenig existent erscheint, aber vielleicht doch -- zumindest in Ansätzen -- häufiger vorhanden ist, als vordergründig anzunehmen. Einige Begründungen für diese Annahme der Verfasserinnen dieser Arbeit werden in den nachfolgenden Ausführungen an entsprechend passender Stelle gegeben. 

Zunächst lohnt sich eine Beantwortung der Frage, was eigentlich die Heterogenität des Feldes der Berufsbildung ausmacht. Da wiederum bundesländerspezifisch eine vielfältige Anzahl unterschiedlichster Regelungen zur Berufsausbildung existiert, die sich in mannigfaltigen Schularten und Bezeichnungen niederschlagen, wird das Bundesland Sachsen zur näheren Betrachtung ausgewählt. Zu dieser können verschiedene Aspekte herangezogen werden, der Versuch einer überblicksartigen Systematisierung erfolgt in den folgenden Punkten.

\subsubsection{Schularten}
\label{sec:Schularten}

Die berufsbildenden Schularten in Sachsen sind die Berufsschulen, Berufsfachschulen, Fachoberschulen, Fachschulen und beruflichen Gymnasien. Diese sind zumeist in Beruflichen Schulzentren zusammengefasst, zumindest soweit sie sich in öffentlicher Trägerschaft befinden. In allen Schularten können berufsbildende Förderschulen eingerichtet werden.

Die Berufsschule wird von Schülern besucht, die eine duale Berufsausbildung in einem der mehr als 360 anerkannten Ausbildungsberufe absolvieren und sich dazu mit einem Arbeitgeber in einem Ausbildungsverhältnis befinden. Sie enthält auch Angebote für behinderte oder benachteiligte Jugendliche. Die Berufsschulzeit dauert in der Regel drei Jahre. 

Berufsfachschulen führen zu einem bundeseinheitlich anerkannten Berufsabschluss und sind häufig im medizinisch-pflegerischen und sozialen Bereich (z. B. Altenpfleger, Gesund- heits- und Krankenpfleger, Notfallsanitäter, Physiotherapeuten, Sozialassistenten) vorzufinden. Die Ausbildung dauert in der Regel 2 -- 3 Jahre. An Berufsfachschulen werden derzeit etwa 40 Bildungsgänge angeboten, welche meist in vollzeitschulischen Formen zu einem Berufsabschluss führen. Das bedeutet, dass sich die Schüler an Berufsfachschulen in der Regel nicht in einem Ausbildungsverhältnis mit einem Arbeitgeber befinden, sondern praktische Anteile der beruflichen Handlungskompetenz durch Praktika, zumeist in verschiedenen Einrichtungen des Berufsfeldes, erworben werden. Ausnahmen bilden jedoch einige Berufe im medizinisch-pflegerischen Bereich, wie z. B. Altenpfleger, Gesundheits- und Krankenpfleger und Notfallsanitäter. In diesen Ausbildungsberufen befinden sich Berufsfachschüler auch in einem Angestelltenverhältnis mit einem Arbeitgeber.
 
An der Fachoberschule können Jugendliche und Erwachsene die Fachhochschulreife erlangen. Die Ausbildung dauert für Schüler mit Realschulabschluss zwei Jahre, für Schüler mit abgeschlossener Berufsausbildung ein Jahr. 
Fachschulen sind Einrichtungen der beruflichen Weiterbildung. Sie bieten Fachkräften mit bereits abgeschlossener Berufsausbildung und beruflichen Erfahrungen länderspezifische Abschlüsse, die sie für Tätigkeiten im mittleren Funktionsbereich zwischen Facharbeitern bzw. Fachangestellten und Hochschulabsolventen befähigen. Fachschulen finden sich in verschiedenen beruflichen Feldern, im Bereich Sozialwesen werden beispielsweise Erzieher und Heilerziehungspfleger in dieser Schulart ausgebildet. 

Schüler mit Realschulabschluss und guten Leistungen können am beruflichen Gymnasium in drei Jahren die allgemeine Hochschulreife erlangen, die zum Studium an allen Hochschulen berechtigt. Sie erhalten neben allgemein bildendem auch berufsbezogenen Unterricht, der sie an die Berufswelt heranführt \footcites[vgl.]{SBSBSSSK2015}[vgl.][4ff]{SMKSK2013}.

\subsubsection{Berufsvorbereitende Maßnahmen}
\label{sec:BerufsvorbereitendeMassnahmen}

Jugendliche, die nach erfolgreichem Abschluss der Oberschule keinen betrieblichen Ausbildungsplatz erhalten oder die Oberschule ohne Hauptschulabschluss beendet haben, können sich an der Berufsschule in einem Berufsgrundbildungsjahr (BGJ) auf die Aufnahme eines Berufsausbildungsverhältnisses oder eine Berufstätigkeit vorbereiten. Sie können eine berufliche Grundbildung in verschiedenen Berufsbereichen erhalten. Damit wird die Berufsschulpflicht erfüllt. Der erfolgreiche Abschluss des BGJ kann als erstes Ausbildungsjahr auf eine nachfolgende Berufsausbildung angerechnet werden.

Das Berufsvorbereitungsjahr (BVJ) hat die Aufgabe, Jugendliche bei der Berufswahl zu unterstützen und auf die Aufnahme einer Berufsausbildung vorzubereiten. Schüler des BVJ erwerben eine berufliche Orientierung in zwei Berufsbereichen (z. B. Holztechnik und Metalltechnik). Bei erfolgreichem Abschluss wird der Hauptschulabschluss zuerkannt.

In Vorbereitungsklassen mit berufspraktischen Aspekten werden Jugendliche und junge Erwachsene mit Migrationshintergrund auf die Aufnahme einer Berufsausbildung oder den Erwerb eines höheren Bildungsabschlusses (z. B. am Beruflichen Gymnasium oder der Fachoberschule) sprachlich vorbereitet. Im Rahmen der Vorbereitungsklasse nehmen die Schüler entsprechend der individuell angestrebten künftigen beruflichen Ausbildung für zwei Monate am Regelunterricht einer berufsbildenden Schule teil \footcite[vgl.][15ff]{SMKSK2013}.

\subsubsection{Zugänge zu beruflicher Bildung}
\label{sec:ZugängeZuBeruflicherBildung}

Ebenso vielfältig wie die einzelnen Schularten sind die Zugangsvoraussetzungen zu diesen. In die Berufsschulen kann aufgenommen werden kann, wer die Vollzeitschulpflicht erfüllt hat, noch berufsschulpflichtig ist und einen Ausbildungsvertrag abgeschlossen hat. Ein bestimmter Schulabschluss wird nicht vorausgesetzt.

Die Ausbildung an den Berufsfachschulen erfordert in der Regel den Realschulabschluss oder einen gleichwertigen Abschluss. Im Bereich Gesundheit und Pflege ist die gesundheitliche Eignung zwingend erforderlich. Bei den Bildungsgängen in der Alten- und Krankenpflege sowie bei der Ausbildung von Hebammen und Notfallsanitätern ist der Nachweis eines Ausbildungsvertrages erforderlich. Jedoch gibt es auch in dieser Schulart Ausnahmeregelungen, da der Zugang zu einigen Berufen auch mit Hauptschulabschluss möglich ist. Ein Beispiel hierfür ist die Krankenpflegehilfe.
 
An den Fachschulen gibt es für die einzelnen Fachbereiche unterschiedliche Aufnahmevoraussetzungen. In der Regel sind eine abgeschlossene Berufsausbildung und der Nachweis einer beruflichen Tätigkeit notwendig \footcite[vgl.][15ff]{SMKSK2013}.

\subsubsection{Schulträger}
\label{sec:Schulträger}

Neben den staatlichen Schulen gibt es in allen Schularten als Ergänzung der sächsischen Bildungslandschaft auch Schulen in freier Trägerschaft, zum Beispiel von privaten oder kirchlichen Organisationen, Vereinen, Gesellschaften oder Privatpersonen. Sie sind für die Schulgestaltung verantwortlich und können insbesondere über eine spezielle pädagogische, religiöse oder weltanschauliche Prägung entscheiden. Die Träger können außerdem Lehr- und Unterrichtsmethoden sowie Lehrinhalte und die Organisation des Unterrichts auch abweichend von den Vorschriften für die öffentlichen Schulen festlegen. Bei den Schulen in freier Trägerschaft werden Ersatz- und Ergänzungsschulen unterschieden. Die aktuellen gesetzlichen Regelungen zu freien Schulen sind im Sächsischen Gesetz über Schulen in freier Trägerschaft (SächsFrTrSchulG) vom 8. Juli 2015 festgeschrieben.

Ersatzschulen sind Schulen in freier Trägerschaft, die als Ersatz für eine im Freistaat Sachsen vorhandene oder grundsätzlich vorgesehene öffentliche Schule dienen. Die Ersatzschule darf in ihren wesentlichen Merkmalen nicht hinter einer öffentlichen Schule zurückstehen. An Ersatzschulen muss das gleiche Bildungsniveau erreicht werden wie an entsprechenden öffentlichen Schulen. Ersatzschulen verwenden daher in der Regel die sächsischen Lehrpläne. Die Ersatzschulen können ein Schulgeld erheben und werden bei Vorliegen der Voraussetzungen nach Ablauf der vierjährigen Wartefrist durch den Freistaat finanziell unterstützt. 

Ergänzungsschulen sind Schulen in freier Trägerschaft, die nicht als Ersatz für öffentliche Schulen dienen. Ergänzungsschulen haben hinsichtlich ihrer Organisation und ihres Bildungsangebots einen schulischen Charakter, sind aber mit keiner Schulart des öffentlichen Schulwesens vergleichbar und stehen damit außerhalb des Schulaufbaus in Sachsen. An ihnen muss das Bildungsniveau einer vergleichbaren öffentlichen Schule nicht erreicht werden. Ergänzungsschulen verwenden daher in der Regel auch keine sächsischen Lehrpläne. Die vergebenen Abschlüsse entsprechen damit auch nicht den staatlichen Abschlüssen, die an öffentlichen Schulen oder Ersatzschulen vergeben werden. Ergänzungsschulen sind demzufolge nicht berechtigt Zeugnisse auszustellen. Schüler einer Ergänzungsschule erhalten zum Schluss ihrer Ausbildung an der Ergänzungsschule eine Bescheinigung über den Schulbesuch oder ein Zertifikat. Ersatzschulen erhalten keine finanzielle Unterstützung durch den Freistaat, die Höhe des Schulgeldes ist daher nicht begrenzt \footcite[vgl.]{SMKSK2015a}. 

Die soeben dargelegten Informationen haben die Heterogenität der berufsbildenden Schulen in Sachsen nur in einigen wichtigen Punkten vorgestellt. Übertragbar sind diese Faktoren auf andere deutsche Bundesländer dahingehend, dass ähnlich vielfältige Strukturmerkmale vorliegen. Deutlich werden sollte, dass es "`die berufsbildende Schule"', also ein einheitliches Konstrukt, bezogen auf die Voraussetzungen hinsichtlich der schulischen und beruflichen Vorbildung oder auch das Alter der Schüler gar nicht gibt. Es ist durchaus vorstellbar, dass an beruflichen Schulzentren und auch an freien Schulen Schüler ohne bisherigen Schulabschluss gemeinsam mit denen, die einen (Haupt-) Schulabschluss nachweisen, aber noch keinen Ausbildungsplatz gefunden haben, eine berufsvorbereitende Maßnahme absolvieren. In anderen Klassen können Personen vom Hauptschulabsolventen, über den Schüler mit Realschulabschluss oder Fachhochschulreife bis hin zu einem Erwachsenen in der dritten Berufsausbildung in allen möglichen Konstellationen aufeinandertreffen. Demzufolge sind Klassen an berufsbildenden Schulen in allen Bereichen oft in ihrer Altersstruktur, in der Struktur der allgemeinen und beruflichen Vorbildung und auch in der Sozialstruktur sehr heterogen, was letztendlich ganz andere pädagogische und zwischenmenschliche Herausforderungen sowie Probleme mit sich bringt als im allgemeinbildenden Bereich. Diese Gegebenheiten wirken selbstverständlich auch direkt in den Bereich der Schulsozialarbeit innerhalb der Berufsbildung hinein und sind deshalb auf keinen Fall zu vernachlässigen.

\subsection{Problemlagen von Schülern an berufsbildenden Schulen}
\label{sec:ProblemlagenVonSchülernAnBerufsbildendenSchulen}

In der gesichteten Literatur zum Thema Schulsozialarbeit im Kontext der Berufsbildung wird das Thema Berufsorientierung und berufsbezogene Jugendbildung meist als ein wichtiges Aufgabenfeld der sozialpädagogischen Jugendhilfe angesehen und ausgeführt. Dadurch entsteht oft der Eindruck, dass mit der Entscheidung für einen Beruf bzw. eine berufliche Fachrichtung der Auftrag erledigt wäre. In kaum einer Publikation finden sich Aussagen zur Notwendigkeit der Schulsozialarbeit an berufsbildenden Schulen und auch Praxisbeispiele sowie Projektbeschreibungen nehmen fast ausschließlich den allgemeinbildenden Schulbereich in den Fokus. Lediglich das Gebiet berufsvorbereitender Bildungsmaßnahmen, z. B. in Form von Berufsvorbereitungsjahren oder Berufsgrundbildungsjahren wird in Veröffentlichungen aus verschiedenen Bundesländern immer wieder erwähnt, erschöpft sich jedoch meist in Beschreibungen der Konzeption und Umsetzung sozialpädagogischer Angebote an bestimmten Schulen.

Fast erscheint es, als würden die Problemlagen der meisten Jugendlichen zum Stillstand kommen, sobald sie die allgemeinbildende Schule verlassen, was in der Realität ganz sicher nicht der Fall ist. 

\begin{quotation}
\noindent
"`Der genauere Blick macht deutlich, dass das Schulsystem in Gestalt der Risikoschüler eine große Gruppe hervorbringt, die gegebenenfalls trotz entsprechender formaler Abschlüsse, nicht über die realen Kompetenzen verfügen, um eine moderne Berufsausbildung absolvieren zu können. Zugleich werden von Seiten des Beschäftigungssystems jedoch nicht all jene aufgenommen, die über die notwendigen Kompetenzen verfügen. [\punkte] Dies zeigt sich insbesondere bei dem zunehmend krisenhaften Übergang von der Schule in die duale Ausbildung."'
\end{quotation}

\noindent
[\punkte] fassen dazu treffend  Braun und Wetzel zusammen \footcite[181]{Braun2006}. Dadurch begründet, müsste eigentlich die Schulsozialarbeit in berufsbildenden Schulen besonders häufig zu finden sein und einen festen Bestandteil der sozialpädagogischen Arbeit darstellen.

Bezogen auf den Bedarf an sozialpädagogischer Unterstützung in der berufsbildenden Schulen lässt sich feststellen, dass die Problemlagen der dortigen jungen Menschen sich wenig bis gar nicht von denen der allgemeinbildenden Schule unterscheiden, jedoch einige strukturell bedingte Unterschiede zu berücksichtigen sind, die im vorangegangenen Punkt bereits ausführlicher erläutert wurden. Es finden sich sowohl Jugendliche mit einem Ausbildungsplatz im dualen System, als auch solche in vollzeitschulischen Ausbildungen sowie junge Menschen ohne Ausbildungsplatz, z. B. in Berufsvorbereitungsjahren. Insbesondere die Jugendlichen ohne Ausbildungsplatz stellen Lehrkräfte und Personal vor zahlreiche Herausforderungen, gelten sie doch vielfach heute als "`berufsunreif"', "`schwer vermittelbar"' und "`lernunwillig"', obwohl sie noch vor wenigen Jahren ohne größere Schwierigkeiten einen Ausbildungsplatz gefunden hätten. Ihre Ausbildungslosigkeit und die damit verbundene Arbeitslosigkeit wird oftmals als individuelles Versagen definiert und die Jugendlichen werden als Benachteiligte stigmatisiert. Die berufsbildende Schule hat hier insbesondere die Aufgabe diese berufsschulpflichtigen Jugendlichen auf eine ungewisse Zukunft vorzubereiten, indem sie sie durch Unterricht, Praktika und sozialpädagogische Maßnahmen mit sozialen und lebenspraktischen Kompetenzen ausstattet \footcite[vgl.][6]{ASSB2011}. 

Obwohl häufig als weniger problembehaftet dargestellt, weisen auch Jugendliche mit Ausbildungsplatz Unterstützungsbedarfe auf, die von Lehrkräften nicht oder nur in sehr begrenztem Maße geleistet werden können. Die berufliche Schule als zentraler Lernort zur Verknüpfung von Theorie und Praxis kämpft hier insbesondere mit einem permanenten Zeitmangel und dadurch mit geringen Möglichkeiten zur Vermittlung sozialer Kompetenzen und zur Unterstützung einer adäquaten Auseinandersetzung mit individuellen Problemen. Die Vorbereitung auf die Abschlussprüfungen prägt den berufsschulischen Alltag so sehr, dass für pädagogische Zielsetzungen und soziale Lerninhalte meist nur wenig Unterrichtszeit zur Verfügung steht. Weiterhin führen deutlich gestiegene theoretische Anforderungen durch die Neuordnung vieler Ausbildungsberufe häufig zu einer Überforderung der Jugendlichen mit schlechteren Schulabschlüssen. Dadurch scheitern viele Auszubildende mit Lernproblemen und schlechten Sprachkenntnissen, z.B. bedingt durch einen Migrationshintergrund, teilweise bereits während der Ausbildung oder auch an den Abschlussprüfungen, so dass die Quoten der Ausbildungsabbrüche und der nicht bestandenen Prüfungen zum Teil sehr hoch sind. Außerdem sind die betriebliche Realität, der volle Arbeitstag, der Umgang mit Vorgesetzten, Kolleginnen, Kundinnen oder Patientinnen und die hohe Verantwortlichkeit für ihr Tun neue Erfahrungen, die für die Jugendlichen zu verarbeiten sind. Dazu kommen oft noch Erlebnisse mit negativen Arbeitsbedingungen, Nichteinhaltung gesetzlicher Schutzbestimmungen, Überstunden, Verrichtung von ausbildungsfremden Arbeiten, einengende oder überfordernde Tätigkeitsbereiche sowie ungerechte Behandlung durch Vorgesetzte oder Kollegen und vieles mehr \footcite[vgl.][7]{ASSB2011}. 

Ähnliche Befunde und Problemlagen wie der Arbeitskreis Berufsschulsozialarbeit in Bayern diagnostizierte auch die Landesarbeitsgemeinschaft Schulsozialarbeit e. V. in Sachsen im Jahre 2004 \footcite[18]{LSS2004}. 

\begin{quotation}
\noindent
"`Von den Berufsschulen werden massive individuelle Probleme und Lernprobleme geschildert. Bei der genaueren Betrachtung dieser äußern sie sich in teilweise gewalthaften Verhaltensauffälligkeiten, Schulverweigerung, mangelnder sozialer Kompetenz und Suchtproblemen. Bei Lernproblemen werden Motivationsprobleme und fehlende Lerntechniken geschildert."' \footcite[18]{LSS2004}
\end{quotation}

\noindent
Insbesondere, aber nicht ausschließlich wurden derartige Feststellungen in den Berufsvorbereitungsjahren gemacht, zu deren Zweck und Besonderheiten bereits Ausführungen erfolgten. Als besonders problematisch wurde dabei die Tatsache angesehen, dass die Jugendlichen teilweise ihr Scheitern bei der Suche nach einem Ausbildungsplatz ausschließlich als persönliches Defizit und nicht als Auswirkung des Lehrstellenmangels betrachten. Dadurch bilden sich in den beruflichen Schulen mitunter konfliktbeladene Klassen, die überwiegend aus leistungsschwachen und unmotivierten Schülern bestehen und aufgrund weitreichender sozialer Benachteiligungen der Schüler eigentlich ein definitives Arbeitsfeld der Schulsozialarbeit im Sinne des § 13 SGB VIII/KJHG wären \footcite[18]{LSS2004}.

Abweichend von den schulartspezifischen Problemlagen sollen auch noch einige Studien betrachtet werden, die sich dem Thema allgemeiner gewidmet haben, um weitere mögliche Themen herauszuarbeiten, die viele Jugendliche heute tangieren und Bedarfe für Schulsozialarbeit bedingen können. Dazu liegt eine Studie der Universität Leipzig aus dem Jahre 2007 vor, in der Problemlagen von Kindern und Jugendlichen mittels einer Befragung von 34 Fachkräften der Kinder- und Jugendhilfe eruiert wurden. Festgestellt wurde dabei, dass die Problemlagen insbesondere in ihrer Komplexität zunehmen, also zunehmend vielfältiger und unberechenbarer geworden sind \footcite[vgl.][142ff]{UniversitaetLeipzig2007}. Dies zeigt sich unter anderem daran, dass die befragten Fachkräfte eine deutliche Zunahme der Gruppe "`schwieriger"' Jugendlicher beschrieben, die durch mehrere soziale und schulische Auffälligkeiten gekennzeichnet sind. Ebenso wurde ein deutliches Ansteigen psychischer und psychosomatischer Störungs- und Krankheitsbilder verzeichnet. Legale und illegale Formen exzessiven Drogengebrauchs stellen die Fachkräfte vor schwierige Probleme, bei denen sowohl Sachfragen (Informationen über Stoffe, Verbreitungsmuster, Wirkungen etc.) als auch Verständnisfragen (Wissen über soziale und jugendkulturelle Hintergründe, Gebrauchsmuster etc.) eine Rolle spielen. 

\begin{quotation}
\noindent
"`Hier wurde hervorgehoben, wie sehr legale wie auch illegale Drogen unter Jugendlichen heute "`alltagsfähig"' geworden sind. Probleme wie das Sinken des Einstiegsalters und eine gesteigerten Suchtgefahr bei Jugendlichen sind Ausdruck dieser Entwicklung."' \footcite[vgl.][142ff]{UniversitaetLeipzig2007}
\end{quotation}

\noindent
Weiterhin konstatierten die Fachkräfte eine deutliche Veränderung der Altersstruktur hilfebedürftiger Jugendlicher. Als Problemlage wurde eine regelrechte "`Überalterung"' des Klientels beschrieben, die durch fehlende Übergänge in Ausbildung und Erwerbstätigkeit entsteht. Demgegenüber sind im Arbeitsfeld auch immer jüngere Jugendliche mit komplexen Problemlagen zu verzeichnen, für die junge bis sehr junge Mutterschaften nur ein Beispiel darstellen. So bilden sich derzeit neue Zielgruppen für die Jugendhilfe und somit auch für die Schulsozialarbeit heraus, die in der Praxis konzeptionelles Umdenken und veränderte Arbeitsformen erfordern \footcite[vgl.][142ff]{UniversitaetLeipzig2007}.
 
Eine Untersuchung zur Schulsozialarbeit in Sachsen aus dem Jahr 2005 durch eine Arbeitsgruppe der Landesarbeitsgemeinschaft Schulsozialarbeit Sachsen e.V. beschäftigte sich ebenfalls mit Problemlagen der Schülerschaft, der Schulen und Möglichkeiten der Schulsozialarbeit. Befragt wurden 78 SozialarbeiterInnen allgemeinbildender Schulen und Sonderschulen. Berufsbildende Schulen waren in der Untersuchung nicht vertreten. Als besonders gravierende Problemlagen und somit Handlungsfelder der Schulsozialarbeit wurden Mobbing, insbesondere in Klassen ab der Klassenstufe 7, Schulverweigerung, Gewalt, Leistungsschwäche, Verhaltensauffälligkeiten, familiäre Probleme (Armut), Integrationsschwierigkeiten von Schülern mit Migrationshintergrund sowie Alkohol- und Drogenprobleme herausgearbeitet. Festgestellt wurde weiterhin, dass keine klar abgrenzbaren Hauptprobleme zu erkennen waren, sondern die jeweiligen Lebensverhältnisse der Betroffenen zu Überforderungssituationen führten, die besonders im Rahmen der Einzelfallhilfe zu thematisieren sind \footcite[vgl.][63ff]{Lang2010}. Anzunehmen ist, dass die, von den Sozialarbeitern benannten, Problemlagen ebenso auf den Bereich der berufsbildenden Schulen übertragen werden können, da die Schüler nach Beendigung ihrer allgemeinbildenden Schullaufbahn in eine berufsvorbereitende oder berufsausbildende Maßnahme überwechseln.

In den vorgestellten Untersuchungen und Berichten ist nunmehr eine Vielzahl von Problemlagen vorgestellt und beschrieben wurden, die jedoch an dieser Stelle nur einen Ausschnitt tatsächlich im berufsbildenden Bereich vorhandener Problemlagen und Herausforderungen repräsentieren können. Alle benannten Probleme sowie auch weitere finden sich im 14. Kinder- und Jugendbericht \footcite[44f]{BundesministeriumFamilie2013} und im Dritten Sächsischen Kinder- und Jugendbericht \footcite[30ff]{SMSSS2009} an verschiedensten Stellen und unter mannigfaltigen Schwerpunkten. Keinesfalls erheben die hier vorgestellten Problemlagen den Anspruch auf Vollständigkeit, was in Anbetracht der individuellen sozialen, schulischen und beruflichen Gegebenheiten der Schüler im berufsbildenden Bereich auch niemals möglich sein kann. Für die vorliegende Arbeit bedeuten die Ausführungen eine wichtige Grundlage zur Konzeption einer eigenen schulspezifischen Erfassung der Problemlagen von Schülern in medizinischen und sozialen Ausbildungen, da für diese spezielle Schülergruppe und die dazugehörigen Schularten keine Untersuchungen existieren bzw. bei der Literaturrecherche gefunden werden konnten. 
