\section{Kritische Diskussion}
\label{sec:KritischeDiskussion}

Die vorliegende Arbeit hat vielfältige Ergebnisse im Interesse der Forschungsfragen geliefert und könnte möglicherweise als durchaus gewinnbringend für den gering repräsentierten Wissenschaftsbereich Berufsbildender Schulen im Fachbereich Gesundheit und Pflege sowie Sozialwesen angesehen werden.

Dennoch sind einige Aspekte der vorangegangenen Ausführungen als kritisch zu betrachten und sollen daher angeführt werden.

Aufgrund der rein exemplarischen Untersuchung am DRK Bildungswerk SN ist keine repräsentative Aussage hinsichtlich der Problemlagen und allgemeinen Beratungs- und Unterstützungsbedarfe von berufsbildenden Schülern, explizit im Bereich Gesundheit, Pflege und Sozialwesen, in Sachsen bzw. deutschlandweit gegeben. Zudem erschwert die Heterogenität zwischen den Bundesländern eine allgemeingültige Antwort bezüglich der untersuchten Ergebnisse. Dennoch sind Parallelen zu anderen beruflichen Schulen, ob in freier oder öffentlicher Trägerschaft, denkbar.

Eine weitere Einschränkung der gegebenen Repräsentativität erfolgt durch die geringe Anzahl der befragten Lehrkräften und Schülern. Aufgrund begrenzter zeitlicher und personeller Ressourcen konnten nur 30\% (bzw. $\approx$ 26,64\%) der Schülergesamtheit und 5 Lehrkräfte befragt werden. Daher kann das realistische Vorliegen von Problemlagen und Beratungs- und Unterstützungsbedarfen anhand der eruierten Ergebnisse nur vermutet und als nicht vollständig untersucht betrachtet werden. 

Desweiteren sind potentielle Fehlerquellen durch die Autorinnen und befragten Probanden denkbar. Da man bei der Konstruktion des Fragebogens auf keinerlei Erfahrungswerte zurückgreifen konnte, sind im Nachhinein betrachtet, bestimmte Details der Umfrage sicher optimierbar. Zudem geben die fehlerhaft ausgefüllten Fragebögen Rückschluss auf mögliche Fehlerquellen. Bei einer weiteren Verwendung des Fragebogens könnte man z. B. die 3 Abschnitte des Bogens farblich unterscheiden umso auf eine Trennung der Abschnitte hinzuweisen. Weiterhin könnte man die Einfach- bzw. Mehrfachnennungen deutlicher hervorheben um so einem fehlerhaften Ankreuzen vorzubeugen. Es wäre auch eine Erweiterung des Fragebogens an sich vorstellbar, im Sinne von zusätzlichen Antwortmöglichkeiten oder Fragestellungen. Zudem wäre eine längere Testphase (Pretest) an Schülern verschiedener Ausbildungsgänge möglich, da diese z. T. eine hohe Heterogenität aufweisen und dementsprechend unterschiedlicher Rahmenbedingungen bedürfen. Weitere Fehlerquellen oder inhaltliche Schwierigkeiten für Schüler hätten so schneller und effizienter erkannt und behoben werden können. Dennoch wurde der genutzte Fragebogen zur Beantwortung der Forschungsfragen, zum Zeitpunkt des Einsatzes, als stimmig und ausreichend eingeschätzt.

Die Konstruktion des Leitfadens für die Lehrerinterviews fand im Rahmen einer universitären Gruppenarbeit statt. Dieser offeriert den Studierenden die Möglichkeit mehr Inhalte zu erfragen als für die Beantwortung der Forschungsfragen notwendig gewesen wäre. Bei einer fokussierteren Betrachtung  der Forschungsfragen könnte der Fragenkatalog dementsprechend verkleinert werden. Dennoch erwiesen sich die "`Zusatzfragen"' als sehr gewinnbringend bei der Erstellung eines Gesamtbildes des Forschungsinteresses am DRK Bildungswerk SN. Im Rahmen der Transkription der aufgezeichneten Interviews wurde zudem das Potential möglicher Nachfragen bei einigen Gesprächspartnern erkannt. Es wird vermutet, das so potentiell weitere erkenntnisreiche Informationen nicht von den Lehrkräften erfragt wurden Dieser Aspekt würde bei folgenden Gesprächen beachtet. Trotz des vorliegenden Forschungsinteresse an sozialpädagogischen und anderweitigen Beratungs- und Unterstützungsangeboten wurde nur ein sozialpädagogisches Angebot von den Lehrpersonen benannt, wie der Ergebnisdarstellung zu entnehmen ist. Da die Autorinnen hier ein Informationsdefizit im Hinblick auf Funktion und Aufgabenbereich eines Schulsozialarbeiters oder auch dem eines Sozialpädagogen vermuten, gibt es Überlegungen diese Kenntnisse bei weiterführenden Gesprächen zu Beginn des Interviews aufzugreifen. Man könnte die Lehrkräfte beispielsweise nach ihrem Begriffsverständnis zu den o. g. Berufszweigen befragen, um ausgehend davon die genannten Beratungs- und Unterstützungsangebote besser in das bekannte Repertoire der Lehrkraft einordnen zu können. Es wäre zwar auch eine Information durch die Interviewer selbst denkbar; diese Möglichkeit könnte jedoch die genannten Angebotsformen massiv beeinflussen.

Die größte Unsicherheit bei der Darstellung der Ergebnisse besteht in der ehrlichen Beantwortung der Fragen durch die gewählten Lehrer und Schüler. Aspekte wie Angst, Verleugnung, soziale Erwünschtheit oder "`Gruppenzwang"' sind nur mögliche Einflussfaktoren auf die wahrheitsgemäße Beantwortung. Diese Faktoren sind jedoch nur schwer nachzuvollziehen und wirken auf jede Studie bzw. Umfrage ein. Daher wird hier nicht von einer zu beeinflussenden, sondern natürlichen Fehlerquelle ausgegangen.