\section{Vorstellung des Forschungsvorhabens}
\label{sec:VorstellungDesForschungsvorhabens}

\subsection{Das erste eigene Forschungsprojekt - Eine Hinführung}
\label{sec:DasErsteEigeneForschungsprojektEineHinführung}

Lehrer sein. OK. Forscher sein. OK. Aber forschende Lehrer?

Als Student an einer deutschen Hochschule ist man am Ende seines Studiums dazu aufgefordert eine Abschlussarbeit zu verfassen, welche die erworbenen Kenntnisse, Methoden und Wissensstrukturen angemessen widerspiegelt. Auch die Autorinnen standen zum Ende ihres 5-jährigen Studiums für das Höhere Lehramt an berufsbildenden Schulen, sprich als angehende Berufsschullehrer, vor dieser abschließenden Herausforderung. Bereits Monate zuvor macht man sich Gedanken, welche Fachrichtung man dafür heranziehen möchte, welches Themengebiet spannend und erforschbar wäre, welche Betreuer sich der Arbeit annehmen würden und wie man diesen Arbeitsprozess überhaupt organisieren könnte? 

Doch zunächst ein Blick zurück ins Jahr 2010. In diesem Jahr haben sich Daniela Wobst und Doreen Stichel dazu entschlossen, ein Studium für das Lehramt an berufsbildenden Schulen im Bereich Gesundheit und Pflege und Sozialpädagogik an der TU Dresden aufzunehmen. Zu diesem Zeitpunkt hatten beide bereits eine Ausbildung absolviert, entsprechende Berufserfahrung gesammelt und waren in verschiedenen Branchen des Gesundheitswesens tätig. Beide hatten sich mit Aufnahme des Studiums das Ziel gesetzt, den bestehenden beruflichen Qualifikationen neue Wissensbereiche, Kompetenzen und berufliche Möglichkeiten hinzuzufügen. Zum Zeitpunkt der Immatrikulation war das Lehramt für berufsbildende Schulen in Bachelor- und Masterstudiengang organisiert, wohingegen im Jahr 2013 das Staatsexamen erneut eingeführt wurde. 

Zurück im Jahr 2015 stand man nun, wie alle Kommilitonen, vor der Frage, welches Thema für eine wissenschaftliche Bearbeitung im Rahmen einer Masterarbeit geeignet wäre. Unabhängig voneinander haben beide Autorinnen bei der Eingrenzung der möglichen Themen besonderes Interesse für den Gegenstand der Schulsozialarbeit gezeigt. Aufgrund einer guten persönlichen Beziehung und positiver Erfahrungswerte in Bezug auf die Erarbeitung gemeinsamer Seminararbeiten entschloss man sich dazu, das bestehende Forschungsinteresse auf dem Gebiet der Schulsozialarbeit zu nutzen und eine gemeinsame Arbeit in die Wege zu leiten. 

Die wissenschaftliche Bearbeitung einer Problemstellung im Rahmen einer Masterarbeit an der Fakultät Erziehungswissenschaften kann durch zwei verschiedene Herangehensweisen erfolgen. Entweder man nutzt den theoretischen Zugang mittels der literaturbezogenen Bearbeitung einer wissenschaftlichen Problemstellung oder den praktischen Zugang in Form eines eigenen Forschungsprojektes.
Aufgrund persönlicher Interessen und Ansichten erschien ein kleines Forschungsprojekt, welches in der Lage ist, eigene Erkenntnisse zu generieren, äußerst attraktiv und zugleich motivierend für den gemeinsamen Forschungsprozess. Zudem entspricht diese Vorgehensweise dem Anspruch an eine Masterprüfung gemäß § 20 der Prüfungsordnung für das Höhere Lehramt an berufsbildenden Schulen nach dem "`der Studierende die fachlichen Zusammenhänge überblickt sowie die Fähigkeit besitzt, wissenschaftliche sowie gegebenenfalls künstlerische Methoden und Erkenntnisse anzuwenden, und die für den Übergang in den für die Befähigung für das Höhere Lehramt an berufsbildenden Schulen vorgeschriebenen Vorbereitungsdienst bzw. eine Promotion notwendigen gründlichen Fachkenntnisse erworben hat. Ebenso wird festgestellt, dass der Studierende über vertiefte fachliche Kenntnisse und berufsfeldbezogene Qualifikationen als Beschäftigungsbefähigung für eine Tätigkeit in Berufsfeldern des öffentlichen oder privaten Bildungssektors verfügt."' \footcite[13]{TUDresden2010}

Angesicht dieser Ausgangslage entschied man sich im Dialog für ein gemeinsames Forschungsprojekt, welches sich dem Thema der Schulsozialarbeit an berufsbildenden Schulen widmen soll. 

Nach einer grundlegenden Sichtung der zum Zeitpunkt der Arbeit veröffentlichten Literatur zum Thema Schulsozialarbeit an berufsbildenden Schulen zeigte sich eine deutliche geringere Repräsentativität sozialpädagogischer Beratungs- und Unterstützungsangebote an berufsbildenden Schulen im Vergleich zu derer an allgemeinbildenden Schulen beispielsweise in Form der Grund- oder Förderschule; ein Umstand der u. a. gesetzliche Gründe hat (siehe Punkt \ref{sec:RechtlicheGrundlagen}) aber auch Verwunderung hervorrief, korrelierte er doch mit persönlichen Vermutungen und Erfahrungen. Aus diesem Grund setzte man sich nun zum Ziel die Probleme und Unterstützungsbedarfe von Schülern reell zu erfassen, um so eine Aussage über den potentiellen Bedarf von Schulsozialarbeit für Schüler an berufsbildenden Schulen formulieren zu können. Aufgrund der begrenzten zeitlichen und personellen Möglichkeiten entschied man sich für eine exemplarische Untersuchung an einer berufsbildenden Schule in Sachsen.
 
Die für die Realisierung des Forschungsprojekts notwendige Kooperationsvereinbarung mit einer berufsbildenden Schule konnte aufgrund der jahrelangen Tätigkeit von Daniela Wobst am DRK Bildungswerk SN und der freundlichen Genehmigung der dortigen Geschäftsleitung in Person von Hr. Vlodrop, Fr. Hösel und Hr. Eckert umgesetzt werden. Die genauen Abläufe sollen nun im Folgenden vorgestellt werden.

\subsection{Problemdefinition}
\label{sec:Problemdefinition}

\subsubsection{Relevanz des Themas}
\label{sec:RelevanzDesThemas}

Da unter Punkt \ref{sec:BedarfeFürSchulsozialarbeit} die Sachlage zum Thema der Schulsozialarbeit an beruflichen Schulen in Sachsen bereits ausführlich erläutert wurde, wird hier auf eine grundlegende Darstellung zur Relevanz des Themas verzichtet. Die folgenden Ausführungen, eine Kurzfassung bisheriger Inhalte, dienen der Hinführung zum Forschungsprojekt.

Es gibt subjektiv viele Problemlagen, denen sich Jugendliche bzw. junge Menschen in Ausbildung gegenüber sehen, dennoch sind kaum Studien, Bedarfsanalysen oder konkrete Angebote von Schulsozialarbeit zur Lösung dieser Probleme vorhanden. Forschungen und Publikationen zum Thema sind zwar sehr vielfältig und zahlreich, darunter auch umfassende Begründungen zu Notwendigkeit und Bedarf, diese richten sich aber nur an den grundschulischen und allgemeinbildenden Bereich. Trotz intensiver Recherche in Fachliteratur und einschlägigen Datenbanken existieren nur sehr wenige Ausführungen, Studien und Bedarfsanalysen für den berufsbildenden Bereich. Eine Ursache hierfür ist gesetzlich bedingt, denn Schulsozialarbeit ist in Sachsen nur in Schulen mit Berufsvorbereitungsjahr vorgeschrieben (siehe Punkt \ref{sec:RechtlicheGrundlagen}). Dem gegenüberzustellen sind jedoch die der sozialwissenschaftlichen Literatur zu entnehmenden vielfältigen Problemlagen von Jugendlichen und jungen Menschen in Ausbildung, die mit gesellschaftlichen und strukturellen sowie institutionellen Gegebenheiten in Zusammenhang stehen. 

Doch inwiefern machen diese Problemlagen den Einsatz von Schulsozialarbeit an berufsbildenden Schulen prinzipiell erforderlich, wenn man die momentan geringen Angebote für diese Schulform beachtet? 

\subsubsection{Forschungsgegenstand}
\label{sec:Forschungsgegenstand}

Im Rahmen der Masterarbeit sollen die Problemlagen und außerunterrichtlichen Be\-rat\-ungs- sowie Unterstützungsbedarfe von Schülern an einer exemplarischen berufsbildenden Schule in Sachsen erfasst werden.

Zusätzlich soll die Sicht der Lehrkräfte hinsichtlich der Problemlagen von Schülern sowie mögliche Beratungs- und Unterstützungsbedarfe durch Lehrkräfte eruiert werden.

\subsubsection{Nutzen der Forschung/Forschungslücke}
\label{sec:NutzenDerForschungForschungslücke}

Lehrkräfte nehmen Problemlagen von Schülern und individuelle Unterstützungsbedarfe (subjektiv) wahr, sehen sich jedoch mit den teilweise komplexen Problemlagen überfordert und bemängeln die unzureichende Unterstützung in dieser Sachlage. Sie fordern eine Rückbesinnung auf ihr Kerngeschäft -- nämlich den Unterricht.

Die geplanten Datenerhebungen aus Schüler- und Lehrerperspektive untersuchen die tatsächlichen Probleme von Schülern und Bedarfe für Schulsozialarbeit an einer berufsbildenden Schule in Sachsen. Abgeleitet werden sollen sowohl die Bedarfe als auch mögliche Beratungs- und Unterstützungsangebote.

\subsubsection{Vorannahmen}
\label{sec:Vorannahmen}

\begin{itemize}
	\item Die Problemlagen von Schülern an berufsbildenden Schulen unterscheiden sich wenig bis gar nicht von denen an allgemeinbildenden Schulen.
	\item Außerunterrichtliche Problemlagen wirken sich tendenziell negativ auf das Unterrichtsgeschehen aus.
	\item Lehrkräfte sehen Problemlagen bei Schülern und erkennen Bedarfe für sozialpädagogische Unterstützungsangebote. 
\end{itemize}

\subsubsection{Forschungsfragen}
\label{sec:Forschungsfragen}

\begin{enumerate}
	\item Welche persönlichen und sozialen (außerunterrichtlichen) Problemlagen haben Schüler des DRK Bildungswerk SN in sozialen und medizinischen Ausbildungsberufen, die das Unterrichtsgeschehen und den Ausbildungserfolg beeinflussen?
	\item Welche persönlichen und sozialen (außerunterrichtlichen) Problemlagen von Schüler nehmen Lehrkräfte des DRK Bildungswerk SN im Bereich Gesundheit, Pflege und Sozialwesen als besondere Belastung für den Unterricht wahr?
	\item Wie schätzen Schüler und Lehrkräfte anhand der (möglichen) subjektiv wahrgenommenen Problemlagen den Bedarf an sozialpädagogischen und anderweitigen Unterstützungs- und Beratungsangeboten ein und wie könnten passende Angebote aussehen?
\end{enumerate}

\noindent
\textbf{Begründung}\\

\noindent
Es existieren subjektiv wahrgenommene Bedarfe an sozialpädagogischen Unterstützungsangeboten für persönliche außerunterrichtliche Probleme, welche das Unterrichtsgeschehen negativ beeinflussen.\\

\noindent
\textbf{Erläuterung der Forschungsfragen}\\

\noindent
Zu Beginn der vorliegenden Arbeit wurde das Prinzip der Schulsozialarbeit in Deutschland bzw. Sachsen als Ausgangspunkt der weiterführenden Überlegungen grundlegend dargelegt. Dennoch wurde, wie bereits am Titel der Arbeit erkennbar, bei der Formulierung der Forschungsfragen auf eine Verwendung dieses Begriffes verzichtet. Diesen Umstand gilt es aus verschiedenen Sichtweisen zu klären.
 
Wie bereits unter Punkt \ref{sec:AllgemeineTheoretischeGrundlagenZurSchulsozialarbeit} erläutert wurde, ist das Wort Schulsozialarbeit ein sehr umfassender Begriff, welchem eine Vielzahl an Maßnahmen, Zielgruppen, Einsatzmöglichkeiten und Zielstellungen zuzuordnen sind. Viele dieser Maßnahmen und Angebote sind jedoch aufgrund gesetzlicher oder schulspezifischer Gegebenheiten nicht im Kontext der berufsbildenden Schulen, explizit am DRK Bildungswerk SN umsetzbar. Das schließt u. a. die am Zielort zu berücksichtigende Abwesenheit eines Schulsozialarbeiters ein. Unabhängig davon sind sowohl sozialpädagogische als auch anderweitige Angebote, die durch die Lehrkräfte aufgezeigt werden können, im Fokus des Forschungsinteresse. Diese können möglicherweise in Auszügen den genannten Angeboten der Schulsozialarbeit zugeordnet werden (siehe Punkt \ref{sec:Methoden}) und so eine Verbindung zwischen Theorie und Praxis aufzeigen. Zudem ist darauf hinzuweisen, dass die Erfassung subjektiver Problemlagen und die Eruierung potentieller Unterstützungsbedarfe von Schülern im Fokus dieser Untersuchung stehen. Konzeptionelle Überlegungen hinsichtlich möglicher sozialpädagogischer oder anderweitiger Angebote werden als Ergebnis der Bedarfsanalyse erst am Ende der Arbeit formuliert und sind als Anregung/Empfehlung für das DRK Bildungswerk SN zu verstehen.
 
Wie im weiteren Verlauf erkennbar, wurde bei der Formulierung der Umfrage und der Interviews ebenfalls auf den Begriff Schulsozialarbeit verzichtet, da dieser für viele Lehrer und Schüler ein zumeist unbekannter und schwer zu definierender Begriff ist. Schulsozialarbeit hätte aus dieser Vermutung heraus zunächst erst als "`Fremdwort"' grundlegend erläutert werden müssen; dem wurde entsagt.

Aus diesen Gründen wurde im Rahmen der Untersuchung komplett darauf verzichtet, den Begriff Schulsozialarbeit zu verwenden und eher die leicht verständliche Formulierung von "`Beratungs- und Unterstützungsangeboten"' gewählt. Darunter werden alle möglichen Angebote mit beratenden und unterstützenden Charakter subsumiert, welche einen Benefit für die Schüler erkennen lassen.

Desweiteren sollen im Rahmen dieses Forschungsprojektes ausschließlich außerunterrichtliche Probleme und Bedarfe eruiert werden. Das bedeutet, dass nur die Problemlagen Betrachtung finden, die unabhängig von Unterrichtsinhalten entstehen, das schulische Miteinander und Erleben jedoch stark beeinflussen können. Beispiel hierfür sind finanzielle oder familiäre Probleme. Es ist in dieser Studie nicht von Interesse, inwieweit Schüler mit ihren Lehrkräften zufrieden sind oder ob Unterrichtsinhalte interessant und abwechslungsreich vermittelt werden; jegliche Problematiken, die innerhalb des Unterrichtsgeschehens entstehen, werden nicht betrachtet. 

Um ein möglichst realistisches und breitgefächertes Abbild der momentanen Probleme und Unterstützungsbedarfe von Berufsschülern in Sachsen zu erhalten, wurde im Vorfeld der Untersuchung beschlossen, zwei verschiedene Perspektiven in Bezug auf das Forschungsinteresse in die Analyse einzubinden. Neben der Schülerperspektive wird auch die Lehrerperspektive bei der Erfassung der Daten berücksichtigt. Beide Parteien haben trotz des gemeinsam erlebten Schulalltages ggf. eine unterschiedliche Wahrnehmung bezüglich der Probleme und Unterstützungsbedarfe von Schülern. Daher betrachtet die erste Forschungsfrage ausschließlich die Schülerperspektive; wo hingegen die zweite Forschungsfrage ausschließlich die Lehrerperspektive in den Fokus stellt. Beide Sichtweisen sollen schließlich mit der dritten Forschungsfrage zusammengeführt und auf eine Schnittmenge hin untersucht werden.

\subsection{Projektplanung}
\label{sec:Projektplanung}

\subsubsection{Zugang zum Forschungsfeld}
\label{sec:ZugangZumForschungsfeld}

Zu Beginn der Projektplanung gab es mehrere Ideen, diese institutionell umzusetzen. In Sachsen gibt es eine Vielzahl von berufsbildenden Schulen unter privater sowie öffentlicher Trägerschaft. Der Gedanke, mehrere öffentliche Schulen zur Generierung einer großen Datenmenge heranzuziehen, wurde aufgrund der geringen zeitlichen Ressourcen negiert. Desweiteren wurden Überlegungen angestellt, öffentliche und private schulische Einrichtung(en) bei der Auswertung der beabsichtigten Ergebnisse zu vergleichen; dieser Aspekt wurde jedoch aufgrund zeitlicher und organisatorischer Hürden ebenso verworfen. Letztendlich wurde beschlossen, dass nur eine exemplarische private schulische Einrichtung zur Projektrealisierung berücksichtigt werden soll. Diese Entscheidung offeriert den Autorinnen bessere personelle, organisatorische und zeitliche Absprachen und ermöglicht zudem eine fokussierte, abgrenzbare und ganzheitlich angelegte Darstellung einer berufsbildenden Schule hinsichtlich der gewählten Problemstellung. Diese exemplarische  Betrachtungsweise ermöglicht zudem konkrete Empfehlungen hinsichtlich Bedarf und Konzeption möglicher Beratungs- und Unterstützungsangebote für eine schulische Einrichtung.

Wie bereits angesprochen, wurde die notwendige schulische Kooperation für dieses Forschungsprojekt durch die freundliche Genehmigung der Geschäftsleitung des DRK Bildungswerk SN ermöglicht, wobei Daniela Wobst, bereits viele Jahre bei dieser Bildungsstätte angestellt, als "`Gatekeeper"' (Türöffner) fungierte. 

\subsubsection[Selektive Erarbeitung im Rahmen einer universitären Gruppenarbeit]{Darlegung über die selektive Erarbeitung relevanter Inhalte im Rahmen einer universitären Gruppenarbeit}
\label{sec:DarlegungÜberDieSelektiveErarbeitungRelevanterInhalteImRahmenEinerUniversitärenSeminararbeit}

Bestandteile der vorliegenden Ausarbeitung beruhen auf einer universitären Gruppenarbeit; ein Umstand der zu klären ist. Zeitgleich mit Beginn der Masterarbeit wurden Daniela Wobst und Doreen Stichel Teil einer universitären Arbeitsgruppe, die bis Ende September 2015 Bestand hat. Diese Gruppe formierte sich im Rahmen des Seminars "`Forschungsfelder"', einem Modul der Fachrichtung Gesundheit und Pflege unter Leitung von Fr. Thümmler. Im Rahmen dieser Veranstaltung ist eine Prüfungsleistung in Form einer Qualitativen Studie zu erbringen. Im gemeinsamen Dialog entschloss man sich, das bereits bestehende Thema der Autorinnen aufzugreifen und die Sicht der Lehrkräfte hinsichtlich der Problemlagen von Schülern sowie mögliche Beratungs- und Unterstützungsbedarfe durch Lehrkräfte in Bezug auf die o. g. Forschungsfrage in Auszügen für die Seminararbeit heranzuziehen bzw. zu bearbeiten. Die Gruppenmitglieder, bestehend aus Janet Kaiser, Juliane Hemmerling, Anne Krause, Daniela Wobst und Doreen Stichel, erarbeiteten somit gemeinsam Inhalte, die auch in die vorliegende Arbeit eingeflossen sind. Inhalte und Arbeitsschritte, die unter Mitwirkung dieser Arbeitsgruppe erbracht wurden, werden im weiteren Verlauf kenntlich gemacht. Da die entsprechende Seminararbeit mit dem vorläufigen Arbeitstitel "`Eine qualitative Einzelfallstudie am DRK Bildungswerk Sachsen"' zum Zeitpunkt der Abgabe dieser Arbeit noch nicht vorlag, kann hiermit nur eine allgemeine Referenz auf diese erfolgen \footcite{Hemmerling2015}. Alle Gruppenmitglieder haben sich jedoch schriftlich damit einverstanden erklärt, das etwaige Inhalte, die unter Mithilfe derer entstanden sind, in der vorliegenden Masterarbeit verwendet werden dürfen (siehe \ref{sec:Sonstiges} Einverständniserklärung Gruppe).

\subsubsection{Universitäre Betreuung}
\label{sec:UniversitäreBetreuung}

Es ist uns ein Anliegen, die fachkundige, freundliche und unterstützende universitäre Betreuung durch Prof. Dr. Gängler, Dipl. Med-Päd. Haupt und Fr. Thümmler bei der Realisierung des Forschungsprozesses zu betonen. Deren hilfreiche Tipps, Anmerkungen und seminargebundenen Inhalte haben zum Gelingen dieses Projektes maßgeblich beigetragen.

\subsubsection{Grundlegende Organisation des Forschungsprojektes}
\label{sec:GrundlegendeOrganisationDesForschungsprojektes}

Im Folgenden sollen die grundlegenden organisatorischen Schritte dargelegt werden, die zur Realisierung des Forschungsprojektes notwendig waren. 

Die ersten Schritte auf dem Weg zur gemeinschaftlichen Abschlussarbeit waren bereits im Februar 2015 vollzogen. Themengebiet, Problemstellung, Forschungsfrage(n) und wesentliche Inhalte der geplanten Untersuchung wurden im stetigen Dialog durch die Autorinnen festgelegt. Ehe jedoch mit der eigentlichen Projektumsetzung begonnen werden konnte, hielt man zunächst Rücksprache mit den universitären Gutachtern, welche sich dem Thema angenommen hatten. Dank dieser kompetenten und inhaltlich ergiebigen Gespräche konnten neue Details und Gedanken in das geplante Forschungsvorhaben einfließen. Nach Abschluss dieser grundlegenden Arbeitsschritte konnte mit der detaillierten Planung der Erhebung begonnen werden. 
Es ist zu beachten, dass einige Arbeitsschritte, welche im Folgenden beschrieben werden, gleichzeitig erfolgten, was eine klare zeitliche Einordnung erschwert.
Dank dem Engagement von Daniela Wobst, u. a. Fachbereichsleiterin der Diätassistenz am DRK Bildungswerk SN, konnte diese langjährige private Bildungseinrichtung in Dresden als Untersuchungsfeld bereits frühzeitig festgelegt werden. Nach ersten mündlichen Absprachen von Frau Wobst mit der Geschäftsführung, unterbreiteten beide Autorinnen zu einem späteren Zeitpunkt ihr genaues Forschungsvorhaben im Rahmen einer kleinen Präsentation vor der gesamten Geschäftsleitung, so dass Ziel, Methoden und Verwendung der gesammelten Daten transparent erläutert werden konnten. Im Laufe dieses freundlichen Gesprächs konnten zudem organisatorische und inhaltliche Fragen geklärt werden und den Autorinnen wurde anschließend Unterstützung in Form der Bereitstellung der zahlreichen Umfragekopien zugesichert. Dank der freundlichen Projektbewilligung durch die Schulleitung konnte man im Anschluss an die betreffenden Lehrer und Schüler herantreten.

Bereits zuvor wurden Überlegungen angestellt, welche wissenschaftlichen Methoden sich als geeignet zur Beantwortung der gewählten Forschungsfragen hinsichtlich der Schü- ler- und  Lehrerperspektive erweisen könnten. Um möglichst wirklichkeitsnahe Aussagen in Bezug auf etwaige Schülerprobleme und Bedarfe unter Berücksichtigung der zur Verfügung stehenden Zeit zu generieren, entschied man sich für zwei verschiedene Methoden. Der Umfragebogen, eine Methode der quantitativen Forschung, ermöglicht das Zusammentragen potentieller Probleme und Bedarfe für Schulsozialarbeit von möglichst vielen Schülern; das Interview hingegen, eine Methode der qualitativen Forschung, dient der Erfassung subjektiver Aussagen von ausgewählten Lehrkräften hinsichtlich der Probleme und Bedarfe ihrer Schüler. Die genauere theoretische Einbettung und praktische Erarbeitung dieser Methoden wird unter Punkt \ref{sec:DieUmfrage} sowie \ref{sec:DasInterview} vertieft dargelegt.

Im nächsten Schritt wandt man sich nun an die Lehrkräfte des DRK Bildungswerk SN. Zu diesem Zweck kontaktierte Daniela Wobst zunächst alle betreffenden Personen über die berufliche E-Mail-Adresse. Über diesen Weg wurden sie über Inhalt, Ziel und Ablauf des universitären Forschungsprojektes an der Schule informiert, über anstehende Besuche in den Klassen bezüglich der Schülerumfragen in Kenntnis gesetzt, aber auch um persönliche Unterstützung in Form der Interviewteilnahme gebeten. Im Anschluss erreichten Daniela Wobst viele freundliche und interessierte Reaktionen diesbezüglich, aber auch zahlreiche Angebote von Lehrkräften, welche bereitwillig an einem Interview teilnehmen würden. Da die betreffenden Pädagogen zumeist in verschiedenen Ausbildungsgängen tätig sind, entschied man sich für eine Auswahl an Personen, welche mit ihrem Erfahrungshorizont die vorliegenden Ausbildungsrichtungen am DRK Bildungswerk SN möglichst komplett abbilden konnten. Somit konnten nach und nach 5 Lehrpersonen kontaktiert und für ein Interview gewonnen werden. Die individuelle Terminvergabe der Interviews erfolgte erneut per E-Mail oder telefonisch, organisiert durch Doreen Stichel. Bereits während dieser Absprachen zeigten die ausgewählten Lehrkräfte großes Interesse an Inhalt, Durchführung und Ergebnis der gesamten Untersuchung.

Nachdem die Lehrer über das Forschungsprojekt in Kenntnis gesetzt wurden, oblag es ihnen, die betreffenden Klassen auf den Besuch von Doreen Stichel oder Daniela Wobst vorzubereiten. Die Autorinnen nahmen vor Beginn der Untersuchung selbst keinen Kontakt zu den Klassen auf.

Ehe mit der Untersuchung begonnen werden konnte, musste zunächst die genaue Zielgruppe der zu befragenden Schüler am DRK Bildungswerk SN untersucht werden. Um ein möglichst realistisches Bild der vorhandenen Problemlagen und potentiellen Unterstützungsbedarfe zu erhalten, galt es, mit der Stichprobe die Grundgesamtheit der schulischen Einrichtung abzudecken. Von der Umfrage ausgeschlossen wurden jedoch die berufsbegleitenden Klassen. Schüler dieser Klassen zeichnen sich durch ein deutlich höheres Durchschnittsalter aus, besitzen bereits einen Beruf, oftmals eine eigene Familie mit Kindern und haben daher gänzlich andere Problemlagen als Schüler, die ihre Erstausbildung machen und zumeist noch im Elternhaus wohnen. Daher wurden, abzüglich der berufsbegleitenden Klassen, alle zum Zeitpunkt der geplanten Umfrage anwesenden Ausbildungsklassen erfasst. Diese wurden vermerkt, in der Personenanzahl erfasst und die für die Befragung nötige Anzahl ermittelt. Die genaue Darlegung dieses Prozesses, untermauert von einigen theoretischen Ausführungen zu Fragekonzeption und Methode, erfolgt unter Punkt \ref{sec:DieUmfrage}. 

\subsubsection{Angestrebte wissenschaftliche Ziele}
\label{sec:AngestrebteWissenschaftlicheZiele}

\textbf{Ziele im Rahmen des Forschungsinteresses}

\begin{itemize}
	\item Die Bearbeitung des Themas dient dem Abgleich von subjektiv wahrgenommenen Problemlagen von Schülern an berufsbildenden Schulen der Forschungsgruppe und vielen Lehrkräften mit den real beschriebenen Problemlagen von Schülern durch Schüler und Lehrkräfte an einer berufsbildenden Schule.
	\item Das Projekt ermöglicht einen konkreten Einblick in reale Problemlagen der Schüler am DRK Bildungswerk SN, welche den Unterricht negativ beeinflussen bzw. denen die Lehrkräfte in ihrem beruflichen Selbstbild potentiell nicht gewachsen sind.
	\item Aus der Erfassung möglicher Problemlagen könnte ein Bedarf für außerunterrichtliche sozialpädagogische Unterstützungsangebote abgeleitet werden.
\end{itemize}

\noindent
\textbf{Angestrebte Ziele für das DRK Bildungswerk Sachsen}
 
\begin{itemize}
	\item Für Schüler mit möglichen außerunterrichtlichen Problemlagen:
	\\
	Ziel ist es, die Lernleistung durch Minderung oder Lösung persönlicher Problemlagen, die die Aufmerksamkeit auf Schule und Unterricht senken, zu steigern.
	\item Entlastung der Lehrer bei möglichen außerunterrichtlichen Problemlagen:
	 \\
	Ziel ist es, die zusätzliche zeitliche und psychische Belastung der Lehrer zu reduzieren. 
	\item Auf einen möglichen Bedarf, durch das Heranziehen von Beratungs- und Unterstützungsangebote entsprechend zu reagieren.
\end{itemize}

\newpage
\subsubsection{Eigene Vorarbeiten und Expertisen}
\label{sec:EigeneVorarbeitenUndExpertisen}

\noindent
\textbf{Gruppenvorstellung}\\

\noindent
Da einige Inhalte dieser Studie auf einer Gruppenarbeit von 5 Personen beruhen, soll die Expertise aller Gruppenmitglieder hiermit auch berücksichtigt werden.

Alle Teilnehmer der Seminargruppe haben eine abgeschlossene Berufsausbildung, welche sich durch 2 Physiotherapeutinnen sowie 1 Diätassistentin, 1 Operationstechnische Assistentin und 1 Medizinische Fachangestellte näher charakterisieren lässt. Alle Beteiligten haben zum Zeitpunkt der Befragung den Abschluss "`Bachelor of Education"' und streben ihren Masterabschluss für das Höhere Lehramt an berufsbildenden Schulen an. Zudem weisen alle die berufliche Fachrichtung Gesundheit und Pflege im Erstfach auf, jedoch verschiedene Zweitfächer.

Neben den praktischen Lehrerfahrungen aus dem Blockpraktikum A, Blockpraktikum B sowie den Schulpraktischen Übungen im Rahmen der universitären Ausbildung besitzen alle Personen individuelle Erfahrungen mit außerunterrichtlichen Problemlagen durch Mitschüler während der eigenen Ausbildung, aber z. T. auch weitere Erfahrungen durch die Betreuung von Praktikanten und durch eigene Lehrtätigkeiten.

Aufgrund dieser Sachlage sind genügend Situationen vertraut, in welchen die Unterrichtstunde nicht für Lerninhalte, sondern zum Klären von außerunterrichtlichen Problemen genutzt wurde. Das wurde sowohl von Lehrern als auch von nicht betroffenen Schüler als belastend empfunden.

Bekannte Ausbildungsschulen bzw. Praktikumsschulen aus dem Erfahrungsbereich der Gruppenmitglieder wiesen bisher keine sozialpädagogischen Unterstützungsangebote auf, obwohl z. T. subjektiv gesehen Bedarf bestanden hätte.\\

\noindent
\textbf{Vorstellung der Autorinnen}\\

\noindent
Die bisher erfolgten "`gruppenverbindenden"' Angaben sollen hiermit um ein paar individuelle Informationen der Autorinnen erweitert werden.\\

\noindent
\underline{Vorbildung Daniela Wobst}
\begin{itemize}
	\item Erfahrungen als Schülerin im Berufsfeld: 1998-2001 Ausbildung zur Diätassistentin
	\item 2001-2003: Praxisanleiterin für Praktikantinnen in Krankenhaus und Rehaklinik
	\item 2003-2010: Lehrkraft für fachpraktischen Unterricht im Fachbereich Diätassistenz sowie Klassenlehrertätigkeit; QM-Beauftragte des DRK Bildungswerk Sachsen mit Aufgabe der Schülerbetreuung (Klassensprechersitzungen, Zusammenkünfte mit Schülervertretern)
	\item seit 2010: weitere unterrichtliche Tätigkeit (fachpraktischer und theoretischer Bereich) sowie Beginn des Lehramtstudiums an der TU Dresden
	\item seit 2014: Fachbereichsleitung der Berufsfachschule für Diätassistenten am DRK Bildungswerk SN
\end{itemize}

\noindent
\underline{Vorbildung Doreen Stichel}
\begin{itemize}
	\item Erfahrungen als Schülerin im Berufsfeld: 2002-2005 Ausbildung zur Physiotherapeutin
	\item 2006-2012: Praktische Erfahrungen im Bereich der physiotherapeutischen Praxis, der stationären Behandlung, Intensivtherapie sowie in der Fitnessbranche
	\item Erfahrung im Umgang mit Praktikanten und Auszubildenden im Rahmen der beruflichen Tätigkeit
	\item 2010: Aufnahme des Lehramtstudiums an der TU Dresden
\end{itemize}

\newpage
\subsubsection{Arbeitsprogramm (inkl. Zeitplan)}
\label{sec:ArbeitsprogrammInklZeitplan}

\begin{longtable}{l|p{9.8cm}}
	
	\textbf{1. Problemdefinition} & \\
	\emph{Januar 2015} &
	\vspace*{-0.6cm}
	\begin{itemize}[nosep,topsep=-0.6cm]
		\item Bildung der Arbeitsgruppe
		\item Erste Literaturrecherche
		\item Entwicklung der Fragestellung/Zielsetzung
	\end{itemize} \\* 
	
	\emph{Februar 2015} & 
	\vspace*{-0.6cm}
	\begin{itemize}[nosep,topsep=-0.6cm]
		\item Suche nach universitären Betreuern
		\item Vorstellung des Vorhabens bei den Betreuern
		\item Ausformulierung der Forschungsfrage
	\end{itemize} \\
	
	\multicolumn{2}{c}{Kurzzeitige Unterbrechung durch das Blockpraktikum B im März 2015} \\*
	
	\textbf{2. Projektplanung} & \textbf{Klärung Projektbewilligung am DRK Bildungswerk Sachsen} \\
	\emph{April 2015} & 
	\vspace*{-0.6cm}
	\begin{itemize}[nosep,topsep=-0.6cm]
		\item Vorstellung des Forschungsvorhabens bei der Geschäftsleitung
		\item Klären möglicher Fragen bezgl. der geplanten Schülerumfrage und der Lehrerinterviews
		\item Besprechen einer möglichen Ergebnispräsentation nach Fertigstellung der Masterarbeit
	\end{itemize} \\ 
	& \textbf{Ressourcenbestimmung Schüler} \\*
	\emph{Mai 2015} & 
	\vspace*{-0.6cm}
	\begin{itemize}[nosep,topsep=-0.6cm]
		\item Welche Ausbildungsrichtungen wollen wir befragen?
		\item Mögliche Ausschlusskriterien bestimmen
		\item Welche Klassen und Schüler stehen im Juni/Juli am DRK Bildungswerk SN zur Verfügung?
		\item Absprache mit den betreffenden Lehrer zum gegebenen Zeitpunkt
	\end{itemize} \\
	& \textbf{Ressourcenbestimmung Lehrer} \\*
	&
	\vspace*{-0.6cm}
	\begin{itemize}[nosep,topsep=-0.6cm]
		\item Eingrenzen geeigneter Lehrkräfte, um das gewählte Ausbildungsspektrum angemessen zu repräsentieren
		\item Anschreiben der Lehrkräfte bezüglich Forschungsvorhaben und Interviewanfrage
		\item Termine für Interviews festlegen
		\item Interviewer und Beisitzer festlegen
	\end{itemize} \\ 
	& \textbf{Erstellung der Umfragematerialien} \\*
	&
	\vspace*{-0.6cm}
	\begin{itemize}[nosep,topsep=-0.6cm]
		\item Umfragebogen (+ Pretest)
		\item Merkzettel zur Studie
	\end{itemize} \\
	& \textbf{Erstellung der Interviewmaterialien} \\*
	& 
	\vspace*{-0.6cm}
	\begin{itemize}[nosep,topsep=-0.6cm]
		\item Leitfragen des Interviews erstellen (+ Pretest)
		\item Datenschutzerklärung
		\item Einverständniserklärung
		\item Postscript
	\end{itemize} \\
	
	\textbf{3. Projektdurchführung} & \textbf{Datenerhebung} \\*
	\emph{Juni 2015} &
	\vspace*{-0.6cm}
	\begin{itemize}[nosep,topsep=-0.6cm]
		\item Durchführung der Schülerumfrage
		\item Durchführung der Lehrerinterviews I.01 -- I.05
	\end{itemize} \\
	& \textbf{Dateneingabe} \\*
	\emph{Juli 2015} &
	\vspace*{-0.6cm}
	\begin{itemize}[nosep,topsep=-0.6cm]
		\item Dateneingabe der Schülerumfragewerte in LimeSurvey
		\item Transkription der Interviews
	\end{itemize} \\
	& \textbf{Zwischenprojektevaluation} \\*
	\emph{Juli 2015} &
	\vspace*{-0.6cm}
	\begin{itemize}[nosep,topsep=-0.6cm]
		\item regelmäßiger Austausch in der Gruppe über Fortschritte und mögliche Probleme im Forschungsprozess
	\end{itemize} \\
	
	\textbf{4. Evaluation} & \textbf{Datenauswertung} \\*
	\emph{August 2015} &
	\vspace*{-0.6cm}
	\begin{itemize}[nosep,topsep=-0.6cm]
		\item ... der Schülerumfrage mittels Statistikausgabe von LimeSurvey und Gesprächen in der Gruppe
		\item ... der Lehrerinterviews mittels Qualitativer Inhaltsanalyse nach Mayring
	\end{itemize} \\
	
	\textbf{5. Projektbericht} & \\*
	\emph{bis Ende August 2015} &
	\vspace*{-0.6cm}
	\begin{itemize}[nosep,topsep=-0.6cm]
		\item Zusammenfügen aller Daten mit dem Ziel der Beantwortung der Forschungsfragen
		\item konzep. Überlegungen für SSA am DRK BWK SN
		\item Verschriftlichung der Ergebnisse
	\end{itemize} \\
	
	\textbf{6. Abgabe Masterarbeit} & \\*
	\emph{15.09.2015} & \\
	
\end{longtable}

\newpage

\subsubsection{Studiendesign und Methoden}
\label{sec:StudiendesignUndMethoden}

Um den vorliegenden Abschnitt nicht gänzlich mit Inhalten zu überladen, werden die Methoden der Umfrage und des Interview zum Zwecke der besseren Gliederungsmöglichkeit unter Punkt \ref{sec:DieUmfrage} sowie \ref{sec:DasInterview} näher erläutert.

\subsubsection{Erwartungshorizont}
\label{sec:Erwartungshorizont}

Aufgrund einer guten Kommunikation zwischen den Autorinnen und den betreffenden Lehrkräften am DRK Bildungswerk SN im Vorfeld der Befragung wird eine reibungslose Durchführung der Schülerumfrage sowie der Lehrerinterviews erwartet. Da Daniela Wobst aktuell am Bildungsinstitut beschäftigt ist, wird auch die organisatorische Gestaltung in Bezug auf Raumplanung, Bereitstellung von Kopien, kurzzeitige Absprachen mit Interviewpartnern oder das Auffinden der entsprechenden Klassen vermutlich problemlos ablaufen. Doch trotz einer sorgfältigen Planung und Organisation der Untersuchung bleiben Unsicherheiten in Bezug auf eine gelingende Durchführung und der zu gewinnenden Datenmenge mittels der genannten Methoden.

Bezüglich der Schülerumfrage werden folgende Bedenken gehegt: Was geschieht, wenn viele Schüler eine Befragung verweigern?; Wie wird mit fehlerhaft ausgefüllten Umfragebögen umgegangen?; Wie verfährt man, wenn keinerlei Probleme und Unterstützungsbedarfe benannt werden?

Im Hinblick auf die Lehrerinterviews gibt es im Vorfeld ebenso Unsicherheiten: Wie reagiert man auf Desinteresse oder vermeintlich unreflektierte Antworten der Gesprächspartner?; Werden die Lehrkräfte Probleme und Unterstützungsbedarfe für die Schüler wahrnehmen und benennen können? 

\subsubsection{Qualitätssicherung}
\label{sec:Qualitätssicherung}

Folgende Maßnahmen dienen der Qualitätssicherung der Forschungsarbeit:
\begin{itemize}
	\item grundlegende theoretische Fundierung, vor Aufnahme der eigenen Untersuchungen und anschließend zur wissenschaftlichen Einordnung gewonnener Erkenntnisse in bereits vorhandene Wissensstrukturen
	\item inhaltliche oder organisatorische Fragen, Aufgaben oder auch entstehende Probleme werden regelmäßig gemeinsam besprochen, diskutiert, reflektiert und evaluiert (das bezieht sich sowohl auf die gemeinsame Arbeit der Autorinnen als auch auf die partielle Zusammenarbeit mit der Seminargruppe)
	\item bei Fragen oder Problemen wird Rücksprache mit den Betreuern gehalten
	\item im Interviewverlauf wird eine Rückkopplung mit dem Interviewpartnern angestrebt, d. h., inwieweit die Interpretation des Interviewers, den Aussagen des Interviewpartners entspricht
	\item Sowohl vor Beginn der Schülerumfrage als auch vor den Lehrerinterviews wird ein Probelauf (Pretest) erfolgen. Dieser Schritt dient sowohl der Erprobung organisatorischer Abläufe als auch der Überprüfung der erarbeiteten Unterlagen (Umfragebogen; Interviewleitfaden) in Bezug auf Logik, Verständnis, Zeitvorgabe und potentielle Fehler. Zudem werden die Probanden im Anschluss um ein ehrliches Feedback gebeten. Die dabei involvierten Schüler und die Lehrperson werden bei der weiteren Datenerhebung nicht berücksichtigt, um keine Ergebnisverzerrung zu verursachen.
\end{itemize}

\subsection{Die Umfrage}
\label{sec:DieUmfrage}

\subsubsection{Hinführung}
\label{sec:Hinführung}

Nachdem soeben die grundlegende Herangehensweise an das Forschungsprojekt dargelegt worden ist, gilt es nun, den Arbeitsprozess bezüglich Konzeption und Erstellung der Schülerumfrage transparent auszuführen. Dazu soll im Folgenden die wissenschaftliche Methode des Fragebogens parallel zur Konzeption der Schülerumfrage in kurzen Zügen erläutert werden.

Der Fragebogen ist ein Instrument, welches aufgrund seiner inflationären Anwendung in den vergangenen Jahren zunehmend Einzug in das Alltagserleben der Menschen gehalten hat. Er findet Verwendung bei Straßenumfragen, TV-Shows, im Interesse der Kundenakquise bei Energie- oder Handyanbietern, zur Sicherung der Erfolgskontrolle bei Wirtschaftsunternehmen oder auch im Rahmen von Abschlussarbeiten von Studenten und Kommilitonen. Bei dieser mannigfaltigen Präsenz von Fragebögen kommt häufig die systematische und wissenschaftliche Ausarbeitung dieser Methode zu kurz \footcite[vgl.][11]{Kallus2010}. Doch mit dem Anspruch an wissenschaftliches Arbeiten gilt es, bei der Konzeption von Fragebögen gewissen Kriterien zu genügen, um so eine nachvollziehbare Systematik frei von subjektiven Deutungen zu konstruieren \footcite[vgl.][9]{Mayer2013}. Im Rahmen dieser Arbeit wurde versucht, diesen wissenschaftlichen Kriterien im Ansatz gerecht zu werden, wenn auch mangelnde Zeit, geringe Erfahrungswerte der Autorinnen und der exemplarische Charakter der Untersuchung das Unterfangen z. T. erschwerten. Zur Beantwortung der folgenden Forschungsfragen wurde der standardisierte Fragebogen als Erhebungsmethode herangezogen:\\
 

\noindent
\textbf{(1)} Welche persönlichen und sozialen (außerunterrichtlichen) Problemlagen haben Schüler des DRK Bildungswerk SN in sozialen und medizinischen Ausbildungsberufen, die das Unterrichtsgeschehen und den Ausbildungserfolg beeinflussen?\\

\noindent
\textbf{(3)} Wie schätzen Schüler (Anm: Lehrkräfte sind hier ausgenommen) anhand der (möglichen) subjektiv wahrgenommenen Problemlagen den Bedarf an sozialpädagogischen und anderweitigen Unterstützungs- und Beratungsangeboten ein und wie könnten passende Angebote aussehen?\\

\noindent
Zunächst wird die relevante Stichprobe ermittelt. Die Grundgesamtheit (also alle zu befragenden Schüler am DRK Bildungswerk SN) betrug zum Zeitpunkt der Umfragekonzeption 679 Schüler. Aus zeitlichen Gründen war es nicht möglich alle Personen zu befragen. Die eingehende Literaturrecherche ergab unterschiedliche bzw. uneindeutige Angaben hinsichtlich der passenden Stichprobengröße; daher wurde im anschließenden Gespräch mit Universitätsangehörigen, in Person von Prof. Dr. Gängler, Fr. Haase und Hr. Bloße, über eine passende Anzahl an Probanden diskutiert. 30\% wurden schließlich als repräsentative Stichprobe festgelegt. Zudem muss man anmerken, je größer die gewählte Stichprobe, desto größer die Annäherung an die wahre Grundgesamtheit und desto weniger Fehler werden in Kauf genommen \footcite[vgl.][65f]{Mayer2013}. 

Nun gilt es festzulegen, welche Personen bzw. Klassen für die Befragung herangezogen werden. Ziel ist es, "`ohne große Fehler Fehler zu machen [\punkte] von der Stichprobe auf die Grundgesamtheit zu schließen."'\footcite[60]{Mayer2013}. Das bedeutet, alle relevanten Merkmale der Grundgesamtheit von Geschlecht, Ausbildungsgang, über unterschiedlich vorliegende Probleme und Unterstützungsbedarfe müssen von der Stichprobe auf die Grundgesamtheit schließen lassen können. Die Stichprobe ist somit ein verkleinertes Abbild der Grundgesamtheit \footcite [vgl.][197]{Kromrey1995}. Daher werden aus allen Ausbildungsgängen je 30\% der Schüler befragt. Die genaue Anzahl der Schüler ist der folgenden "`Konzeption der Schülerbefragung"' zu entnehmen, welche im Vorfeld der Umfrage u. a. auch mit den Gutachtern besprochen wurde und zum Zwecke der vorliegenden Arbeit nur leicht verändert wurde.

Im weiteren Verlauf wurde die Zufallsauswahl (Random-Verfahren) als zufallsgesteuertes Auswahlverfahren festgelegt, da so die Repräsentativität durch das Verfahren selbst gewährleistet wird \footcite[vgl.][60]{Mayer2013}. Wie bereits erwähnt, werden bei der Umfrage stets ganze Klassen befragt. Die zu berücksichtigenden Fragebögen werden anschließend von einer neutralen Person aus der Gesamtheit der Bögen entsprechend des Ausbildungsganges gezogen. Diese vorherige Unterteilung in Teilgruppen (hier in Form von Fachrichtungen) wird auch als Klumpenauswahl (Cluster Sampling) bezeichnet \footcite[vgl.][63]{Mayer2013}.

\newpage
\subsubsection{Konzeption der Schülerbefragung}
\label{sec:KonzeptionDerSchülerbefragung}

\textbf{Allgemeines}
\begin{itemize}
	\item Vorstellung des DRK Bildungswerk SN: siehe Punkt \ref{sec:BerufsbildendeSchulenInSachsenEineExemplarischeVorstellungAmDRKBildungswerkSachsen} (dementsprechend wird hier auf eine erneute Darstellung der Ausbildungsgänge verzichtet)
	\item Forschungsziel ist die Befragung einer Schülermenge, welche die Grundgesamtheit der Schülerschaft am DRK Bildungswerk SN widerspiegelt.
	\item Befragt werden soll mit einer Zufallsauswahl (Random-Verfahren).
	\item Genaues Verfahren: Klumpen-Auswahl (Cluster Sampling), d.h. die Aufteilung der Grundgesamtheit in Teilgruppen (in diesem Fall Fachrichtungen) 
	\item Begründung: Diese Auswahl repräsentiert unserer Meinung nach die verschiedenen Fachrichtungen am besten, die sich hinsichtlich der Zugangsvoraussetzungen, Altersstruktur und der Ausbildungsanforderungen deutlich unterscheiden.
\end{itemize}

\textbf{Beschreibung der Stichprobenauswahl}
\begin{itemize}
	\item Aktuell gibt es am DRK Bildungswerk SN insgesamt 679 Schüler. Dies ist etwas weniger als zunächst angenommen, da in der Rettungsassistenz und in der Altenpflege im März 2015 insgesamt 4 Klassen verabschiedet wurden.
	\item Ausschlusskriterium: Die berufsbegleitenden Klassen werden bei der Befragung nicht berücksichtigt, da sie nur einen Tag pro Woche zum Unterricht in der Schule sind und sich ihre Altersstruktur sowie Zusammensetzung von den Vollzeitklassen deutlich unterscheiden. Hinzu kommt, dass potentielle Unterstützungsangebote durch sie kaum bis gar nicht genutzt werden könnten.
	\item Abzüglich der berufsbegleitenden Schüler verbleiben 578 Schüler.
	\item Befragt werden sollen nach Absprache mit o.g. Dozenten 30\% der Grundgesamtheit. Das ergibt eine zu befragende Stichprobe von 175 Schülern.
	\item 578 Schüler gesamt -- Stichprobe ergibt 175 Schüler mit folgender Aufteilung (gewichtet nach der Verteilung der jeweiligen Fachrichtung, davon jeweils 30\%). Dadurch erhalten wir ein relativ realistisches Bild der Gesamtgruppe. 
		\begin{itemize}
			\item 32 Personen aus der Altenpflege 
			\item 3 Personen aus der Diätassistenz
			\item 68 Personen aus der Erzieherausbildung 
			\item 13 Personen aus der Heilerziehungspflege 
			\item 21 Personen aus der Physiotherapie
			\item 25 Personen aus der Rettungsassistenz und Notfallsanitäter
		\end{itemize}
	\item Trotz der zum Teil geringen Personenzahl werden jeweils ganze Klassen befragt. Damit entfällt eine komplizierte Vorauswahl, nach welchen Kriterien die Antworten der Schüler berücksichtigt werden und es erfolgt keine Selektion von ausgewählten Schülern. Zusätzlich entfallen ganz praktische Probleme wie z.B. Was passiert, wenn von den Vorausgewählten am Befragungstag nur ein Teil anwesend ist? Zudem können jeweils ganze Klassen persönlich eingewiesen und motiviert werden.
	\item Für die Bedarfsanalyse unberücksichtigt bleiben Geschlecht, Alter und Schulabschluss. Diese Daten werden jedoch für mögliche statistische Auswertungen oder weiterführende Studien bzw. zur Fehleranalyse mit aufgenommen.
	\item Die Fragebögen werden jeweils in die ganze Klasse gegeben, anonym ausgefüllt und aus dem Klassensatz werden dann zufällig die auszuwertenden Bögen, in Höhe der jeweiligen errechneten Stichprobengröße gezogen. Zur Wahrung der Durchführungsobjektivität erfolgt dieser Vorgang durch eine neutrale Person, die dem Projekt nicht angehörig ist. 
	\item Die ausgewählten Bögen werden ausgewertet (LimeSurvey).
	\item Zusätzlich zur Schülerbefragung werden Leitfadeninterviews mit 5 ausgewählten Lehrkräften (Klassenlehrern), darunter die Vertrauenslehrerin, durchgeführt. Diese Personen haben aufgrund ihrer Lehrtätigkeit Erfahrungen in den verschiedensten Berufsfeldern der Gesundheit und Pflege bspw. in der Physiotherapie, Krankenpflegehilfe, Heilerziehungspflege, Erzieher, Altenpflege aber auch bei den Notfallsanitätern und Rettungsassistenten.
	\item Die Interviews erweitern die Informationen hinsichtlich wahrgenommener Problemlagen und möglicher Unterstützungsangebote für Schüler.
\end{itemize}

\subsubsection{Inhaltliche Konzeption}
\label{sec:InhaltlicheKonzeption}

Aufgrund persönlicher und beruflicher Erfahrungen im Umgang mit Schülern entschied man sich einen relativ leicht verständlichen, übersichtlichen und knappen Fragebogen zu erstellen. Nur so ist, aus subjektiven Erfahrungen abgeleitet, zu gewährleisten, dass alle Schüler, unabhängig von Alter und Vorbildung, die Fragebögen verstehen und bearbeiten können und nur so ist der Einsatz im laufenden Unterricht, ohne organisatorische Schwierigkeiten und mit Einverständnis der Lehrer, effektiv möglich. Alle Überlegungen zur inhaltlichen Ausgestaltung des Fragebogens erfolgten während der andauernd-kommunikativen Zusammenarbeit der Autorinnen. Die komplette Schülerumfrage ist dem Anhang unter Punkt \ref{sec:Schülerumfrage} zu entnehmen.

Zu Beginn der Umfrage werden die Schüler von Doreen Stichel bzw. Daniela Wobst bezüglich Sinn, Zweck, Aufbau und Zielstellung aufgeklärt und eingewiesen; dennoch beginnt die Umfrage selbst mit einem erläuternden Textabschnitt, welcher die Zielstellung der kommenden Fragen erneut klar benennt. Der grundlegende Aufbau der Umfrage gliedert sich in 3 Abschnitte, wobei Abschnitt 1: 3 Fragen; Abschnitt 2: 8 Fragen und Abschnitt 3: 4 Fragen beinhaltet. Die Schüler beantworten somit 15 Fragen, wobei der Fokus auf dem Abschnitt 2 liegt. Dieser dient der Ermittlung der persönlichen Problemlagen und thematisiert die Annahme außerunterrichtlicher Unterstützungsangebote. 

Der \textbf{erste Abschnitt} umfasst die allgemeinen Angaben zur Person. Die darin erfragten Inhalte wie Geschlecht, Alter und der höchste bisher erreichte Bildungsabschluss dienen nicht der Beantwortung der Forschungsfragen, sondern werden eher aus statistischen Gründen und persönlichem Interesse erfragt. Zudem haben diese Fragen einen aufschließenden und motivierenden Charakter zu Beginn der Umfrage. Die Angaben zur Person werden in der Ergebnisdarstellung erwähnt, erfahren jedoch keine weitere Interpretation, da sie keine Relevanz für das Forschungsinteresse haben.

Der \textbf{zweite Abschnitt} repräsentiert den Kern des Forschungsinteresses in Bezug auf die Schülerperspektive, da hier die Ermittlung der persönlichen Problemlagen und die potentielle Annahme von außerunterrichtlichen Unterstützungsangeboten ermittelt wird. Im Anschluss an die grundlegende Frage, inwieweit Schüler früher oder aktuell von ausbildungsbeeinflussenden Problemlagen betroffen waren, erfolgt die Differenzierung dieser potentiell vielschichtigen Problemlagen. Den Schülern wird neben der Nennung verschiedener Problemlagen auch die Möglichkeit eingeräumt, keine (ausbildungsbeeinflussenden) Probleme zu haben. Wie bereits unter Punkt \ref{sec:ProblemlagenVonSchülernAnBerufsbildendenSchulen} dargelegt, haben Berufsschüler z. T. ähnliche Probleme wie Schüler an allgemeinbildenden Schulen. Dennoch weichen viele Problemlagen aufgrund der heterogenen Struktur in berufsbildenden Klassen, beispielhaft erkennbar an der Altersspanne, unterschiedlicher Vorbildung und Familienstand, z. T. stark von denen "`klassischer"' Schüler ab. Mit Hilfe eingehender Literaturrecherche, die sich in Bezug auf Problemlagen an Berufsschulen recht bedeckt hält und persönlicher Erfahrungen im Lehrberuf hat man sich mit Unterstützung einer Expertenrunde am DRK Bildungswerk SN auf 19 Problemlagen beschränkt, die das Schulerleben zumeist negativ beeinflussen. Aus subjektiver Sicht decken diese Bereiche einen Großteil der aktuellen Problemlagen und Anforderungsbereiche von Berufsschülern ab; ohne Anspruch auf  Vollständigkeit. Um allen Schülern einen leichten Zugang zur Beantwortung der Frage(n) zu ermöglichen, wurde auf eine komplizierte Benennung der Problemlagen durch Fachvokabular verzichtet und eine alltagssprachliche Formulierung gewählt.\\

\newpage
\noindent
Die gewählten Begriffe gilt es nun möglichst genau und eindeutig zu bestimmen:\\

\noindent
\underline{Arbeitsdefinitionen:}
\begin{enumerate}
	\item \textbf{Konflikte in der Klasse:} Eine Unterrichtsklasse vereinigt Menschen mit unterschiedlichen Persönlichkeiten, welche verschiedenen Meinungen und Interessenlagen hervorbringt. Dieser Umstand kann zu Konflikten in Form von Meinungsverschiedenheiten führen, die verbal wie körperlich ausgefochten werden und unterschiedlichste Ausmaße annehmen können. Diese Belastung kann zu einer Beeinträchtigung einzelner oder mehrerer Personen führen.
	\item \textbf{Mobbing:} Mobbing umschreibt das andauernde Schikanieren von Mitschülern, was seelische Verletzungen zur Folge haben kann. Es kann direkt und offen, aber auch über soziale Netzwerke wie Facebook ausgeübt werden.
	\item \textbf{Diskriminierung:} Diskriminierung in der Klasse umschreibt das Schikanieren von Mitschülern, welche ein vermeintlich "`ausgrenzendes"' Merkmal besitzen. Das könnte z. B. die ethnische Herkunft, Religionsangehörigkeit, sexuelle Orientierung oder eine bestimmte Lebensweise sein.
	\item \textbf{Gewalt in der Schule:} Dieser Punkt umschreibt jegliche körperliche Gewalt gegen Mitschüler z.B. in Form von treten, schubsen, schlagen etc. und beeinträchtigt das Schulerleben stets negativ.
	\item \textbf{Probleme aufgrund von un-/entschuldigten Fehlzeiten:} Fehlzeiten kommen aufgrund verschiedenster Konstellationen zustande. Diese beinhalten bspw. schulische Demotivation, Verschlafen, Probleme mit öffentlichen Verkehrsmitteln, familiäre Verpflichtungen (z. B. Kinder in die Kita), berufliche Überschneidungen (wenn Schüler einen Nebenjob ausüben (müssen)), lange Anfahrtswege uvm. Bei gehäuftem Auftreten dieser entschuldigten oder auch unentschuldigten Fehlzeiten kann es zu Problemen mit der Ausbildung kommen, da hier eine Höchstfehlzeit festgeschrieben und somit eine Gefahr der Versetzung oder des Abschlusses möglich ist.
	\item \textbf{Krankheitsbedingte Probleme:} Schüler können, unabhängig von ihrem zumeist jungen Alter, eine Vielzahl an Krankheiten aufweisen. Dieses persönliche Problem führt meist zu weiteren Problemen wie Fehlzeiten, Konflikten in der Klasse oder zu Schwierigkeiten im Praktikum. Diese können somit die Ausbildung und das Schulerleben erschweren.
	\item \textbf{Konflikte im häuslichen Umfeld bzw. familiäre Probleme:} Jeder Schüler entstammt unterschiedlichen familiären Strukturen. Diese können bezgl. menschlicher Wärme, Unterstützung, Alltagsstrukturen, Konflikten und Anforderungen z. T. stark voneinander abweichen. Diese Probleme haben ebenfalls Zugang zum Schulgeschehen. 
	\item \textbf{Überhöhter Alkoholkonsum:} Die meisten Berufsschüler befinden sich am Übergang vom Jugendlichen zum Erwachsen, mitsamt der einhergehenden neuen Herausforderungen und Möglichkeiten. Viele Schüler testen dabei ihre Grenzen aus und treten auch über diese hinweg. Der Konsum von Alkohol oder auch Drogen führt jedoch meist zu weiteren Problemen wie Fehlzeiten, Versetzungsgefährdung (durch schlechte Leistungen) oder auch zu Konflikten in der Klasse. Alkohol und Drogen sind an Schulen grundlegend nicht gestattet; daher drohen bei Nachweis Disziplinarstrafen oder auch die Kündigung.
	\item \textbf{Drogenkonsum:} siehe 8
	\item \textbf{Mediale Süchte (TV, Spiele, Smartphone, soziale Netzwerke etc.):} Der Gebrauch von modernen Kommunikationstechnologien gehört für viele Schüler zum normalen Alltag. Die massive Nutzung dieser Techniken kann jedoch zu Suchtverhalten führen, was wiederum andere Probleme wie Fehlzeiten oder Demotivation hervorrufen kann.
	\item \textbf{Finanzielle Probleme/Schulden:} Ähnlich wie bereits unter Punkt 8 angeführt, befinden sich die meisten Schüler in einer Art Übergangsphase vom Jugendlichen hin zum Erwachsenen in der Grenzen ausgetestet und z. T. überschritten werden; der Umgang mit Geld ist ein Lernprozess, der ebenfalls zu Schwierigkeiten führen kann. Es gibt aber auch weitere beispielhafte Ursachen für Geldprobleme: Jugendliche können armen Verhältnissen entstammen und haben keinen direkten Zugang zu Geld; andere haben aufgrund übersteigerter Ansprüche keine finanziellen Ressourcen und weitere haben anderweitige Verpflichtungen wie z. B: eigene Kinder. Zudem ist anzumerken, dass diese vollzeitschulische Ausbildung Schulgeld kostet und das Gros der Schüler über kein oder wenig "`Lehrgeld"' verfügt.
	\item \textbf{Probleme in der praktischen Ausbildung/Praktikum:} Diese Probleme können vielseitig sein und sich beispielsweise wie folgt äußern: es gibt persönliche Differenzen mit Ausbildern, konträre Auffassungen von Praktikumsinhalten zwischen Schüler und Ausbildungsstätte, unangemessene Anforderungen im Praktikum oder auch physische wie psychische Belastungssituationen. Einige Schüler stellen bei Problemen in der praktischen Ausbildung ihre generelle Ausbildung infrage und gefährden einen Abschluss.  
	\item \textbf{Probleme bei der Wissensaneignung/fehlende Lernstrategien:} Manche Schüler haben nie gelernt, wie man lernt und haben daher Probleme bei der Wissensaneignung und somit mit der Ausbildung.
	\item \textbf{Fehlende Motivation in Bezug auf Schule/Ausbildung:} Schüler können aus den verschiedensten Gründen heraus für die Ausbildung ungenügend motiviert sein: einige wurden zu dieser Ausbildung überredet; manche hatten ggf. eine falsche Vorstellung vom Berufsbild; andere sind durch mangelnde Erfolgserlebnisse demotiviert und weitere haben aktuell mehr Interesse an anderen Lebensbereichen.
	\item \textbf{Versetzungsgefährdung und/oder schlechte Leistungen, die den Ausbildungserfolg gefährden:} Bedingt durch andere Probleme kann es zu schlechten schulischen Leistungen kommen, die einem erfolgreichen Abschluss entgegenstehen. 
	\item \textbf{Überforderung in der Ausbildung (zu hohes fachliches Anforderungsniveau):} Schüler sind mit dem fachlichen Anforderungsniveau bspw. mit Stoffmenge oder Fachvokabular überfordert.
	\item \textbf{Überbelastung (vielfältige Aufgaben und Anforderungen in der Ausbildung sind trotz Bemühen nicht zu bewältigen):} Schüler können die vielfältigen Anforderungen in der Ausbildung trotz Bemühen nicht bewältigen, beispielsweise in Form von Praktika, regelmäßigem Lernen, Klausuren, Hausaufgaben und Präsentationen, welche gegebenenfalls parallel zu weiteren individuellen Anforderungsbereichen im Privatleben erledigt werden müssen.
	\item \textbf{Prüfungsangst:} Prüfungsangst beinhaltet die (Versagens-)Angst vor dem Nichtbestehen von Prüfungsleistungen, die für den erfolgreichen Abschluss der Ausbildung notwendig sind. Diese Angst kann weitere Ängste generieren wie Zukunftsangst, finanzielle Sorgen, familiäre Konsequenzen etc.
	\item \textbf{Zukunftsängste in Bezug auf Arbeitsplatz:} Arbeitslosigkeit und Hartz IV sind aktuelle Themen, denen sich Jugendliche frühzeitig stellen müssen. Der erfolgreiche Abschluss wird daher nicht unbedingt mit einer Anstellung und finanzieller Absicherung assoziiert, sondern wirft eher neue Fragen und Sorgen auf.
\end{enumerate}

\noindent
Nachdem die Schüler ihre potentiellen Probleme per Ankreuzen dargelegt haben, geht es im Weiteren um die Inanspruchnahme möglicher Unterstützungsangebote für die betreffende Person. Es wird zunächst erfragt, ob Gespräche mit bestimmten Personen in der zurückliegenden Ausbildungszeit in Anspruch genommen wurden. Da Personen aus dem Familien- oder Freundeskreis oftmals die ersten Ansprechpartner für betroffene Schüler darstellen, werden diese neben schulischen Personen wie Klassenlehrer, Schulleiter und anderen Lehrkräften ebenso zur Auswahl gestellt. Im folgenden Punkt wird das Augenmerk auf die Inanspruchnahme des Vertrauenslehrers gelegt. Die Schüler weisen mit ihrer Beantwortung darauf hin, ob dieses bestehende Konzept der Unterstützung am DRK Bildungswerk SN angenommen wird und weisen gleichzeitig mögliche Gründe für eine Nichtinanspruchnahme auf. Neben der Ermittlung der Problemlagen hat es sich diese Studie zur Aufgabe gemacht, etwaige Unterstützungsbedarfe zu eruieren und darauf gestützt mögliche weiterführende Beratungs- und Unterstützungsbedarfe am DRK Bildungswerk SN zu konzipieren bzw. anzuregen. Um den Erfolg möglicher Angebote vorab einschätzen zu können, wird im nächsten Punkt ermittelt, ob Schüler überhaupt solche Angebote bei Bedarf in Anspruch nehmen würden. Darauf aufbauend schließt der zweite Abschnitt der Schülerumfrage mit der Frage nach dem Personenkreis, von welchem ein Unterstützungsangebot angenommen werden würde, und wie die Rahmenbedingungen dieser Angebote aussehen sollten, auch wenn man selbst diese vielleicht nicht nutzen würde. Mit dem zweiten Abschnitt können so die subjektiven Problemlage und individuellen Beratungs- und Unterstützungsangebote der Schüler erfasst werden. 

Im Anschluss erfolgte der \textbf{dritte und letzte Abschnitt} der Befragung. Dieser soll die subjektiv wahrgenommenen allgemeinen Probleme und Unterstützungsbedarfe der Mitschüler am DRK Bildungswerk SN erfassen. Die 4 Fragen, die dafür herangezogen werden, sind den Schülern im Wortlaut bereits bekannt, da sie bereits im zweiten Abschnitt für die persönliche Erfassung von Problemen und Bedarfen genutzt wurden. Die Schüler sollen jedoch jetzt den allgemeinen Bedarf für Beratungs- und Unterstützungsangebote für Mitschüler am DRK Bildungswerk SN benennen und anschließend die Problemlagen aufzeigen, für die es Angebote geben sollte. Zum Schluss wird erneut der Personenkreis erfragt, der diese Angebote bereitstellen sollte, aber auch die passenden Rahmenbedingungen eruiert. Ziel ist, das Selbst- und Fremdbild der Schüler zu überprüfen und so ein möglichst realistisches Bild der Probleme und Bedarfe aller Schüler zu gewinnen. Seitens der Autorinnen wird der Verdacht gehegt, dass viele Schüler gegebenenfalls keine eigenen Probleme und Bedarfe benennen, diese aber verstärkt für ihre Mitschüler aufzeigen. Es wäre ebenfalls denkbar, dass eigene Probleme und Bedarfe auf andere Schüler projiziert werden, da man sich diese persönlich nicht eingestehen möchte. Dieser Aspekt ist aber im Rahmen der aktuellen Untersuchung nicht aufzuklären und bleibt hypothetisch. 

Die Erstellung des Fragebogens war aufgrund der bisher geringfügigen Erfahrungswerte in diesem Wissenschaftszweig eine echte Herausforderung für die Autorinnen und unterlag einem länger andauernden Prozess, wobei die Ergebnisse regelmäßig verworfen und neu überdacht wurden.
 
Bei der Formulierung der Fragen, in Bezug auf Verständlichkeit und Komplexität, wurde versucht, die unterschiedlichen Voraussetzungen der zu befragenden Schüler zu berücksichtigen. Zudem wurden die folgenden 5  Aspekte bei der Formulierung der Fragen bedacht, welche dem Werk von Kallus zu entnehmen sind \footcite[vgl.][63-66]{Kallus2010}. Diese sollen hier kurz vorgestellt werden:

\begin{itemize}
	\item \textbf{Lesbarkeit und klares Design:} Die Aussagen sollten gut lesbar und als Einheit erkennbar sein. 
	\item \textbf{Verständlichkeit:} Die Aussagen sollten durch eine klare und eindeutige Wortwahl leicht verständlich formuliert werden. Es ist auf eine neutrale Formulierung zu achten; keine Suggestivfragen!
	\item \textbf{Einfache Beantwortbarkeit:} Die Fragen sollten leicht zu beantworten sein und keine komplexen Gedankenwege von den Schülern verlangen. Komplexe Sachverhalte sind auf ein einfacheres Niveau "`hinunterzubrechen"'.
	\item \textbf{Neutraler Bezug zum Personen-Lebensumfeld/-kontext:} Die Aussagen sollen einen Bezug zum Erleben und Verhalten der antwortenden Personen herstellen. Dem wurde mittels Einleitung sowie einleitenden Sätzen bei entsprechend komplexerer Fragen entsprochen.
	\item \textbf{Eindeutigkeit und Klarheit:} Fragen sollten möglichst so formuliert werden, dass sie auch ohne andere Fragen aus dem Kontext zu beantworten wären.
\end{itemize}

\noindent
Zum Schluss überlegte man sich verschiedene Wege der Umfragedurchführung -- zur Diskussion standen die Online-Umfrage und die klassische Umfrage auf Papier. Aufgrund des begrenzten Zeitfensters und negativer Erfahrungen im Vorfeld in Bezug auf die aktive Teilnahme an Online-Umfragen am DRK Bildungswerk SN wurde diese Idee zugunsten der Umfrage auf Papier entschieden.

\noindent
\textbf{Merkzettel zur Studie}\\

\noindent
Ergänzend zur Umfrage wurde ein kleiner Merkzettel für die Schüler erstellt. Trotz einer klaren Einweisung und Belehrung vor Beginn der Umfrage soll so auch nach Beendigung der Umfrage die Möglichkeit gegeben sein, sich an die für das Projekt verantwortlichen Personen wenden zu können (Hinweis: Dieses Angebot wurde bis zur Abgabe der vorliegenden Arbeit nicht in Anspruch genommen.). Neben grundlegenden Informationen zur Umfrage werden auch die Kontaktdaten von Daniela Wobst und Doreen Stichel auf dem Zettel vermerkt. Dieses Schriftstück in A5-Größe wird an jeden beteiligten Schüler ausgegeben und ist dem Anhang beigefügt (siehe Punkt \ref{sec:Schülerumfrage}). \\

\noindent
\textbf{Pretest}\\

\noindent
Ein entwickeltes Instrument, in diesem Fall die Schülerumfrage, sollte vor dem konkreten Einsatz getestet (pilotiert) werden. Diese Pilotierung dient der Eignungsfeststellung bei der Zielgruppe, deckt etwaige Verständnisprobleme auf und dient der Erprobung des später geplanten Ablaufs. Dieser Testdurchlauf sollte bei einer Probandengruppe aus der Grundgesamtheit erfolgen \footcite[vgl.][275]{Krueger2014}. Aus diesem Grund wurde vor Beginn der Schülerumfrage ein Pretest, im  Sinne des wissenschaftlichen Arbeitens, von den Autorinnen durchgeführt. Die entsprechenden Schüler wurden zufällig ausgewählt und hatten vor Testbeginn keine Vorkenntnis, warum dieser kurzfristige Termin anberaumt wurde. Zu Beginn wurden die 10 Schüler über Sinn und Zweck der Untersuchung aufgeklärt und begannen anschließend mit der Umfrage. Im Anschluss an die Bearbeitungszeit wurden die Fragebögen nicht sogleich eingesammelt, da die Schüler mit Hilfe des Bogens nun Anmerkungen und Feedback an die Autorinnen richten konnten. Dabei wurden inhaltliche und optische Bestandteile des Fragebogens besprochen, Verständnisprobleme erfragt aber auch das subjektive Interesse an den Inhalten diskutiert. Aufgrund des mehrheitlichen positiven Feedbacks wurden im Anschluss keine tiefgreifenden Änderungen an der Umfrage vorgenommen. Es erfolgten lediglich kleinere Anpassungen bei der Formulierung der Fragestellungen und Ergänzungen zu Antwortmöglichkeiten. Die geplanten 20 Minuten Bearbeitungszeit konnten nach dieser Untersuchung auf 10-15 Minuten reduziert werden. Dieser Pretest hatte somit einen sehr konstruktiven und motivierenden Charakter und bestärkte die Autorinnen für die reale Umsetzung dieses Instrumentes.

\subsection{Das Interview}
\label{sec:DasInterview}

Wie bereits eingangs erwähnt, ist ein Teil der Materialien, welche für die vorliegende Masterarbeit herangezogen wurde, im Rahmen einer universitären Gruppenarbeit in der Fachrichtung Gesundheit und Pflege unter Leitung von Fr. Thümmler erarbeitet worden. Die im Folgenden beschriebene Erarbeitung des Interviewleitfadens für die Befragung der Lehrkräfte am DRK Bildungswerk SN fand durch diese fünfköpfige Arbeitsgruppe statt. Diese Zusammenarbeit währte mehrere Wochen und kann daher nur an wesentlichen Schritten verdeutlicht werden.

\noindent
Grundlegendes Ziel der Interviews ist die Beantwortung der folgenden Forschungsfragen:\\

\noindent
\textbf{(2)} Welche persönlichen und sozialen (außerunterrichtlichen) Problemlagen von Schüler nehmen Lehrkräfte des DRK Bildungswerk SN im Bereich Gesundheit, Pflege und Sozialwesen als besondere Belastung für den Unterricht wahr?\\

\noindent
\textbf{(3)} Wie schätzen Lehrkräfte (Anm.: Schüler sind hier ausgenommen) anhand der (möglichen) subjektiv wahrgenommenen Problemlagen den Bedarf an sozialpädagogischen und anderweitigen Unterstützungs- und Beratungsangeboten ein und wie könnten passende Angebote aussehen?\\

\noindent
Um die subjektive Wahrnehmung der Lehrkräfte hinsichtlich der Problemlagen ihrer Schüler und deren Auswirkungen auf den Schulalltag beschreiben zu können, wurde als Studiendesign die qualitative Einzelfallstudie gewählt. Da die vorliegende Arbeit die Perspektive der Schüler und Lehrer am DRK Bildungswerk SN untersucht, bildet der Forschungsschwerpunkt eine Schnittstelle zwischen Berufsbildungs- und Sozialforschung.

Als Erhebungsmethode zur Datengewinnung wurde das leitfadengestützte Interview ausgewählt. Im Gegensatz zum narrativen Interview, welches ein offenes Erzählverfahren darstellt, werden durch das leitfadengestützte Interview konkrete Aussagen über einen Gegenstand zum Ziel der Datenerhebung \footcite[vgl.][37]{Mayer2013}. Kennzeichnend für ein Leitfadeninterview sind offen formulierte Fragen, denen der Befragte frei antworten kann. Durch den konsequenten Einsatz des Leitfadens können Gespräche vorab strukturiert und daraus resultierend eine Vergleichbarkeit der Ergebnisse der Einzelinterviews ermöglicht werden \footcites[vgl.][112]{Flick1999}[vgl.][376f]{Friebertshaeuser1997}. Dennoch offeriert diese Form der Erhebung dem Interviewer einen gewissen individuellen  Freiraum in der Befragung. "`Diese Einzelentscheidungen, die nur in den Interviewsituationen selbst getroffen werden können, verlangen vom Interviewer ein großes Maß an Sensibilität für den konkreten Interviewverlauf und für den Interviewten."' \footcite[113]{Flick1999} 

Die Stichprobe der Interviewpartner wurde in einer Vorab-Festlegung "`absichtsvoll"' begründet. D. h. die Personen wurden nach bestimmten Kriterien aus der Grundgesamtheit der zur Verfügung stehenden Lehrkräfte festgesetzt \footcite[vgl.][39]{Mayer2013}. Wie bereits in der Unterrichtsplanung erwähnt, stellte sich eine größere Anzahl von Lehrkräften per E-Mail freiwillig zur Verfügung an diesen Interview teilzunehmen. Die Autorinnen entschieden sich für 5 Personen, die Unterschiede in Geschlecht, Alter, Lehrerfahrung und Fachrichtungszugehörigkeit aufwiesen. Trotz der geringen Probandenzahl sollte so ein möglichst breites Erfahrungswissen der Lehrkräfte in Bezug auf Problemlagen und Unterstützungsbedarfe der Schüler erschlossen werden. Ungeachtet der Bestrebungen der Autorinnen war eine größere Stichprobenerhebung in der Kürze der Zeit nicht zu realisieren.

Aufgrund der o. g. spezifisch formulierten Forschungsfragen waren Bestandteile des Leitfadens vorab bereits festgelegt. Ziel der Befragung war die umfassende Berücksichtigung des zu behandelnden Realitätsausschnittes in Form der Erfassung von Problemlagen und Unterstützungsbedarfen der Schüler. Daran anknüpfend galt es nun, Themenkomplexe zu konzipieren, denen Nachfrage-Themen zugeordnet sind \footcite[vgl.][45]{Mayer2013}. Es war eine Herausforderung, den Leitfaden unter Beachtung aller interessierenden Themen nicht zu überladen, da sonst eine nicht zu bewältigende Fülle von Daten die Folge gewesen wäre. In einem längeren Denk- und Arbeitsprozess wurden folgende Fragen und etwaige Unterfragen erarbeitet (siehe Punkt \ref{sec:Lehrerinterviews}). Diese sollen nun in Kurzform erläutert werden:

\begin{enumerate}
	\item Eröffnungsfrage: Beschreiben Sie eine Klasse die Ihnen jetzt spontan einfällt.
	\begin{itemize}
		\item Diese Frage dient als freier Einstieg, um die Lehrkraft für die weitere Befragung zu öffnen und zu motivieren. Zudem können bereits hier Hinweise auf Probleme der Schüler genannt werden z. B. aufgrund der Heterogenität in der Klasse oder in Bezug auf Mitarbeit.
	\end{itemize}
	\item Welche Problemlagen nehmen Sie hauptsächlich als ausbildungsbeeinflussend bei Ihren Lernenden wahr?
	\begin{itemize}
		\item Hier werden die konkreten Problemlagen der Schüler in Erfahrung gebracht, welche zur Beantwortung der Forschungsfragen benötigt werden.
	\end{itemize}
	\item Wie und in welchem Umfang beeinflussen die Schülerprobleme Ihren Unterricht?
	\begin{itemize}
		\item Durch diese Frage kann eruiert werden inwieweit die benannten Probleme der Schüler Einfluss auf die Ausbildung haben und damit auch auf den Arbeitsalltag des Lehrers. Die gewonnenen Ergebnisse untermauern die ausbildungsbeeinflussende Gestalt der genannten Problemlagen.
		\item Sollten hier wenige Anmerkungen generiert werden, sind Nachfragen möglich wie z. B. "`Wie war Ihr persönlicher Umgang mit den Erlebnis(sen)?"'; "`Gibt es besonders prägnante Erinnerungen an bestimmte Problemlagen?"'
	\end{itemize}
	\item Mit welchen Erwartungen kommen die SchülerInnen auf Sie zu?
	\begin{itemize}
		\item Aussagen die hier gewonnen werden, sind nicht primär für die Beantwortung der Forschungsfragen notwendig, offerieren jedoch erste Unterstützungsbedarfe, die Schüler an ihre Lehrer herantragen oder von ihnen einfordern.
		\item Eine Nachfrage wäre inwieweit die Lehrkräfte um konkrete Hilfe von den Schülern gebeten werden oder ob sich diese eher "`Dinge von der Seele reden"' wollen.
	\end{itemize}
	\item Inwieweit sehen Sie sich persönlich den Anforderungen der Schüler an Sie gewachsen? 
	\begin{itemize}
		\item Unterfrage: Wie lassen sich diese Anforderungen mit Ihrem Selbstverständnis/Rollenverständnis als Lehrkraft in Einklang bringen?
		\item Aussagen die für diesen Themenkomplex benannt werden, geben Auskunft über einen möglichen externen Unterstützungsbedarf falls sich die Lehrkräfte den Anforderungen der Schüler nicht gewachsen fühlen. Zudem ermöglicht es einen Einblick in das (Vertrauens-) Verhältnis zwischen Lehrer und Schüler. Mögliche Unterstützungsbedarfe für Lehrkräfte können hier ebenso eruiert werden, stehen jedoch nicht im Fokus der vorliegenden Arbeit und sind daher maximal Grundlage für weitere Überlegungen.
		\item Die Frage nach dem persönlichen Umgang der Lehrperson bezüglich der Problemlagen der Schüler wurde nach dem Pretest den vorhandenen Unterfragen hinzugefügt. Sie dient primär dem persönlichen Forschungsinteresse der Autorinnen.
	\end{itemize}
	\item Inwieweit wünschen Sie sich Angebote, die sie hinsichtlich des Umgangs mit solchen Problemlagen der SchülerInnen unterstützen?
	\begin{itemize}
		\item Unterfrage: Welche außerunterrichtlichen Beratungs- und Unterstützungsangebote können Sie sich vorstellen? Wie könnten sich diese auf das Unterrichtsgeschehen auswirken?
		\item Diese Frage ist bedeutend für die Beantwortung der Forschungsfragen und zeigt neben der Nennung möglicher Unterstützungsangebote auch den Kenntnisstand der Lehrkraft in Bezug auf sozialpädagogische Beratungs- und Unterstützungsangebote auf.
		\end{itemize}
	\item Auch wenn es nicht explizit im Fragebogen angegeben ist, wird am Ende des Interviews dem Gesprächspartner nochmal die Gelegenheit gegeben, etwaige Ideen, Vorschläge oder Anmerkungen, die im Laufe des Gesprächs aufgekommen sind, loszuwerden. So endet das Interview für den Befragten nicht so abrupt und das Gespräch kann angenehm ausgleiten.
\end{enumerate}

\noindent
\textbf{Pretest}\\

\noindent
Analog zur Schülerumfrage wurde nach der Erstellung des Leitfadens im Sinne des wissenschaftlichen Arbeitens ein Pretest mit einer weiteren Lehrperson des DRK Bildungswerkes SN durchgeführt. Dieses Testgespräch diente dazu, potentielle Unklarheiten in der Fragestellung, technische Schwierigkeiten aber auch den Fragestil des Interviewers zu prüfen und im Anschluss gegebenenfalls abzuändern. Das Gespräch verlief freundlich und motivierte die Autorinnen für den konkreten Einsatz bei den kommenden Lehrkräften. Im gemeinsamen Gespräch wurden anschließend kleinere Änderungen bei der Formulierung der Fragen vorgenommen und die Frage nach dem persönlichen Umgang der Lehrkräfte mit den Problemlagen der Schüler als interessante Steuerungsfrage in den Leitfaden übernommen. Die vorab ausgearbeiteten Fragen wurden nicht weiter verändert, da der Pretest insgesamt ein zufriedenstellendes Resultat lieferte. 

Anzumerken ist jedoch, dass mit Beginn der Interviews stets 2 Aufnahmegeräte, in Form von Smartphones, bereitlagen, da im Pretest die Aufnahme aus ungeklärten Gründen plötzlich abbrach und so das Gespräch leider ungenutzt verhallte.\\

\noindent
\textbf{Datenschutzerklärung, Einverständniserklärung und Postscript}\\

\noindent
Die interviewbegleitenden Dokumente in Form der Datenschutzerklärung, Einverständniserklärung und des Postscriptes entstammen in ihrer grundlegenden Gestaltung den Seminarmaterialien von Frau Thümmler, welche im Rahmen des Seminars "`Forschungsfelder"' an der TU Dresden bereit gestellt wurden. Diese wurden anschließend nach persönlichen Gusto leicht abgeändert und durchgängig bei allen 5 Interviews verwendet.
