\section{Vorstellung des Forschungsvorhabens}
\label{sec:VorstellungDesForschungsvorhabens}

\subsection{Das erste eigene Forschungsprojekt - Eine Hinführung}
\label{sec:DasErsteEigeneForschungsprojektEineHinführung}

Lehrer sein. OK. Forscher sein. OK. Aber forschende Lehrer?

Jeder Student an deutschen Hochschulen muss am Ende seines Studiums eine Abschlussarbeit schreiben die seine erworbenen Kenntnisse, Methoden und Wissensstrukturen angemessen widerspiegelt. Auch wir standen zum Ende des 5-jährigen Studiums für das Höhere Lehramt an berufsbildenden Schulen, sprich als angehende Berufsschullehrer, vor dieser abschließenden Herausforderung. Bereits Monate zuvor macht man sich Gedanken welche Fachrichtung man dafür heranziehen möchte, welches Themengebiet spannend und erforschbar wäre, welche Betreuer sich der Arbeit annehmen würden und wie man diesen Arbeitsprozess überhaupt organisieren könnte? 

Doch wagen wir zunächst einen Schritt zurück ins Jahr 2010. In diesem Jahr haben sich Daniela Wobst und Doreen Stichel dazu entschlossen ein Studium für das Lehramt an berufsbildenden  Schulen im Bereich Gesundheit und Pflege und Sozialpädagogik an der TU Dresden aufzunehmen. Zu diesem Zeitpunkt hatten beide bereits eine Ausbildung absolviert, entsprechende Berufserfahrung gesammelt und waren in verschiedenen Branchen des Gesundheitswesens tätig. Unabhängig voneinander hatte man sich mit Aufnahme des Studiums das Ziel gesetzt den bestehenden beruflichen Qualifikationen, neue Wissensbereiche, Kompetenzen und berufliche Möglichkeiten  hinzuzufügen. Zum Zeitpunkt der Immatrikulation war das Lehramt für berufsbildende Schulen in Bachelor- und Masterstudiengang organisiert, wohingegen im Jahr 2013 wieder das Staatsexamen eingeführt wurde. Aus diesem Grund handelt es sich bei dieser Abschlussarbeit um eine Masterarbeit und um keine andere Form des Abschlusses.

Zurück im Jahr 2015 stand man nun, wie alle Kommilitonen, vor der Frage welches Thema für eine wissenschaftliche Bearbeitung im Rahmen einer Masterarbeit geeignet wäre.  Unabhängig voneinander haben die Autorinnen bei der Eingrenzung der Themenwahl besonderes Interesse für den Gegenstand der Schulsozialarbeit gezeigt. Aufgrund einer guten persönlichen Beziehung und positiver Erfahrungswerte in Bezug auf die Erarbeitung gemeinsamer Seminararbeiten, entschloss man sich dazu, das bestehende Forschungsinteresse auf dem Gebiet der Schulsozialarbeit zu nutzen und eine gemeinsame Arbeit in die Wege zu leiten. 

Die wissenschaftliche Bearbeitung einer Problemstellung im Rahmen einer Masterarbeit an der Fakultät Erziehungswissenschaften kann durch zwei verschiedene Herangehensweisen erfolgen. Entweder man nutzt den theoretischen Zugang mittels einer literaturbezogenen Bearbeitung einer wissenschaftlichen Problemstellung oder den praktischen Zugang in Form eines eigenen Forschungsprojektes.
Aufgrund persönlicher Interessen und Ansichten erschien ein kleines Forschungsprojekt, welches in der Lage ist eigene Erkenntnisse zu generieren, äußerst attraktiv und zugleich motivierend für den Forschungsprozess an sich. Zudem entspricht diese Vorgehensweise dem Anspruch an eine Masterprüfung gemäß §20 der Prüfungsordnung für das Höhere Lehramt na berufsbildenden Schulen nach dem "`der Studierende die fachlichen Zusammen-hänge überblickt sowie die Fähigkeit besitzt, wissenschaftliche sowie gegebenenfalls künstlerische Methoden und Erkenntnisse anzuwenden, und die für den Übergang in den für die Befähigung für das Höhere Lehramt an berufsbildenden Schulen vorgeschriebenen Vorbereitungsdienst bzw. eine Promotion notwendigen gründlichen Fachkenntnisse erworben hat. Ebenso wird festgestellt, dass der Studierende über vertiefte fachliche Kenntnisse und berufsfeldbezogene Qualifikationen als Beschäftigungsbefähigung für eine Tätigkeit in Berufsfeldern des öffentlichen oder privaten Bildungssektors verfügt."' (Quelle bla bla)

Angesicht dieser Ausgangslage entschied man sich im gemeinsamen Dialog für ein gemeinsames Forschungsprojekt welches sich dem Thema der Schulsozialarbeit an berufsbildenden Schulen widmen soll. 


\subsection{Problemdefinition}
\label{sec:Problemdefinition}

\subsubsection{Relevanz des Themas}
\label{sec:RelevanzDesThemas}

Da bereits unter Punkt \ref{sec:SchulsozialarbeitAnBerufsbildendenSchulenInSachsen} die Sachlage zum Thema der Schulsozialarbeit an sächsischen beruflichen Schulen ausführlich erläutert wurde, soll hier nur noch eine Kurzfassung darlegt werden um auf die Bedeutsamkeit des Themas hinzuweisen. (Quellen: siehe Daniela)

Es gibt subjektiv viele Problemlagen, denen sich Jugendliche bzw. junge Menschen in Ausbildung gegenüber sehen, dennoch sind kaum Studien, Bedarfsanalysen oder konkrete Angebote von SSA zur Lösung dieser Probleme vorhanden. Forschungen und Publikationen zum Thema SSA sind zwar sehr vielfältig und zahlreich, darunter auch umfassende Begründungen zu Notwendigkeit und Bedarf von SSA, diese richten sich aber nur an den grundschulischen und allgemeinbildenden Bereich. Trotz intensiver Recherche in Fachliteratur und einschlägigen Datenbanken existieren nur sehr wenige Ausführungen, Studien und Bedarfsanalysen für den berufsbildenden Bereich. Eine Ursache hierfür ist gesetzlich bedingt, denn SSA ist in Sachsen nur in Schulen mit Berufsvorbereitungsjahr vorgeschrieben.(Quelle) Dem gegenüberzustellen sind jedoch die der sozialwissenschaftlichen Literatur zu entnehmenden vielfältigen Problemlagen von Jugendlichen und jungen Menschen in Ausbildung, die mit gesellschaftlichen und strukturellen sowie institutionellen Gegebenheiten in Zusammenhang stehen. 

Doch inwiefern machen diese Problemlagen den Einsatz von SSA an BBS prinzipiell erforderlich, wenn man dabei die momentan geringen Angebote für diese Schulform beachtet? 

\subsubsection{Forschungsgegenstand}
\label{sec:Forschungsgegenstand}

Im Rahmen der Masterarbeit sollen die Problemlagen und außerunterrichtlichen Beratungs- sowie Unterstützungsbedarfe von Schülern an einer exemplarischen berufsbildenden Schule in Sachsen erfasst werden.

Zusätzlich soll die Sicht der Lehrkräfte hinsichtlich der Problemlagen von Schülern sowie mögliche Beratungs- und Unterstützungsbedarfe durch Lehrkräfte eruiert werden.

Erläuterung warum nur eine private Schule....
eingrenzung auf drk bwk..erläuterung genau dazu später

\subsubsection{Nutzen der Forschung/Forschungslücke}
\label{sec:NutzenDerForschungForschungslücke}

Lehrkräfte nehmen Problemlagen von Schülern und individuelle Unterstützungsbedarfe (subjektiv) wahr, sehen sich jedoch mit den teilweise komplexen Problemlagen überfordert und bemängeln die unzureichende Unterstützung in dieser Sachlage. Sie fordern eine Rückbesinnung auf ihr Kerngeschäft -- nämlich den Unterricht 
Die Erhebungen (Schülersicht und Lehrersicht) untersuchen die tatsächlichen Probleme und Bedarfe an einer exemplarischen berufsbildenden Schule in Sachsen.

Abgeleitet werden sollen sowohl die Bedarfe, als auch mögliche Beratungs- und Unterstützungsangebote.

\subsubsection{Vorannahmen}
\label{sec:Vorannahmen}

Die Problemlagen von SchülerInnen an berufsbildenden Schulen unterscheiden sich wenig bis gar nicht von denen der allgemeinbildenden Schule.

Außerunterrichtliche Problemlagen wirken sich tendenziell negativ auf das Unterrichtsgeschehen aus
Lehrkräfte sehen Problemlagen bei Schülern und erkennen Bedarfe für sozialpädagogische Unterstützungsangebote. 

\subsubsection{Forschungsfragen}
\label{sec:Forschungsfragen}

\begin{enumerate}
	\item \textbf{Schülerperspektive:}
	Welche persönlichen und sozialen (außerunterrichtlichen) Problemlagen haben SchülerInnen des DRK BWK SN in sozialen und medizinischen Ausbildungsberufen, die das Unterrichtsgeschehen und den Ausbildungserfolg beeinflussen?
	\item \textbf{Lehrerperspektive:}
	Welche persönlichen und sozialen (außerunterrichtlichen) Problemlagen von SchülerInnen nehmen Lehrkräfte des DRK BWK SN im Bereich Gesundheit und Pflege als besondere Belastung für den Unterricht wahr?
	\item \textbf{Verbindung der Schüler- und Lehrerperspektive:}
	Wie schätzen SchülerInnen und Lehrkräfte anhand der (möglichen) subjektiv wahrgenommenen Problemlagen den Bedarf an sozialpädagogischen Unterstützungs- und Beratungsangeboten und anderen Angeboten ein und wie könnten passende Angebote aussehen?
\end{enumerate}

Begründung: Es existieren subjektiv wahrgenommene Bedarfe an sozialpädagogischen Unterstützungsangeboten für persönliche außerunterrichtliche Probleme, welche das Unterrichtsgeschehen negativ beeinflussen.

Erläuterung FF: was heißt außerunterrichtlich für uns?

\subsection{Projektplanung}
\label{sec:Projektplanung}

\subsubsection{Zugang zum Forschungsfeld}
\label{sec:ZugangZumForschungsfeld}

\subsubsection{Darlegung über die selektive Erarbeitung relevanter Inhalte im Rahmen einer universitären Seminararbeit}
\label{sec:DarlegungÜberDieSelektiveErarbeitungRelevanterInhalteImRahmenEinerUniversitärenSeminararbeit}

\subsubsection{Universitäre Betreuung}
\label{sec:UniversitäreBetreuung}

Gängler, Haupt, Thümmler, Vorerfahrungen

\subsubsection{Organisation}
\label{sec:Organisation}

\subsubsection{Angestrebte wissenschaftliche Ziele}
\label{sec:AngestrebteWissenschaftlicheZiele}

\begin{itemize}
	\item Die Bearbeitung des Themas dient dem Abgleich von subjektiv wahrgenommenen Problemlagen von Schülern an berufsbildenden Schulen der Forschungsgruppe und vielen Lehrkräften mit den real beschriebenen Problemlagen von Schülern durch Schüler und Lehrkräfte an einer berufsbildenden Schule.
	\item Das Projekt ermöglicht einen konkreten Einblick in reale Problemlagen der Schülern DRK BWK SN, welche den Unterricht negativ beeinflussen bzw. denen die Lehrkräfte in Ihrem beruflichen Selbstbild potentiell nicht gewachsen sind.
	\item Aus der Erfassung möglicher Problemlagen könnte ein Bedarf für außerunterrichtliche sozialpädagogische Unterstützungsangebote abgeleitet werden.
\end{itemize}

\textbf{Angestrebte Ziele für das DRK BWK SN:}
 
\begin{itemize}
	\item Für Schüler mit möglichen außerunterrichtlichen Problemlagen:
	\\
	Ziel ist es, die Lernleistung durch Minderung oder Lösung persönlicher Problemlagen, die die Aufmerksamkeit auf Schule und Unterricht senken, zu steigern.
	\item Entlastung der Lehrer bei möglichen außerunterrichtlichen Problemlagen:
	 \\
	Ziel ist es, die zusätzliche zeitliche und psychische Belastung der Lehrer zu reduzieren. 
	\item Auf einen möglichen Bedarf, durch das Heranziehen von Beratungs- und Unterstützungsangebote entsprechend reagieren.
\end{itemize}

\subsubsection{Eigene Vorarbeiten und Expertisen}
\label{sec:EigeneVorarbeitenUndExpertisen}

\textbf{Gruppenvorstellung}

Da einige Zuarbeiten dieser Arbeit auf einer Gruppenarbeit von 5 Personen beruhen, soll die Expertise aller Gruppenmitglieder hiermit auch berücksichtigt werden.

Alle Teilnehmer der Seminargruppe haben eine abgeschlossene Berufsausbildung, welche sich durch 2 Physiotherapeutinnen sowie 1 Diätassistentin, 1 Operationstechnische Assistentin und 1 Medizinische Fachangestellte näher charakterisieren lässt. Alle Beteiligten haben zum Zeitpunkt der Befragung den Abschluss "`Bachelor of Education"' und streben ihren Masterabschluss für das Höhere Lehramt an berufsbildenden Schulen an. Zudem weisen alle die berufliche Fachrichtung Gesundheit und Pflege im Erstfach auf, jedoch verschiedene Zweitfächer.

Neben den praktische Lehrerfahrungen im Block A, Block B und während der Schulpraktischen Übungen besitzen alle Personen individuelle Erfahrungen mit außerunterrichtlichen Problemlagen durch Mitschüler während der eigenen Ausbildung aber z.T. auch weitere Erfahrungen durch die Betreuung von Praktikanten und von eigenen Lehrtätigkeiten.

Aufgrund dieser Sachlage sind somit genügend Situationen vertraut, wo die Unterrichtstunde nicht für Lerninhalte, sondern zum Klären von außerunterrichtlichen Problemen genutzt wurde, was sowohl für die Lehrer als auch für die nicht betroffenen Schüler belastend empfunden wird.

An den der Gruppe bekannten Ausbildungsschulen, als auch an den Praktikumsschulen (der Gesundheit und Pflege) während des Studiums gab es keine sozialpädagogischen Unterstützungsangebote, obwohl durch eigene Erfahrungen potentieller Bedarf bestanden hätte.\\

\noindent
\textbf{Vorstellung der Autorinnen}

Die bisher erfolgten "`gruppenverbindenden"' Angaben, sollen hiermit noch um ein paar individuelle Informationen der Autorinnen erweitert werden.\\

\underline{Vorbildung Daniela Wobst:}\\
\begin{itemize}
	\item Erfahrungen als Schülerin im Berufsfeld: 1998-2001 Ausbildung zur Diätassistentin
	\item 2001-2003: Praxisanleiterin für Praktikantinnen in Krankenhaus und Rehaklinik
	\item 2003-2010: Lehrkraft für fachpraktischen Unterricht in den Fachbereichen Diätassistenz, Altenpflege und Krankenpflegehilfe sowie Klassenlehrertätigkeit, QM-Beauftragte des DRK BWK SN mit Aufgabe der Schülerbetreuung (Klassensprechersitzungen, Zusammenkünfte mit Schülervertretern)
	\item seit 2010: weitere unterrichtliche Tätigkeit (fachpraktischer und theoretischer Bereich) sowie Beginn des Lehramtstudiums an der TU Dresden
	\item seit 2014: Fachbereichsleitung der Berufsfachschule für Diätassistenten am DRK BW SN
\end{itemize}

\underline{Vorbildung Doreen Stichel:}\\
\begin{itemize}
	\item Erfahrungen als Schülerin im Berufsfeld: 2002-2005 Ausbildung zur Physiotherapeutin
	\item 2006-2012: Praktische Erfahrungen im Bereich der physiotherapeutischen Praxis, der stationären Behandlung, Intensivtherapie sowie in der Fitnessbranche
	\item Erfahrung im Umgang mit Praktikanten und Auszubildenden im Rahmen der beruflichen Tätigkeit
	\item Aufnahme des Lehramtstudiums an der TU Dresden
\end{itemize}
\newpage

\subsubsection{Arbeitsprogramm (inkl. Zeitplan)}
\label{sec:ArbeitsprogrammInklZeitplan}

\begin{longtable}{l|p{9.8cm}}
	
	\textbf{1. Problemdefinition} & \\
	\emph{Januar 2015} &
	\vspace*{-0.6cm}
	\begin{itemize}[nosep,topsep=-0.6cm]
		\item Bildung der Arbeitsgruppe
		\item Erste Literaturrecherche
		\item Entwicklung der Fragestellung/Zielsetzung
	\end{itemize} \\* 
	
	\emph{Februar 2015} & 
	\vspace*{-0.6cm}
	\begin{itemize}[nosep,topsep=-0.6cm]
		\item Suche nach universitären Betreuer
		\item Vorstellung des Vorhabens bei den Betreuern
		\item Ausformulierung der Forschungsfrage
	\end{itemize} \\
	
	\multicolumn{2}{c}{Kurzzeitige Unterbrechung durch das Blockpraktikum B im März 2015} \\*
	
	\textbf{2. Projektplanung} & \textbf{Klärung Projektbewilligung am DRK BWK SN} \\
	\emph{April 2015} & 
	\vspace*{-0.6cm}
	\begin{itemize}[nosep,topsep=-0.6cm]
		\item Vorstellung des Forschungsvorhabens bei der Geschäftsleitung
		\item Klären möglicher Fragen bezgl. der geplanten Schülerumfrage und der Lehrerinterviews
		\item Besprechen einer möglichen Ergebnispräsentation nach Fertigstellung der Masterarbeit
	\end{itemize} \\ 
	& \textbf{Ressourcenbestimmung Schüler} \\*
	\emph{Mai 2015} & 
	\vspace*{-0.6cm}
	\begin{itemize}[nosep,topsep=-0.6cm]
		\item Welche Ausbildungsrichtungen wollen wir befragen?
		\item Mögliche Ausschlusskriterien bestimmen
		\item Welche Klassen und Schüler stehen im Juni/Juli am DRK BWK SN zur Verfügung?
		\item Absprache mit den betreffenden Lehrer zum gegebenen Zeitpunkt
	\end{itemize} \\
	& \textbf{Ressourcenbestimmung Lehrer} \\*
	&
	\vspace*{-0.6cm}
	\begin{itemize}[nosep,topsep=-0.6cm]
		\item Eingrenzen geeigneter Lehrkräfte um das gewählte Ausbildungsspektrum angemessen zu repräsentieren
		\item Anschreiben der Lehrkräfte bezüglich Forschungsvorhaben und Interviewanfrage
		\item Termine für Interviews festlegen
		\item Interviewer und Beisitzer festlegen
	\end{itemize} \\ 
	& \textbf{Erstellung der Umfragematerialien} \\*
	&
	\vspace*{-0.6cm}
	\begin{itemize}[nosep,topsep=-0.6cm]
		\item Umfragebogen (Pretest)
		\item Merkzettel zur Studie
	\end{itemize} \\
	& \textbf{Erstellung der Interviewmaterialien} \\*
	& 
	\vspace*{-0.6cm}
	\begin{itemize}[nosep,topsep=-0.6cm]
		\item Leitfragen des Interviews erstellen (Pretest)
		\item Datenschutzerklärung
		\item Einverständniserklärung
		\item Postscript
	\end{itemize} \\
	
	\textbf{3. Projektdurchführung} & \textbf{Datenerhebung} \\*
	\emph{Juni 2015} &
	\vspace*{-0.6cm}
	\begin{itemize}[nosep,topsep=-0.6cm]
		\item Durchführung der Schülerumfragen
		\item Durchführung der Lehrerinterviews I.01 -- I.05
	\end{itemize} \\
	& \textbf{Dateneingabe} \\*
	\emph{Juli 2015} &
	\vspace*{-0.6cm}
	\begin{itemize}[nosep,topsep=-0.6cm]
		\item Dateneingabe der Schülerumfragewerte in LimeSurvey
		\item Transkription der Interviews
	\end{itemize} \\
	& \textbf{Zwischenprojektevaluation} \\*
	\emph{Juli 2015} &
	\vspace*{-0.6cm}
	\begin{itemize}[nosep,topsep=-0.6cm]
		\item regelmäßiger Austausch in der Gruppe über Fortschritte und mögliche Probleme im Forschungsprozess
	\end{itemize} \\
	
	\textbf{4. Evaluation} & \textbf{Datenauswertung} \\*
	\emph{August 2015} &
	\vspace*{-0.6cm}
	\begin{itemize}[nosep,topsep=-0.6cm]
		\item ... der Schülerumfrage mittels Statistikausgabe von LimeSurvey und Gesprächen in der Gruppe
		\item ... der Lehrerinterviews mittels Qualitativer Inhaltsanalyse nach Mayring
	\end{itemize} \\
	
	\textbf{5. Projektbericht} & \\*
	\emph{bis Ende August 2015} &
	\vspace*{-0.6cm}
	\begin{itemize}[nosep,topsep=-0.6cm]
		\item Zusammenfügen aller Daten mit dem Ziel der Beantwortung der Forschungsfragen
		\item Konzeptüberlegungen für SSA am DRK BWK SN
		\item Konzeptüberlegungen für SSA am DRK BWK SN
		\item Verschriftlichung der Ergebnisse
	\end{itemize} \\
	
	\textbf{6. Abgabe Masterarbeit} & \\*
	\emph{15.09.2015} & \\
	
\end{longtable}

\newpage

\subsubsection{Studiendesign und Methoden}
\label{sec:StudiendesignUndMethoden}

\subsubsection{Erwartungshorizont}
\label{sec:Erwartungshorizont}

\subsubsection{Qualitätssicherung}
\label{sec:Qualitätssicherung}








	
	










