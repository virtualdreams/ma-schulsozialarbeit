\section{Vorstellung des Forschungsvorhabens}
\label{sec:VorstellungDesForschungsvorhabens}

\subsection{Das erste eigene Forschungsprojekt - Eine Hinführung}
\label{sec:DasErsteEigeneForschungsprojektEineHinführung}

Lehrer sein. OK. Forscher sein. OK. Aber forschende Lehrer?

Jeder Student an deutschen Hochschulen muss am Ende seines Studiums eine Abschlussarbeit schreiben die seine erworbenen Kenntnisse, Methoden und Wissensstrukturen angemessen widerspiegelt. Auch wir standen zum Ende des 5-jährigen Studiums für das Höhere Lehramt an berufsbildenden Schulen, sprich als angehende Berufsschullehrer, vor dieser abschließenden Herausforderung. Bereits Monate zuvor macht man sich Gedanken welche Fachrichtung man dafür heranziehen möchte, welches Themengebiet spannend und erforschbar wäre, welche Betreuer sich der Arbeit annehmen würden und wie man diesen Arbeitsprozess überhaupt organisieren könnte? 

Doch wagen wir zunächst einen Schritt zurück ins Jahr 2010. In diesem Jahr haben sich Daniela Wobst und Doreen Stichel dazu entschlossen ein Studium für das Lehramt an berufsbildenden Schulen im Bereich Gesundheit und Pflege und Sozialpädagogik an der TU Dresden aufzunehmen. Zu diesem Zeitpunkt hatten beide bereits eine Ausbildung absolviert, entsprechende Berufserfahrung gesammelt und waren in verschiedenen Branchen des Gesundheitswesens tätig. Beide hatten sich mit Aufnahme des Studiums das Ziel gesetzt den bestehenden beruflichen Qualifikationen, neue Wissensbereiche, Kompetenzen und berufliche Möglichkeiten hinzuzufügen. Zum Zeitpunkt der Immatrikulation war das Lehramt für berufsbildende Schulen in Bachelor- und Masterstudiengang organisiert, wohingegen im Jahr 2013 das Staatsexamen erneut eingeführt wurde. 

Zurück im Jahr 2015 stand man nun, wie alle Kommilitonen, vor der Frage welches Thema für eine wissenschaftliche Bearbeitung im Rahmen einer Masterarbeit geeignet wäre. Unabhängig voneinander haben beide Autorinnen bei der Eingrenzung der möglichen Themen besonderes Interesse für den Gegenstand der Schulsozialarbeit gezeigt. Aufgrund einer guten persönlichen Beziehung und positiver Erfahrungswerte in Bezug auf die Erarbeitung gemeinsamer Seminararbeiten, entschloss man sich dazu, das bestehende Forschungsinteresse auf dem Gebiet der Schulsozialarbeit zu nutzen und eine gemeinsame Arbeit in die Wege zu leiten. 

Die wissenschaftliche Bearbeitung einer Problemstellung, im Rahmen einer Masterarbeit an der Fakultät Erziehungswissenschaften, kann durch zwei verschiedene Herangehensweisen erfolgen. Entweder man nutzt den theoretischen Zugang mittels der literaturbezogenen Bearbeitung einer wissenschaftlichen Problemstellung oder den praktischen Zugang in Form eines eigenen Forschungsprojektes.
Aufgrund persönlicher Interessen und Ansichten erschien ein kleines Forschungsprojekt, welches in der Lage ist eigene Erkenntnisse zu generieren, äußerst attraktiv und zugleich motivierend für den gemeinsamen Forschungsprozess. Zudem entspricht diese Vorgehensweise dem Anspruch an eine Masterprüfung gemäß §20 der Prüfungsordnung für das Höhere Lehramt an berufsbildenden Schulen nach dem "`der Studierende die fachlichen Zusammen-hänge überblickt sowie die Fähigkeit besitzt, wissenschaftliche sowie gegebenenfalls künstlerische Methoden und Erkenntnisse anzuwenden, und die für den Übergang in den für die Befähigung für das Höhere Lehramt an berufsbildenden Schulen vorgeschriebenen Vorbereitungsdienst bzw. eine Promotion notwendigen gründlichen Fachkenntnisse erworben hat. Ebenso wird festgestellt, dass der Studierende über vertiefte fachliche Kenntnisse und berufsfeldbezogene Qualifikationen als Beschäftigungsbefähigung für eine Tätigkeit in Berufsfeldern des öffentlichen oder privaten Bildungssektors verfügt."' (/TODO)

Angesicht dieser Ausgangslage entschied man sich im Dialog für ein gemeinsames Forschungsprojekt welches sich dem Thema der Schulsozialarbeit an berufsbildenden Schulen widmen soll. 

Nach einer grundlegenden Sichtung, der zum Zeitpunkt der Arbeit veröffentlichten Literatur zum Thema Schulsozialarbeit an berufsbildenden Schulen, zeigte sich eine deutliche geringere Repräsentativität sozialpädagogischer Beratungs- und Unterstützungsangebote an berufsbildenden Schulen im Vergleich zu derer an allgemeinbildenden Schulen bspw. in Form der Grund- oder Förderschule; ein Umstand der u.a. gesetzliche Gründe hat (/TODO) aber auch Verwunderung hervorrief, korrelierte er doch mit persönlichen Vermutungen und Erfahrungen. Aus diesem Grund setzte man sich nun zum Ziel die Probleme und Unterstützungsbedarfe von Schülern reell zu erfassen umso eine Aussage über den potentiellen Bedarf von SSA für Schüler an BBS formulieren zu können. Aufgrund der begrenzten zeitlichen und personellen Möglichkeiten, entschied man sich für eine exemplarische Untersuchung an einer BBS in Sachsen.
 
Die für die Realisierung des Forschungsprojekts notwendige Kooperationsvereinbarung mit einer exemplarischen Schule in Sachsen konnte aufgrund der jahrelangen Tätigkeit von Daniela Wobst am DRK Bildungswerk Sachsen und der freundlichen Genehmigung der Geschäftsleitung des DRK Bildungswerk Sachsen in Form von Hr. Vlodrop, Fr. Hösel und Hr. Eckert realisiert werden, wobei die genauen Umstände im Folgenden detailliert vorgestellt werden.

\subsection{Problemdefinition}
\label{sec:Problemdefinition}

\subsubsection{Relevanz des Themas}
\label{sec:RelevanzDesThemas}

Da bereits unter Punkt \ref{sec:SchulsozialarbeitAnBerufsbildendenSchulenInSachsen} die Sachlage zum Thema der Schulsozialarbeit an sächsischen beruflichen Schulen ausführlich erläutert wurde, soll hier nur noch eine Kurzfassung darlegt werden um auf die Bedeutsamkeit des Themas hinzuweisen. (/TODO)

Es gibt subjektiv viele Problemlagen, denen sich Jugendliche bzw. junge Menschen in Ausbildung gegenüber sehen, dennoch sind kaum Studien, Bedarfsanalysen oder konkrete Angebote von SSA zur Lösung dieser Probleme vorhanden. Forschungen und Publikationen zum Thema SSA sind zwar sehr vielfältig und zahlreich, darunter auch umfassende Begründungen zu Notwendigkeit und Bedarf von SSA, diese richten sich aber nur an den grundschulischen und allgemeinbildenden Bereich. Trotz intensiver Recherche in Fachliteratur und einschlägigen Datenbanken existieren nur sehr wenige Ausführungen, Studien und Bedarfsanalysen für den berufsbildenden Bereich. Eine Ursache hierfür ist gesetzlich bedingt, denn SSA ist in Sachsen nur in Schulen mit Berufsvorbereitungsjahr vorgeschrieben.(/TODO) Dem gegenüberzustellen sind jedoch die der sozialwissenschaftlichen Literatur zu entnehmenden vielfältigen Problemlagen von Jugendlichen und jungen Menschen in Ausbildung, die mit gesellschaftlichen und strukturellen sowie institutionellen Gegebenheiten in Zusammenhang stehen. 

Doch inwiefern machen diese Problemlagen den Einsatz von SSA an BBS prinzipiell erforderlich, wenn man die momentan geringen Angebote für diese Schulform beachtet? 

\subsubsection{Forschungsgegenstand}
\label{sec:Forschungsgegenstand}

Im Rahmen der Masterarbeit sollen die Problemlagen und außerunterrichtlichen Beratungs- sowie Unterstützungsbedarfe von Schülern an einer exemplarischen berufsbildenden Schule in Sachsen erfasst werden.

Zusätzlich soll die Sicht der Lehrkräfte hinsichtlich der Problemlagen von Schülern sowie mögliche Beratungs- und Unterstützungsbedarfe durch Lehrkräfte eruiert werden.

\subsubsection{Nutzen der Forschung/Forschungslücke}
\label{sec:NutzenDerForschungForschungslücke}

Lehrkräfte nehmen Problemlagen von Schülern und individuelle Unterstützungsbedarfe (subjektiv) wahr, sehen sich jedoch mit den teilweise komplexen Problemlagen überfordert und bemängeln die unzureichende Unterstützung in dieser Sachlage. Sie fordern eine Rückbesinnung auf ihr Kerngeschäft -- nämlich den Unterricht.

Die geplanten Datenerhebungen aus Schüler- und Lehrerperspektive untersuchen die tatsächlichen Probleme von Schülern und Bedarfe für SSA an einer berufsbildenden Schule in Sachsen.

Abgeleitet werden sollen sowohl die Bedarfe, als auch mögliche Beratungs- und Unterstützungsangebote.

\subsubsection{Vorannahmen}
\label{sec:Vorannahmen}

Die Problemlagen von Schülern an berufsbildenden Schulen unterscheiden sich wenig bis gar nicht von denen der allgemeinbildenden Schule.

Außerunterrichtliche Problemlagen wirken sich tendenziell negativ auf das Unterrichtsgeschehen aus
Lehrkräfte sehen Problemlagen bei Schülern und erkennen Bedarfe für sozialpädagogische Unterstützungsangebote. 

\subsubsection{Forschungsfragen}
\label{sec:Forschungsfragen}

\begin{enumerate}
	\item \textbf{Schülerperspektive:}
	Welche persönlichen und sozialen (außerunterrichtlichen) Problemlagen haben Schüler des DRK Bildungswerk SN in sozialen und medizinischen Ausbildungsberufen, die das Unterrichtsgeschehen und den Ausbildungserfolg beeinflussen?
	\item \textbf{Lehrerperspektive:}
	Welche persönlichen und sozialen (außerunterrichtlichen) Problemlagen von Schüler nehmen Lehrkräfte des DRK Bildungswerk SN im Bereich Gesundheit und Pflege als besondere Belastung für den Unterricht wahr?
	\item \textbf{Verbindung der Schüler- und Lehrerperspektive:}
	Wie schätzen Schüler und Lehrkräfte anhand der (möglichen) subjektiv wahrgenommenen Problemlagen den Bedarf an sozialpädagogischen Unterstützungs- und Beratungsangeboten und anderen Angeboten ein und wie könnten passende Angebote aussehen?
\end{enumerate}

Begründung: Es existieren subjektiv wahrgenommene Bedarfe an sozialpädagogischen Unterstützungsangeboten für persönliche außerunterrichtliche Probleme, welche das Unterrichtsgeschehen negativ beeinflussen.

/TODO Erläuterung FF: was heißt außerunterrichtlich für uns? warum wird aus sp angeboten allg. ber.u.unterst.angebote? warum lehrer und schülerperspektive

\subsection{Projektplanung}
\label{sec:Projektplanung}

\subsubsection{Zugang zum Forschungsfeld}
\label{sec:ZugangZumForschungsfeld}

Zu Beginn der Projektplanung gab es mehrere Ideen diese institutionell umzusetzen. In Sachsen gibt es eine Vielzahl von BBS unter privater wie öffentlicher Trägerschaft. Der Gedanke mehrere öffentliche Schulen zur Generierung einer großen Datenmenge heranzuziehen, wurde aufgrund der geringen zeitlichen Ressourcen negiert. Desweiteren wurde überlegt eine öffentliche und eine private schulische Einrichtung bei der Auswertung der beabsichtigten Ergebnisse zu vergleichen; dieser Aspekt wurde jedoch aufgrund zeitlicher und organisatorischer Hürden ebenso verworfen. Letztendlich wurde beschlossen, das nur eine exemplarische private schulische Einrichtung zur Projektrealisierung berücksichtigt werden soll. Diese Entscheidung offeriert den Autorinnen bessere personelle, organisatorische und zeitliche Absprachen und ermöglicht zudem eine fokussierte, abgrenzbare und ganzheitlich angelegte Darstellung einer BBS hinsichtlich der gewählten Problemstellung. Diese exemplarische  Betrachtungsweise ermöglicht zudem konkrete Empfehlungen hinsichtlich Bedarf und Konzeption möglicher Beratungs- und Unterstützungsangebote für eine schulische Einrichtung.
Wie bereits angesprochen, wurde die notwendige schulische Kooperation für dieses Forschungsprojekt durch die freundliche Genehmigung der Geschäftsleitung des DRK Bildungswerk SN ermöglicht, wobei Daniela Wobst, bereits viele Jahre bei dieser Bildungsstätte angestellt, als Gatekeeper fungierte. 

\subsubsection[Selektive Erarbeitung im Rahmen einer universitären Gruppenarbeit]{Darlegung über die selektive Erarbeitung relevanter Inhalte im Rahmen einer universitären Gruppenarbeit}
\label{sec:DarlegungÜberDieSelektiveErarbeitungRelevanterInhalteImRahmenEinerUniversitärenSeminararbeit}

Bestandteile der vorliegenden Ausarbeitung beruhen auf einer universitären Gruppenarbeit; ein Umstand der zu klären ist. Zeitgleich mit Beginn der Masterarbeit wurden Daniela Wobst und Doreen Stichel Teil einer universitären Arbeitsgruppe die bis Ende September 2015 Bestand hat. Diese Gruppe formierte sich im Rahmen des Seminars "`Forschungsfelder"', einem Modul der Fachrichtung Gesundheit und Pflege, unter Leitung von Fr. Thümmler. Im Rahmen dieser Veranstaltung ist eine Prüfungsleistung in Form einer Qualitativen Studie zu erbringen. Im gemeinsamen Dialog entschloss man sich, das bereits bestehende Thema der Autorinnen aufzugreifen und die Sicht der Lehrkräfte hinsichtlich der Problemlagen von Schülern sowie mögliche Beratungs- und Unterstützungsbedarfe durch Lehrkräfte in Bezug auf die o.g. Forschungsfrage in Auszügen(!) für die Seminararbeit heranzuziehen bzw. zu bearbeiten. Die Gruppenmitglieder, bestehend aus Janet Kaiser, Juliane Born, Anne Krause, Daniela Wobst und Doreen Stichel erarbeiteten somit gemeinsam Inhalte die auch in die vorliegende Arbeit eingeflossen sind. Inhalte und Arbeitsschritte die unter Mitwirkung dieser Arbeitsgruppe erbracht wurden, werden im weiteren Verlauf kenntlich gemacht. Da die entsprechende Seminararbeit mit dem vorläufigen Arbeitstitel "`Eine qualitative Einzelfallstudie am DRK Bildungswerk Sachsen"' /TODO Quelle zum Zeitpunkt der Abgabe dieser Arbeit noch nicht vorlag, kann hiermit nur eine allgemeine Referenz auf diese erfolgen. Alle Gruppenmitglieder haben sich jedoch schriftlich damit einverstanden erklärt, das etwaige Inhalte die unter Mithilfe derer entstanden sind, in der vorliegenden Masterarbeit verwendet werden dürfen (s. \ref{sec:Sonstiges} Einverständniserklärung Gruppe).

\subsubsection{Universitäre Betreuung}
\label{sec:UniversitäreBetreuung}

Es ist uns ein Anliegen, die fachkundige, freundliche und unterstützende universitäre Betreuung durch Prof. Dr. Gängler, Fr. Haupt und Fr. Thümmler bei der Realisierung des Forschungsprozesses zu betonen. Deren hilfreiche Tipps, Anmerkungen und seminargebundenen Inhalte haben zum Gelingen dieses Projektes maßgeblich beigetragen. Jeder war zudem kompetenter Ansprechpartner bei inhaltlichen Fragen.

\subsubsection{Grundlegende Organisation des Forschungsprojektes}
\label{sec:GrundlegendeOrganisationDesForschungsprojektes}

Im Folgenden sollen die grundlegenden organisatorischen Schritte dargelegt werden, die zur Realisierung des Forschungsprojektes notwendig waren. Die detaillierten Ausführungen zu bestimmten Themen sind den darauffolgenden Punkten zu entnehmen /TODO. 

Die ersten Schritte auf dem Weg zur gemeinschaftlichen Abschlussarbeit waren bereits im Februar 2015 vollzogen. Themengebiet, Problemstellung, Forschungsfrage(n) und wesentliche Inhalte der geplanten Untersuchung wurden im stetigen Dialog durch die Autorinnen festgelegt. Ehe jedoch mit der eigentlichen Projektumsetzung begonnen werden konnte, hielt man zunächst Rücksprache mit den universitären Gutachtern, welche sich dem Thema angenommen hatten. Dank dieser kompetenten und inhaltlich ergiebigen Gespräche konnten neue Details und Gedanken in das geplante Forschungsvorhaben einfließen. Nach Abschluss dieser grundlegenden Arbeitsschritte konnte mit der detaillierten Planung der Erhebung begonnen werden. 
Es ist zu beachten, das einige Arbeitsschritte welche im folgenden beschrieben werden, gleichzeitig erfolgten, was eine klare zeitliche Einordnung erschwert.
Dank dem Engagement von Daniela Wobst, u.A. Fachbereichsleiterin der Diätassistenz am DRK Bildungswerk SN konnte diese langjährige private Bildungseinrichtung in Dresden als Untersuchungsfeld bereits frühzeitig festgelegt werden. Nach ersten mündlichen Absprachen von Frau Wobst mit der Geschäftsführung, unterbreiteten beide Autorinnen zu einem späteren Zeitpunkt ihr genaues Forschungsvorhaben im Rahmen einer kleinen Präsentation vor der gesamten Geschäftsleitung, so das Ziel, Methoden und Verwendung der gesammelten Daten transparent erläutert werden konnten. Im Laufe dieses freundlichen Gesprächs konnten zudem organisatorische und inhaltliche Fragen geklärt werden und den Autorinnen wurde anschließend Unterstützung in Form der Bereitstellung der zahlreichen Umfragekopien zugesichert. Dank der freundlichen Projektbewilligung durch die Schulleitung konnte man im Anschluss an die betreffenden Lehrer und Schüler herantreten.

Bereits zuvor wurden Überlegungen angestellt, welche wissenschaftlichen Methoden sich als geeignet zur Beantwortung der gewählten Forschungsfragen hinsichtlich der Schüler- und  Lehrerperspektive erweisen könnten. Um möglichst wirklichkeitsnahe Aussagen in Bezug auf etwaige Schülerprobleme und Bedarfe, unter Berücksichtigung der zur Verfügung stehenden Zeit, zu generieren, entschied man sich für zwei verschiedene Methoden. Der Umfragebogen, eine Methode der quantitativen Forschung, ermöglicht das Zusammentragen der potentieller Probleme und Bedarfe für SSA von möglichst vielen Schülern. Das Interview hingegen, eine Methode der qualitativen Forschung, dient der  Erfassung subjektiver Aussagen von ausgewählten Lehrkräften hinsichtlich der Probleme und Bedarfe ihrer Schüler. Die genauere theoretische Einbettung und praktische Erarbeitung dieser Methoden in Form des Umfragebogens bzw. des Interviewleitfadens wird unter \ref{sec:DieUmfrage} sowie \ref{sec:DasInterview} vertieft dargelegt.

Im nächsten Schritt wand man sich nun an die Lehrkräfte des DRK Bildungswerk SN. Zu diesem Zweck kontaktierte Daniela Wobst zunächst alle betreffenden Personen über die berufliche E-Mail-Adresse. Über diesen Weg wurden sie über Inhalt, Ziel und Ablauf des universitären Forschungsprojektes an der Schule informiert, über anstehende Besuche in den Klassen bezgl. der Schülerumfragen in Kenntnis gesetzt, aber auch um persönliche Unterstützung in Form der Interviewteilnahme gebeten. Im Anschluss erreichten Daniela Wobst viele freundliche und interessierte Reaktionen in Bezug auf das Thema der Untersuchung aber auch zahlreiche Angebote von Lehrkräften welche bereitwillig an einem Interview teilnehmen würden. Da die betreffenden Pädagogen zumeist in verschiedenen Ausbildungsgängen tätig sind, entschied man sich für eine Auswahl an Personen welche mit ihrem Erfahrungshorizont die vorliegenden Ausbildungsrichtungen am DRK Bildungswerk SN möglichst komplett abbilden konnten. Somit konnten nach und nach 5 Lehrpersonen kontaktiert und für ein Interview gewonnen werden. Die individuelle Terminvergabe der Interviews erfolgte erneut per E-Mail oder telefonisch, organisiert durch Doreen Stichel. Bereits während dieser Absprachen zeigten die ausgewählten Lehrkräfte großes Interesse an Inhalt, Durchführung und Ergebnis der gesamten Untersuchung.
Nachdem die Lehrer über das Forschungsprojekt in Kenntnis gesetzt wurden, oblag es ihnen die einzelnen Klassen auf den Besuch von Doreen Stichel oder Daniela vorzubereiten. Mit den Klassen selbst wurde vor Umfragebeginn kein Kontakt aufgenommen.


\subsubsection{Angestrebte wissenschaftliche Ziele}
\label{sec:AngestrebteWissenschaftlicheZiele}

\begin{itemize}
	\item Die Bearbeitung des Themas dient dem Abgleich von subjektiv wahrgenommenen Problemlagen von Schülern an berufsbildenden Schulen der Forschungsgruppe und vielen Lehrkräften mit den real beschriebenen Problemlagen von Schülern durch Schüler und Lehrkräfte an einer berufsbildenden Schule.
	\item Das Projekt ermöglicht einen konkreten Einblick in reale Problemlagen der Schülern DRK Bildungswerk SN, welche den Unterricht negativ beeinflussen bzw. denen die Lehrkräfte in Ihrem beruflichen Selbstbild potentiell nicht gewachsen sind.
	\item Aus der Erfassung möglicher Problemlagen könnte ein Bedarf für außerunterrichtliche sozialpädagogische Unterstützungsangebote abgeleitet werden.
\end{itemize}

\textbf{Angestrebte Ziele für das DRK Bildungswerk SN:}
 
\begin{itemize}
	\item Für Schüler mit möglichen außerunterrichtlichen Problemlagen:
	\\
	Ziel ist es, die Lernleistung durch Minderung oder Lösung persönlicher Problemlagen, die die Aufmerksamkeit auf Schule und Unterricht senken, zu steigern.
	\item Entlastung der Lehrer bei möglichen außerunterrichtlichen Problemlagen:
	 \\
	Ziel ist es, die zusätzliche zeitliche und psychische Belastung der Lehrer zu reduzieren. 
	\item Auf einen möglichen Bedarf, durch das Heranziehen von Beratungs- und Unterstützungsangebote entsprechend reagieren.
\end{itemize}

\subsubsection{Eigene Vorarbeiten und Expertisen}
\label{sec:EigeneVorarbeitenUndExpertisen}

\textbf{Gruppenvorstellung}

Da einige Inhalte dieser Studie auf einer Gruppenarbeit von 5 Personen beruhen, soll die Expertise aller Gruppenmitglieder hiermit auch berücksichtigt werden.

Alle Teilnehmer der Seminargruppe haben eine abgeschlossene Berufsausbildung, welche sich durch 2 Physiotherapeutinnen sowie 1 Diätassistentin, 1 Operationstechnische Assistentin und 1 Medizinische Fachangestellte näher charakterisieren lässt. Alle Beteiligten haben zum Zeitpunkt der Befragung den Abschluss "`Bachelor of Education"' und streben ihren Masterabschluss für das Höhere Lehramt an berufsbildenden Schulen an. Zudem weisen alle die berufliche Fachrichtung Gesundheit und Pflege im Erstfach auf, jedoch verschiedene Zweitfächer.

Neben den praktischen Lehrerfahrungen aus dem Blockpraktikum A, Blockpraktika B sowie den Schulpraktischen Übungen im Rahmen der universitären Ausbildung, besitzen alle Personen individuelle Erfahrungen mit außerunterrichtlichen Problemlagen durch Mitschüler während der eigenen Ausbildung aber z.T. auch weitere Erfahrungen durch die Betreuung von Praktikanten und durch eigene Lehrtätigkeiten.

Aufgrund dieser Sachlage sind somit genügend Situationen vertraut, wo die Unterrichtstunde nicht für Lerninhalte, sondern zum Klären von außerunterrichtlichen Problemen genutzt wurde, was sowohl für die Lehrer als auch für die nicht betroffenen Schüler belastend empfunden wird.

An den der Gruppe bekannten Ausbildungsschulen, als auch an den Praktikumsschulen (der Gesundheit und Pflege) während des Studiums gab es keine sozialpädagogischen Unterstützungsangebote, obwohl durch eigene Erfahrungen potentieller Bedarf bestanden hätte.\\

\noindent
\textbf{Vorstellung der Autorinnen}

Die bisher erfolgten "`gruppenverbindenden"' Angaben, sollen hiermit noch um ein paar individuelle Informationen der Autorinnen erweitert werden.\\

\underline{Vorbildung Daniela Wobst:}\\
\begin{itemize}
	\item Erfahrungen als Schülerin im Berufsfeld: 1998-2001 Ausbildung zur Diätassistentin
	\item 2001-2003: Praxisanleiterin für Praktikantinnen in Krankenhaus und Rehaklinik
	\item 2003-2010: Lehrkraft für fachpraktischen Unterricht in den Fachbereichen Diätassistenz, Altenpflege und Krankenpflegehilfe sowie Klassenlehrertätigkeit, QM-Beauftragte des DRK Bildungswerk SN mit Aufgabe der Schülerbetreuung (Klassensprechersitzungen, Zusammenkünfte mit Schülervertretern)
	\item seit 2010: weitere unterrichtliche Tätigkeit (fachpraktischer und theoretischer Bereich) sowie Beginn des Lehramtstudiums an der TU Dresden
	\item seit 2014: Fachbereichsleitung der Berufsfachschule für Diätassistenten am DRK Bildungswerk SN
\end{itemize}

\underline{Vorbildung Doreen Stichel:}\\
\begin{itemize}
	\item Erfahrungen als Schülerin im Berufsfeld: 2002-2005 Ausbildung zur Physiotherapeutin
	\item 2006-2012: Praktische Erfahrungen im Bereich der physiotherapeutischen Praxis, der stationären Behandlung, Intensivtherapie sowie in der Fitnessbranche
	\item Erfahrung im Umgang mit Praktikanten und Auszubildenden im Rahmen der beruflichen Tätigkeit
	\item 2010: Aufnahme des Lehramtstudiums an der TU Dresden
\end{itemize}

\newpage

\subsubsection{Arbeitsprogramm (inkl. Zeitplan)}
\label{sec:ArbeitsprogrammInklZeitplan}

\begin{longtable}{l|p{9.8cm}}
	
	\textbf{1. Problemdefinition} & \\
	\emph{Januar 2015} &
	\vspace*{-0.6cm}
	\begin{itemize}[nosep,topsep=-0.6cm]
		\item Bildung der Arbeitsgruppe
		\item Erste Literaturrecherche
		\item Entwicklung der Fragestellung/Zielsetzung
	\end{itemize} \\* 
	
	\emph{Februar 2015} & 
	\vspace*{-0.6cm}
	\begin{itemize}[nosep,topsep=-0.6cm]
		\item Suche nach universitären Betreuer
		\item Vorstellung des Vorhabens bei den Betreuern
		\item Ausformulierung der Forschungsfrage
	\end{itemize} \\
	
	\multicolumn{2}{c}{Kurzzeitige Unterbrechung durch das Blockpraktikum B im März 2015} \\*
	
	\textbf{2. Projektplanung} & \textbf{Klärung Projektbewilligung am DRK Bildungswerk SN} \\
	\emph{April 2015} & 
	\vspace*{-0.6cm}
	\begin{itemize}[nosep,topsep=-0.6cm]
		\item Vorstellung des Forschungsvorhabens bei der Geschäftsleitung
		\item Klären möglicher Fragen bezgl. der geplanten Schülerumfrage und der Lehrerinterviews
		\item Besprechen einer möglichen Ergebnispräsentation nach Fertigstellung der Masterarbeit
	\end{itemize} \\ 
	& \textbf{Ressourcenbestimmung Schüler} \\*
	\emph{Mai 2015} & 
	\vspace*{-0.6cm}
	\begin{itemize}[nosep,topsep=-0.6cm]
		\item Welche Ausbildungsrichtungen wollen wir befragen?
		\item Mögliche Ausschlusskriterien bestimmen
		\item Welche Klassen und Schüler stehen im Juni/Juli am DRK Bildungswerk SN zur Verfügung?
		\item Absprache mit den betreffenden Lehrer zum gegebenen Zeitpunkt
	\end{itemize} \\
	& \textbf{Ressourcenbestimmung Lehrer} \\*
	&
	\vspace*{-0.6cm}
	\begin{itemize}[nosep,topsep=-0.6cm]
		\item Eingrenzen geeigneter Lehrkräfte um das gewählte Ausbildungsspektrum angemessen zu repräsentieren
		\item Anschreiben der Lehrkräfte bezüglich Forschungsvorhaben und Interviewanfrage
		\item Termine für Interviews festlegen
		\item Interviewer und Beisitzer festlegen
	\end{itemize} \\ 
	& \textbf{Erstellung der Umfragematerialien} \\*
	&
	\vspace*{-0.6cm}
	\begin{itemize}[nosep,topsep=-0.6cm]
		\item Umfragebogen (+ Pretest)
		\item Merkzettel zur Studie
	\end{itemize} \\
	& \textbf{Erstellung der Interviewmaterialien} \\*
	& 
	\vspace*{-0.6cm}
	\begin{itemize}[nosep,topsep=-0.6cm]
		\item Leitfragen des Interviews erstellen (+ Pretest)
		\item Datenschutzerklärung
		\item Einverständniserklärung
		\item Postscript
	\end{itemize} \\
	
	\textbf{3. Projektdurchführung} & \textbf{Datenerhebung} \\*
	\emph{Juni 2015} &
	\vspace*{-0.6cm}
	\begin{itemize}[nosep,topsep=-0.6cm]
		\item Durchführung der Schülerumfrage
		\item Durchführung der Lehrerinterviews I.01 -- I.05
	\end{itemize} \\
	& \textbf{Dateneingabe} \\*
	\emph{Juli 2015} &
	\vspace*{-0.6cm}
	\begin{itemize}[nosep,topsep=-0.6cm]
		\item Dateneingabe der Schülerumfragewerte in LimeSurvey
		\item Transkription der Interviews
	\end{itemize} \\
	& \textbf{Zwischenprojektevaluation} \\*
	\emph{Juli 2015} &
	\vspace*{-0.6cm}
	\begin{itemize}[nosep,topsep=-0.6cm]
		\item regelmäßiger Austausch in der Gruppe über Fortschritte und mögliche Probleme im Forschungsprozess
	\end{itemize} \\
	
	\textbf{4. Evaluation} & \textbf{Datenauswertung} \\*
	\emph{August 2015} &
	\vspace*{-0.6cm}
	\begin{itemize}[nosep,topsep=-0.6cm]
		\item ... der Schülerumfrage mittels Statistikausgabe von LimeSurvey und Gesprächen in der Gruppe
		\item ... der Lehrerinterviews mittels Qualitativer Inhaltsanalyse nach Mayring
	\end{itemize} \\
	
	\textbf{5. Projektbericht} & \\*
	\emph{bis Ende August 2015} &
	\vspace*{-0.6cm}
	\begin{itemize}[nosep,topsep=-0.6cm]
		\item Zusammenfügen aller Daten mit dem Ziel der Beantwortung der Forschungsfragen
		\item Konzeptüberlegungen für SSA am DRK BWK SN
		\item Verschriftlichung der Ergebnisse
	\end{itemize} \\
	
	\textbf{6. Abgabe Masterarbeit} & \\*
	\emph{15.09.2015} & \\
	
\end{longtable}

\newpage

\subsubsection{Studiendesign und Methoden}
\label{sec:StudiendesignUndMethoden}

Dieser Abschnitt wird zum Zwecke der eindeutigen Darstellung unter Punkt \ref{sec:DieUmfrage} sowie \ref{sec:DasInterview} näher erläutert.

\subsubsection{Erwartungshorizont}
\label{sec:Erwartungshorizont}

\subsubsection{Qualitätssicherung}
\label{sec:Qualitätssicherung}

Folgende Maßnahmen dienen der Qualitätssicherung der Forschungsarbeit:\\
\begin{itemize}
	\item grundlegende theoretische Fundierung vor Aufnahme der eigenen Untersuchungen und anschließend zur wissenschaftlichen Einordnung gewonnener Erkenntnisse in bereits vorhandene Wissensstrukturen
	\item inhaltliche oder organisatorische Fragen, Aufgaben oder auch entstehende Probleme  werden regelmäßig in der Gruppe besprochen, diskutiert, reflektiert und evaluiert (das bezieht sich sowohl auf die gemeinsame Arbeit der Autorinnen als auch auf die partielle Zusammenarbeit mit der Seminargruppe)
	\item bei Fragen oder Problemen wird Rücksprache mit den Betreuern gehalten
	\item Im Interviewverlauf wird eine Rückkopplung mit dem Interviewpartnern angestrebt, d.h. ob die eigene Interpretation dem Gesagten entspricht
	\item Sowohl vor Beginn der Schülerumfrage als auch vor den Lehrerinterviews wird ein Probelauf (=Pretest) erfolgen. Dieser Schritt dient sowohl der Erprobung organisatorischer Abläufe als auch der Überprüfung der erarbeiteten Unterlagen (Umfragebogen; Interviewleitfaden) in Bezug auf Logik, Verständnis, Zeitvorgabe und potentielle Fehler. Zudem werden die Probanden im Anschluss um ein ehrliches Feedback gebeten. Die dabei involvierten Schüler und der einzelne Lehrer werden bei der weiteren Datenerhebung nicht berücksichtigt um keine Ergebnisverzerrung zu verursachen.
\end{itemize}

\subsection{Die Umfrage}
\label{sec:DieUmfrage}

\subsubsection{Theoretischer Hintergrund}
\label{sec:TheoretischerHintergrund}

\subsubsection{Praktische Umsetzung} 
\label{sec:PraktischeUmsetzung}

\textbf{Merkzettel zur Studie}\\
Ergänzend zur Umfrage wurde ein kleiner Merkzettel für die Schüler erstellt. Trotz einer klaren Einweisung und Belehrung vor Beginn der Umfrage soll so auch nach Beendigung der Umfrage die Möglichkeit gegeben sein, sich an die für das Projekt verantwortlichen Personen wenden zu können. Neben grundlegenden Informationen zur Umfrage werden somit auch Kontaktdaten von Daniela Wobst und Doreen Stichel auf dem Zettel vermerkt. Dieses Schriftstück in A5-Größe wird an jeden beteiligten Schüler ausgegeben und ist dem Anhang beigefügt \ref{sec:Schülerumfrage}.

Die nun folgende Konzeption der Schülerbefragung wurde im Vorfeld der Umfrage gemeinsam von den Autorinnen erarbeitet und mit den Gutachtern besprochen.

\textbf{Allgemeines:}\\
\begin{itemize}
	\item Da die Vorstellung des DRK Bildungswerk SN bereits unter Punkt /TODO erfolgte, wird auf erneute Darstellung der Ausbildungsgänge im Folgenden verzichtet.
	\item Forschungsziel ist die Befragung einer Schülermenge welche die Grundgesamtheit der Schülerschaft am DRK Bildungswerk SN widerspiegelt.
	\item Befragt werden soll mit einer Zufallsauswahl (Random-Verfahren) um /TODO
	\item Genaues Verfahren: Klumpen-Auswahl (Cluster Sampling), d.h. die Aufteilung der Grundgesamtheit in Teilgruppen (in diesem Fall Fachrichtungen) /TODO
	\item Begründung: Diese Auswahl repräsentiert unserer Meinung nach die verschiedenen Fachrichtungen am besten, die sich hinsichtlich der Zugangsvoraussetzungen, Altersstruktur und der Ausbildungsanforderungen deutlich unterscheiden.
\end{itemize}

\textbf{Beschreibung der Stichprobenauswahl}\\
\begin{itemize}
	\item Aktuell gibt es am DRK Bildungswerk SN insgesamt 679 Schüler. Dies ist etwas weniger als zunächst angenommen, da in der Rettungsassistenz und in der Altenpflege im März 2015 insgesamt 4 Klassen verabschiedet wurden.
	\item Ausschlusskriterium: Die berufsbegleitenden Klassen werden bei der Befragung nicht berücksichtigt, da sie nur einen Tag pro Woche zum Unterricht in der Schule sind und sich ihre Altersstruktur sowie Zusammensetzung von den Vollzeitklassen deutlich unterscheiden. Hinzu kommt, dass potentielle Unterstützungsangebote durch sie kaum bis gar nicht genutzt werden könnten.
	\item Abzüglich der berufsbegleitenden Schüler verbleiben 578 Schüler.
	\item Befragt werden sollen nach Mayer (/TODO S. 66) 30\% der Grundgesamtheit, was eine Aussagesicherheit von 95\% und eine Stichprobenfehlerquote von 6,5\% ergeben würde. Das entspricht einer zu befragenden Stichprobe von 175 Schülern.
	\item 578 Schüler gesamt -- Stichprobe ergibt 175 Schüler mit folgender Aufteilung (gewichtet nach der Verteilung der jeweiligen Fachrichtung, davon jeweils 30\%) Dadurch erhalten wir ein relativ realistisches Bild der Gesamtgruppe. 
		\begin{itemize}
			\item 32 Personen aus der Altenpflege 
			\item 3 Personen aus der Diätassistenz
			\item 68 Personen aus der Erzieherausbildung 
			\item 13 Personen aus der Heilerziehungspflege 
			\item 21 Personen aus der Physiotherapie
			\item 25 Personen aus der Rettungsassistenz und Notfallsanitäter
		\end{itemize}
	\item Trotz der zum Teile geringen Personenzahl werden jeweils ganze Klassen befragt. Damit entfällt eine komplizierte Vorauswahl nach welchen Kriterien die Antworten der Schüler berücksichtigt werden und es erfolgt keine Selektion von ausgewählten Schülern. Zusätzlich entfallen ganz praktische Probleme wie z.B. Was passiert, wenn von den Vorausgewählten am Befragungstag nur ein Teil anwesend ist? Zudem können jeweils ganze Klassen persönlich eingewiesen und motiviert werden.
	\item Für die Bedarfsanalyse unberücksichtigt werden Geschlecht, Alter und Schulabschluss. Diese Daten werden jedoch für mögliche statistische Auswertungen oder weiterführende Studien bzw. zur Fehleranalyse mit aufgenommen.
	\item Die Fragebögen werden jeweils in die ganze Klasse gegeben, anonym ausgefüllt und aus dem Klassensatz werden dann zufällig die auszuwertenden Bögen gezogen, in Höhe der jeweiligen errechneten Stichprobengröße. Zur Wahrung der Durchführungsobjektivität erfolgt dieser Vorgang durch eine neutrale Person in Form einer Studienkollegin, die dem Projekt nicht angehörig ist. Durch dieses Verfahren verringert sich die Gefahr von Selektions- und Durchführungs-BIAS (/TODO klären).
	\item Die ausgewählten Bögen werden ausgewertet (/TODO s. datenauswertung/Verweis)
	\item Zusätzlich zur Schülerbefragung werden Leitfadeninterviews mit 5 ausgewählten Lehrkräften (Klassenlehrern), darunter der Vertrauenslehrer, durchgeführt. Diese Personen haben aufgrund ihrer Lehrtätigkeit Erfahrungen im Bereich der Fachschule, Krankenpflegehilfe, Altenpflege, Notfallmedizin? und der Therapieberufe./TODO richtig?
	\item Diese dienen ebenfalls der Bedarfsermittlung aus Sicht der Lehrkraft und der möglichen praktischen Umsetzung der Unterstützungsangebote.
\end{itemize}

\subsection{Das Interview}
\label{sec:DasInterview}

\subsubsection{Theoretischer Hintergrund}
\label{sec:TheoretischerHintergrund}

\subsubsection{Praktische Umsetzung}
\label{sec:PraktischeUmsetzung}










	
	










