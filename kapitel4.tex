\section[Vorstellung des DRK Bildungswerk Sachsen]{Berufsbildende Schulen in Sachsen - Eine exemplarische Vorstellung am DRK Bildungswerk Sachsen}
\label{sec:BerufsbildendeSchulenInSachsenEineExemplarischeVorstellungAmDRKBildungswerkSachsen}

\subsection{Allgemeines}
\label{sec:Allgemeines}

Die in diesem Abschnitt vorgestellten allgemeinen Informationen zum DRK Bildungswerk SN dienen dazu, einen ersten Überblick über die für die eigenen Forschungsprojekte ausgewählte Bildungseinrichtung zu geben. Da eine der Verfasserinnen der vorliegenden Arbeit bereits seit 2003 dort als fest angestellte Mitarbeiterin beschäftigt ist, wurden für diesen Abschnitt, mit Ausnahme des Punktes "`Historische Entwicklung"', keine Quellen verwendet. Vielmehr dienten eigenes Wissen und langjährige Erfahrung, die Recherche in den schuleigenen Dokumenten und die Absprache mit den Kollegen dazu, alle relevanten Informationen zusammen zutragen.

Das DRK Bildungswerk SN ist eine gemeinnützige GmbH und als solche eine 100-prozentige Tochtergesellschaft des DRK Landesverbandes Sachsen. Innerverbandlich wird das Bildungswerk als sogenanntes "`Kompetenzzentrum für die berufliche und verbandsspezifische Aus-, Fort- und Weiterbildung"' des Deutschen Roten Kreuzes in Sachsen bezeichnet. Innerhalb dieses Aufgabenbereiches werden Ausbildungs-, Fortbildungsdienst- und Serviceleistungen nicht nur für Einrichtungen des DRK sondern auch für externe Interessenten erbracht. Die sich daraus ableitenden immanenten Haupttätigkeitsfelder sind der Betrieb von staatlich anerkannten Berufsfachschulen und Fachschulen im sozialen, pflegerischen und medizinischen Bereich sowie die Unterstützung der Ausprägung von Handlungskompetenz bei haupt- und ehrenamtlich tätigen Personen im Deutschen Roten Kreuz und bei externen Kunden. Zur Erfüllung der Bildungsaufgaben kooperiert das DRK Bildungswerk SN mit  Unternehmen, Bildungsträgern, Universitäten, Behörden und Verbänden der Freien Wohlfahrtspflege.

Das DRK Bildungswerk SN bietet seit über 20 Jahren jungen Erstauszubildenden und Umschülern die Möglichkeit, in den fünf staatlich anerkannten Berufsfachschulen und in der staatlich anerkannten Fachschule, im Fachbereich Sozialwesen in zwei verschiedenen Fachrichtungen, einen Gesundheits-, Pflege- oder sozialen Beruf zu erlernen. Dazu stehen am Schulstandort Bremer Straße 10d in Dresden, nach einem Umzug von einem anderen Standort im Jahr 2011, modern ausgestattete Unterrichtsräume und zahlreiche Fachkabinette, wie z. B. Pflegekabinette, Kreativräume, Gymnastik- und physiotherapeutischer Behandlungsraum, Lehrküche und SAN-Arena zur Verfügung. In allen Fachrichtungen wird ständig daran gearbeitet, eine qualifizierte, wissenschaftlich fundierte, praxis- und handlungsorientierte Berufsausbildung nach modernen Lehr- und Lernmethoden zu gewährleisten, obwohl sich dies aufgrund unterschiedlicher gesetzlicher Vorgaben und Lehrplanregelungen für die einzelnen Berufe durchaus als Herausforderung erweist. Die einzelnen Fachrichtungen, welche derzeit im DRK Bildungswerk SN nach den gesetzlichen Regelungen der Verordnung des Sächsischen Staatsministeriums für Kultus über die Berufsfachschule im Freistaat Sachsen (Schulordnung Berufsfachschule -- BFSO) ausbilden, sind: Altenpflege, Diätassistenz, Krankenpflegehilfe, Notfallsanitäter- bzw. Rettungsassistenz und Physiotherapie. Die Ausbildung für Erzieher und Heilerziehungspfleger erfolgt nach den Regelungen der Verordnung des Sächsischen Staatsministeriums für Kultus und des Sächsischen Staatsministeriums für Umwelt und Landwirtschaft über die Fachschule im Freistaat Sachsen (Schulordnung Fachschule -- FSO). Über die konkreten Ausbildungsformen und Anzahlen der Schüler in den einzelnen Fachrichtungen wird im Punkt \ref{sec:MitarbeiterSchülerUndFortbildungsteilnehmer} detaillierter berichtet. Alle genannten Berufsfachschulen und die Fachschule befinden sich im Status der staatlich anerkannten Ersatzschulen.

Die berufliche Fort- und Weiterbildung wird vorrangig ebenfalls im Tätigkeitsbereich Gesundheit, Pflege und Sozialwesen angeboten und richtet sich mit dem Ziel der Aufrechterhaltung, Verbesserung und kontinuierlichen Weiterentwicklung einer kompetenten Berufstätigkeit sowohl an Absolventen der vorherigen Berufsausbildungen und an Fach- und Hilfskräfte im DRK sowie bei verschiedensten öffentlichen und privaten Trägern. Neben beruflichen Fort- und Weiterbildungen werden auch zahlreiche weitere Schulungs- und Bildungsmaßnahmen angeboten, beispielsweise für Teilnehmer am Bundesfreiwilligendienst oder im Rahmen der Ersten Hilfe und des Sanitätsdienstes bis hin zu Man\-age\-ment- und IT-Themen. 

\subsection{Historisches}
\label{sec:Historisches}

Das DRK Bildungswerk SN entwickelte sich in der Nachwendezeit aus der "`Bezirksakademie für Gesundheits- und Sozialwesen"', in der von 1960 bis 1992, am Standort Schevenstraße 3 in Dresden, Weiterbildungen für Krankenschwestern und –pfleger, Fachkrankenschwesternausbildungen, Fachärzteaus- und -weiterbildungen, Pharmazieausbildungen sowie die Orthopädieschuhmacherausbildung angeboten wurden. Die Bezirksakademie für Gesundheits- und Sozialwesen stand in der Nachwendezeit vor strukturellen, finanziellen und fachlichen Problemen, so dass das Land Sachsen auf der Suche nach einem geeigneten privaten Träger an das Deutsche Rote Kreuz herantrat. Sehr schnell wurde 1991 die berufsbegleitende Ausbildung zum Heilerziehungspfleger im Schulgebäude Schevenstraße, noch unter der Trägerschaft der Bezirksakademie, begonnen. 1992 erhielt das DRK durch Ministeriumsentscheid den Auftrag zur Übernahme der Trägerschaft für die Bezirksakademie. Durch den DRK Landesverband Sachsen sowie die DRK Schwesternschaft wurde am 24.09.1992 die Gründung des "`DRK-Bildungswerk für soziale und pflegerische Berufe Sachsen e. V."' beschlossen.  

Bereits 1993 fanden die ersten Erweiterung des Bildungswerkes statt, als der zusätzliche Schulstandort Wilthen mit dem Ziel, Aus-, Fort- und Weiterbildungsmaßnahmen für Mitarbeiterinnen und Mitarbeiter in gesundheitlichen und pflegerischen Bereichen anzubieten, errichtet wurde. Am gleichen Standort erhielt die bereits langjährig dort angesiedelte Landesrettungsschule die staatliche Anerkennung als Berufsfachschule für Rettungsassistenten unter der Trägerschaft des DRK Landesverbandes Sachsen e. V. Zusätzlich  begann im gleichen Jahr die damalige Fachschulausbildung zum Altenpfleger in Dresden (Schulstandort Kaitzer Straße 2) und in Leipzig und es erfolgte der Umzug der Heilerziehungspflege auf die Goetheallee 39. 1994 wurden erste Rettungsassistenten in Wilthen ausgebildet und der Weiterbildungsbereich für Krankenschwestern und -pfleger in den verschiedensten Disziplinen, für Stationsleiter, Heimleiter, Pflegedienstleiter sowie Orthopädieschuhmachermeister am Ausbildungsstandort Dresden aufrechterhalten. Ende des Jahres wurden die Fachkrankenschwesteraus- und -weiterbildungen von den Krankenhäusern übernommen und die Orthopädieschuhmachermeister verlegten ihre Ausbildung an eine private Schule. 1995 erhielt das DRK Bildungswerk SN die schulaufsichtliche Genehmigung zum Errichten und Betreiben der Fachschule für Sozialwesen Fachrichtung Heilerziehungspflege, 1996 folgte diese für die Fachrichtung Altenpflege. Zusätzlich wurden Berufsfachschulen zur Ausbildung in der Krankenpflege, Krankenpflegehilfe und Physiotherapie an verschiedenen Schulstandorten beantragt und genehmigt. Jedoch wurde nur in der Physiotherapie die Ausbildung wirklich begonnen. 1998 kam als weitere Fachrichtung die Diätassistenz hinzu. Als 2003 die Ausbildung in der Altenpflege bundeseinheitlich neu geregelt wurde, wurde die 2-jährige Fachschulausbildung durch eine 3-jährige Berufsfachschulausbildung ersetzt und somit Erstausbildungen zum Altenpfleger möglich. Zusätzlich wurden erstmals auch Praxisanleiter an der Berufsfachschule für Altenpflege ausgebildet. 2004 übernimmt das DRK Bildungswerk SN die Trägerschaft der Landesrettungsschule in Wilthen und integriert die Berufsfachschulausbildung zum Rettungsassistenten in ihr Portfolio. 2006 wurden alle bisherigen Dresdener Fachrichtungen am Standort Haydnstraße 39 in Dresden vereinigt und die Berufsfachschule für Altenpflege in Leipzig geschlossen. Durch einen Rechtsformwechsel wurde 2008 aus dem DRK-Bildungswerk für soziale und pflegerische Berufe Sachsen e. V. das DRK Bildungswerk SN gemeinnützige GmbH als 100\%ige Tochtergesellschaft des DRK Landesverbandes Sachsen e. V., zusätzlich wurde die Gründung der Abteilung für berufliche und verbandsspezifische Aus-, Fort- und Weiterbildung vollzogen. Im darauffolgenden Jahr erweiterte die Berufsausbildung zum Erzieher das Angebot der fachschulischen Vollzeitausbildungen. 2010 erfolgte der Umzug der Berufsfachschule für Rettungsassistenten an den Standort des Bildungswerks in Dresden, worauf auch dieser seine Kapazitätsgrenzen erreichte. Daher wurde 2011 der Umzug aller Fachrichtungen und der Abteilung Aus-, Fort- und Weiterbildung in das DRK Zentrum Sachsen auf der Bremer Straße 10d in Dresden realisiert, wo zusätzlich die neu geregelte Ausbildung zum Krankenpflegehelfer startete und weitere berufsbegleitende Angebote konzipiert wurden. 2014 wurde schließlich die Ausbildung im Fachbereich Rettungsassistenz durch die neue Ausbildung zum Notfallsanitäter abgelöst, so dass nunmehr alle eingangs genannten Fachrichtungen in ihrer Entwicklung benannt und erläutert wurden \footcite[vgl.]{DRKBS2015}.

\subsection{Mitarbeiter, Schüler und Fortbildungsteilnehmer}
\label{sec:MitarbeiterSchülerUndFortbildungsteilnehmer}

Zum Zeitpunkt der durchgeführten Studien im Mai und Juni 2015 waren am DRK Bildungswerk SN insgesamt 40 Mitarbeiter in Vollzeit und Teilzeit zur Realisierung der eingangs erläuterten Aufgaben beschäftigt. Von diesen sind drei der Geschäftsführung zuzuordnen, 28 Beschäftigte entfallen auf den pädagogischen Bereich, sechs sind in der allgemeinen Schulverwaltung und dem Geschäftsführungssekretariat tätig und drei Mitarbeiter betreuen fachlich und organisatorisch den Bereich der Fort- und Weiterbildung. Zusätzlich wird die Absicherung des Unterrichts in allen schulischen und außerschulischen Abteilungen durch eine Vielzahl von externen Honorardozenten gewährleistet, mit denen das DRK Bildungswerk SN teilweise bereits sehr langjährig zusammenarbeitet. 

Am Stichtag 10.05.2015 lernten im Schuljahr 2014/15 insgesamt 679 Schüler, aufgeteilt auf 32 Klassen, in den einzelnen berufsfachschulischen und fachschulischen Ausbildungsrichtungen. Davon entfielen 152 Schüler auf die Altenpflege, 9 auf die Diätassistenz, 268 auf die Erzieher, 35 auf die Heilerziehungspflege, 43 auf die Krankenpflegehilfe, 70 auf die Physiotherapie und 82 auf die Rettungsassistenz/Notfallsanitäter. 101 Schüler aus der Gesamtheit lernten in berufsbegleitenden Ausbildungsgängen der Fachbereiche Altenpflege, Erzieher und Heilerziehungspflege. 

Zur Darstellung der Teilnehmerzahlen hinsichtlich der Fort- und Weiterbildung wird auf die Zahlen des Kalenderjahres 2014 zurückgegriffen, da diese Maßnahmen selten dem Schuljahresturnus folgen, sondern häufig in Eintages- oder Mehrtagesveranstaltungen bis hin zu mehrmonatigen Schulungsprogrammen absolviert werden. Dazu wurden im genannten Jahr 450 Veranstaltungen mit insgesamt 4200 Teilnehmern durchgeführt.

\subsection{Schülerstruktur und Besonderheiten}
\label{sec:SchülerstrukturUndBesonderheiten}

In diesem Punkt werden einige Voraussetzungen und Hintergrundinformationen beschrieben, welche für die Konzeption und Durchführung der eigenen Forschungsarbeiten relevant sind. 

Bedingt durch die unterschiedlichen gesetzlichen Grundlagen der einzelnen Ausbildungsberufe ergeben sich auch im DRK Bildungswerk SN die bereits in Punkt \ref{sec:BerufsbildendeSchulen} ausgeführten sehr heterogenen Zusammensetzungen der Schülerstruktur. Diese Heterogenität betrifft einige Fachrichtungen mehr, andere weniger, ist jedoch insgesamt sehr präsent und führt vielfach zu unterrichtlichen und außerunterrichtlichen Problemen, auf die im Forschungsteil noch näher einzugehen sein wird. Hinsichtlich der Altersstruktur, der allgemeinbildenden Voraussetzungen und Schulabschlüsse sowie der beruflichen Vorbildung sind alle denkbaren Konstellationen vertreten. Während derzeit bei den Diätassistenten, Krankenpflegehelfern, Physiotherapeuten und Rettungsassistenten/Notfallsanitätern mit Ausnahmen Erstauszubildende zu verzeichnen sind, ist dies bei Altenpflegern, Erziehern und Heilerziehungspflegern nicht der Fall. Insgesamt steigt jedoch der Anteil an Schülern mit anderer beruflicher Vorbildung oder langjähriger fachfremder Berufstätigkeit und demzufolge mit höherem Lebensalter, bis hinein ins 4. oder 5. Lebensjahrzehnt, an. Die Altenpflege ist bereits langjährig als klassischer Umschulungsberuf bekannt, so dass dort ebenfalls junge Erstauszubildende mit älteren Umschülern, die durch die Arbeitsagenturen gefördert werden, vertreten sind. Im Bereich der Fachschule, welche schulrechtlich dem Weiterbildungsbereich zugeordnet wird (vgl. Punkt \ref{sec:BerufsbildendeSchulen}), müssen alle Schüler einen förderlichen Grundberuf oder einen nicht förderlichen mit entsprechender beruflicher Tätigkeit nachweisen, so dass auch hier die Altersstrukturen und Vorkenntnisse sehr verschieden sind. Es ist deutlich zu beobachten, dass es Schülern, die die allgemeinbildende Schule schon viele Jahre oder Jahrzehnte verlassen haben, oft schwer fällt, sich wieder in der Schülerrolle zurecht zu finden und erfolgreiche Lernstrategien anzuwenden bzw. ihre Lernmotivation aufrecht zu erhalten.

Auch in Bezug auf die allgemeinen Schulabschlüsse, welche Voraussetzungen für den Zugang zu den einzelnen Berufen sind, gibt es eine breite Varianz. Für die Fachrichtungen Krankenpflegehilfe und Rettungsassistenz (auslaufend) wird der Hauptschulabschluss oder jeder höherwertige Schulabschluss zugrunde gelegt, für die Fachrichtungen Altenpflege, Diätassistenz, Notfallsanitäter und Physiotherapeuten mindestens der mittlere Bildungsabschluss. Auch bei Erziehern und Heilerziehungspflegern gibt es dahingehend eine Streuung in der gesamten Bandbreite, da die jeweiligen individuellen Vorberufe oder sogar Studienabschlüsse die notwendige allgemeine schulische Vorbildung definieren. Demzufolge ist das Leistungsniveau in einzelnen Klassen sehr unterschiedlich und Probleme mit leistungsbedingter Über- und Unterforderung sind durchaus sehr präsent. Beide Aspekte, also sowohl die Altersstruktur und damit verbundenen verschiedenen Erfahrungen, Werte und Einstellungen, als auch die Varianz hinsichtlich der schulischen Voraussetzungen stellen für Lehrkräfte und Schüler gleichermaßen eine große Herausforderung dar. Soziale und persönliche Unterschiede und Problemlagen sind selbstverständlich im DRK Bildungswerk SN genauso wie in der Gesamtgesellschaft repräsentiert und wirken ebenfalls erheblich in das Unterrichtsgeschehen ein. 

Zusätzlich zu den oben genannten Schwerpunkten sind auch finanzielle Aspekte bei der Analyse der Schülerstruktur nicht außer Acht zu lassen. Das DRK Bildungswerk SN als freier Schulträger erhebt für die einzelnen Ausbildungen Schulgelder in unterschiedlicher Höhe. In den vollzeitschulischen Ausbildungsgängen Diätassistenz, Erzieher, Heilerziehungspflege, Krankenpflegehilfe, Physiotherapie und Rettungsassistenz verfügen die Schüler in der Regel über kein Einkommen und erhalten keine Ausbildungsvergütung. Zahlreiche Schüler finanzieren die Ausbildung über ihre Eltern bzw. Partner oder erhalten, in Abhängigkeit von der Einkommenssituation der Familie, Schüler- oder Meister-BAföG (nur Erzieher und Heilerziehungspfleger). Die Schüler in der Altenpflege und dem Bereich Notfallsanitäter haben einen Arbeitgeber ("`quasi-duale Ausbildung"'), ebenso wie die Teilnehmer der berufsbegleitenden Ausbildungsgänge bei Erziehern und Heilerziehungspflegern. Um eine staatliche Förderung von Schülern in einzelnen Fachrichtungen zu gewährleisten, unterhält das DRK Bildungswerk SN seit 2009 ein QM-System nach DIN EN ISO 9001-2008 und ist zertifizierter Träger nach AZAV (Akkreditierungs- und Zulassungsverordnung Arbeitsförderung). Somit besteht eine Berechtigung für einzelne zertifizierte Ausbildungsmaßnahmen (Altenpfleger Vollzeit, Altenpfleger verkürzt, Erzieher Vollzeit, Erzieher verkürzt, Rettungsassistenten Vollzeit) Bildungsgutscheine der Arbeitsagenturen entgegen zu nehmen. Schüler mit Bildungsgutschein erhalten Leistungen zur Sicherung ihres Lebensunterhaltes von den Arbeitsagenturen, müssen aber in der Regel die Ausbildung im angestrebten Beruf um ein Drittel verkürzen, was wiederum zu fachlichen Kompetenzproblemen und Überforderung führen kann.

Zur Unterstützung der Schüler hinsichtlich ihrer schulischen und außerschulischen (noch näher zu erforschenden und einzugrenzenden) Problemlagen gibt es derzeit wenige bis keine Angebote im DRK Bildungswerk SN. In einzelnen Fachrichtungen werden sporadisch themen- bzw. fächerbezogene Nachhilfeangebote gemacht, teilweise unterstützen sich die Schüler auch gegenseitig durch Lern- oder praktische Übungsgruppen außerhalb der Schulzeit. Weiterhin stehen allen Schülern ihre Klassenlehrer und auch Fachbereichsleiter oder Schulleiter für fachliche und persönliche Fragen in unterschiedlichem Maße zur Verfügung. In Abhängigkeit von der Größe des jeweiligen Fachbereiches und der Anzahl der Schüler können sich Lehrkräfte und Leiter in unterschiedliche Qualität und Quantität um "`ihre"' Schüler kümmern, wobei jedoch auch immer das persönliche Verhältnis eine große Rolle spielt. Zusätzlich ist eine für die gesamte Bildungseinrichtung zuständige Vertrauenslehrerin eine mögliche Ansprechpartnerin für alle Schüler. Nach eigenen Aussagen treten jedoch bisher nur wenige Schüler an sie heran, was sich auch mit theoretischen Befunden zur Nutzung von Vertrauenslehrer-Angeboten deckt und ursächlich im Rahmen der eigenen Forschungsarbeit untersucht werden soll. Angebote der klassischen Schulsozialarbeit gibt es derzeit nicht. Hausintern können Lehrkräfte und Schüler jedoch zusätzliche Möglichkeiten der Beratung und Betreuung bei Bedarf nutzen (z. B. Schwangerenberatung und psychologische Beratung im DRK Landesverband, Bereich soziale Arbeit).  

\subsection{Perspektiven}
\label{sec:Perspektiven}

Abschließend zur Vorstellung der Bildungseinrichtung soll ein kurzer Blick in die Zukunft die Ausführungen abrunden. Ab dem Schuljahr 2015/16 plant das DRK Bildungswerk SN das Portfolio der Ausbildungsberufe um den Bildungsgang Sozialassistenten zu erweitern. Die dazugehörige Berufsfachschule befindet sich derzeit in der Genehmigungsphase. Weiterhin ist der Aufbau zusätzlicher Schulstandorte mit unterschiedlichen Fachrichtungen in den Regionen Leipzig und Chemnitz geplant, um hier insbesondere den Anforderungen an die Aus-, Fort- und Weiterbildung von Fachpersonal der DRK Kreisverbände gerecht zu werden.

Durch die Geschäftsführung und die Schulleitungen wurde in den zurückliegenden Monaten verstärkt die Erkenntnis gewonnen, dass sich derzeit weitreichende Veränderungen in der Schülerstruktur, im fachlichen wie auch im sozialen und persönlichen Bereich ergeben, die zunehmend in die Unterrichtsvorbereitung, -durchführung und –nachbereitung hineinwirken und verschiedene Ressourcen der Lehrkräfte zunehmend binden. Daher gab es bereits Überlegungen, ob und in welcher Form unterstützende Maßnahmen für die Schüler angeboten werden können. Diese sollen vertieft und ausgebaut und in den nächsten Monaten in durchführungsfähige Angebote und Projekte umgewandelt werden. Ergänzend befindet sich ein sogenanntes "`Development-Center"' in Kooperation mit der TU Dresden in Planung, wodurch eine Eingangsdiagnostik hinsichtlich fachlicher, sozialer und methodischer Ressourcen und Schwächen bereits im Bewerbungsverfahren erprobt werden soll. Damit wird anvisiert, über mehrere Testverfahren die individuellen fachlichen und persönlichen Unterstützungsbedarfe der Schüler frühzeitig in Erfahrung zu bringen, um mit möglichen Unterstützungs- und Beratungsangeboten Problemen entgegenzuwirken oder diese bearbeiten zu können. Zusätzlich soll auch der Anteil der für die Ausbildungsberufe ungeeigneten Schüler reduziert werden, indem ihre fehlende Eignung oder Motivation bereits im Bewerbungsverfahren erkannt wird. Im Rahmen einer noch konzeptionell auszuarbeitenden Form der Berufsorientierung könnten diese Bewerber hinsichtlich anderer Berufsausbildungen oder Vorbereitungsmaßnahmen beraten werden. Die Ergebnisse der vorliegenden Arbeit werden in die ausgeführten Perspektivprojekte einfließen und hoffentlich zu deren Gelingen beitragen. 