\section{Projektdurchführung}
\label{sec:Projektdurchführung}

\subsection{Durchführung der Schülerumfrage}
\label{sec:DurchführungDerSchülerumfrage}

Im Anschluss an Konzeption und Pilotierung erfolgte im Juni 2015 schließlich der praktische Einsatz der Schülerumfrage am DRK Bildungswerk SN mittels der "`Umfrage zum Bedarf außerunterrichtlicher Unterstützungsangebote am DRK Bildungswerk Sachsen"' (siehe Punkt \ref{sec:Schülerumfrage}). Wie zuvor bemerkt, wurden alle Lehrkräfte bereits im Vorfeld über Sinn und Zweck der geplanten Untersuchung per E-Mail aufgeklärt. Die Schüler selbst wurden nicht kontaktiert. 

Aufgrund ausbildungsbezogener schulexterner Praktika, wurden die im Zeitraum der geplanten Umfrage zur Verfügung stehenden verschiedenen Ausbildungsklassen bereits vor Beginn der Schülerbefragung durch die Autorinnen eruiert. Alle zu berücksichtigenden Ausbildungsrichtungen konnten dank dieser Planung innerhalb von ca. 4 Wochen problemlos befragt werden. Die Auswahl der Klassen erfolgte dabei rein zufällig. Aus organisatorischen Gründen wurden die Lehrkräfte, welche zum Zeitpunkt der Befragung in den ausgewählten Klassen unterrichtend tätig waren, zuvor bereits über unser Erscheinen in Kenntnis gesetzt. Bei Bedarf wurde die Umfrage so zu Beginn der Unterrichtsstunde oder am Ende dieser durchgeführt. Der Unterrichtsbeginn stellte sich jedoch aus Gründen der Qualitätssicherung als zweckdienlicher heraus, da die Schüler zu diesem Zeitpunkt aufnahmefähiger und motivierter erschienen, was zum Ende der Stunde eher nicht der Fall war. Diese Erkenntnis hatte zur Folge, dass im Anschluss versucht wurde, die restlichen Klassen stets zu Unterrichtsbeginn zu befragen. Alle involvierten Lehrer erwiesen sich dabei als aufgeschlossen, freundlich, verständnisvoll und interessiert am Forschungsgegenstand.  Einige offerierten zudem klares Interesse am Ergebnis der Untersuchung. 

Die Befragung einer Klasse wurde stets durch eine Autorin durchgeführt, wobei sich beide Verfasserinnen dieser Tätigkeit annahmen.
Die Schüler wurden zunächst über die eigene Person und das Forschungsansinnen informiert. Desweiteren wurde die Freiwilligkeit der Teilnahme und die anonymisierte Verarbeitung der erhobenen Daten für rein wissenschaftliche Zwecke thematisiert. Die zu befragenden Schüler wurden anschließend explizit nach ihrem Einverständnis hinsichtlich ihrer Teilnahme am Projekt befragt. Alle Schüler waren mit der Umfragedurchführung und anschließenden anonymisierten Verarbeitung der erhobenen Daten einverstanden. Nur ein Schüler entschied sich am Ende der Umfragebearbeitung, ohne Angabe von Gründen, gegen eine Abgabe des Fragebogens. Dieser Umstand wurde respektiert. Nach Einwilligung zur Teilnahme wurden die Schüler in wenigen Worten in die Teilbereiche der Umfrage eingeführt. Diese Vorgehensweise beruhte auf Erfahrungen des Pretest und sollte den Schülern das leichtere Erfassen und Beantworten der Fragen ermöglichen. Insbesondere die Unterscheidung in Hinblick auf die Einschätzung persönlicher Problemlagen und Beratungs- und Unterstützungsbedarfe von der Einschätzung der Problemlagen und Bedarfe anderer Schüler am DRK Bildungswerk Sachsen wurde betont. Zudem wurde auf die Varianz in den Antwortmöglichkeiten hingewiesen. Bevor schließlich mit der Umfrage begonnen werden konnte, wurde der Merkzettel zur Studie an die Schüler ausgegeben; dieser beinhaltet eine kleine Zusammenfassung zum Forschungsprojekt und die Kontaktdaten der Autorinnen in Form der E-Mail-Anschrift (siehe Punkt \ref{sec:Schülerumfrage}). Sollten die Schüler im Nachhinein Fragen zum Forschungsprojekt entwickeln, könnten sie sich so an die Referentinnen wenden. Diese Option wurde bis zur Fertigstellung der vorliegenden Arbeit nicht genutzt. Jetzt konnte die Umfrage beginnen. Bei Fragen oder Verständnisschwierigkeiten konnten sich die Schüler jederzeit an den Referenten wenden. Die Beantwortung der Umfrage dauerte je nach Klasse zwischen 10-20 Minuten und erfolgte zumeist in Stillarbeit. Alle involvierten Schüler wirkten wohlwollend und motiviert im Geschehen. Probleme gab es keine.

Nach Beendigung der Umfrage wurden die Umfragebögen eingesammelt und gezählt, um sich so einen aktuellen Stand über die notwendigen Umfragebögen pro Ausbildungsrichtung zu verschaffen.

\subsection{Durchführung der Lehrerinterviews}
\label{sec:DurchführungDerLehrerinterviews}

Ab Mitte Juni 2015 konnten die 5 individuell vereinbarten Termine (I.01 - I.05) mit den Lehrkräften am DRK Bildungswerk SN innerhalb von 2 Wochen durchgeführt werden. Es ist zu betonen, dass dieses Engagement freiwillig und außerberuflich erfolgte und keinerlei positive wie negative Folgen für die Personen hatte. Die den verschiedenen Ausbildungsrichtungen zugehörigen Lehrkräfte wiesen Unterschiede in Geschlecht, Alter, beruflicher Zugehörigkeit, familiären Hintergrund, Erfahrungswissen und in der Dauer des Angestelltenverhältnisses am DRK Bildungswerk SN auf. Es handelte sich dabei um drei Damen und zwei Herren im Alter von 26 bis 45 Jahren. Die betreffenden Personen unterrichteten zum Zeitpunkt der Befragung im Bereich der Physiotherapie, Krankenpflegehilfe, Heilerziehungspflege, Erzieher, Altenpflege aber auch bei den Notfallsanitätern und Rettungsassistenten. Somit waren den Interviewpartnern insgesamt viele verschiedene Ausbildungsklassen bekannt, welche bei den Ausführungen berücksichtigt werden konnten.

Die Interviews fanden in einem Seminar- bzw. Konferenzraum am DRK Bildungswerk SN statt und verliefen ungestört.
Neben dem betreffenden Interviewpartner, waren stets zwei Personen in Gestalt des Interviewers (Doreen Stichel) und des Beisitzers (wechselnd besetzt von Juliane Hemmerling, Anne Krause und Daniela Wobst) im Raum anwesend. Diese Konstellation ergab sich aus der "`Doppelbelegung"' der ersten 3 Interviews. I.01 - I.03 wurden im Rahmen der universitären Forschungsgruppe durchgeführt und ausgewertet. Die Ergebnisse dieser 3 Interviews wurden daher sowohl für die Gruppenseminararbeit als auch für die vorliegende Masterarbeit genutzt. Die verbleibenden 2 Interviews (I.04 - I.05) oblagen in Durchführung und Auswertung alleinig den Autorinnen. Daniela Wobst hat im Vorfeld darauf hingewiesen, nur bedingt Teil der Interviewdurchführung sein zu wollen, aufgrund ihrer persönlichen Bindung zu allen Beteiligten. Eine ungewollte persönliche Beeinflussung der Interviewpartner sollte so reduziert werden.

Alle Lehrkräfte haben einer Befragung freiwillig und ohne Druck zugestimmt und wurden im Vorfeld über die anonymisierte und wissenschaftliche Verwendung ihrer Daten aufgeklärt. Die betreffende Einverständniserklärung sowie die Datenschutzerklärung zum Projekt wurden vor Gesprächsbeginn von den Interviewteilnehmern gelesen und gegengezeichnet. Die Interviews wurden mit Zustimmung der Lehrkräfte per Smartphone aufgezeichnet. 

Nach Begrüßung und Aufklärung der Teilnehmer erfolgte zunächst die Aufnahme der soziodemografischen Daten ehe das eigentliche Interview begann. Alle Interviews (I.01 - I.05) waren von einer offenen, freundlichen und aufgeschlossenen Atmosphäre geprägt, wobei Unterschiede in der Beantwortung der Fragen, abhängig von Persönlichkeit und Interesse der Gesprächspartner erkennbar waren. Eine Lehrperson formulierte eher knappe Antworten, wohingehend ein Anderer in seinen Aussagen deutlich mehr zu erzählen hatte; manche Personen waren im Gespräch besonders offen, engagiert und interessiert, andere hingegen zurückhaltender und eher distanziert. Nichtsdestotrotz verliefen alle Interviews sehr angenehm und waren stets von gegenseitigem Respekt geprägt. Die Dauer der Gespräche variierte von 30-60 Minuten. Nach Beendigung und Verabschiedung der Interviewteilnehmer wurde ein Postscript ausgefüllt, in welchem Aspekte wie Stimmung, Sympathie, der erste Eindruck aber auch Ideen zur Beantwortung der Forschungsfragen vermerkt wurden. Alle verwendeten Unterlagen, welche bei der Durchführung der Interviews genutzt wurden, sind dem Anhang zu entnehmen (siehe Punkt \ref{sec:Lehrerinterviews}).