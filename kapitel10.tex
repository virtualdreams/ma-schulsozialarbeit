\section[Konzeptionelle Überlegungen für das DRK Bildungswerk Sachsen]{Aus dem Forschungsprojekt resultierende konzeptionelle Überlegungen für Beratungs- und Unterstützungsangebote am DRK Bildungswerk Sachsen}
\label{sec:AusDemForschungsprojektResultierendeKonzeptionelleÜberlegungenFürBeratungsUndUnterstützungsangeboteAmDRKBildungswerkSachsen}

Wie bereits der Titel der vorliegenden Arbeit vermuten lässt, kann und soll an dieser Stelle keine Konzeption im Sinne eines planungs- und durchführungsfähigen Angebotes für das DRK Bildungswerk SN erarbeitet werden. Ob überhaupt die hier aufgezeigten Schwerpunkte in irgendeiner Art und Weise zur Anwendung gebracht werden können und die Ergebnisse des beschriebenen Forschungsvorhabens dazu genutzt werden, hängt von vielen institutionellen Rahmenbedingungen ab und muss im Vorfeld einer möglichen Konzeption genau überdacht werden. Die hier zu machenden Ausführungen können dazu Denkanstöße geben und bedenkenswerte Optionen, auch unter Einbeziehung von Erfahrungswerten anderer berufsbildender Schulen, aufzeigen. Selbstverständlich sollen dabei die eruierten Ergebnisse der Schüler- und Lehrerbefragungen berücksichtigt werden. Nicht berücksichtigt werden an dieser Stelle die von den Lehrkräften benannten Ideen und Möglichkeiten fachlicher Unterstützung der Schüler, da der Schwerpunkt der Arbeit von vornherein im Bereich persönlicher und sozialer Problemlagen und außerunterrichtlicher Angebote lag. 

Wie in den theoretischen Ausführungen zur Schulsozialarbeit bereits angeklungen, liegt der Schwerpunkt möglicher Angebote im berufsbildenden Bereich tendenziell in Varianten der Beratung und Unterstützung, methodisch umgesetzt mittels Einzelfallhilfe, die mit verschiedenen Formen von problembezogener Beratung und individueller Begleitung der Schüler einen hohen Stellenwert einnimmt. Zusätzlich wäre auch die Methode der sozialen Gruppenarbeit, insbesondere bei klassen- und gruppenbezogenen Problemstellungen, wie z. B. Mobbing, möglich. Anhand der Fülle der ermittelten Problemlagen bei Schülern im DRK Bildungswerk SN und dem wahrgenommenen Bedarf für Unterstützung und Beratung sollten diese beiden Methoden nicht unberücksichtigt bleiben \footcite[vgl.][10ff]{LSS2004}. Die Einzelfallhilfe hat vor allem die Funktion einer Begleitung bei der Gestaltung des Übergangs in das Arbeitsleben und bei der Lösung individueller Konflikte und Defizite \footcite[vgl.][74]{Stuewe2015}. Konzeptionell ist also am ehesten von einem oder mehreren möglichen Projekten mit einer problembezogenen fürsorgerischen Ausrichtung auszugehen (siehe Punkt \ref{sec:Konzeptionen}). Die Zielgruppe bilden daher vorrangig (sozial) benachteiligte und/oder individuell beeinträchtigte Schüler mit individuellen Problemlagen \footcite[vgl.][25f]{Speck2006}. 

Wer möglicherweise die Einzelfallhilfe oder soziale Gruppenarbeit bzw. außerunterrichtliche Beratungs- und Unterstützungsangebote im DRK Bildungswerk SN implementieren und anbieten könnte, kann aktuell nicht vorausgesagt werden. Da die Einstellung eines ausgebildeten Schulsozialarbeiters derzeit nicht geplant ist, wäre es ratsam, diese Haltung entweder zu überdenken oder die Einzelfallhilfe in Form von außerunterrichtlichen Unterstützungs- und Beratungsangeboten bspw. durch interessierte und speziell fortgebildete Lehrkräfte anzubieten bzw. vorerst modellhaft zu erproben. In diesem Zusammenhang ist durchaus zu berücksichtigen, dass die Annahme solcher Angebote und die Aufnahme von Verbindungen zu Beratungs- oder Vertrauenslehrern, nach den Erfahrungen aus Berichten oder Evaluationen von Schulsozialarbeit, nicht immer besonders hoch sind \footcite[vgl.][17f]{LSS2004}. Die Gründe dafür sind nicht bekannt, allerdings weisen die Ergebnisse der Schülerbefragung, in denen ein Großteil der Schüler die Nutzung von Beratungs- und Unterstützungsangeboten und Angeboten der Vertrauenslehrer im schulischen Kontext für sich selbst ablehnt, auf eine vergleichbare Situation im DRK Bildungswerk SN hin. 

Hinsichtlich der Erreichbarkeit sprach sich ein Großteil der  befragten Schüler für eine ständige Erreichbarkeit im Schulalltag aus, was ohnehin mit institutionell bedingten Schwierigkeiten behaftet ist und bei einer Realisierung von Angeboten mittels fortgebildeter Lehrkräfte nahezu unmöglich sein dürfte. Keinesfalls empfohlen wird für den berufsbildenden Bereich aber auch eine Ausdehnung von Angeboten in den  Nachmittag \footcite[vgl.][17f]{LSS2004}.

\begin{quotation}
\noindent
"`Einige Schulsozialarbeiter beklagten einen Interessenkonflikt, da die Lehrkräfte die Schüler während ihres Unterrichts ungern zur Beratung gehen lassen und die Schüler hingegen ihre Freizeit nicht unbedingt für das Beratungsangebot "`opfern"' wollen."' \footcite[93]{Ganser2004}
\end{quotation}

\noindent
Solche Ergebnisse sollten in die konzeptionellen Überlegungen in jedem Falle einfließen. Dennoch sprechen die Befunde zur sozialpädagogischen Arbeit immer wieder auch dafür, niederschwellige Angebote mit verschiedensten Möglichkeiten der Kontaktaufnahme durch Schüler zu realisieren, keine festen Sprechzeiten anzubieten, sondern eher eine dahingehend offene Kultur zu pflegen und breite Präsenz innerhalb der Schule zu zeigen \footcite[vgl.][48]{Essers2012}. Diese Erkenntnisse sprechen deutlich gegen eine Übernahme der Beratungs- und Unterstützungsfunktion durch Lehrkräfte was durch die Ergebnisse der Schülerbefragung (Angebote sollten vorwiegend von schulfremden Beratern mit ständiger Erreichbarkeit während des Schultages gemacht werden) unterstützt wird. .

Die Themen Vertrauen, Vertraulichkeit und Schweigepflicht scheinen für den berufsbildenden Bereich eine besondere Bedeutung zu besitzen, was die Ergebnisse der Schüler- und Lehrerbefragung verdeutlichen und andere Autoren ebenso bestätigen \footcite[vgl.][49]{Essers2012}. Hier könnten verschiedene vertrauensbildende Maßnahmen, wie sie bspw. von den befragten Lehrkräften in Form von erlebnispädagogischen Angeboten für ganze Klassen beschrieben wurden, durchaus hilfreich sein, zumindest wenn Vertrauensprobleme und Konflikte innerhalb des Klassengefüges auftauchen. Jedoch ist auch das derzeitige Angebot einer Vertrauenslehrerin, die einem überwiegenden Teil der Schülerschaft nicht bekannt zu sein scheint bzw. bei persönlichen Problemlagen nicht angesprochen werden würde, zu überdenken. Den Bekanntheitsgrad, z. B. mittels Besuchen und Vorstellungen in den einzelnen Klassen zu erhöhen, erscheint unproblematisch. Beachtenswert ist jedoch auch der Ansatz, die Vertrauenslehrertätigkeit auf mehrere Schultern zu verteilen und aus den einzelnen Fachbereichen Vertrauens- oder Beratungslehrer zu generieren. Diese sollten selbstverständlich die Aufgabe freiwillig übernehmen und dafür zeitliche sowie räumliche Ressourcen, z. B. in Form von Abminderungsstunden zur Verfügung gestellt bekommen. Denkbar wäre es auch, mehrere Vertrauens- und Beratungslehrer in den Fachbereichen für die Schüler zur Wahl zu stellen, um eine frühzeitige Beteiligung der Lernenden zu gewährleisten und das Problem des Bekanntheitsgrades von vornherein zu umgehen. 

Das Thema Netzwerkarbeit scheint für mögliche Beratungs- und Unterstützungsangebote ebenfalls ein wichtiges zu sein, welches auf mehreren Ebenen betrachtet werden kann. Einige Lehrkräfte betonten in den Interviews die teilweise notwendige Weitervermittlung von Schülern an andere Institutionen bzw. Kooperationspartner. Diese können sehr zahlreich sein und von Ämtern, über Beratungsstellen, Psychologen und Ärzte bis hin zu Partnern der praktischen Ausbildung reichen \footcites[vgl.][49ff]{Essers2012}[vgl.][21]{NiedersaechsischesKultusministerium2004}.

\begin{quotation}
\noindent
"`Der Aufbau eines solchen Netzwerkes mit außerschulischen Partnern erfordert Kontaktfreudigkeit, Flexibilität und Organisationsfähigkeit."' \footcite[21]{NiedersaechsischesKultusministerium2004}
\end{quotation}

\noindent
In erster Linie bedarf diese Netzwerkarbeit aber auch zeitlicher Ressourcen, wenn sie gewinnbringend für beratungs- und unterstützungsbedürftige Schüler sein soll, weshalb hier wiederum die Leistbarkeit solcher Angebote durch Lehrkräfte im DRK Bildungswerk SN neben der unterrichtlichen Tätigkeit in Frage gestellt werden muss. Bezogen auf die bessere und intensivere Zusammenarbeit mit Praxiseinrichtungen, die auf Basis der von Schülern angegebenen zahlreichen Problemlagen in Bezug auf die praktische Ausbildung sowie auf der von Lehrern dargelegten Bedarfe erfolgen sollte, sind jedoch ausgewählte und erfahrene Lehrkräfte als  Netzwerkpartner zu befürworten.

Hinsichtlich möglicher sozialer Gruppenarbeit als schulsozialarbeitsbezogener Methode haben die durchgeführten Befragungen (speziell durch die Themen Mobbing und Konflikte innerhalb der Klassen) durchaus entsprechende Bedarfe offengelegt. Dazu wäre es z. B. vorstellbar eine Fachkraft bzw. eine speziell fortgebildete Lehrkraft in Form eines "`Mobbingbeauftragten"' im DRK Bildungswerk SN zu implementieren oder auch externe Angebote in Anspruch zu nehmen. Alle literaturbezogenen Befunde weisen darauf hin, dass das Thema Mobbing, direkt oder über soziale Netzwerke, immer größere Bedeutung erlangt und Lehrkräfte im Umgang damit häufig an ihre Grenzen stoßen. Mit bestimmten Methoden der sozialen Gruppenarbeit könnten von einer spezialisierten Fachkraft im unterrichtlichen Rahmen thematische Einheiten gestaltet und Konflikte bearbeitet werden. Vorstellbare Effekte davon könnten positive Auswirkungen auf den Unterricht und den Lernerfolg der Schüler sowie besseres Klassenklima und die Förderung der Sozialkompetenz, als wichtige berufsbezogene Kompetenz, sein \footcites[vgl.][51]{Essers2012}[vgl.][20]{NiedersaechsischesKultusministerium2004}. Weiterhin wäre die interdisziplinäre Projektarbeit eine Methode der sozialen Gruppenarbeit, die durch Fachkräfte oder weitergebildete Lehrkräfte innerhalb des DRK Bildungswerk SN positive Entwicklungsmöglichkeiten für die Schüler bereithalten könnte. 

Nach den bisher genannten Möglichkeiten für außerunterrichtliche Beratungs- und Unterstützungsangebote sollen an dieser Stelle noch einige Ideen und Ansätze benannt werden, die den Befragungen entstammen, jedoch weniger Theoriebezüge aufweisen.

Besonders naheliegend wäre es, im DRK Bildungswerk SN innerverbandliche Ressourcen zu nutzen, die bspw. im Fachbereich Jugendhilfe des DRK Landesverbandes zu finden sind. Die fachliche Expertise des Referates soziale Arbeit könnte eine wertvolle Hilfe bei Entscheidungs- und Umsetzungsprozessen hinsichtlich möglicher Angebote leisten. Weiterhin wäre auch eine Intensivierung der Zusammenarbeit im Sinne einer verstärkten Netzwerkarbeit mit DRK-Einrichtungen zu befürworten. 

In den Interviews mit Lehrkräften wurden immer wieder Fortbildungsbedarfe thematisiert, die sinnvolle Unterstützung für die unterrichtenden Berufspädagogen sein könnten. Insbesondere die Themen Supervision und kollegiale Fallberatung kamen mehrfach zur Sprache. Da davon auszugehen ist, dass rein schulorganisatorisch bedingt, Klassenlehrer meist die ersten Ansprechpartner von Schülern bei Problemen sind und es auch bleiben werden, sind solche Fortbildungen ausdrücklich zu befürworten. In den Interviews zeigt sich, dass Lehrkräfte allein schon durch die unterrichtlichen Verpflichtungen in sehr engem Kontakt mit Schülern stehen und mit deren Problemlagen konfrontiert werden. Das würde sich auch nicht ändern, wenn die Zuständigkeiten und personellen Ressourcen innerhalb des DRK Bildungswerkes SN eine "`Überweisung"' oder "`Weiterleitung"' von "`Problemfällen"' zur Bearbeitung ermöglichen würden. Daher sollten entsprechende Fortbildungsmöglichkeiten und die regelmäßige Implementierung des kollegialen Austausches nicht unberücksichtigt bleiben.

Im Abschnitt \ref{sec:Perspektiven} wurde das geplante "`Development-Center"' im Sinne einer strukturierten und kompetenzbezogenen Schülerauswahl bereits kurz vorgestellt. Dadurch könnten sich positive Aspekte hinsichtlich der Prävention von Orientierungslosigkeit und Demotivation der Schüler ergeben. Einige Lehrkräfte gaben in den Interviews an, dass Schüler vermehrt Berufe im Bereich von Gesundheit, Pflege und Sozialwesen ergreifen, ohne dass sie selbst das eigentlich wollen oder wissen, welche Kompetenzen, Fähigkeiten und Fertigkeiten sowie beruflichen Tätigkeiten eigentlich von ihnen verlangt werden. Insbesondere in diesem speziellen Feld, wo Beruf immer auch ein Stück "`Berufung"' sein muss, um den Anforderungen gerecht zu werden, sind solche Entwicklungen fatal und das "`Absitzen"' von Unterrichtszeiten sowie das Erlernen eines Berufes, weil die Eltern diesen als "`arbeitsmarktsicher"' (vgl. Lehrerinterviews) erachten, nicht förderlich. Daher könnten eine theoriegeleitete  und verbesserte Schülerauswahl und die Orientierung auf eventuell passendere Berufe durchaus positive Effekte nach sich ziehen, sofern die Bewerbersituation dies haushaltsbedingt überhaupt zulässt. 

Nunmehr sind mögliche, für die Verfasserinnen relevante, Ansätze für außerunterrichtliche Beratungs- und Unterstützungsangebote am DRK Bildungswerk SN dargelegt, welche z. T. durch theoretische Grundlagen der Schulsozialarbeit begründet werden konnten. Eine Grundsatzentscheidung sowie die Auswahl möglicher praktikabler Ansätze und die Konzeption von Angeboten obliegen nun der Geschäftsführung, wobei die Entscheidung für oder gegen eine sozialpädagogische Fachkraft bzw. einen Schulsozialarbeiter eine der relevantesten sein wird. Die Verfasserinnen plädieren nach der umfangreichen Beschäftigung mit der Thematik ausdrücklich dafür. 

